%%
%% Beginning of file 'sample62.tex'
%%
%% Modified 2018 January
%%
%% This is a sample manuscript marked up using the
%% AASTeX v6.2 LaTeX 2e macros.
%%
%% AASTeX is now based on Alexey Vikhlinin's emulateapj.cls 
%% (Copyright 2000-2015).  See the classfile for details.

%% AASTeX requires revtex4-1.cls (http://publish.aps.org/revtex4/) and
%% other external packages (latexsym, graphicx, amssymb, longtable, and epsf).
%% All of these external packages should already be present in the modern TeX 
%% distributions.  If not they can also be obtained at www.ctan.org.

%% The first piece of markup in an AASTeX v6.x document is the \documentclass
%% command. LaTeX will ignore any data that comes before this command. The 
%% documentclass can take an optional argument to modify the output style.
%% The command below calls the preprint style  which will produce a tightly 
%% typeset, one-column, single-spaced document.  It is the default and thus
%% does not need to be explicitly stated.
%%
%%
%% using aastex version 6.2
\documentclass{aastex62}

\usepackage{graphicx}	% Including figure files
\usepackage{amsmath}	% Advanced maths commands
\usepackage{amssymb}	% Extra math symbols
\usepackage{capt-of}

%% The default is a single spaced, 10 point font, single spaced article.
%% There are 5 other style options available via an optional argument. They
%% can be envoked like this:
%%
%% \documentclass[argument]{aastex62}
%% 
%% where the layout options are:
%%
%%  twocolumn   : two text columns, 10 point font, single spaced article.
%%                This is the most compact and represent the final published
%%                derived PDF copy of the accepted manuscript from the publisher
%%  manuscript  : one text column, 12 point font, double spaced article.
%%  preprint    : one text column, 12 point font, single spaced article.  
%%  preprint2   : two text columns, 12 point font, single spaced article.
%%  modern      : a stylish, single text column, 12 point font, article with
%% 		  wider left and right margins. This uses the Daniel
%% 		  Foreman-Mackey and David Hogg design.
%%  RNAAS       : Preferred style for Research Notes which are by design 
%%                lacking an abstract and brief. DO NOT use \begin{abstract}
%%                and \end{abstract} with this style.
%%
%% Note that you can submit to the AAS Journals in any of these 6 styles.
%%
%% There are other optional arguments one can envoke to allow other stylistic
%% actions. The available options are:
%%
%%  astrosymb    : Loads Astrosymb font and define \astrocommands. 
%%  tighten      : Makes baselineskip slightly smaller, only works with 
%%                 the twocolumn substyle.
%%  times        : uses times font instead of the default
%%  linenumbers  : turn on lineno package.
%%  trackchanges : required to see the revision mark up and print its output
%%  longauthor   : Do not use the more compressed footnote style (default) for 
%%                 the author/collaboration/affiliations. Instead print all
%%                 affiliation information after each name. Creates a much
%%                 long author list but may be desirable for short author papers
%%
%% these can be used in any combination, e.g.
%%
%% \documentclass[twocolumn,linenumbers,trackchanges]{aastex62}
%%
%% AASTeX v6.* now includes \hyperref support. While we have built in specific
%% defaults into the classfile you can manually override them with the
%% \hypersetup command. For example,
%%
%%\hypersetup{linkcolor=red,citecolor=green,filecolor=cyan,urlcolor=magenta}
%%
%% will change the color of the internal links to red, the links to the
%% bibliography to green, the file links to cyan, and the external links to
%% magenta. Additional information on \hyperref options can be found here:
%% https://www.tug.org/applications/hyperref/manual.html#x1-40003
%%
%% If you want to create your own macros, you can do so
%% using \newcommand. Your macros should appear before
%% the \begin{document} command.
%%
\newcommand{\vdag}{(v)^\dagger}
\newcommand\aastex{AAS\TeX}
\newcommand\latex{La\TeX}

%% Tells LaTeX to search for image files in the 
%% current directory as well as in the figures/ folder.
\graphicspath{{./}{figures/}}

%% Reintroduced the \received and \accepted commands from AASTeX v5.2
\received{January 1, 2018}
\revised{January 7, 2018}
\accepted{\today}
%% Command to document which AAS Journal the manuscript was submitted to.
%% Adds "Submitted to " the arguement.
\submitjournal{ApJ}

%% Mark up commands to limit the number of authors on the front page.
%% Note that in AASTeX v6.2 a \collaboration call (see below) counts as
%% an author in this case.
%
%\AuthorCollaborationLimit=3
%
%% Will only show Schwarz, Muench and "the AAS Journals Data Scientist 
%% collaboration" on the front page of this example manuscript.
%%
%% Note that all of the author will be shown in the published article.
%% This feature is meant to be used prior to acceptance to make the
%% front end of a long author article more manageable. Please do not use
%% this functionality for manuscripts with less than 20 authors. Conversely,
%% please do use this when the number of authors exceeds 40.
%%
%% Use \allauthors at the manuscript end to show the full author list.
%% This command should only be used with \AuthorCollaborationLimit is used.

%% The following command can be used to set the latex table counters.  It
%% is needed in this document because it uses a mix of latex tabular and
%% AASTeX deluxetables.  In general it should not be needed.
%\setcounter{table}{1}

%%%%%%%%%%%%%%%%%%%%%%%%%%%%%%%%%%%%%%%%%%%%%%%%%%%%%%%%%%%%%%%%%%%%%%%%%%%%%%%%
%%
%% The following section outlines numerous optional output that
%% can be displayed in the front matter or as running meta-data.
%%
%% If you wish, you may supply running head information, although
%% this information may be modified by the editorial offices.
\shorttitle{PhD summary}
\shortauthors{Moore}
%%
%% You can add a light gray and diagonal water-mark to the first page 
%% with this command:
% \watermark{text}
%% where "text", e.g. DRAFT, is the text to appear.  If the text is 
%% long you can control the water-mark size with:
%  \setwatermarkfontsize{dimension}
%% where dimension is any recognized LaTeX dimension, e.g. pt, in, etc.
%%
%%%%%%%%%%%%%%%%%%%%%%%%%%%%%%%%%%%%%%%%%%%%%%%%%%%%%%%%%%%%%%%%%%%%%%%%%%%%%%%%

%% This is the end of the preamble.  Indicate the beginning of the
%% manuscript itself with \begin{document}.

\begin{document}

\title{PhD Project Summary}

%% LaTeX will automatically break titles if they run longer than
%% one line. However, you may use \\ to force a line break if
%% you desire. In v6.2 you can include a footnote in the title.

%% A significant change from earlier AASTEX versions is in the structure for 
%% calling author and affilations. The change was necessary to implement 
%% autoindexing of affilations which prior was a manual process that could 
%% easily be tedious in large author manuscripts.
%%
%% The \author command is the same as before except it now takes an optional
%% arguement which is the 16 digit ORCID. The syntax is:
%% \author[xxxx-xxxx-xxxx-xxxx]{Author Name}
%%
%% This will hyperlink the author name to the author's ORCID page. Note that
%% during compilation, LaTeX will do some limited checking of the format of
%% the ID to make sure it is valid.
%%
%% Use \affiliation for affiliation information. The old \affil is now aliased
%% to \affiliation. AASTeX v6.2 will automatically index these in the header.
%% When a duplicate is found its index will be the same as its previous entry.
%%
%% Note that \altaffilmark and \altaffiltext have been removed and thus 
%% can not be used to document secondary affiliations. If they are used latex
%% will issue a specific error message and quit. Please use multiple 
%% \affiliation calls for to document more than one affiliation.
%%
%% The new \altaffiliation can be used to indicate some secondary information
%% such as fellowships. This command produces a non-numeric footnote that is
%% set away from the numeric \affiliation footnotes.  NOTE that if an
%% \altaffiliation command is used it must come BEFORE the \affiliation call,
%% right after the \author command, in order to place the footnotes in
%% the proper location.
%%
%% Use \email to set provide email addresses. Each \email will appear on its
%% own line so you can put multiple email address in one \email call. A new
%% \correspondingauthor command is available in V6.2 to identify the
%% corresponding author of the manuscript. It is the author's responsibility
%% to make sure this name is also in the author list.
%%
%% While authors can be grouped inside the same \author and \affiliation
%% commands it is better to have a single author for each. This allows for
%% one to exploit all the new benefits and should make book-keeping easier.
%%
%% If done correctly the peer review system will be able to
%% automatically put the author and affiliation information from the manuscript
%% and save the corresponding author the trouble of entering it by hand.

\correspondingauthor{Keavin Moore}
\email{keavin.moore@gmail.com}

%\author[0000-0002-0786-7307]{Greg J. Schwarz}
%\affil{American Astronomical Society \\
%2000 Florida Ave., NW, Suite 300 \\
%Washington, DC 20009-1231, USA}

\author{Keavin Moore}
\affiliation{Department of Earth \& Planetary Sciences, McGill University, 3450 rue University, Montr\'{e}al, QC H3A 0E8, Canada}
\affiliation{McGill Space Institute, McGill University, 3550 rue University, Montr\'{e}al, QC H3A 2A7, Canada}
\nocollaboration

%% Note that the \and command from previous versions of AASTeX is now
%% depreciated in this version as it is no longer necessary. AASTeX 
%% automatically takes care of all commas and "and"s between authors names.

%% AASTeX 6.2 has the new \collaboration and \nocollaboration commands to
%% provide the collaboration status of a group of authors. These commands 
%% can be used either before or after the list of corresponding authors. The
%% argument for \collaboration is the collaboration identifier. Authors are
%% encouraged to surround collaboration identifiers with ()s. The 
%% \nocollaboration command takes no argument and exists to indicate that
%% the nearby authors are not part of surrounding collaborations.

%% Mark off the abstract in the ``abstract'' environment. 
\begin{abstract}

250-word limit for abstract. Fill this in later.

\end{abstract}

%% Keywords should appear after the \end{abstract} command. 
%% See the online documentation for the full list of available subject
%% keywords and the rules for their use.
\keywords{exoplanets, habitability, M-dwarfs}

%% From the front matter, we move on to the body of the paper.
%% Sections are demarcated by \section and \subsection, respectively.
%% Observe the use of the LaTeX \label
%% command after the \subsection to give a symbolic KEY to the
%% subsection for cross-referencing in a \ref command.
%% You can use LaTeX's \ref and \label commands to keep track of
%% cross-references to sections, equations, tables, and figures.
%% That way, if you change the order of any elements, LaTeX will
%% automatically renumber them.
%%
%% We recommend that authors also use the natbib \citep
%% and \citet commands to identify citations.  The citations are
%% tied to the reference list via symbolic KEYs. The KEY corresponds
%% to the KEY in the \bibitem in the reference list below. 

\section{Introduction} \label{sec:intro}

My project will involve the habitability of planets orbiting M-dwarfs throughout the lifetime of the system, motivated by the deep-water cycle and atmospheric loss of volatiles. ‘Dune worlds’ are essentially planets with very low, but not zero, surface water content; these small amounts of water will effect the silicate weathering thermostat (a process which maintains the habitability of Earth) due to the little water but large surface area of land. These planets may have atmospheres which are too oxidized, a negative for life, meaning the planet is technically habitable, but inhospitable. The following questions can arise: 1) are these ‘Dune worlds’ potentially intermittently habitable, and 2) do M-dwarfs primarily produce Dune worlds? {\bf Note that we start our simulations after the (potentially long-lived for planets around M-dwarfs (e.g., \citealt{schaefer16}, \citealt{wordsworth18}) magma ocean stage, i.e., when the planet has solidified. We also assume that plate tectonics has initiated.} \\

The planet is much closer to its host star than we are to the Sun, and M-dwarfs don’t get brighter at the same rate as the Sun. Since there is such a small amount of water, there is no chance of a ‘runaway greenhouse’ on these worlds either, and there is no ice albedo feedback (as likely the low surface water inventory will be liquid); because of this, the habitable zone around the star could be much wider. Through simulations of the evolution of the star/planet system, I will attempt to constrain the habitability of these so-called ‘Dune worlds’ around the much more common M-dwarf stars, which are thought to be much more amenable to life than Solar-type stars. \\

I presented a poster at CASCA 2019, titled "Water Cycling and Atmospheric Loss on M-Earths", showing my first year research progress (mainly the seafloor pressure dependent model results, with and without loss). {\bf See poster for details, and a great summary of the project itself.} The poster presented the following questions to an interested reader.
\begin{enumerate}
    \item Can Earth-like planets orbiting M-dwarfs (``M-Earths") retain a surface ocean while losing water to space?
    \item Can water stored in the mantle rejuvenate a desiccated surface?
    \item Are planets with both land and ocean common? Waterworlds? Dry, desiccated planets?
\end{enumerate}

{\bf H\textsubscript{2}O photolysis predominantly occurs through the equation $\mathrm{H}_2\mathrm{O} + \mathrm{h}\nu \rightarrow \mathrm{H} + \mathrm{OH}$, but other processes (with quantum yields two orders of magnitude lower) are $\mathrm{H}_2\mathrm{O} + \mathrm{h}\nu \rightarrow \mathrm{H}_2 + \mathrm{O}(^1\mathrm{D})$ and $\mathrm{H}_2\mathrm{O} + \mathrm{h}\nu \rightarrow 2\mathrm{H} + \mathrm{O}(^3\mathrm{P})$.} \citep{wordsworth13} Note: $\mathrm{O}(^1\mathrm{D})$ is the first excited state of the oxygen atom, while $\mathrm{O}(^3\mathrm{P}$ is the ground state. From Wikipedia, "The quantum yield of a radiation-induced process is the number of times a specific event occurs per photon absorbed by the system."

\section{Notes from Papers} \label{sec:notes}

{\bf I will likely take more specific notes in my notebook while reading papers. They will be included here.}

A great explanation of the stabilizing feedback of the carbonate-silicate cycle is in \citet{alibert14}: "If the surface temperature decreases, the removal of CO\textsubscript{2} through silicate weathering is suppressed, while CO\textsubscript{2} is still released into the atmosphere by volcanoes. This increases the CO\textsubscript{2} concentration in the atmosphere and eventually the temperature. On the contrary, if the surface temperature increases, then silicate weathering increases, which decreases the CO\textsubscript{2} atmospheric concentration and greenhouse effect, thus lowering the surface temperature." \\

The carbonate cycle, and silicate weathering, are well-explained in \citet{sleep01}, and should be re-read if needed. "On Earth today, ridges vent mainly into the ocean, while arcs vent mainly into the air." \\

"A planet with no N\textsubscript{2} and low $p$CO\textsubscript{2} would be extremely susceptible to water loss... At higher values of $p$CO\textsubscript{2}, the increase in surface pressure outstrips the increase in the saturation vapor pressure of water, so the atmosphere actually becomes more stable against water loss." \citep{kasting93} \\

"The magma ocean stage on close-in rocky exoplanets may be extremely long-lived... M-dwarfs may stay in a [saturated XUV flux] phase for roughly a gigayear... Gliese 1132b typically short lifetimes of magma oceans... {\bf No regassing after magma ocean solidifies.}... Fig. 5 shows solidification times in Myr -- Earth $6.2 \times 10^{-4}$, around M-dwarf, solidification $<$1000 yr (based on Gliese 1132b model)." \citep{schaefer16} \\

"For planets that are not yet in a runaway state, the rate of water loss is constrained by the supply of H\textsubscript{2}O to the high atmosphere. A key factor in this is the temperature of the coldest region of the atmosphere or cold trap, which limits the local H\textsubscript{2}O mixing ratio by condensation. When cold trap temperatures are low, the bottleneck in water loss becomes diffusion of H\textsubscript{2}O through the homopause, rather than the rate of H\textsubscript{2}O photolysis or hydrogen escape to space." \citep{wordsworth13} \\

"Because of the efficiency of H\textsubscript{2}O and H\textsubscript{2} photolysis, H dominates H\textsubscript{2} as the escaping species unless the deep atmosphere is reducing, which we assume is not the case here." \citep{wordsworth13} \\

"Coronal mass ejection from highly active young stars may also erode substantial quantities of... CO\textsubscript{2} and N\textsubscript{2} if their magnetic moments are weak... we note that in the case of planets in close orbits around M-stars, in particular, our results are contingent on the presence of a sufficiently strong magnetic field to guard against direct loss of the primary atmospheric component." \citep{wordsworth13} \\

"It is therefore almost impossible for atmospheres to be forced into a moist stratosphere regime by impact heating without significant erosion also occurring... impacts will only cause substantial water loss if they also remove significant amounts of CO\textsubscript{2} and/or N\textsubscript{2} from the planet's atmosphere." \citep{wordsworth13} \\

"In cases where surface liquid H\textsubscript{2}O is a significant fraction of the planetary mass ($20-30$ Earth oceans), volatile outgassing can become suppressed by overburden pressure (Kite et al. 2009, Elkins-Tanton 2011), and interior mechanisms involving clathrate hydrate formation may become important (Levi et al. 2013)." \citep{wordsworth13} \\

{\bf All of sections 4 and 5 of \citet{wordsworth18} have great explanations of magma ocean-atmosphere exchange and atmosphere-interior exchange (after solidification), including lots of background and references.}

{\bf Find more details on ``cold-trap mechanism'' mentioned in Wordsworth papers, specifically related to CO\textsubscript{2} and abiotic O\textsubscript{2} build-up. Read introduction of \citet{wordsworth13}}.

\subsection{Planetary Surface Water Regimes}

"{\bf Dune planets, $\lesssim 1$\% of Earth's surface water}... Models begin with 20, 40, 60 cm depth of water on surface... $\sim$4 cm in atmosphere in runaway greenhouse threshold... aqua planet, ocean of water 2.7 km deep (Fig. 11); land planets, 2.7 m and 27 m global inventories." \citep{abe11} \\

"10 times Earth's water mass fraction is a waterworld (0.02\% to 0.1\%); 10 times to 1000 times Earth, "waterworlds" \citep{kite18}... {\bf Waterworlds: $>$10 surface oceans (conservatively chosen).}" \citep{abbot12} {\bf Note: Our current model uses a higher ocean basin covering fraction, could hold more surface water?} \\

From \citet{komacek16}: Fig. 2 says waterworld when $\tilde{d_w} = 2.85$, corresponding to $d_w \times 4$ km $= 11.4$ km. (Only corresponds to 3.6 TO, which is VERY low.) {\bf Possibly different waterworld boundaries for each model, see Fig. 3. This could be very important later.} \\

\citet{wordsworth13} use "waterworld" to define "a planet with enough surface liquid water to prevent subaerial land, but not so much H\textsubscript{2}O as to inhibit volatile outgassing." {\bf Our current model assumes the same thing for cycling.} \\

"On the Earth, there would be no land if the ocean mass were three times larger than the present (i.e., 0.023\% of the Earth's mass..." {\bf (VERY low waterworld boundary!} "It turns out that high-pressure ices exist for an Earth-like planet with an ocean of more than $\sim$20 to 100 terrestrial oceans, depending on surface temperature... High-pressure ices form in their simulations if $M_{ocean} \gtrsim 45$ TO." \citep{nakayama19} {\bf High pressure ice limit = 100 TO} \\

"A waterworld state is likely stable \citep{wordsworth13} since water loss rates would be low because the atmosphere would be  CO\textsubscript{2}-rich due to the lack of silicate weathering feedback." \citep{komacek16} Note that the authors calculate whether or not the planet is a waterworld at the end of the simulations (for all 3 models), based on their equation (31):
\begin{equation}
    \tilde{d}_{o,max} \approx \frac{d_{o,max,\oplus}}{d_{w,\oplus}} \tilde{g}^{-1}
\end{equation}
and the condition that a planet is a waterworld if $\tilde{d}_w > \tilde{d}_{o,max}$.

\subsection{Stagnant Lid}
"The standard criterion is that a stagnant lid is obtained when the surface velocity is equal or less than 1\% of the rms mantle viscosity... Rayleigh number $>10^8$ stagnant lid? Would require a very hot mantle temperature." \citep{noack14} {\bf Note: More description in notebook, however, doesn't give a good result for stagnant lid limit.} \\

"A sufficiently low effective friction coefficient ($\lesssim 0.1$) is critical for the operation of plate tectonics on super-Earths." \citep{korenaga10}

"Super-Earths could be in a stagnant lid or episodic regime, but Earth-like planets (i.e., similar radius/mass) may have plate tectonics or be episodic, not completely stagnant lid (see Fig. 3) if high enough yield stresses are present." \citep{oneill07} \\

"{\bf The stagnant lid regime develops only when the viscosity contrast between the surface and interior exceeds about $10^4$.}" \citep{reese98} Occurs when mantle driving forces do not exceed lithospheric yield strength (from Wikipedia). {\bf So: stagnant lid when $\eta_{lithosphere} \gtrsim 10^4 \eta_{mantle}$, OR $\eta_{mantle} \gtrsim 10^4 \eta_{lithosphere}$. Assuming $\eta_{lithosphere} = 10^{22}$ Pa s (as in \citet{liu16}, based on \citet{flesch00}, conservatively.) Then, stagnant lid if $\eta_{mantle} < 10^{18}$ Pa s or $\eta_{mantle} > 10^{26}$ Pa s.}\\

"An initially hotter upper mantle (by reducing the initial upper thermal boundary layer thickness) does not significantly affect the core-mantle boundary (CMB) lids, although it helps to ignite earlier upper mantle convection." \citep{stamenkovic12} (See also Section 6, point 3.) {\bf Possible reason to start with high mantle temp (3200 K) in our models?} The Peclet number is used to determine the viscosity regime, as in other studies (see Fig. 13). \\

\section{Notes from Talks}

Sean Raymond:
\begin{itemize}
    \item No Jupiter formed? Likely to get super-Earths.
    \item On Earth, $M_{water} \sim 0.1$\% $M_{\oplus}$.
    \item {\bf Water was delivered to Earth by same population that was implanted into asteroid belt as C-types (but not directly from these asteroids, as originally proposed).}
\end{itemize}

Evgenya Shkolnik:
\begin{itemize}
    \item M-dwarf $\rightarrow$ $10^6$ red photons for each UV.
    \item {\bf $\rightarrow$ far UV \& near UV $=$ photodissociation}
    \item {\bf $\rightarrow$ XUV $=$ atmospheric escape}
    \item Extreme ultra-violet (EUV) corrodes upper atmosphere
    \item Stars spin down over time, become less active
    \item Late M's very active, even later in their life
    \item Superflare on M-dwarf (young early M stars) $\sim 200$ times the regular flux, likely happens every day
    \item Young M-dwarfs flare 100-1000 times more frequently and more violently than older M-dwarfs; many more flares in the UV than in the optical
\end{itemize}

Curtis Williams:
\begin{itemize}
    \item Upper mantle dry from solar wind irradiated volatiles (based on MOR data)
    \item Lower mantle in line with solar wind irradiated volatiles, also solar nebular gas based on more recent data
    \item {\bf Proto-Earth must have rapidly accreted volatiles from nebular gas; volatiles were dissolved into magma ocean from a nebular atmosphere}
    \item Solar nebula likely dispersed by $\sim$ 4 million years
    \item Winds can cause magma oceans to white cap; white cap acts to grab volatiles from atmosphere and mix into magma ocean (Peter Olsen, 2019)
\end{itemize}

Bekki Dawson:
\begin{itemize}
    \item From Kepler, super-Earths common around 1/4 to 2/3 of stars
    \item Close-in super-Earths may form as gas disk is dissipating
    \item Large amount of solid material required close to star; more likely that they migrate from outer disk; both of these lead to a diversity of compositions
\end{itemize}

\section{Notes from Comps}
References for numbers/variables from proposal are indicated in my comps notes! Look there before looking elsewhere. \\

How can we prove this is a good model?
\begin{itemize}
    \item Two-box captures gross estimates of water partitioning over Gyr timescales
    \item We can try to reproduce Earth using plausible initial conditions \& solar fluxes (still major uncertainties)
    \item Try to reproduce Venus/Mars (although water loss history uncertain?)
    \item Venus D/H ratio 100 times greater than Earth, so it lost substantial water (H would have been preferentially lost over D); stagnant lid of Venus still probably allows resurfacing events
    \item Mars possibly has water trapped in mantle, either preferentially stored by cycling or tectonics stopped early \& remaining surface water lost to space; Mars speculated to have lost atmosphere by impact erosion
    \item {\bf Plate tectonics may be history-dependent, so might be stuck making assumptions about initial tectonic mode (but probably stagnant lid)}
\end{itemize}

Additional observations?
\begin{itemize}
    \item Disk-integrated rotational multi-band photometry of Earth (``changing colours of pale blue dot'') encodes information about continents, oceans, \& clouds
    \item ``Single pixel'' maps could be used to construct coarse longitudinal maps using a 5-10 m space telescope \& starshade; could identify continents of a tectonically active planet
    \item {\bf Continental crust formation requires multiple melting events to get dense, floating continents on top of mantle; not clear that tectonics was required to create continental crust on Earth, but could be a reasonable indicator for plate tectonics on another planet (e.g. Stern 2016)}
    \item Geophysical constraints: could say something about surface water, but you would be unable to say anything definitive about e.g. tectonic state, water in mantle
\end{itemize}

Possible JWST measurements?
\begin{itemize}
    \item JWST will be able to measure transit spectroscopy of M-Earths (mid-resolution) $\rightarrow$ atmosphere/water vapour
    \item High-resolution near-IR spectrscopy with ground-based telescopes $rightarrow$ Doppler shift of planet features allows extraction from spectrum which includes many similar molecules to Earth's atmosphere
    \item Eclipse measurements will allow us to probe dayside temperature (JWST) $\rightarrow$ surface temperature
    \item JWST phase measurements: dayside typically hotter \& brighter; amplitude of phase variations sensitive to presence of fluid at surface of planet; planets with atmospheres \& oceans will have muted thermal phase variations as compared to barren rock
\end{itemize}

Why might hydration equations be important?
\begin{itemize}
    \item Degassing removes volatiles from mantle \& increases its viscosity. {\bf Higher viscosity reduces heat flow}, but mantle wants to maintain heat flow, viscosity, \& convective vigor.
    \item {\bf Mantle hotter and cools more slowly when volatile-dependent rheology.}
    \item Regassing will add volatiles \& decrease viscosity... cool the mantle.
\end{itemize}

Aluminum-26 early on?
\begin{itemize}
    \item Yes! Extinct now (half-life 0.73 Myr), but it may have been a heat source for rapid heating \& differentiation of planetary bodies
    \item Comes from nearby SN (likely), probably true for solar system formation
    \item Paper by Lichtenberg+ (2019) discusses importance of 26-Al heating for water budget determination of rocky planets
    \item {\bf More 26-Al early $=$ less water on planets (dehydrated)}
\end{itemize}

Condensation of stellar nebula to rocky planets? How close to Earth-like will it be?
\begin{itemize}
    \item Planets that form in situ in HZs of M-dwarfs could be small \& dry; T-Tauri phase means water can't condense there.
    \item Planets that form beyond snow line could form with sufficient volatiles, migrate inwards; but may be unable to shed thick H/He envelope (if so, never terrestrial and probably not habitable)
    \item High surface temperatures may inhibit plate tectonics
    \item Planets in HZ with water may have different compositions to Earth because they formed beyond snow line and migrated inwards; could have large inventories of reducing species, since H-rich species (like CH4 and NH3) condense at large distances from host star \& are easily accreted during planet formation
    \item Outgassing of reduced compounds could lead to quick removal of atmospheric O2
    \item Planets may not have magma ocean phase; may not have rocky surfaces, but water \& ice extending to greater depths
    \item {\bf Thick H/He envelope could be beneficial if it takes the brunt of XUV early on to prevent water loss, then dissipates}
\end{itemize}

Waterworlds still habitable?
\begin{itemize}
    \item Yes! And life on Earth may have started at hydrothermal vents in ocean.
    \item BUT classical definition of habitability requires silicate weathering thermostat to regulate climate, so waterworlds may not have such a stable climate. ($+$ high CO2 in atmosphere)
\end{itemize}

T-dependent degassing/regassing?
\begin{itemize}
    \item {\bf If all water goes into melt, there should be MORE water in the melt (e.g., 10\% of melt should have 10 times more water, 1/0.1); same number of molecules going into melt, so it should be ENRICHED relative to entire mantle}
    \item Since our depth of melting is 60 km, should get 10\% melting; 1\% for every 3 km, max extent 20\%, mean extent 10\% (Langmuir \& Broecker 2012)
    \item {\bf Can improve this parameterization using the melt fraction from Schaefer \& Sasselov 2015}
\end{itemize}

Core? Water sequestration?
\begin{itemize}
    \item There could definitely be primordial water (from accretion, NOT late delivery by comets) trapped in the core; unconstrained, since we can't measure its composition.
    \item {\bf Hirschmann (2006) suggests there may be substantial water sequestered at core-mantle boundary, which would affect convection scheme of mantle (but also unconstrained).}
    \item {\bf Schaefer \& Sasselov (2015) do consider additional heat transfer upwards into mantle from core (neglecting radioactive decay); assume core isothermal; excluded from Komacek \& Abbot for simplicity}
    \item {\bf They instead calculate $Ra_{crit} = 0.28 Ra^{0.21}$ for the lower mantle, while retaining upper mantle $Ra_{crit} = 1100$; mixed heatinf; convective instability determined by LOCAL conditions rather than entire mantle}
    \item Model we use reproduces Earth well, but Schaefer \& Sasselov (2015) explopre this other convection regime for application to other planets; {\bf they find mantle cools more rapidly \& significantly more water degassed from interior}
\end{itemize}

Core/mantle ratio?
\begin{itemize}
    \item Core could contain substantially more water (due to its availability during formation from solar nebula), but I'm uncertain about efficient transfer mechanisms to provide contributions to deep-water cycle
    \item Possibly plumes from core-mantle boundary could incorporate some into melt?
\end{itemize}

scipy.integrate, what does it use? Timestep?
\begin{itemize}
    \item We use ``vode'', a real-valued ordinary differential equation solver, which uses an IMPLICIT Adams method for non-stiff equations, but will switch to backward-differentiation formulae \& adjust internal timestep for stiff problems
    \item Equations solved using matrix algebra
    \item {\bf Implicit scheme good for parameters dependent upon one another at same level, as in our coupled integrations, and usually more stable (albeit more complicated) than explicit (explicit good for values independent of one another)}
\end{itemize}

Enthalpy of melting may cool the mantle
\begin{itemize}
    \item Interesting -- effect of enthalpy in mantle accounted for in SS2015 model through added volatiles decreasing mantle viscosity, but not explicitly included
    \item Would be interesting to investigate further with respect to recrystallization, compare to radioactive heating
\end{itemize}

Radioactive elements in mantle providing heating?
\begin{itemize}
    \item $^{238}$U, $^{235}$U, $^{232}$Th, $^40$K
    \item Nominal bulk silicate Earth contains $\sim 21$ ppb U (McDonough \& Sun 1995), currently assumed for our model, average value from KA16
\end{itemize}

Subduction/plate tectonics; throughout Earth's history, style of plate tectonics, early Earth tectonics
\begin{itemize}
    \item Possible that Earth began in stagnant lid and transitioned into plate tectonics, but no clear record when it began
    \item {\bf Subduction speeds may change our results; for example, faster subduction could carry H2O deeper into mantle}
\end{itemize}

Thermostat impact of CO2 without life?
\begin{itemize}
    \item CaCO3 deposition can still occur in oeans, albeit at slower rate
    \item Must be at equilibrium to maintain climate
\end{itemize}

Crust different than granite on early Earth
\begin{itemize}
    \item If mafic, provides a CO2 sink (reduced rocks)
    \item If felsic (like granite today), doesn't help to sequester CO2
    \item Can go through literature and place constraints on Earth's crust based on early Earth
    \item Early Earth crust may have been ultramafic after solidifying from magma ocean \& better at drawing down CO2
\end{itemize}

Assumptions about thermostat okay, but actually modelling it may be difficult
\begin{itemize}
    \item May be too poorly constrained
    \item We only assume that it operates to regulate climate; may be easier to just think about surface temperature instead of directly calculating
\end{itemize}

Continental crust vs. oceanic crust
\begin{itemize}
    \item Continental crust today is felsic granite, made of mainly feldspar, quartz, mica, amphiboles, $\sim 35$ km thickness
    \item Oceanic crust is mafic basalt, made of mainly feldspar, pyroxene, olivine, $\sim 6$ km depth
\end{itemize}

Earth may be at cusp of having plate tectonics
\begin{itemize}
    \item {\bf May require cool surface temps but M-Earths could be quite hot for much of their evolution (pre-MS)}
    \item Stagnant lid may develop deep fractures, allowing deeper hydration of crust; could be locking away water in interior by resurfacing
    \item {\bf Resurfacing means mantle gets so hot that it forces its way to the surface, covering most or all of surface in magma \& cools to create new surface (suggested to have occurred on Venus)}
\end{itemize}

Concentration of volatiles from formation/accretion only? How likely is volatile late delivery? Earth may be atypical (late delivery by comets)
\begin{itemize}
    \item M-Earths that form in situ in HZ of M-dwarfs may be relatively volatile-depleted
    \item If they form beyond snow line \& migrate in, they can accrete significant volatiles
    \item Earth water may have also come from late-stage accretion of planetesimals (Raymond et al. 2004)
    \item How likely? I will search literature, but if abundant volatiles in outer proto-planetary disk, it should at least be possible in other planetary systems
\end{itemize}

Cl, Fl affects atmospheric chemistry, different H flux
\begin{itemize}
    \item Will be interesting to investigate further if we choose to include atmospheric composition
\end{itemize}

Composition of rocky planets
\begin{itemize}
    \item Mg-Fe ratio controls H locked away
    \item {\bf We don't know anything about their compositions, but the metallicity of the host star may be a decent proxy}
\end{itemize}

Ca content of planets will affect silicate weathering
\begin{itemize}
    \item Another reason to look into formation/migration history to determine likely compositions (but seems like quite a stretch)
\end{itemize}

Can we predict what an M-dwarf planet would look like, based on Earth, distance from star, bulk composition?
\begin{itemize}
    \item May have different bulk composition than Earth, but no real way to measure this currently
    \item {\bf Metallicity of host star would be a decent proxy; low metallicity M-dwarfs are rare, so good chance there is enough material in protoplanetary disk for composition similar to Earth (but abundances of Mg, Fe, Ca could be different}
\end{itemize}

Early Earth, different tectonics? Atmosphere?
\begin{itemize}
    \item Earth probably began in stagnant lid (but no evidence)
    \item Hadean/Archean likely when first continents formed \& plate tectonics began to operate ($\sim 2.5$ Ga, Langmuir \& Broecker)
    \item Early atmosphere likely lacked free O2, likely reduced which contained abundant CH4 (low in today's atmos), which would have provided greenhouse warming; also NH3, H2
    \item CH4 could have been photodissociated and H lost to space; unfortunately not able to measure this
    \item Reduced C would mostly be stored in surface rocks
    \item Oxygen gradually built up as life evolved and photosynthesis made O waste product, becoming more efficient with time
    \item CH4 highly reactive with oxygen (think of burning natural gas)
    \item {\bf Early atmos likely formed from outgassing of reduced gases from interior}
    \item {\bf Materials that formed Earth were highly reduced -- abundant H, C, Fe in solar nebula, which are oxygen-hungry}
    \item Today's atmosphere is oxidizing, $\sim21$\% oxygen (and only in this steady state because of processing by life)
\end{itemize}

Silicate weathering?
\begin{itemize}
    \item CO2 extremely important in regulating Earth climate
    \item Most CO2 stored in limestone (CaCO3)
    \item Rain on rocks causes weathering, releasing chemicals which are transferred to ocean
    \item Ca \& Si released from CaSio3 important, as both make up organism shells of CaCO3 and SiO2
    \item Most Ca \& Si from breakdown of feldspars, pyroxenes, other mafic materials in continental crust
    \item CaSiO3 ``wollastonite'' broken down by rainwater \& CO2 dissolved in soils
    \item ions moved through soils to streams, then oceans
    \item Organisms then make shells (CaCO3 could precipitate from seawater organically before)
    \item Shells become sediment on plate and are subducted
    \item High T \& P of interior causes minerals to break down -- ``metamorphism''
    \item CaSiO3 returned to mantle, CO2 released by convergent margin volcanoes into atmosphere
    \item {\bf Note: interior degassing of CO2 NOT impacted by surface temp, so its rate doesn't change during the stabilizing feedback}
    \item Feedback (also works opposite): increased CO2 in atmos or increased solar luminosity $\rightarrow$ warmer atmos, more H2O in atmos, more acidic rainwater $\rightarrow$ increased weathering $\rightarrow$ increased flux of Ca2+ to oceans $\rightarrow$ precipitation of CaCO3 lowers CO2
\end{itemize}

Isotope fractionation due to atmospheric loss (D/H)?
\begin{itemize}
    \item When H2O photodissociated, lighter H escapes preferentially over D
    \item D/H ratio of Earth similar to chondrites, suggesting it lost little water
    \item D/H ratio of Venus 100 times greater than Earth, so it must have lost substantial water (must have had 1000 times more than currently)
\end{itemize}

Signature from mantle?
\begin{itemize}
    \item Mantle has 18O/16O ($\delta^{18}$O) of 5 per mil (parts per thousand), compared to seawater standard (which is zero)
    \item Rocks influenced by weather with a liquid ocean (i.e., on the surface) shows $\delta^{18}$O $>5$ per mil; low temperature water cycle evidence
    \item {\bf Can distinguish between sedimentary and mantle-derived rocks this way}
\end{itemize}

Silicate feedback vs. radiative feedback, grad level explanation?
\begin{itemize}
    \item Studies show solar forcing must be greater than CO2 forcing to cause same global mean surface temp change
    \item Didn't get a chance to read, but Modak et al. (2016) seem to explain mechanism
    \item They note 2.25\% increase in solar irradiance produces same long-term global warming as doubling CO2
\end{itemize}

CO2 effect on cycling?
\begin{itemize}
    \item Shouldn't directly affect our degassing/regassing mechanisms, but our thermal evolution parameterization depends on contrast b/w surface temp and mantle temp for convection in Ra
    \item More CO2 would increase surface temp, decrease temp contrast, lower Ra, slightly less convection?
\end{itemize}

Ices in surface reservoirs?
\begin{itemize}
    \item Less liquid water available for cycling
    \item Should have effect on hybrid mode, which explicitly depends on surface water
    \item Ice $+$ water will have an overall lower density than all water, so seafloor pressure may be lower (since $P = g d_w \rho_w$
    \item Hybrid should decrease MORE water is P decreased, but would need to thaw ice before it is available for cycling/loss to space
\end{itemize}

Plate tectonics? What fraction?
\begin{itemize}
    \item Schaefer \& Elkins-Tanton (2018) note that plate tectonics may not be common outcome for M-Earths
    \item Temperate surface conditions (relatively cool) are favourable for plate tectonics, so more cool planets means more likely to have plate tectonics, BUT may be history-dependent and difficult to constrain
\end{itemize}

Planet formation: phase transition b/w ocean \& atmosphere?
\begin{itemize}
    \item Earth's surface temperatures and pressures seem to be close to triple point of water, allowing all 3 phases to exist simultaneously
    \item Likely had some control by early climate of Earth, and suggests planets in the same temp/pressure range could experience the same thing; phases critical to deep water cycle, climate (through greenhousen effect), storage in ice
    \item This may be interesting to investigate, but could be too much; I just assume a similar atmospheric mass to Earth to allow triple point for now
\end{itemize}

Plate tectonics/water in mantle? (No ocean limit, i.e., atmosphere to mantle)?
\begin{itemize}
    \item No ocean would require high surface temp to evaporate all water into atmosphere
    \item Cycling into mantle would require deposition of water vapour to surface somehow (snow, ice), BUT higher surface temps
    \item I don't think it's possible to get water into mantle from atmos, but you could degas water from mantle by volcanism/resurfacing
\end{itemize}

Snowball Earth $rightarrow$ path-dependence, start with ice? Multiple equilibria
\begin{itemize}
    \item Earth's oceans completely freezing would stop silicating weathering/chemical erosion
    \item CO2 would build up in atmosphere until warm enough to melt ice $rightarrow$ driven by Earth's internal heat, insensitive to surface temp. $\rightarrow$ still plate tectonics
    \item Runaway $\rightarrow$ growth of ice leads to increase in Earth's albedo, which reflects away more sunlight
    \item As nothing removing CO2, it builds up again \& warms ice to melt, exposing less reflective sea \& land
    \item Eventually weathering begins again \& allows CO2 to be regulated back to stable levels
    \item Occurs on Myr timescales
    \item {\bf Orbital variations of Earth's distance from Sun (eccentric orbit) \& variation in tilt induce climate change, and may have caused ice ages}
    \item Two stable equilibria, snowball \& non-snowball (``slush'', with ice but exposed ocean at equator, apparently not stable) dependent on initial conditions
\end{itemize}

Initial condition sensitivity $rightarrow$ two stable equilibria in our models?
\begin{itemize}
    \item So far, only 1 stable equilibrium
    \item Will be interesting to do more sensitivity analyses as we make the model more complicated
\end{itemize}

Arc volcano depth constant?
\begin{itemize}
    \item Loss of water as plate is subducted causing partial melting of overlying mantle
    \item Chlorite becomes unstable at approximately 40-60 km depth
    \item Volcanic arcs always form when subducting plate reaches 100 km due to the temperature-pressure conditions there
    \item {\bf Water highly soluble in magma, so as it boils off the plate it contributes to the partial melt, leading to explosive volcanism (like Mt. St. Helens)}
    \item Earthquakes occur as the plate is subducted $rightarrow$ Benioff zones on inclined angle of subducting slab, because subduction occurs at subduction zones or convergent margins
    \item {\bf We could account for additional degassing back to the surface from mantle by arc volcanism $rightarrow$ would it be tied to our ``regassing efficiency factor'' to determine how much ends up in melt instead of being subducted?}
\end{itemize}

Subduction speed? Efficiency? Dependence on seafloor pressure?
\begin{itemize}
    \item We can vary our subduction speed either through varying the spreading rate in our models, or by changing the amount of water subducted into the deep mantle
    \item Faster subduction should allow water to be transported deeper into the mantle
    \item Dependence of seafloor pressure in literature debated -- we can test both negative \& positive dependence for degassing to see results
\end{itemize}

Same water cycling between mantle and surface?
\begin{itemize}
    \item {\bf MOR basalts indicate that they are derived from upper mantle source region with 50-200 ppm $rightarrow$ this seems to have remained constant over billions of years, suggesting subduction of hydrated oceanic crust is depositing water in MORB source region}
    \item Hirschmann (2006) notes that while much water is subducted into deep mantle, upper mantle supply could be derived from partial melts from various sources (e.g., water advecting up from transition zone)
\end{itemize}

Early evidence of water on Earth?
\begin{itemize}
    \item Zircons suggest there was a robust water cycle on Earth \& liquid water present 4.4 Ga
    \item Heavy oxygen fractionation gives $\delta^{18}$O of 6-7 per mil (the Jack Hills zircons), greater than the mantle $\delta^{18}$O $=5$ per mil
    \item {\bf Rocks affected by a water cycle will have heavier oxygen (Langmuir \& Broecker 2012)}
\end{itemize}

How are degassing/regassing rates measured today for Earth?
\begin{itemize}
    \item Degassing measured from ratio of H2O/Ce (Cerium) at mid-ocean ridges to be $2\times10^{11}$ kg/yr) (Hirschmann \& Kohlstedt 2012)
    \item Water \& Ce behave similarly during partial melting of mantle, but adundance of Ce known more because it is easier to analyze \& its abundance less susceptible to modification during \& after volcanic eruptions
    \item Regassing estimated based on studies of serpentinite (olivine $+$ water) subducted into deep mantle (Schmidt \& Poli 1998), and computed based on water contents measurements of sediments \& crust subducted at subduction zones to be $\sim 1.8\times10^{12}$ kg/yr (Jarrard 2003)
    \item Measurements from 41 subduction zones; spread of values of $[0.7-2.9]\times10^{12}$ kg/yr
\end{itemize}

HZ location of M-dwarfs over time?
\begin{itemize}
    \item Moves monotonically inwards for up to 1 Gyr during pre-MS contraction phase of M-dwarf
    \item Then stable for a long time while M-dwarf on MS (billions of years)
\end{itemize}

Atmospheric loss: How will we go about making the mass loss more sophisticated than merely sucking water out of the ocean? {\bf NOTE: Nick and I discussed this, and it seems needlessly complicated. Probably end up using CLAUSIUS-CLAPEYRON instead}
\begin{itemize}
    \item We will add a third box, atmosphere, from which water will be lost
    \item We need flux to get water into atmosphere $\rightarrow$ evaporation? (more complicated than a simple equation, since surface temperature should not be at boiling point or all surface water would evaporate!)
    \item Could use Boltzmann distribution for water molecules in ocean $rightarrow$ some near surface will have enough energy to evaporate $\rightarrow$ more if surface temp. increases
    \item We can calculate surface temp. using M-dwarf evolution models (stellar evolution: Baraffe et al. 1998, 2015; XUV evolution: Ribas et al. 2005) to determine radiation incident on planet $\rightarrow$ also need planet albedo (Earth = 0.3) and orbital distance
    \item Can calculate incident XUV flux, which we can plug into energy-limited mass loss rate of Luger \& Barnes (2015)
    \item {\bf Energy-limited mass loss assumes all O reacts with surface sink, so it doesn't build up in atmosphere}
    \item {\bf We can also include diffusion-limited mass loss of Luger \& Barnes (2015), which accounts for O building up in atmosphere (inefficient surface sinks) which H must diffuse through to escape; H can also drag some O with it, if the outflow is strong enough}
\end{itemize}

In what way might you expect those changes to affect the mass loss?
\begin{itemize}
    \item The mass loss of water should be significant early in the evolution due to high initial XUV flux from host M-dwarf
    \item {\bf Energy-limited will be limited by XUV flux, so an M-Earth around a SMALLER M-dwarf will experience MORE water loss due to HIGHER XUV}
    \item Likewise, a planet orbiting closer to its star will have more water transported to atmosphere \& available to be lost due to higher surface temps
    \item Our loss will be limited by transport of water from surface to atmosphere as well as XUV flux
    \item {\bf Luger \& Barnes (2015) indicate planets can lose oceans worth of water throughout HZ during early evolution, and we could make our planets more robust to loss by beginning with all water in the mantle \& allowing it to be degassed first to surface, as in McGovern \& Schubert (1989) and Schaefer \& Sasselov (2015), and speculated to have happened for Earth}
    \item {\bf BUT we will still test partitioning between surface and interior, since steam atmosphere could collapse into an ocean after magma ocean solidification}
    \item Diffusion-limited mass loss rates should end up being lower than energy-limited, since H escape to space is bottlenecked by buildup of O, given the same XUV flux incident from host star
\end{itemize}

{\bf Nick's notes from comprehensive exam itself:} \\
\begin{itemize}
    \item Runaway greenhouse $\rightarrow$ H2O in exosphere $rightarrow$ loss
    \item Constant freeboard: explanations other than constant ocean volume (e.g., isostatic adjustment)
    \item Feedbacks? $P$-dependent is stabilizing feedback? Yes it is!
    \item Glacier reservoir to sequester water?
    \item How many oceans in cryosphere? Hint: sea level rise due to climate change
    \item Multiple equilibria? No, not yet. {\bf If multiple equilibria, need to consider integration method}
    \item Habitability? Could extend to include waterworlds
    \item Are waterworlds known? Can't constrain yet. Why not? Mass-radius, and density measurements degenerate
    \item Bulk chemical composition of planets? Metallicity of host star is most useful.
    \item Short-lived nucleides? $^26$Al important for initial heating \& dehydration (Lichtenberg+ 2019)
    \item Which parameters are well constrained? Which matter less? Albedo could be constrained (but does it matter?); water on surface well constrained for Earth, interior not so much; {\bf H2O in melt $\neq$ H2O in mantle}
    \item Telescope observations? Which ones would constrain model? Eclipses $\rightarrow$ surface temp.; transits $\rightarrow$ H2O in upper atmosphere
    \item Timescale of water loss? Time-variable timesteps? Former is Myr-Gyr; {\bf variable timesteps possible, scipy.integrate already does this, but could be integrated more carefully}
    \item Models based on Earth, is Earth typical? It's a good starting point, since we sort of understand it
    \item Formation theories predict similar compositions for G-dwarfs, what abhout Kepler?!
    \item {\bf Polluted white dwarfs show evidence of terrestrial planets}
    \item How could equations change for stagnant lid? Volcanism for degassing; resurfacing event, all volatiles to surface, what about regsassing? Did resurfacing event happen on Earth? Any evidence? We can't know, since geologic activity has erased any evidence
    \item How did Earth transition from stagnant lid to plate tectonics? Unknown, but probably occurred $~2.5$ Ga
    \item What changed in Earth's mantle? It cools! More viscous, less vigorous convection, why plate tectonics?! Add H2O during resurfacing event?
    \item Currently testable hypotheses of these models? Earth as sanity check; presence of H2O on M-Earth
    \item Atmospheric water in atmosphere after 5 Gyr?
    \item How to know that planet went through desiccated phase? D/H ratio? DHO for exoplanets?!
    \item Can melt NOT degas? Are there examples on Earth where melt doesn't regas?
    \item How would you incorporate cases where seafloor pressure exceeds threshold pressure?
    \item Scenario with (temporarily) zero degassing, should be upper limit on surface water
    \item Needs to take/TA climate physics, or equivalent course in EPS/ATOC?
    \item Evaporative cooling (latent heat)?
    \item How much more water vapour at Venus compared to Earth? e.g., all absorbed power evaporates water; energy balance
    \item Exposed land and carbon cycle? CO2 would be complicated, is this coupling between water and carbon cycles? {\bf Abbot \& Cowan model this.}
    \item Set-up of model is deterministic, at what point does it become non-deterministic (chaotic)?
    \item Could we predict which M-Earths have atmospheres?
    \item Compositional difference between Mars and Earth? Main element that is different? Iron (Fe)
    \item {\bf Why is Fe important to water cycle? Oxidize Fe, FeO, release H, loss to space. Functionally equivalent to photodissociation of H2O}
    \item Bulk composition of M-Earths? We don't know!
    \item Bulk composition affects mantle water sequestering
    \item Weathering thermostat? Are there others? Seafloor weathering, {\bf several others (ask Vincent about this!)}
    \item Best stars for HZ? Are M-dwarfs the best? YES, because of stellar demographics (also planetary demographics); more temperate terrestrial planets; nearby, M-dwarf, solar [Fe/H], plate tectonics
    \item Model starts at mantle temp 3200 K? What state is it in? Fully molten?
    \item Relate cooling of mantle to time to reach steady state?
    \item How do we know initial mantle temperature (or should we vary)?
    \item How would we know whether planet is in steady state?
    \item Evidence for early oceans on Earth? Zircons!
    \item Why should your paper be required reading for exoplaneteers? Capture overall partitioning of water? and physical processes
    \item Transit spectroscopy!
    \item Clouds? Can be inferred from eclipse measurements, but also from...
    \item Exoplanet observables can be predicted in near future?
\end{itemize}

\section{Reference Summaries} \label{sec:refsum}

\subsection{Indication of insensitivity of planetary weathering behaviour and habitable zone to surface land fraction \citep{abbot12}}
A climate model is used to study the effect of land fraction on the weathering rate, and thus the width of the habitable zone; waterworlds tend to have narrower habitable zones. The authors find that planets with even a little bit of land benefit from the silicate weathering stabilizing feedback. Waterworlds may also reach a point of "self-arrest", in which they reach a moist greenhouse state and begin losing water, eventually exposing continents and activating silicate weathering, which could lead to an Earth-like habitable planet. {\bf Lots of good silicate weathering background/theory throughout this paper.}

\subsection{Habitable zone limits for dry planets \citep{abe11}}
A general circulation model, based on the Earth, is used to show that ``land planets'' (or Dune worlds), have wider habitable zones than ``aqua planets'', such as the Earth. Obliquity is important for insolation, as much of the water may remain at the poles of a land planet. It is also possible that aqua planets may evolve into land planets without a runaway greenhouse phase. Land planets are resistant to both complete freezing and to a runaway greenhouse. {\bf Section 4 describes hydrogen escape and volatile cycling on a ``land planet'', comparing to an ``aqua planet'' like Earth. Overall, lots of interesting information throughout paper.}

\subsection{Characterization of the hydrospheres of TRAPPIST-1 planets \citep{acuna21}}

\subsection{How hospitable are space weather affected habitable zones? The role of ion escape \citep{airapetian17}}
The impact of XUV flux on the ion escape of oxygen and nitrogen is calculated for Earth-size and super-Earth planets orbiting G, K, and M dwarfs, and applied to Proxima Cen b and GJ 832b. The results indicate that Earth-like exoplanets around M and active K stars, which are exposed to high XUV, can possibly have their atmospheres entire removed in a few to a few hundred Myr (given a low replenishment rate), making them uninhabitable. \\
{\it NOTES:} \\
{\bf ``Thermal escape models suggest taht exoplanetary atmospheres around active K-M stars should undergo massive hydrogen escape, while heavier species including oxygen will accumulate forming an oxidizing atmosphere.''} \\
{\bf ***``We show that non-thermal oxygen ion escape could be as important as thermal, hydrodynamic H escape in removing the constituents of water from exoplanetary atmospheres under supersolar XUV irradiation.''***} \\
{\bf ***``Our models suggest that the atmospheres of a significant fraction of Earth-like exoplanets around M dwarfs and active K stars exposed to high XUV fluxes will incur a significant atmospheric loss rate of oxygen and nitrogen, which will make them uninhabitable within a few tens to hundreds of Myr, given a low replenishment rate from volcanism or cometary bombardment.''***} \\
{\bf ``During a flare, the solar flux in X-ray UV (XUV, 20-300 Angstroms) band increases by a factor of 1000 and in the extreme UV (EUV, 300-1215 Angstroms) band by up to a factor of 20 (ref).''} \\
{\bf ***``At heights greater than 200 km, such high XUV and EUV (abbreviated here as XUV) and particles precipitating from the magnetosphere ionize gas, producing photo- and secondary electrons, an important contributor to outflows of charged atmospheric particles on Earth, Venus, and Mars (refs).''***} \\
{\bf ***``The X-ray luminosities from young and active M dwarfs are over two orders of magnitude greater than the Sun, but are characterized by compact climatological HZs (CHZs) because of their low luminosities (ref)... Exoplanets around red dwarfs should be exposed to XUV fluxes up to two orders of magnitude larger than those around active solar-type G and K stars.''***} \\
{\bf ``The XUV flux from the young Sun and GJ 832 are comparable in magnitude and shape at wavelengths shorter (and including) Ly$\alpha$ emission line. This suggests the contribution of X-type flare activity flux is dominant in the ``quiescent'' fluxes from the young Sun and inactive M dwarfs.''} \\
{\bf ***``It is reasonable to assume that the young Sun and quiet dwarfs should produce frequent coronal mass ejections, or CMEs (and associated solar energetic particle events).''***} \\
{\bf ``The dynamic pressure from CMEs can significantly modify the planetary magnetic field and cause energy dissipation in its polar regions (ref).''} \\
{\bf ***``XUV radiation induces non-thermal heating via photoabsorption and photoionization raising the temperature of the exosphere, and therefore its pressure scale height. At high XUV fluxes, this process initiates hydrodynamic atmospheric escape of neutral atmospheric species, with the loss rate dependent on the molecular mass of atmospheric species.''***} \\
{\bf ``For the environments of active solar-type stars and M dwarfs, much of the hydrogen likely escapes from a planet's atmosphere during a system's early evolution, leaving behind an atmosphere enriched in heavier elements such as N and O, which are difficult to remove unless dense and fast stellar winds or the processes of photochemical escape are invoked (refs).''} \\
{\bf ***``Models of atmospheric ionization and loss via non-thermal mechanisms are critical for predicting the evolution of oxygen and nitrogen-rich atmospheres as well as the efficiency of atmospheric loss of water as a critical factor of exoplanetary habitability.''} \\
{\bf ***``In the region above an Earth-size planet's exobase, the layer where collisions are negligible, the incident XUV flux ionizes atmospheric atoms and molecules and produces photoelectrons.''***} \\
{\bf ***``The upward propagating photoelectrons outrun ions in the absence of a radially directed polarization electric field and forms the charge separation between electrons and atmospheric ions. Thus, a radially directed polarization electric field is established that enforces the quasi-neutrality and zero radial current.''***} \\
{\bf ***``For ionospheric ions with energies over 10 eV, the polarization electric field cancels a substantial part of the Earth's gravitational potential barrier, greatly enhancing the flux of escaping ions and forming an ionospheric outflow.''***} \\
{\bf ``In this study, we did not consider the processes of photolysis of water molecules that can provide atomic oxygen through formation of hydroxyl molecules and hydrogen atoms that thermally escape from the atmosphere. In essence, we assume that water in the lower atmosphere is photolyzed into H, O, and O2 before reaching the upper atmospheric domain of our simulations.''} \\
{\bf ``O$^+$ ions form due to photoionization of atomic oxygen via photons with wavelengths $\sim 300-600$ Angstroms and collisions with photoelectrons.''} \\
{\bf ``Figure 2 suggests that the photoelectron flux increases approximately linearly with the input XUV flux.''} \\
{\bf ***``For the same XUV flux, we find that as we increase the exobase temperature by a factor of 2, the resulting O$^+$ outflow rates increase by a factor of 10.''***} \\
{\bf ``Figure 3 shows that the mass loss of oxygen ions increases roughly linearly with the solar flux.''} \\
{\bf ***Equation (1) gives the mass loss rate as a function of XUV flux from the models of this paper. ``This loss rate is comparable to the thermal loss of hydrogen at XUV fluxes $\sim20$ times the Sun, at the inner edge of the CHZ if radiative cooling and the transition from hydrodynamic to Jeans escape are accounted for an $\sim1$ Earth-mass planet.''**} \\
{\bf ***``Oxygen ion loss at the rate prescribed by Equation (1) will remove Earth's oxygen at the surface pressure of 1 bar, $\sim1.2 \times 10^{18}$ kg atmosphere within 24.2 Myr if it is not replenished by CO2 through tectonic and volcanic outgassing.''***} \\
{\bf ``For an XUV flux $10$ times the Sun, the super-Earth orbiting GJ 832 with a 1 bar Earth-like atmospheric composition could lose oxygen through O$^+$ escape within 50 Myr.''} \\
{\bf ***``M dwarfs remain magnetically active over Gyr timescales. For low luminosity mid- and late-M dwarfs possessing compact CHZs ($<0.1$ AU), desiccation and overall atmospheric loss present challenging conditions for planetary habitability at the timescales from a few to a few hundred Myr.''***} \\
{\bf ``If Proxima b is super-Earth size, full mantle convection becomes less efficient, so that heat in deep interiors is transferred by conduction rather than convection. This reduces the efficiency of volcanic activity and the magnetic dynamo in massive terrestrial-type planets (ref).''} \\
{\bf ***``The escape time of a 1 bar oxygen-rich atmosphere on Proxima Cen b is expected to be $\sim10$ Myr.''***} \\
{\bf ``Radiative forcing should play an important role in oxygen ion loss via enhanced ionization and heating of the lower atmosphere, and in the formation of ambipolar electric fields in the upper atmosphere.''} \\
{\bf ``Moore et al.(1997) showed that a solar wind pressure pulse caused significant additional ionospheric outflow at Earth.''} \\
{\bf ``Increased temperature enhances the pressure scale height, and thus provides an additional source of ionospheric heating (besides collisional heating due to radiation and precipitating particles) that will increase the escape rate of oxygen, nitrogen, and heavy ions by a factor of 3-5 with respect to the quiet magnetosphere (refs).''} \\
{\bf ``Charge exchange due to a dense and fast stellar wind will provide an efficient process for additional loss of hydrogen, oxygen, and nitrogen from planetary atmospheres (refs).''} \\
{\bf ``Proxima Cen b appears to have a low probability of being habitable.''} \\
{\bf ***``We conclude that atmospheres of Earth-size planets within respective space weather-affected HZs should be vulnerable to the high XUV fluxes, making close-in planets around low luminosity M dwarfs uninhabitable within a few to hundred Myr, assuming minimal replenishment from volcanic outgassing and bombardment by comets.''***} \\
{\bf ``Our scaling law of the mass-loss rate of O$^+$ with the incident XUV flux suggests that terrestrial-type planets may be habitable if they are located $\geq 0.3$ AU or at the outer edges of CHZ, and therefore a very efficient greenhouse warming should be available to support liquid surace water.''} \\
{\bf ``Occurrences rates and XUV fluxes imply that the best candidates for truly habitable Earth-like planets will be those in the CHZs around middle-aged (over 1 Gyr) mid-K to G dwarfs that provide mild space weather environments.''} \\

\subsection{On the radius of habitable planets \citep{alibert14}}
A maximum radius is estimated, for both Earth-like and super-Earth planets, for which a planet can be habitable based on the conditions that liquid water can be present on the surface, and that an ice layer does not exist at the base of the ocean (lots of water in ocean causes high pressure at bottom). A five-layer model of the planet is computed for various masses, and fairly simple planet/atmospheric compositions. {\bf My current code uses the maximum estimate of 50 Earth oceans for a 1 Earth mass planet, but Fig. 4 and other text may indicate a higher capacity, although the model doesn't include possible interior water storage. For example, the model indicates up to 100 Earth oceans possible on surface.}

\subsection{A terrestrial planet candidate in a temperate orbit around Proxima Centauri \citep{anglada16}}
Proxima Centauri b discovery paper.

\subsection{H2O storage capacity of olivine at 5-8 GPa and consequences for dehydration partial melting of the upper mantle \citep{ardia12}}

\subsection{Stellar Proton Event-induced surface radiation dose as a constraint on the habitability of terrestrial exoplanets \citep{atri20}}

\subsection{Stellar flares versus luminosity: XUV-induced atmospheric escape and planetary habitability \citep{atri20b}}

\subsection{New evolutionary models for pre-main sequence and main
sequence low-mass stars down to the hydrogen-burning limit \citep{baraffe15}}
Evolution models for low-mass stars, including M-dwarfs. These models are widely used throughout the literature, and {\bf could be good for me to use once I start modelling the M-dwarf luminosity in-depth.} 

\subsection{Tidal locking of habitable exoplanets \citep{barnes17}}
Two equilibrium tide models are used to simulate the tidal evolution of planets in the habitable zones (HZ) of FGKM stars, including \textit{Kepler} and \textit{TESS} candidates. Half of \textit{Kepler} targets may be tidally-locked; all \textit{TESS} candidates may be tidally-locked within 1 Gyr. Tidal locking strongest around M-dwarfs, and possible for FGK HZ planets. Eccentric planets may be tidally-locked, but not synchronously-rotating, causing libration. 

\subsection{The habitability of Proxima Centauri b I: evolutionary scenarios \citep{barnes18}}
The evolution of Proxima Centauri b is studied using the VPLANET simulation, including modules accounting for Galactic effects, orbital and rotational evolution of planets within the system (including tidal effects), stellar evolution, atmospheric evolution, and geophysical evolution. Proxima Centauri b could likely have surface water today in three scenarios: a large initial water inventory, low XUV escape efficiency, or an initial hydrogen envelope. The latter is the most likely habitable case, in which the XUV radiation erodes the envelope instead of the surface water, retaining some surface oceans. {\bf Lots of good information for project highlighted throughout this paper.}

\subsection{Magma ocean evolution of the TRAPPIST-1 planets \citep{barth20}}
A new magma ocean module, MagmOc, coupling magma ocean and atmospheric evolution, is added to the popular VPLanet code, and compared with previous magma ocean models. MagmOc is then used to investigate GJ 1132b and early Earth, as well as the main goal of the HZ TRAPPIST-1 planets -- e, f, g. Recent studies have suggested T1e, f, g could be up to several wt\% water, although they are in close proximity to their host star and should experience rapid atmospheric escape. The results, for all planets studied, indicate that at most 3-5\% of the initial water becomes locked in the solidified mantle after magma ocean solidification, and that, for Earth's present-day water content, it must have started quite wet (10-100 TO). A high initial water content leading to a thick steam atmosphere also prolongs the magma ocean phase, prolonging the oxygen sink and preventing abiotic O2 build-up in the atmosphere. {\bf This will be a great reference to come back to when 1D atmosphere is implemented in my own model.} \\

\subsection{Disequilibrium mantle melting \citep{bedard89}}
A geophysics paper, outlining that disequilibrium melting (partial equilibration) is expected, due to long equilibration times compared to the relatively fast flow through dykes and channels. If local equilibrium is maintained, then the mantle should act as an ion exchange column; otherwise, trace elements will move faster through the channels.

\subsection{Stagnant lids and mantle overturns: Implications for Archaean tectonics, magmagenesis, crustal growth, mantle evolution, and the start of plate tectonics \citep{bedard18}}

\subsection{Water vapor on the habitable-zone exoplanet K2-18b \citep{benneke19}}
The authors report the detection of water vapour in the atmosphere of a sub-Neptune orbiting an M3 star. K2-18b receives comparable flux as the Earth receives from the Sun, so it is possible that there are liquid water clouds that could precipitate. The planet likely has a thick H/He envelope; this extended atmosphere is ideal for follow-up transit observations.

\subsection{Water loss from terrestrial planets orbiting ultracool dwarfs: implications for the planets of TRAPPIST-1 \citep{bolmont16}}
1D radiation-hydrodynamic simulations used to estimate water loss during early evolution of planets around ultra-cool dwarfs (including brown dwarfs), indicating that sufficient water may be present to condense once planet has migrated into the HZ (sometimes staying there for $\sim$ 1 Gyr. Results also applied to TRAPPIST-1 planets for comparison. \\
{\it NOTES:} \\
{\bf ``For solar-type stars, the flux in the whole XUV range is two to five times higher than in the X-ray \citep{ribas05}. We thus multiply the value corresponding to the X-ray range by 5. This constitutes an upper limit of what one might expect, indeed for active UCDs the factor can be lower than 2 (e.g., for TRAPPIST-1 it is 1.78). We consider here that it can be used as a proxy for the whole XUV range.''} \\

\subsection{Are exoplanetestimals differentiated? \citep{bonsor20}}

\subsection{Host-star and exoplanet compositions: a pilot study using a wide binary with a polluted white dwarf \citep{bonsor21}}

\subsection{Seafloor weathering controls on atmospheric CO\textsubscript{2} and global climate \citep{brady97}}
Seafloor weathering on Earth appears to be thermally-dependent instead of chemically; basalt weathering in seawater-like conditions is temperature-dependent. Seafloor weathering affects the climate, albeit less than continental weathering, and may have regulated the climate during the continent-free Archean. {\bf This paper has some good ocean geochemistry.}

\subsection{Molybdenum isotopic evidence for the late accretion of outer Solar System material to Earth \citep{budde19}}
While Earth formed in the ``dry'' silicate region of the protoplanetary disk, it still has an abundance of volatiles, including water, delivered at some point during its early evolution. The authors find that the molybdenum isotopic composition of Earth's primitive mantle falls between that of carbonaceous and non-carbonaceous materials, suggesting that Earth's volatiles and water were delivered and accreted late in evolution (i.e., after the core was formed), ``probably through the Moon-forming impact.'' {\bf Since this is a Nature article, I only have access to the abstract.}

\subsection{Water delivery to dry protoplanets by hit-and-run collisions \citep{burger20}}

\subsection{Persistence of flare driven atmospheric chemistry on rocky habitable zone worlds \citep{chen21}}

\subsection{The evolution of solar flux from 0.1 nm to 160 $\mu$m: quantitative estimates for planetary studies \citep{claire12}}

\subsection{The interaction of Venus-like, M-dwarf planets with the stellar wind of their host star \citep{cohen15}}

\subsection{Variable water input controls evolution of the Lesser Antilles volcanic arc \citep{cooper20}}
Water subducted into the mantle influences magma production, continental crust formation, and earthquakes, but is difficult to measure directly. Boron trace elements and isotopic fingerprints of melt inclusions of Atlantic subduction zones are studied here, revealing that serpentine is the dominant supplier of subducted water. The data also indicate the seismic/volanic evolution of the arc is strongly influenced by the structure and hydration of subducted crust, based on volcanic data above fracture zones. \\
{\it NOTES:} \\
{\bf ``Water hosted in hydrous phases within the subducting plate will be released as the slab sinks into the mantle and warms up. As the water migrates out of the slab, the stress on faults is reduced, causing earthquakes. At the same time, the addition of water to the overlying mantle wedge reduces the solidus temperature, which may enhance melting.''} \\
{\bf ``Hydration of lithosphere formed by intermediate or fast spreading occurs mainly in the mafic crust through faults that form as the plate bends into the trench.''} \\
{\bf ``Slow spreading produces highly tectonized oceanic lithosphere with relatively thin mafic crust, pronounced faults and sections of upper-mantle material exposed at the seafloor.''} \\
{\bf ``The transform faults at slow-spreading ridges, which manifest as fracture zones in mature oceanic crust, are more seismically active and penetrate to greater depths than lithosphere formed by faster spreading. These large scale faults provide pathways for seawater and low/medium-temperature alteration including hydration of the mantle mineral olivine to serpentine.''} \\
{\bf ***``Serpentine, in the form of antigorite, can hold up to 13 wt\% structural water, at least double the water capacity of hydrated mafic crust.''} \\
{\bf ***``Subduction of serpentinized mantle lithosphere has the potential to supply substantial volumes of fluid to magmatic arcs.''} \\
{\bf ``Arc magmas produced through mantle melting induced by serpentine-derived fluids have significantly higher $\delta^{11}$B values (up to $+18$ per mil) than mantle from MORB sources ($\sim -7.1$ per mil). Fluids derived from subducted sediments also have a different distinct chemical signature.''} \\
{\bf ``Water contents of melt inclusions are affected by differentiation processes during crustal storage and thus are a poor proxy for primary magmatic water contents. Water content will increase in a melt undergoing undersaturated crystallization, remain constant under water-saturated conditions and be lost from melt during late-stage degassing.''} \\
{\bf ``Further modification of water in melt inclusions can occur because of post-entrapment crystallization and or/diffusive water loss.''} \\
{\bf ``Our data indicate that enhanced fluid mixing of the mantle wedge is associated with higher magma production in the Lesser Antilles volcanic arc.''} \\
{\bf ***``Our observations provide strong evidence that a heterogeneous distribution of serpentine in subducting mantle lithosphere exerts a primary control on along-arc variations in mantle wedge hydration and seismicity and may also influence crustal structure and magmatic productivity of volcanic arcs.''} \\

\subsection{Alien maps of an ocean-bearing world \citep{cowan09}}

\subsection{Water cycling between ocean and mantle: Super-Earths need not be waterworlds \citep{cowan14}}
Two-box model used to study the deep water cycle, particularly the exchange between surface and mantle and its steady-state solutions. This model uses seafloor pressure as the dependent-variable for both regassing (subduction of slabs, transporting water into mantle) and degassing (at mid-ocean ridges or volcanic arcs, back to the surface), which depends on surface gravity. Super-Earths have shallower ocean basins + more water in the mantle, but are more likely to be waterworlds; however, this study finds they are much less susceptible to inundation than expected. Tectonically active planets should have oceans + exposed continents, enabling the silicate weathering thermostat required for habitability. This model is expanded upon in \citep{komacek16}.

\subsection{Geoscience for understanding habitability in the solar system and beyond \citep{dehant19}}
An overview of the geoscience relating to surface habitability of exoplanets, encompassing the entire planet from the core out to the atmosphere. Topics covered include planet formation, role of impacts, planetary interiors (including mantle convection), atmospheric evolution (including escape), the carbon and water cycles on Earth, as well as interactions between the atmosphere, geosphere, and interiors of planets, and possible biosignatures. Overall, this is a thorough review of the coupling between planetary surfaces and interiors (and possible biospheres), {\bf and should be read again at some point.} \\
{\it NOTES:} \\
{\bf ``Earth-like planets probably have masses $\leq 5 M_{\oplus}$ (refs). There is little direct evidence for rocky super-Earths so far in the observation of exoplanets.''} \\
{\bf ``Heat loss through lithosphere is enhanced by the existence of plate tectonics, if the latter occurs.''} \\
{\bf ``The early magma ocean stage is expected to distibute volatile species between the interior and coexisting atmosphere (refs).''} \\
{\bf ``The presence of a steam atmosphere protects the lower atmosphere from harsh incoming radiation and hence influences atmospheric escape rates, photochemistry, and climate during the critical early stage in Earth's history when EUV fluxes were up to $\sim100$ times that of today.''} \\
{\bf ``Some studies suggest planets with plate tectonics have larger outgassing rates compared to stagnant lids, important for atmospheric formation.''} \\
{\bf ``It has been proposed that Earth's plate tectonics is the remnant of magma ocean convection and thus present for most of the planet's history (ref), while others argue plate tectonics have only existed since 2.5-3.2 Ga (refs).''} \\
{\bf ``Oxidizing conditions (like those in Earth's mantle) favour efficient volatile outgassing of the mantle by melting.''} \\
{\bf ``Oxygen fugacity of the mantle is critical to habitability, as it can strongly influence outgassing and thus atmospheric formation.''} \\
{\bf ``The volatile content of the surface environment, particularly the presence of liquid water is thought to have a large feedback on the interior, for example by favoring plate tectonics (e.g., ref). Partial melting depletes the mantle in volatiles and extracts water from the interior to the surface, where it is released via volcanism (e.g., refs).''} \\
{\bf ``Subduction of water is crucial to the production of continents, as it reduces the viscosity of the mantle. With the emergence of a large continental area, weathering rates increase.''} \\
{\bf ``The fraction of emerged continents should strongly depend on initial conditions (e.g., mantle temp, initation time of plate tectonics) as well as the weathering rate (ref).''} \\
{\bf ``Mantle partial melting also means mantle depletion. Water and CO2, especially, are incompatible species and go into the melt easily.''} \\
{\bf ``Hydrodynamic escape is an important process for establishing volatile inventories, particularly for small planets ($<0.8 M_{\oplus}$) as they would be unable to retain their inventory of volatiles like water.''} \\
{\bf ``In the present-day solar system, non-thermal escape mechanisms constitute the bulk of the measured escape rates on terrestrial planets.''} \\
{\bf ``Some recent models indicate that losses from magnetospheres could be enhanced near the magnetic poles and cusps (ref), thus limiting further protection offered by magnetic fields.''} \\
{\bf ``Impacts can create a large input of volatiles in the atmosphere from the melting they cause during impact and through the volatiles that they carry.''} \\
{\bf ``Small impactors are able to deliver volatiles on vaporization but are usually less efficient at creating degassing of the planet.''} \\
{\bf ``Plate tectonics also has a strong influence on the continuous existence and diversity of volcanism. This is linked to subduction of hydrated material, which is released in the mantle wedge and triggers arc volcanism close to the subduction zone.''} \\
{\bf ``The biological enhancement of weathering rates has been argued to lengthen the lifespan of the biosphere (ref).''} \\
{\bf ``Water that is released in subduction zones at a depth of 100-200 km induces partial melting and therefore the production of new continental crust (refs). Importantly, this causes a positive feedback cycle, since more continental crust is subjected to weathering.''} \\
{\bf ``The origin of water on Earth is reasonably well established, and comes mainly from carbonaceous chondrites and degassing.''} \\
{\bf ``Examples of simultaneous biosignatures are e.g., the species pair (CH4 and O2). Without life on Earth to re-supply them, these species would react and be removed to form carbon dioxide and water (ref).''} \\


\subsection{Water delivery to the TRAPPIST-1 planets \citep{dencs19}}

\subsection{Is Proxima Centauri b habitable? A study of atmospheric loss \citep{dong17}}

\subsection{Role of planetary obliquity in regulating atmospheric escape: G-dwarf vs. M-dwarf Earth-like exoplanets \citep{dong19}}

\subsection{Atmospheric escape from TOI-700 d: Venus versus Earth analogs \citep{dong20}}
The atmospheric retention of an Earth-sized planet orbiting an early-M dwarf, TOI-700 d, are assessed using a magnetohydrodynamic simulation of the stellar wind, accounting for planetary atmospheric composition and magnetic field. If unmagnetized, it is possible for TOI-700 d to rapidly lose its atmosphere or retain it for many Gyr, depending on atmospheric composition. A magnetized TOI-700 d with Earth-like atmosphere could probably persist on timescales of $\sim$Gyrs, as the magnetic field reduces the loss rate by roughly an order of magnitude. \\
{\bf ``We employ state-of-the-art magnetohydrodynamic models to simulate the stellar wind and the associated rates of atmospheric escape, mainly considering the planetary atmospheric composition and magnetic field.''} \\
{\it NOTES:} \\
{\bf ***``TOI-700 d is nevertheless capable of retaining a 1 bar atmosphere over gigayear timescales for certain regions of parameter space.''} \\
{\bf ***``Our simulations show that the unmagnetized TOI-700 d with a 1 bar Earth-like atmosphere could be stripped away rather quickly (<1 Gyr), while the unmagnetized TOI-700 d with a 1 bar CO2-dominated atmoshere could persist for many billions of years; we find that the magnetized Earth-like case falls in between these two scenarios.''} \\
{\bf ``As with our solar system, all the planets in the TOI-700 system are outside the region where stellar wind speed is equal to the fast magnetosonic speed; this means the stellar wind environment of the planets is always `superfast'.''} \\
{\bf ***``The upper atmosphere of a Venus-like planet is cooler than that of an Earth-like planet due to efficient CO2 cooling (ref).''} \\
{\bf ***``The upper atmosphere of a Venus analog is cooler than its Earth-like counterpart due to the efficient CO2 cooling caused by 15 $\mu$m emission (ref). In consequence, the exobase of a Venus-like planet is situated lower than that of an Earth-analog, therby making the former more tightly confined.''} \\
{\bf ***``The extent of the atmospheric reservoir that is susceptible to erosion by stellar wind is smaller for the Venus analog as compared to an Earth-like atmosphere.''} \\
{\bf ***``The presence of the global magnetic field reduces the atmospheric loss rate by roughly one order of magnitude relative to the unmagnetized case (see Table 1).''} \\
{\bf ***``In this work, unmagnetized Earth-like worlds are characterized by the highest atmospheric ion escape rates.''} \\
{\bf ``In the case of an unmagnetized Earth-like planet, our simulations imply that a 1 bar atmosphere may undergo complete depletion in $<1$ Gyr.''} \\
{\bf ``When a magnetic field is included and the composition held fixed, the corresponding time for depletion increases to a few Gyr.''} \\
{\bf ``If an Earth-like atmosphere is replaced with a Venus analog, we estimate that the timescale over which a 1 bar atmosphere is lost will be $>10$ Gyr.''} \\
{\bf ***``Depending on the composition and magnetic field, TOI-700 d may possess a sizable atmosphere or be devoid of it.''} \\
{\bf ``The escape rates were probably higher when the star was younger due to intense stellar winds and EUV fluxes (refs).''} \\
{\bf ``Still unclear: the extent of H2O photolysis (ref), driven by the higher incident X-ray and UV fluxes at TOI-700 d (ref) that might produce a thick oxygen atmosphere for this specific planet, as pointed out by Lingam et al.(2020).''} \\
{\bf ***``We showed that retaining a 1 bar atmosphere over Gyr timescales is potentially feasible for TOI-700 d under certain circumstances. This improves the prospects for long-term habitability of this world, and possibly for similar exoplanets around quiescent early M dwarfs.''} \\

\subsection{The occurrence of potentially habitable planets orbiting M dwarfs estimated from the full \textit{Kepler} dataset and an empirical measurement of the detection sensitivity \citep{dressing15}}
Occurrence rates for Earth-size planets and super-Earths orbiting M-dwarfs are estimated using the full \textit{Kepler} data set and a custom pipeline, along with occurrence rates for Earth-size planets and super-Earths per M dwarf HZ using two limiting cases: the classical HZ \citep{kasting93} and the recent Venus/early Mars HZ; the latter gives higher rates. They find an occurrence rate of roughly 2.5 planets per M dwarf, with radii $1-4 R_{\oplus}$ and periods $<200$ days.

\subsection{Exsolved volatiles in magma reservoirs \citep{edmonds18}}
Exsolved volatiles coexist with magma in crustal magma reservoir; this is reviewed at the corresponding crustal temperatures and pressures, along with the implications of these volatiles for the fluid dynamics of the magma reservoir. The authors outline what kind of exsolved volatiles could exist with crustal magmas, distinguishing between decompression-driven (``first-boiling'') and crystallization-driven (``second-boiling'') exsolution. Models are created to assess the impact of volatile content on erupted mass and compared with data. Magma-volatile separation is then assessed, with the results that in crystal-poor melts, exsolved volatiles are convectively separated, while in crystal-rich melts, these volatiles can become separated from the parent magma. {\bf This review would be good for learning details about magma and eruptions, not fully included here.} \\
{\it NOTES:} \\
{\bf ``Magmas close to the surface of Earth, stored within crustal reservoirs, contain exsolved volatiles. This exsolved phase is made up of water and carbon dioxide primarily.''} \\
{\bf ``These volatiles are dominantly sourced from the mantle, but may also be assimilated from crustal rocks and fluids (ref) and are enriched in silicate melts as they cool and fractionate.''} \\
{\bf ``Basaltic melts can reach several wt\% H2O for magma in arc settings (ref).''} \\
{\bf An exsolved volatile phase lowers the bulk density and increases the buoyancy of magmas (ref), influences phase equilibria (refs), as well driving convection (refs) and promoting magma mingling and mixing through overturn (refs).''} \\
{\bf ``Our understanding of the form and organisation of crustal magma reservoirs has evolved in recent years, towards a picture of trans-crustal, mush-dominated, heterogeneous magma reservoirs which may be long-lived relative to inter-eruption timescales.''} \\
{\bf ``Studies indicate that permeable, outgassing pathways through crystal-rich magma may be maintained from depth... The accumulation of an exsolved volatile in the roof zone of such a protracted reservoir may play a critical role in triggering and sustaining eruptions.''} \\
{\bf ``Large volcanic `outgassing events' may relieve reservoir pressure and prevent magmatic eruptions.''} \\
{\bf ``Accumulation of an exsolved volatile phase may also have important implications for the interpretation of ground displacements at the surface through the formation of a highly compressible region of the magma reservoir (refs), which may buffer volume changes and suppress ground deformation.''} \\
{\bf ***``Magmas exsolve water and carbon dioxide as they rise through the crust, cool and crystallize.''} \\
{\bf ***``Melts can reach saturation with respect to an exsolved volatile phase during decompression (termed ``first boiling'') or during isobaric cooling and crystallization (of volatile-poor crystals) in a magma storage area (``second boiling'') leading to the formation of an exsolved magmatic volatile phase (MVP).''} \\
{\bf ``Exsolution of water from silicate melt during decomprression-induced degassing, or first boiling, results in a lowering of the solidus, owing to a decrease in the entropy of mixing caused by the loss of water from the melt.''} \\
{\bf ``Decompression of magma is therefore usually accompanied by extensive degassing-induced crystallization (refs), which has the effect of promoting further volatile exsolution and increasing the bulk viscosity of the magma (ref).''} \\
{\bf ``Second boiling takes place in magma storage areas in the crust, where magmas cool beneath their solidus temperatures and crystallize.''} \\
{\bf ``Rapid rise of magma through dykes to the surface will experience dominantly first boiling vesiculation and expansion.''} \\
{\bf ***``The composition of the evolving exsolved magmatic volatile phase will depend on the initial volatile content of the magma, the pressure, temperature, melt composition, and oxidation state.''} \\
{\bf ``Fig. 2 illustrates the effect of CO2 on the composition and pressure at which volatile-saturation of H2O occurs (ref).''} \\
{\bf ``Fig. 3 shows the relationship between the melt H2O content, the mass of the exsolved volatile phase, and the degree of crystallization at 200 MPa for a range of initial water contents. Substantial fractions of exsolved volatiles may be generated during isobaric crystallization.''} \\
{\bf ``A degassing path is one whereby CO2 is lost at high pressure, followed by H2O at low pressure (ref).''} \\
{\bf ``For ocean island magmas, the exsolved volatile phase at this depth is highly CO2-rich, owing to the much lower bulk H2O concentrations; in arcs, the deep exsolved volatile phase is more hydrous.''} \\
{\bf ``The existence of hydrous magmas deep in the crust lends support to the notion that magma reservoirs from the mid-crust upward contain substantial fractions of exsolved water as supercritical fluid or low density vapour at low pressures.''} \\
{\bf ``Plutonic rocks exposed at the surface, however, present only scant evidence of an exsolved magmatic volatile phase, despite abundant petrological evidence of hydrous components in the melt (e.g., amphibole). However, there is good evidence for magmatic exsolved volatile phase saturation in plutonic igneous rocks (ref).''} \\
{\bf ``In MOR and hotspot settings, basaltic melts typically contain little H2O; for these magmas, saturation in an exsolved fluid phase occurs in the mid-crust but the fluid is initially very CO2-rich; water does not exsolve from the melt in significant quantities until the magma is within a few hundred meters of the surface (ref) or until after significant amounts of isobaric crystallization has taken place.''} \\
{\bf ***``As well as the primary volatile species, H2O and CO2, sulfur is also present at bulk concentrations of up to 0.5 wt\% in arc magmas (refs). `Excess' sulfur emissions during large explosive volcanic eruptions may lead to tropospheric cooling owing to the influence of sulfate aerosols on the radiation budget of Earth's atmosphere (ref).''} \\
{\bf ``Sulfur partitions strongly into aqueous exsolved volatile phase at crustal pressures for water-rich, relatively evolved magmas (refs).''} \\
{\bf ``An exsolved volatile phase that evolves in a magma reservoir prior to eruption is highly variable in sulfur content, dependent on oxygen fugacity, magma composition and temperature.''} \\
{\bf ``Large amounts of exsolved volatiles co-existing with magmas at depth will buffer the C/S ratio of volcanic gases observed at the surface.''} \\
{\bf ``The presence and production of an exsolved volatile phase in crustal magma reservoirs can lead to a number of different effects on rheology, compressibility, and on eruption duration and style, depending on a range of factors, including whether the magma reservoir is sealed or open to the atmosphere, the crystal content and viscosity of the melt, and on whether the magma reservoir is composed of a single magma body or multiple bodies of melt with different composition and gas content.''} \\
{\bf ***``In a magma that contains dissolved volatiles, cooling and crystallization will induce saturation in an exsolved volatile phase (second boiling). The exsolved phase is typically of lower density than the melt phase and this process therefore leads to an increase in volume and hence pressure of the magma body as it deforms the surrounding crust.''} \\
{\bf ***``With a volatile-saturated melt, the magma may be of order 10 time more compressible than a magma with no exsolved volatiles.''} \\
{\bf ``Second boiling during cooling and crystallization, generating a magmatic exsolved volatile phase, can produce enough pressure in the magma reservoir to trigger an eruption (refs).''} \\
{\bf ``It is seen that once the magma is volatile-saturated, the continued exsolution of volatiles with cooling leads to a significant expansion of the melt-crystal-volatile mixture and a large associated increase in the pressure.''} \\
{\bf ``The very large compressibility of the exsolved volatile phase can accommodate large changes in volume for relatively small changes in pressure. This leads to much larger erupted volumes for volatile-rich magmas (refs).''} \\
{\bf ``The eruption duration increases with the compressibility of the magma, consistent with increased mass erupted for volatile-saturated magma.''} \\
{\bf ``Historical eruptions also suggest that the gradual decompression of a deep magma reservoir exerts a long term control on the eruption rate.''} \\
{\bf ``Magma intrusion has often been cited as a trigger for eruption (refs), and there is frequently petrological gand geochemical evidence of magma mixing and/or heating shortly before eruptions (refs).''} \\
{\bf ``If an intrusion of hot, relatively dense magma occurs below an existing body of cooled, evolved melt, the new lower layer of magma may inflate through cooling, crystallization and exsolution of volatiles, but remain dense relative to the overlying layer.''} \\
{\bf ``It is possible, however, for this underplating magma to drive eruption of the overlying magma (refs).''} \\
{\bf ``The accumulation of bubbles in the lower layer of melt will lower the density of this layer and tend to promote mixing or overturn of the magmas.''} \\
{\bf ``In some cases, the separation of volatiles from the melt may lead to a more continuous `open' style of degassing of the volcanic system. This can serve to regulate the pressure in the chamber and may tend to suppress larger, explosive eruptions.''} \\
{\bf ``Magma bodies may spend much of their time in the crust near their solidus temperature, in a partially or entirely crystalline state (refs). Volatile-rich basalts rising up into the crust will tend to underplate and interact with such crystalline magma bodies (refs).''} \\
{\bf ``Accumulations of exsolved volatiles may lead to large emissions of volcanic gases during eruptions.''} \\
{\bf ``The chlorinity of the exsolved volatile phase is critical for its metal-carrying capacity (ref).''} \\
{\bf ``The `tonnage' of metals that may be accumulated migt then be expected to be proportional to the water content of the magma, which will determine the mass fraction of the exsolved volatile phase, its chlorinity, which will depend on the melt chlorine content, temperature and pressure, the overall size of the magma reservoir and perhaps just as importantly, the crystallinity and strucure of the reservoir.''} \\
{\bf ***``Most magmas, particulary those in subduction zone settings, are in equilibrium with an exsolved vapour phase through much of the evolution from the mid-crust to the surface and this exsolved volatile phase plays a fundamental role in magma differentiation, eruption triggering and the formation of hydrothermal ore deposits.''} \\
{\bf ``If a series of fractures or channels open up to allow the exsolved volatiles to leak from the melt-crystal mixture, this can stabilize the vertical density profile of the magma reservoir over time''} \\

\subsection{Volcanism and volatile recycling on a one-plate planet: Applications to Venus \citep{elkins07}}

\subsection{Ranges of atmospheric mass and composition of super-Earth exoplanets \citep{elkins08}}

\subsection{Surface and ocean habitability of TRAPPIST-1 planets under the impact of flares \citep{estrela20}}

\subsection{A nearly water-saturated mantle transition zone inferred from mineral viscosity \citep{fei17}}
Mantle transition zone (MTZ) viscosity inferred from a measurement of dislocation mobility (strength and ductility of crystal structures) of bridgmanite and ringwoodite, believed to make up the MTZ. Ringwoodite should contain 1-2 weight \% water, and the MTZ should be water-saturated.

\subsection{Evolutionary models for ultracool dwarfs \citep{fernandes19}}

\subsection{On the XUV luminosity evolution of TRAPPIST-1 \citep{fleming20}}
{\bf XXXX This paper and its results will be important for our stellar evolution! Check Table 2 for TRAPPIST-1 MCMC-derived parameters. XXXX}
Markov Chain Monte Carlo methods are used to determine the XUV evolution of TRAPPIST-1, strongly dependent on priors used. TRAPPIST-1 planets likely received $10^3$-$10^4$ times the XUV flux currently received on Earth during the star's 1 Gyr pre-MS evolution, and have likely received consistent high XUV flux since then, meaning that the high levels of volatile/water loss on HZ planets e, f, g would require large initial volatile/water inventories. Note that TRAPPIST-1 is a late M-dwarf, an "ultracool dwarf." {\bf The stellar evolution is simulated using the STELLAR module of VPlanet, using \citet{baraffe15} models. The XUV evolution is simulated using the models of \citet{ribas05}.} \\
{\it NOTES:} \\
{\bf ``TRAPPIST-1's planetary system likely experienced a persistent and extreme XUV flux environment (for several Gyr), potentially driving significant atmospheric erosion and volatile loss.''} \\
{\bf ***``High-energy stellar radiation originates from the corona via the heating of magnetically confined plasma (ref). The stellar magnetic field is likely generated via differential rotation within the stellar convective envelope (Parker 1955), linking rotation to stellar activity and XUV emission.''***} \\
{\bf ***``Stellar activity evolution is characterized by two distinct phases. First, in the saturated phase, young, rapidly rotating stars (Ro $\lesssim 0.1$) maintain a constant $L_X/L_{bol} \approx 10^{-3}$ (refs). Then, at longer rotation periods and larger Ro, stars transition to the unsaturated phase in which $L_X/L_{bol}$ exponentially decays over time (refs).''***} \\
{\bf ***``Recent work has shown that the stellar dynamo processes that generate magnetic fields and drive XUV emission in fully convective M dwarfs follow the same evolution with Ro as described above for solar-type stars (Wright \& Drake 2016; Wright et al. 2018). We can therefore apply this model to examine XUV evolution of individual fully convective stars.''***} \\
{\bf ***``We simulate TRAPPIST-1's stellar evolution using the STELLAR module in VPLanet (Barnes et al. 2020), which performs a bicubic interpolation over mass and age of the Baraffe et al. (2015) stellar evolution tracks.''**} \\
{\bf ``STELLAR simultaneously evolves a star's radius, effective temperature, radius of gyration, $L_{XUV}$, and rotation rate in addition to $L_{bol}$. Each VPLanet simulation using STELLAR lasts about 10 s.''} \\
{\bf ``Duration of the saturated phase is likely longer for later-type stars (refs), with fully convective M dwarfs potentially remaining active throughout their lifetimes ($t_{sat} \gtrsim 7$ Gyr, refs).''} \\
{\bf ***``Since Wright \& Drake (2016) found that the X-ray evolution of fully convective stars is qualitatively similar to that of partially convective FGK stars, we adopt the $\beta_{XUV}$ exponent distribution of late K dwarfs from the Jackson et al. (2012) sample as our prior (see paper for prior).''***} \\
{\bf ***``The decline in our MCMC implies that ultracool dwarfs like TRAPPIST-1 likely remain saturated for many gigayears.''***} \\
{\bf ***``Our analysis strongly disfavors short saturation timescales, with only a 0.5\% chance that $t_{sat} \leq 1$ Gyr, the saturation timescale adopted by \citep{luger15} in their analysis of water loss from exoplanets orbiting in the habitable zone of late M dwarfs.''***} \\
{\bf ``There is a 40\% chance that TRAPPIST-1 is still in the high $L_{XUV}/L_{bol}$ saturated phase today, suggesting that the TRAPPIST-1 planets could have undergone prolonged volatile loss.''} \\
{\bf ``The luminosities of ultracool dwarfs do not significantly change duing the main sequence \citep{baraffe15}.''} \\
{\bf ***``We suggest that studies of volatile loss from planets orbiting ultracool dwarfs model the long-term $L_{XUV}$ evolution of the host star, or at least assume saturation timescales of $t_{sat} \gtrsim 4$ Gyr.''***} \\
{\bf ***``The high-energy fluxes incident on the innermost planets throughout the pre-MS phase were probably large enough for atmospheric mass loss to be recombination-limited and scale as $\dot{m} \sim F_{XUV}^0.6$ (ref) as opposed to the oft-assumed energy-limited escape ($\dot{m} \sim F_{XUV}$; Watson et al. 1981; ref), potentially inhibiting volatile loss.''***} \\
 
\subsection{Dynamics of the Pacific-North American plate boundary in the western United States \citep{flesch00}}

\subsection{Generation of plate-like behaviour and mantle heterogeneity from a spherical, viscoplastic convection model \citep{foley09}}

\subsection{The conditions for plate tectonics on super-Earths: inferences from convection models with damage \citep{foley12}}

\subsection{The role of plate tectonic-climate coupling and exposed land area in the development of habitable climates on rocky planets \citep{foley15}}

\subsection{Whole planet coupling between climate, mantle, and core: implications for rocky planet evolution \citep{foley16}}
A review of the physics and geology important for habitability of terrestrial planets, focusing on studies of silicate planets with iron cores here in the Solar System. Water and carbon dioxide are said to be the most important greenhouse gases; carbon dioxide can regulate a cool climate, allowing plate tectonics to occur, but silicate weathering is required to avoid hot climates. Plate tectonics is coupled with the generation of a magnetic field to protect a planet's surface from harmful stellar radiation, maintaining high heat flow out of the core. The effects of the atmosphere are also outlined, including volatile content and initial atmospheric composition. The aforementioned magnetic field can prevent total atmospheric escape.

\subsection{The dependence of planetary tectonics on mantle thermal state: applications to early Earth evolution \citep{foley18a}}
A model using grain size reduction to investigate the mobile lid regime (i.e., where stresses lead to the mantle failing and developing lithospheric shear zones) of the early Earth to explain the origin of plate tectonics. Plate tectonics are not seen on other rocky Solar System bodies (e.g., Venus, Mars), which instead show a stagnant lid. High mantle temperatures may instead lead to a ``sluggish lid'' regime with drip-like subduction, which can explain some early Earth geologic records (e.g., changes in rate of continental crust formation). The grain size reduction model does not favour early Earth in the stagnant lid regime. {\bf This could be an important reference for determining what happens when mantle temperature falls below the solidus, and what happens to degassing/regassing rates.}

\subsection{Carbon cycling and habitability of Earth-sized stagnant lid planets \citep{foley18b}}
The negative feedback provided by the carbon cycle is critical to Earth maintaining a habitable climate, and is facilitated by plate tectonics. The carbon cycle on a stagnant-lid Earth-like planet is modelled, including thermal evolution, CO2 degassing, both continental and seafloor weathering, and crustal growth; essentially, the authors check whether the weathering is in the ``supply-limited'' regime. The results indicate that plate tectonics may not be necessary for habitability, as stagnant lid planets can maintain a stable climate, for a range of initial radiogenic heating rates and total CO2 budgets, for 1-5 Gyr, after which volcanism ceases. The results provide constraints on when a stagnant-lid planet may lose its habitability as a function of age, important because planetary heat budget and age may be constrained in the near-future. In summary, a planet's initial carbon budget and radiogenic heating budget are critical to the habitability of stagnant lid planets, with a higher heating budget prolonging volcanism/CO2 degassing. {\bf This is likely an important reference to read again, since it is quite complex.} \\
{\it NOTES:} \\
{\bf ``Plate tectonics also leads to the creation of continents, which increase the area of exposed land and also enhance weathering (ref).''} \\
{\bf ``In the stagnant-lid regime, the lithosphere is rigid and nearly immobile, and no subduction occurs; convection only takes place between the rigid lid, in the form of drip-like downwellings from the base of the lid (refs). Recycling of volatiles is thus seemingly limited.''} \\
{\bf ``Stagnant lid planets still experience volcanism, which can release mantle CO2 to the atmosphere, create fresh rock and topography at the surface, and potentially even allow for some surface recycling through burial by lava flows (refs).''} \\
{\bf ``The upper limit to CO2 drawdown occurs when weathering immediately and completely alters all available fresh rocks, as soon as it is supplied to the weathering zone -- ``supply-limited. When weathering is not supply-limited, a feedback that stabilizes temperate surface temps is possible.''} \\
{\bf ``We assume the mantle has an oxidation state similar to Earth's mantle, such that CO2 is a primary outgassing product of mantle volcanism.''} \\
{\bf ``Melt migration through a thick stagnant lid is not well understood, and melt could stall during its ascent through the lid, or even at the lid base. Melt stalling would slow mantle cooling and prolong volcanism, but also reduce outgassing.''} \\
{\bf ``Our temperature equation ignores advective heat transport, which occurs when crust newly generated by volcanism pushes the pre-existing crust down into the mantle. At small Peclet numbers, advection is negligible.''} \\
{\bf ``Note that melt production, and hence crustal production and CO2 outgassing, is continuous and global as long as volcanism is active, as melt is constantly generated by upwelling mantle into a global melt zone.''} \\
{\bf ``Mantle melting and volcanism produce a crust at the surface that grows over time. The lower crust may founder and sink into the mantle, preventing the crust from becoming so thick that it influences the thickness of the stagnant lid, and thus heat flow and velocity.''} \\
{\bf ``We assume that crust delaminates when it reaches the base of the thermal lithosphere as dictated by convective instability, as in this region the viscosity contrast between lithosphere and mantle is low.''} \\
{\bf ``Incompatible elements, including heat-producing elements, are preferentially incorporated into the melt phase during mantle melting. As a result, the crust becomes enriched in heat-proudcing elements, and the mantle becomes depleted.''} \\
{\bf ``Carbon is lost from the mantle during volcanism and deposited in the crust via weathering. As carbonated crust is buried by subsequent lava flows, temperature and pressure conditions are typically high enough for crustal decarbonation to occur (ref).''} \\
{\bf ``The dependence of weathering rate (used in this model) on surface temperature and atmospheric CO2 content allows a balance between weathering and degassing to be quickly established.''} \\
{\bf ``The depth at which crustal decarbonation occurs determines whether carbonated crust is recycled into the mantle by convective foundering or whether the crust degasses its carbon back to the atmosphere before it founders.''} \\
{\bf ``CO2 is incompatible, and is therefore preferentially lost from the soloid to the melt, analogous to the heat-producing elements.''} \\
{\bf ``Volcanic degassing of the mantle causes a net transfer of carbon from the mantle to the crust over the lifetime of a stagnant lid planet.''} \\
{\bf ``A planet would have to have a mantle CO2 budget $\approx 4-110$ times larger than Earth's for supply-limited weathering to prevail due to mantle degassing (ref). Supply-limited weathering will occur at a lower total planetary CO2 budget when the CO2 predominantly resides in the crust as opposed to the larger-volume mantle.''} \\
{\bf ``Transferring carbon from mantle to crust via volcanism, and continually burying this crust through the metamorphic decarbonation depth, is thus an effective means of sustaning high CO2 degassing rates.''} \\
{\bf ``Even with a high radiogenic heating budget, a planet with a low CO2 inventory will quickly become globally frozen, even though volcanism can continue for Gyrs.''} \\
{\bf ``The likelihood of planets meeting the CO2, radiogenic heating budget, and age constraints for maintaining habitable climate conditions in the stagnant lid regime appears to be relatively high.''} \\
{\bf ``Larger planets have a smaller surface-area-to-volume ratio and thus retain their heat for longer, all else being equal. Therefore, increasing planet size is expected to prolong mantle melting and volcanism. However, melt will be formed at higher pressures due to the larger pressure gradient on a massive planet and could end up being too dense to erupt.''} \\
{\bf ``\citep{noack17} argue that planets larger than $\approx2-3 M_{\oplus}$ will experience no degassing because dense melt will prevent eruption.''} \\ 
{\bf ``Core cooling would also lead to the formation of plumes, which could sustain volcanism even after pressure release volcanism due to passive mantle upwelling has ceased, although plumes are likely not enough to maintain habiatable conditions.''} \\


\subsection{Far-ultraviolet activity levels of F, G, K, and M dwarf exoplanet host stars \citep{france18}}
Archive and newly obtained observations of dwarf stars are used to study the correlation between far-UV (FUV) and extreme-UV (EUV) radiation, and to calibrate the flux relationship; the authors present an equation for EUV flux based on observations of the bolometric flux and 4 ionic lines (FUV, favouring N V and Si IV lines). {\bf Useful for atmospheric loss around M dwarfs?} \\
{\it NOTES:} \\
{\bf ***``We demonstrate that UV activity-rotation relation in the full F-M star sample is characterized by a power-law decline (with index $\alpha approx -1.1$), starting at rotation periods $gtrsim 3.5$ days.''***} \\
{\bf ``We find that star-planet interactions (SPIs) are not a statistically significant contributor to the observed UV activity levels.''} \\
{\bf ``The success of planet searches employing radial velocity techniques and transit photometry has demonstrated that $\sim300-400$ stars in the solar neighbourhood ($d < 50$ pc) host confirmed planetary systems.''} \\
{\bf ``The UV stellar spectrum is required to understand HZ atmospheres, as it both drives and regulates atmospheric heating and chemistry on Earth-like planets and is critical to the long-term stability of terrestrial atmospheres.''} \\
{\bf ***``Extreme-UV (EUV; $100 \lesssim \lambda \lesssim 911$ Angstroms) photons from the central star drive thermospheric heating, and this may lead to significant atmospheric escape (refs).''***} \\
{\bf ***``Ionization by EUV photons and the subsequent loss of atmospheric ions to stellar wind pick-up can also drive extensive atmospheric mass-loss on geologic timescales (refs).''***} \\
{\bf ***``Stellar far-ultraviolet (FUV) observations serve as a means for predicting the ionizing (EUV) flux from cool stars, either through the use of solar scaling relations (ref) \citep{youngblood16} or more detailed differential emission measure techniques (ref).''***} \\
{\bf ``Magnetic fields play a crucial role in protecting surface life from damaging high-energy particles from stellar winds and coronal mass ejections (ref), as well as promoting the long-term stability of planetary atmospheres (ref). Searches for exoplanetary magnetic fields have not yielded any firm detections to date (ref).''} \\
{\bf ``The presence of a planetary magnetic field may induce interactions that can generate enhanced flare activity (ref). Enhanced flare activity in star-planet systems (ref) appears promising and phase-resolved observations may provide more direct clues to the properties of exoplanetary magnetism.''} \\
{\bf ***``Tidal (gravitational) SPIs may alter the rotational evolution of the host star and the orbital evolution of the planet (ref). In this way, tides may significantly affect the stellar activity level.''***} \\
{\bf ``SPI signals could be expected to correlate with mass of the planet over the semi-major axis of the planet (or other proportionalities between the dissipated power and the star-planet system configuration).''} \\
{\bf ***Table 3 shows the flux measurements for planet host stars in the sample; Table 4 shows the same for the non-planet host stars***} \\
{\bf ***``Comparing the FUV activity indices with the stellar rotation periods, we observe a ``saturated'' plateau followed by a roughly continuous, power-law decline in UV activity. The transition between high-UV activity and intermediate-to-low UV activity occurs around rotation periods of 3.5 days.''***} \\
{\bf ``The transition to the low-activity UV state takes place at shorter rotation periods for cool stars as a whole, relative to M-dwarf only samples. This indicates that warmer stars ``turn-over'' to a lower activity level at shorter rotation periods than for M stars.''} \\
{\bf ``There is essentially no discernible difference between the FUV activity levels of the planet-hosting and non-planet-hosting samples. This supports the assertion made above that we are observing an age spread of a single stellar population as opposed to two distinct planet-hosting and non-planet-hosting groups.''} \\
{\bf ***Equation (4) allows calculation of the EUV flux based on ion flux. ``We therefore favour N V and Si IV asthe best proxies for fractional EUV flux.''***} \\
{\bf ***``We determine that (1) the EUV fluxes follow a power-law relationship with the FUV transition region activity indices over a wide range of spectral types and rotation periods and (2) with an estimate of a star's bolometric luminosity and a measurement of one of the higher temperature FUV emission lines, the stellar EUV flux in the 90-360 Angstrom band can be estimated to roughly a factor of two.''***} \\
{\bf ***``We do not suggest extending these relations to the X-ray wavelengths (5-100 Angstroms).''***} \\
{\bf ``The results suggest a common picture where the overall XUV + FUV (5-1800 Angstroms) flux decreases by one to two orders of magnitude as the stars age from $\sim0.5$ to 5 Gyr.''} \\
{\bf *** Figure 6: ``One observes a $\sim$two order of magnitude decline in the EUV emission strength as cool stars move from the saturated activity regime at rotation periods $\lesssim 3.5$ days to the presumably older population at rotation periods $gtrsim 20$ days.'' This Figure also shows that M dwarfs ``maintain higher relative EUV activity levels than solar-type stars for a given rotation period (a proxy for age)''*** } \\
{\bf ***``For terrestrial atmospheres, increasing the EUV flux to levels estimated for the young Sun ($\sim 1$ Gyr; ref) can increase the temperature of the thermosphere by a factor of $\gtrsim 10$ \citep{tian08}, potentially causing significant and rapid atmospheric mass-loss.''***} \\
{\bf ***``The issue of increased EUV irradiance and the atmospheric stability of rocky planets (ref) is even greater for M dwarfs, where the EUV irradiance levels of even field-age stars (ages $\sim2-6$ Gyr) are predicted to drive runaway oxidation as many Earth oceans worth of hydrogen are lost (refs).''} \\
{\bf ``Our results suggest that stars with more massive and close-in planets emit more UV photons from their chromospheres and transition regions relative to their bolometric luminosity.''} \\
{\bf ``The results also confirm the general trend between $\log$ SPI versus $\log F_{ion}/F_{bolom}$ identified for M dwarfs by (ref), with the caveat that our lager sample identifies significnat stellar and observational biases that may drive this result.''} \\
{\bf ***``We conclude that the SPI does not play and explicit role in shaping the distribution of UV activity indices in our sample.''***} \\
{\bf ***``The bimodal distribution in FUV activity level can be explained by a sample bias: exoplanet host stars bright enough to obtain UV observations largely come from radial-velocity surveys that specifically select for low-activity stars.''***} \\
{\bf ***``A note of caution for researchers modelling exoplanetary atmospheres: by selecting stellar irradiance levels based solely on samples of exoplanet host stars, one is underestimating the flux levels seen earlier in that planet's evolution by an order of magnitude or more.''***} \\
{\bf ``We found a significant correlation ($\sim 99$\% confidence) between the presence of massive, short-period planets and stellar activity as indicated by enhanced FUV emission. However, astronomical and observational biases complicate the direct connection of the enhanced UV activity with the planetary system.''} \\

\subsection{The high-energy radiation environment around a 10 Gyr M dwarf: habitable at last? \citep{france20}}
Study of GJ 699 indicate that flares may be the controlling factor on atmospheric retention/habitability at late times, if terrestrial atmospheres can survive the initial high-loss induced by the early stellar evolution, or if it can obtain a secondary atmosphere.

\subsection{The California-Kepler survey VII: Precise planet radii leveraging GAIA DR2 reveal the stellar mass dependence of the planet radius gap \citep{fulton19}}

\subsection{Improved angular momentum evolution model for solar-like stars II. Exploring the mass dependence \citep{gallet15}}
A model is presented for the rotational evolution of 0.5 to 1.1 $M_{Sun}$ stars, including pre-MS and ZAMS spin-up, based on three key parameterizations: star-disk interaction, wind braking, and core-envelope decoupling. Various free parameters are adjusted to match recent observations. Lower mass stars have a wide dispersion of rotational velocities at ZAMS compared to Solar-like stars. The stars of all masses appear to have three distinctive evolution regimes: a slowly varying spin rate during first few Myr, then rapid acceleration towards the ZAMS, then braking during longer timescales on the MS.

\subsection{Stability of CO2 atmospheres on desiccated M dwarf exoplanets \citep{gao15}}

\subsection{On the magnetic protection of the atmosphere of Proxima Centauri b \citep{garcia17}}
The ionospheric outflows of H$^+$ and O$^+$ are calculated by placing the Earth at Proxima Centauri b's orbit and accounting for the enhanced EUV flux from the host M dwarf. The results reinforce the sensitivity of O$^+$ to increased EUV, and indicate that ion escape could completely remove the atmosphere of an Earth-like planet around Proxima Centauri, {\bf an `old' M dwarf which is past the pre-MS phase}. \\
{\it NOTES:} \\
{\bf ***``Planetary magnetic fields can prevent direct stripping away of the planetary atmosphere by the stellar wind, but ion escape can still occur at the magnetic poles. This process, the polar wind, is well known to occur at Earth and may have contributed to the habitability of Earth's early atmosphere.''***} \\
{\bf ***``The polar wind is highly variable and sensitive to both ionizing radiation and geomagnetic activity. The higher ionizing radiation levels of M dwarfs at habitable zone distances are expected to increase the polar wind by orders of magnitude and, instead of helping create a habitable atmosphere, may strip away enough volatiles to render the planet inhospitable.''***} \\
{\bf ***``We show that an Earth-like planet orbiting at the distance of Proxima b would not survive the escape of its atmosphere at that location, and therefore the pathway to habitability for Proxima b requires a very different atmospheric history than that of Earth.''***} \\
{\bf ``Proxima Centauri is characterized by high magnetic activity and flaring (refs), resulting in a large and highly variable EUV and X-ray flux at the closely orbiting HZ planet.''} \\
{\bf ``A magnetized planet is also subject to atmospheric erosion at the magnetic poles, where open field lines extend from the planet's surface and into the stellar wind, providing an open pathway for ionospheric escape (ref).''} \\
{\bf ***``Ionospheric plasma processes provide additional energy for atmospheric escape and often act differently on different ion species. As such, they can regulate the composition through allowing escape of ions that may otherwise be gravitationally bound.''***} \\
{\bf ``The acceleration of ionized oxygen into Earth's magnetosphere indicates that additional energization processes are effective for ions that enable their escape.''} \\
{\bf ``The incident EUV spectrum influences several of these processes, both by ionizing the neutral atmosphere to provide the ionospheric source population and by heating the ionosphere and neutral atmosphere.''} \\
{\bf ***``Ionospheric escape is believed to have played a role in the pathway to habitability of the atmosphere of early Earth, when the Sun's flare activity and the resulting ion outflow were much larger \citep{airapetian17}.''***} \\
{\bf ``Similar results are likely for the TRAPPIST-1 planets: TRAPPIST-1 appears to have weaker Ly$\alpha$ emission relative to its bolometric luminosity than Proxima by a factor of $\sim 2$ but stronger X-ray emission by a factor of $\sim 3$ (refs).''} \\
{\bf ***``Since coronal emission from M dwarfs such as Proxima appears to have what amounts to a continuous flaring component (ref), the choice of which constitutes a typical level of emission is somewhat subjective.''***} \\
{\bf ***``We expect that enhanced EUV, as well as enhanced driving from stellar wind interactions with the planetary magnetic field, will increase neutral thermospheric temperatures (ref). In the context of an Earth-like upper atmosphere subject to intense stellar flaring, it is likely that NO production would help cool down the atmosphere and prevent hydrodynamic escape \citep{tian08} (other ref too).''***} \\
{\bf ``The superthermal electron population enhances the heating and ambipolar electric field that enhances upward ion acceleration along magnetic field lines.''} \\
{\bf ***``Superthermal electron effects, as well as ion production and heating, enable escape of ionospheric ions, particularly heavier species like O+ that may not be able to escape through purely hydrodynamic processes.''***} \\
{\bf ``Wave-particle interactions and electrons precipitating from the magnetosphere both occur in Earth's ionosphere and enhance ionospheric outflow.''} \\
{\bf ``The ionization rate for Proxima b is significantly higher than for Earth due to both the higher EUV flux and electron fluxes that lead to increased secondary production.''} \\
{\bf ``The heating rate likewise increases due to both the increase in superthermal fluxes and the increase in thermal electron density.''} \\
{\bf ``The number density for H$^+$, O$^+$, and thermal electrons increase by at least an order of magnitude relative to Earth, and the outgoing fluxes of these three species also increases.''} \\
{\bf ***``At Earth, there is little to no escape of O$^+$ without some additional energy input in the form of electromagnetic waves or particle precipitation from the magnetosphere. At Proxima b, the EUV flux alone results in significant outgoing fluxes, while increasing the thermospheric temperature increases fluxes further.''***} \\
{\bf ***``The simulations shown in this Letter are consistent with that picture of the sensitivity of O$^+$ to enhanced EUV radiation, much larger than H$^+$.''***} \\
{\bf ``Studies at Earth have also indicated that O$^+$ fluxes are enhanced by charged particle precipitation (ref), magnetic field-aligned currents (ref), and wave-particle interactions (ref).''} \\
{\bf ``Both electromagnetic flux and energetic particle flux associated with flares can increase the lower-altitude ionization and heating rates and thereby indirectly enhance ion escape.''} \\
{\bf ``Our simulations show that a factor of four increase in thermospheric temperature increases the outflowing O$^+$ fluxes by more than an order of magnitude.''} \\
{\bf ***``For the O+ and H+ escape fluxes for $3 \times T$, the total mass loss rate results in a loss of the entire mass of the atmosphere over 365 Myr.''***} \\
{\bf ``We have carried out full simulations of the effect of the thermospheric temperature on escape, rather than relying on outflow scaling with thermospheric temperature as done in \citet{airapetian17}.''} \\
{\bf ***``The presence of a magnetosphere is not enough to prevent significant atmospheric loss through ionospheric heating and escape.''***} \\
{\bf ***``A path to habitability for Proxima b would be the replenishment of its atmosphere through outgassing or precipitation of volatiles.''***} \\
{\bf ***``The large escape rates compared to Earth indicate that, even in the case of an Earth-like intrinsic magnetic field, the evolution of the atmosphere of Proxima b would be very different from that of the Earth.''***} \\

\subsection{Evolution of Earth-like planetary atmospheres around M dwarf stars: assessing the atmospheres and biospheres with a coupled atmosphere biogeochemical model \citep{gebauer18}}
First study to use a model coupling the biosphere, geosphere, and atmosphere of an Earth-like planet orbiting in the HZ of the M-dwarf star, AD Leo, is used to investigate oxygen production and levels, beginning with an Archean-like Earth atmosphere. Smaller oxygen productivity levels are required to oxygenate the atmosphere, and thus a "Great Oxidation Event", akin to that on early Earth, may have occurred earlier in the planet's evolution, strengthening the possibility of the presence of both O\textsubscript{2} and O\textsubscript{3}. {\bf Read again later after learning more geochemistry.}

\subsection{Characterizing atmospheres of transiting Earth-like exoplanets orbiting M dwarfs with James Webb Space Telescope \citep{gialluca21}}

\subsection{Seven temperature terrestrial planets around the nearby ultracool dwarf star TRAPPIST-1 \citep{gillon17}}
Discovery paper for the now well-known TRAPPIST-1 system of seven planets orbiting very close to a late M-dwarf star. Three planets are believed to be in the habitable zone.

\subsection{The TRAPPIST-1 JWST community initiate \citep{gillon20}}

\subsection{The habitability of stagnant-lid Earths around dwarf stars \citep{godolt19}}
Earth is the only known planet with plate tectonics; as such, stagnant lid planets must be studied. A mantle convection model and a 1D cloud-free climate model, along with simple irradiation models for dwarf stars, are used to determine the H\textsubscript{2}O and CO\textsubscript{2} outgassing rates and HZ limits for planets orbiting F, G, K, and M-dwarf stars. Specifically, planets orbiting M-dwarfs, which are very active early in their evoluton, can potentially lose all their water during an early magma ocean phase, and secondary outgassing, once the planet has solidified, can rejuvenate the surface water, as long as the mantle is not too dry. {\bf Read this again -- lots of good info on outgassing, atmospheric loss, M-dwarf HZs, etc.}

\subsection{Atmospheric escape processes and planetary atmospheric evolution review \citep{gronoff20}}
{\it NOTES:} \\
{\bf``Photodissocation reactions seldom considered in evaluating escape rates since production of fast enough particles to escape is small with respect to ion recombination processes.''} \\
{\bf ``Ion escape is believed to be one of the major sources of atmospheric escape... at exoplanets around M-dwarfs.''} \\
{\bf ``The main problem of the energy limited escape approximation is that it is too often applied to rocky exoplanets while concealing these major limitations: 1. The escape regime is not evaluated; 2. The atmospheric profile is not evaluated; 3. Atmospheric composition not taken into account, H assumed to be only species (presence of cooling species like CO2 can change escape regime); 4. Only photo-ionization heating taken into account; 5. Non-thermal processes not addressed.''} \\
{\bf ``Energy-limited escape models can be interesting for studying H-rich rocky planets early in their histories, for which the escape of H may not have been diffusion-limited but energy-limited (Tian et al., 2005), however energy-limited escape is less relevant to more comprehensive habitability studies.''} \\
{\bf ``All modelling studies to date have indicated that the H2 upwelling into the lower thermosphere, combined with additional H2 produced in the thermosphere, sets the eventual planetary escape flux of H2.''} \\
{\bf ``Lower mass stars are expected to be saturated (in X-ray luminosity vs. bolometric luminosity), and therefore comparitively more active and UV and X-ray bright, than higher mass stars for much longer.''} \\
{\bf ``There are signs from the extant record of early Earth history that plate tectonics may not be the upper limit for crustal recycling rates on the Earth. Instead, the Earth may have experienced a `heat-pipe' phase (W. B. Moore \& Webb, 2013). In this phase, persistent mafic to ultramafic volcanism regularly re-surfaced the Earth. Both crustal material and surface water are cycled back into the mantle through repeated eruption and burial of older flows. The heat pipes were associated with greater eruptive volumes of volcanic material as well as faster crustal recycling than plate tectonics.''} \\
{\bf Runaway greenhouse of Venus explained very well with respect to Earth (Section 4.2.2.4 -- READ AGAIN!} \\
{\bf ``(Conclusions of Garcia-Sage et al. (2017)) means that, to sustain habitability in the sense of liquid water existing at the surface, rocky exoplanets would require a large amount of volatiles in their initial inventory, and that they should not lose them in the active young years of their host star.''} \\
{\bf ``While they are the easiest targets for detecting habitable exo-atmospheres with instruments such as JWST, planets in the HZ of red dwarfs may not be able to sustain them and therefore would be the worst target.''} \\
{\bf `` The Alfven surface might serve as an inner limit at which the HZ can be placed for a given stellar system.''} \\
{\bf ``Calculation of Joule heating of upper atmosphere required for case where EUV \& X-ray stellar radiation is much higher than the Earth case, and for different atmospheric compositions.''} \\
{\bf ``It was believed that a planetary magnetic field was shielding its atmosphere from escape, until observations showed that the escape rate at Earth is higher than at Venus and Mars... The thermospheres of Venus and Mars are called cryospheres because of the cooling effect of CO2: their thermospheric temperature is extremely low, effectively shielding them from several escape processes.''} \\
{\bf ``Recent developments in study of the coupling between stellar wind, magnetospheres, and ionospheres challenge the idea of a protective effect of magnetospheres on atmospheric erosion.''}
{\bf ``The impact of intermittent stellar events such as flares, CMEs, and SEPS on atmospheric escape is seldom taken into account... major problem for studies of close-in exoplanets.''}
{\bf ``Overall it should not be forgotten that the inventory of volatiles, which has been estimated from the density of the exoplanet, will define the lifetime of a habitable world with large escape rates.''} \\
{\bf ``(Atmospheric modelling) will help prevent poor estimates based on energy-limited escape, that do not take diffusion limitation into account.''} \\

\subsection{Earth as a proxy exoplanet: deconstructing and reconstructing spectrophotometric light curves \citep{gu20}}

\subsection{Atmospheric dynamics on terrestrial planets: the seasonal response to changes in orbital, rotational, and radiative timescales \citep{guendelman19}}

\subsection{Lithologic controls on silicate weathering regimes of temperate planets \citep{hakim20}}

\subsection{AQUA: A collection of H2O equations of state for planetary models \citep{haldemann20}}

\subsection{Emergence of two types of terrestrial planet on solidificaiton of magma ocean \citep{hamano13}}

\subsection{On the possible evolutionary history of the water ocean on Venus \citep{hara20}}
1D radiative-convective atmosphere model

\subsection{Imaging the dusty substructures due to terrestrial planets in planet-forming disks with ALMA and the next generation Very Large Array \citep{harter20}}

\subsection{\textit{Ariel} planetary interiors White Paper \citep{helled21}}
Observational prospects and potential challenges in linking planetary atmospheres to bulk interior composition.

\subsection{Partitioning of water during melting of the Earth's upper mantle at H2O-undersaturated conditions \citep{hauri06}}

\subsection{The mass-luminosity relation from end to end \citep{henry04}}

\subsection{The atmospheres of rocky exoplanets I. Outgassing of common rock and the stability of liquid water \citep{herbort20}}
Modelling of outgassing of various terrestrial species, using a thermo-chemical equilibrium model for the crust and near-crust atmosphere rocks and condensates, and multiple possible planetary bulk compositions (Bulk Silicate Earth, continental crust, MORB, CI chondrite, polluted WD), indicates that if phyllosilicates are present and not saturated (especially at lower temperatures), they can remove water from the gas phase (very effectively like a sponge, potentially removing all water from the atmosphere), inhibiting the condensation and thus presence of liquid surface water. These were neglected in previous studies, but are clearly quite important. The model also outputs the composition and oxidation state of the near-crust atmosphere at various temperatures for each planet bulk composition, as well as condensates as planetary temperature evolves. \\
{\it NOTES:} \\
{\bf ``For temperatures between $\sim 600$ K and $\sim 3500$ K, the near-crust atmospheres of all considered total element abundances are mainly composed of H2O, CO2, SO2 and in some cases of O2 and H2.''} \\
{\bf ***``For temperatures $lesssim 500$ K, only N2-rich or CH4-rich atmospheres remian.''} \\
{\bf ``The inclusion of phyllosilicates as potential condensed species is crucial for lower temperatures, as they can remove water from the gas phase below about 700 K and inhibit the presence of liquid water.''} \\
{\bf ***``H2O, O2, and CH4 are natural products from the outgassing of different kinds of rocks that had time to equilibrate. These are discussed as biomarkers, but do emerge naturally as a result of the thermodynamic interaction between crust and atmosphere. Only the simultaneous detection of all three molecules might be a sufficient biosignature, as it is inconsistent with chemical equilibrium.''} \\
{\bf ``To date, it is impossible to determine whether the surface material has been processed by plate tectonics or by radiation and stellar winds on an exposed surface only.''} \\
{\bf ``We note that phyllosilicate materials form already in planet-forming disks (ref) and may contribute substantially to the water delivery to Earth.''} \\
{\bf ***``Phyllosilicates should be an integral part of equilibrium condensation models as they are known in geology to form effectively in a wide temperature range on relatively short timescales as condensates directly from the gas phase, but also as alterations in silicate rocks exposed to water vapour (refs).''} \\
{\bf ``The crust of Earth differentiated into two different parts, continental and oceanic crust. The oceanic crust is geologically younger and consists mainly of basalt whereas continental crust is much older, less dense, and consists mainly of granite.''} \\
{\bf ``Bulk Silicate Earth (BSE) is an approximation for the composition of the Earth excluding its core. This leads to a composition that is rich in MgO and FeO-bearing silicates (mafic), but relatively poor in C, N, F, P, S, Cl, K.''} \\
{\bf ``Both the (ref) model and our model show that H2O and O2 are the most abundant gas species in a wide temperature range.''} \\
{\bf ``Chondrites are believed to be remainders of the formation period of the planets and thus can provide insights into the composition of the planets at very early stages.''} \\
{\bf ``The volatile elements H, C, N are significantly more abundant in CI chondrites than continental crust, BSE, and MORB.''} \\
{\bf ***``The CI chondrite composition is the only rocky element composition considered in this paper that produces liquid and solid water without the need for additional hydrogen and oxygen. CI chondrites are hydrated so much that even after the formation of phyllosilicates, there is still some water left to condense.''} \\
{\bf ``white dwarfs, the burnt-out cores of low-mass stars which become visible only after terminal ejection of a planetary nebula, usually have pure hydrogen-helium atmospheres, because all elements heavier than He settle down quickly ($\sim10^5$ yrs) in the extremely strong gravitational field of the object (refs). Nevertheless, some WDs show an enrichment in elements like O, Mg, Al, Si, Ca, Fe, and C (e.g., refs).''} \\
{\bf ``The observed relative metal abundances in polluted WDs are comparable to Earth's composition with some deviations. The most plausible reason for such enrichment with heavy elements in a WD atmosphere is the accretion of planetesimals or planets from the WD's planetary system.''} \\
{\bf ``A reducing atmosphere is expected for planets with increased total iron abundance in the crust, or could be caused by the late delivery of iron-rich bodies to the planet surface.''} \\
{\bf ***``Crucial for water formation is the presence of hydrogen. With additional H and O, the formation of liquid and solid water succeeds in our models.''} \\
{\bf ``Phyllosilicates can act as a reservoir for capturing a certain amount of water. By adding sufficient hydrogen AND oxygen, it is possible to saturate the phyllosilicates and to have liquid and solid water as stable condensates.''} \\
{\bf ``The existence of phyllosilicates is thermodynamically impossible in the hot core and overwhelming part of the mantle. The large amount of water that had once been present in this matter is likely to have been driven out into the crust and atmosphere during planet evolution. This way, there is plenty of water available to saturate the phyllosilicates in the crust AND to have excess water to form an ocean.''} \\
{\bf ``The temperature window in which O2 is the dominant gas species becomes narrower for lower pressures.''} \\
{\bf ``Especially the timescales for the phyllosilicate formation are very interesting as small rocks can be hydrated on timescales of planetary evolution which is in agreement with conclusions of (ref).''} \\
{\bf ``The near-crut atmosphere above a magma ocean consists, in all cases, mainly of H2O, O2, CO2, and SO2.''} \\
{\bf ***``It is hypothesized that most of the Earth's water is trapped in phyllosilicates in the wet mantle transition zone (ref), but this zone is not saturated with water.''} \\
{\bf ***``From the results of this work, the upper crust of a (cooling) planet needs to be saturated in phyllosilicates in order to allow for the stability of liquid water on the surface.''} \\
{\bf ***``The detection of gaseous water in the atmosphere is not conclusive for the existence of liquid water, as the phyllosilicates are able to incorporate all potential liquid water in phase equilibrium.''} \\

\subsection{Water, melting, and the deep Earth H\textsubscript{2}O cycle \citep{hirschmann06}}
A review of the deep water cycle, focusing on the current H\textsubscript{2}O reservoirs within the mantle (other reviews focus on oceanic lithospheric subduction beneath arc volcanoes). While the transition zone could contain large amounts of water, the upper mantle likely contains less, with the lower mantle having a very low storage capacity. Water is crucial for characterizing the interior convective flow of a planet, and thus may be present in solids, hydrous fluids, or melts. The steady-state freeboard (i.e., level of oceans w.r.t. continents) reflects the steady water budget between the surface and mantle. If the transition zone has a large storage of water, it has either been that way for much of Earth's history, or has been hydrated at the expense of the other mantle reservoirs and not the oceans. {\bf Read again later after learning more geochemistry.}

\subsection{The H/C ratios of Earth's near-surface and deep reservoirs, and consequence for deep Earth volatile cycles \citep{hirschmann09}}

\subsection{Water in Earth's mantle \citep{hirschmann12}}

\subsection{Bifurcation in the growth of continental crust \citep{honing19a}}
The Earth's current rate of continental growth may be dominated by positive feedbacks and thus be balanced at an unstable "bifurcation point", with two stable points of smaller and larger continental volumes. The results imply that continental growth is strongly dependent on initial planetary conditions, and that exoplanets with plate tectonics should show a wide variety of land-water ratios even with similar ages, sizes, and total water inventories. The growth of continental crust is important to account for in planetary evolution studies.

\subsection{Carbon cycling and interior evolution of water-covered plate
tectonics and stagnant lid planets \citep{honing19b}}
The long-term carbon cycle for water-covered plate tectonics and stagnant lid planets are modelled and compared (note: no surface erosion). Seafloor-weathering rate and atmospheric CO\textsubscript{2} strongly affect climate evolution. Plate tectonics planets tend to reach a steady state of atmospheric CO2 (dependent on the crustal production rate), while the steady state for stagnant-lid planets involves all carbon depleted from the mantle and existing in the atmosphere/crust. Either tectonic regime could lead to long-term habitability; CO\textsubscript{2} levels decrease at late times, due to mantle cooling, which could counteract increasing surface temperature due to increasing solar luminosity with time and maintain long-term habitability; this decrease occurs for plate tectonics through an increasing fraction of subduction zones avoiding crustal decarbonation as the mantle cools, and for stagnant lid planets through an increasing decarbonation depth. Initial mantle temperature dependence more important for stagnant lid planets, as is the assumed weathering model (CO2-dependent vs. temperature-dependent). Mantle temperature/thermal convection parameterization more complicated (i.e., more accurate) than that of \citep{komacek16}, the same as that used in \citep{schaefer15}; however, both plate tectonics and stagnant lid models included. \\
{\it NOTES:} \\
{\bf ``The negative feedback is mainly controlled by temperature-dependent continental weathering, which requires emerged land (ref).''} \\
{\bf ``In the early evolution of the Earth, the continents were largely flooded, resulting in a strongly reduced erosion rate (ref), affecting the efficiency of continental weathering.''} \\
{\bf ``A negative feedback mechanism that works for water-covered oceanic crust is provided by seafloor weathering (refs).''} \\
{\bf ***``The major difference with respect to continental weathering is that seafloor weathering does not depend on the erosion rate to provide fresh surface that can be weathered, but rather on the supply rate of fresh basaltic crust by volcanism.''} \\
{\bf ``The way seafloor weathering behaves is debated. Seafloor weathering was commonly believed to directly depend on the atmospheric CO2 partial pressure (refs). However, more recent studies argue that it mainly depends on the pore-space temperature, which is turn is controlled by the planetary surface temperature (refs).''} \\
{\bf ``Large amounts of atmospheric CO2 allow liquid water to exist on the planetary surface farther away from the star.''} \\
{\bf ``It remains unclear as to whether or not carbon recycling can occur on stagnant-lid planets and to what extent this can regulate their atmospheric CO2 partial pressure.''} \\
{\bf ``The production of fresh crust by volcanism provides a potential carbon sink, which could establish a negative feedback similar to seafloor weathering (ref).''} \\
{\bf ``Volcanic eruptions may bury carbonated crust, which then sinks deeper into the mantle. Increasing pressure and temperature with depth affect the stability of carbonates.''} \\
{\bf ***``The release of CO2 back into the atmosphere caused by the increase in temperature with depth is known as metamorphic decarbonation.''} \\
{\bf ``For stagnant-lid planets with an Earth-like total CO2 budget, carbon cycling could provide negative feedback avoiding a supply-limited regime, which would occur through a complete carbonation of the crust.''} \\
{\bf ***``On planets without plate tectonics, the production of fresh crust occurs via hotspot volcanism. In the presence of water, the basaltic crust can be carbonated and subsequently buried by new volcanic eruptions.''} \\
{\bf ``The CO2 sink on a stagnant-lid planet is directly coupled to the production rate of new crust and hence to the CO2 source and not the erosion rate.''} \\
{\bf ``The negative feedback may keep these planets habitable if fresh rock is supplied at a sufficiently large rate and seafloor weathering is temperature dependent.''} \\
{\bf ``The minimum CO2 budget of a planet required to recover from a snowball state crucially depends on whether or not exchange of CO2 between ocean and atmosphere is possible.''} \\
{\bf ``The temperature profile of a subduction zone depends on several parameters such as subduction angle and plate speed and varies from one subduction zone to another (ref).''} \\
{\bf ***``In the present-day Earth, most subduction zones are sufficiently cold to avoid decarbonation, which enables carbon recycling into the mantle.''} \\
{\bf ***``In the early Earth, most subduction zones were hot, decarbonation was efficient, and recycling of carbon into the mantle was difficult (refs).''} \\
{\bf ***``For a stagnant-lid planet, carbon recycling into the mantle could potentially occur through continuous volcanism, burial and sinking of carbonated crust, eventually followed by delamination.''} \\
{\bf ***``After the planet has cooled sufficiently for mantle melting to cease, volcanism and mantle degassing will cease as well; no fresh crustal rock can be produced and weathering will greatly slow down. The atmospheric CO2 partial pressure is then no longer controlled by interior-surface processes and will tend to remain constant.''} \\
{\bf ``We note that in steady state, the atmospheric carbon concentration does not depend on the crustal production rate. This is because the ingassing rate and degassing rate both depend on crustal production rate in the same manner.''} \\
{\bf ``There is a linear relationship between pore-space temperature and surface temperature (see ref).''} \\
{\bf ***``The plate speed increases with the convection strength as derived from boundary layer theory (e.g., ref), which would yield a larger plate speed when Eath was hotter, but at the same time it decreases with crustal thickness, which has also been larger in the past (ref), possibly causing sluggish plate tectonics (ref).''} \\
{\bf ``An increasing fraction of subduction zones avoid decarbonation as the mantle cools.''} \\
{\bf ``Degassing on stagnant-lid planets occurs at hot spots. Partial melting, which ultimately causes degassing, occurs at the transition from the lid to the upper mantle.''} \\
{\bf ***``The presence of water reduces the solidus temperature. However, large amounts of water would be required to substantially reduce the solidus. A solidus reduction of 100 K would already require a bulk water concentration of more than 200 ppm \citep{katz03}.''} \\
{\bf ***``As long as the mantle temperature of a stagnant lid planet is large enough to cause partial melting to take place regionally, metamorphic decarbonation will occur globally.''} \\
{\bf ``It has been argued that gravitational instabilities could recycle the crust into the mantle (refs), which could cause sinking and decarbonation of part of the crust for planets in their late evolution even if partial melting no longer takes place. (Neglected in this model)''} \\
{\bf ``If partial melting and decarbonation occur in stagnant-lid planets, Eqs. 16 and 17 imply that a combined steady state can only be reached if the mantle degassing rate is zero, which would be the case if all of the carbon were stored in the atmosphere and crust.''} \\
{\bf ***``It is unlikely that stagnant-lid planets can reach this steady state. Rather, the combined degassed CO2 reservoir, which is distributed between the crust and atmosphere, will increase with time as long as volcanism is active.''} \\
{\bf ``Since we do not model the water cycle in this paper, we keep the mantle water concentration constant.''} \\
{\bf ***``For plate tectonics, as the mantle cools and an increasing fraction of subduction zones avoid decarbonation, the equilibrium atmospheric CO2 partial pressure decreases.''} \\
{\bf ***``Differences due to different initial mantle temperatures vanish at $\approx 1$ Gyr.''} \\
{\bf ``We show that the atmospheric CO2 obtained from the steady-state models are stable fixed points towards which the atmospheric CO2 partial pressure evolves.''} \\
{\bf ``The differences of initial conditions after 1 Gyr are small, and differences of initial conditions at the age of solar system have vanished.''} \\
{\bf ***``We note that the convergence rate is determined by the crustal production rate, which we keep constant. If the carbon cycle operated faster during early evolution, equilibrium would be reached faster accordingly.''} \\
{\bf ***`From Fig. 5, it becomes apparent that the scaling used for seafloor weathering largely determines the evolution of the atmospheric CO2 and thereby the surface temperature.''} \\
{\bf ***``For CO2-dependent weathering, the surface temperature increases with both luminosity and mantle temperature. Therefore, a cooling mantle can partly compensate for the increasing luminosity, thereby limiting the net change of surface temperature.''} \\
{\bf ``The equilibrium surface temperature is still a function of the degassing rate, which in turn is sensitive to the total carbon reservoir.''} \\
{\bf ***``As long as the complete crustal carbonation limit is not reached, the surface temperature does not vary significantly for the pore-space temperature dependent weathering model.''} \\
{\bf ***``For stagnant lid planets, neglecting a minor adjustment owing to initial conditions during the first few Myr, the degassing rate closely follows the mantle temperature evolution. After $\sim$1 Gyr, the effect of initial mantle temperature vanishes. Following that, the degassing rate strongly decreases, and between 4 and 5 Gyr, volcanism ceases completely.''} \\
{\bf ``After 2-3 Gyr, the atmospheric CO2 reservoir decreases despite an increase of the crustal CO2 reservoir. This is caused by an increase of the decarbonation depth, which results in a reduced decarbonation rate.''} \\
{\bf ***``The pore-space temperature dependent weathering model implies an extremely strong weathering feedback and hardly allows any CO2 to remain in the atmosphere.''} \\
{\bf ***``We note that independent of weathering scaling, the crustal CO2 reservoir steadily increases as long as partial melting takes place, whereas the atmospheric reservoir decreases in the late evolution as decarbonation depth increases. This qualitatively differs from stagnant lid models that do not account for weathering and decarbonation, in which the atmospheric CO2 partial pressure steadily increases (e.g., \citet{tosi17}.''} \\
{\bf ``For the CO2-dependent weathering model, the atmospheric CO2 is greatly affected by the initial mantle temperature. This is because any CO2 that is degassed from the mantle cannot be recycled back.''} \\
{\bf ***``Since the crustal reservoir is continuously supplied to the atmosphere via decarbonation, the atmospheric CO2 partial pressure throughout the entire evolution crucially depends on the initial mantle temperature.''} \\
{\bf ``Since the decarbonation depth depends on the surface temperature, the difference in the atmospheric CO2 partial pressure grows larger.''} \\
{\bf ``Differences in initial mantle temperature of 100 K can cause large differences in the surface temp at 4.5 Gyr of up to 50 K if CO2-dependent weathering model is used.''} \\
{\bf ``In contrast, the pore-space temperature dependent weathering model hardly allows any CO2 to remain in the atmosphere, independently of the initial mantle temperature. In the late evolution, the remnant CO2 in the atmosphere is depleted and reaches zero.''} \\
{\bf ``Particularly in the CO2-dependent weathering model, the atmospheric CO2 of a stagnant-lid planet is strongly affected by the accumulated degassed CO2, which in turn depends on mantle rheology and oxidation state.''} \\
{\bf ***``On the one hand, a large mantle viscosity delays mantle cooling, but on the other hand it results in a more sluggish convection, which reduces the rate of melt production.''} \\
{\bf ***``The oxygen fugacity impacts the evolution of the atmospheric CO2 partial pressure and the surface temperature even more strongly than the choice of the initial mantle temperature. This is because a greater oxygen fugacity enhances the rate at which CO2 is degassed without enhancing melt prodution rate.''} \\
{\bf ***``For planets with a water content of more than 100 times that of Earth and with a water-layer of several hundred km, the long-term carbon cycle becomes unimportant in stabilizing the climate (ref).''} \\
{\bf ``The positive feedback connecting surface temperature, decarbonation, and atmospheric CO2 affects the habitability of both model planets.''} \\
{\bf ``The cooling mantle in the late evolution reduces the atmospheric CO2, regardless of tectonic mode.''} \\
{\bf ***``The results indicate that a temperature-dependent weathering rate stabilizes the surface temperature more strongly than CO2-dependent.''} \\
{\bf ``Since we found that the initial mantle temperature impacts the atmospheric CO2 of stagnant-lid planets, in particular if a weak weathering feedback (i.e., CO2-dependent) is used, it could also play a role in the ability of a planet to avoid a permanent snowball state.''} \\
{\bf ***``The early Earth may have been in a stagnant-lid regime (ref). Changes in the tectonic mode may be caused by changes of the surface temperature (ref), which would establish a feedback to the climate (ref).''} \\
{\bf ***``It is not clear whether an increasing planetary mass would make plate tectonics more likely due to a larger shear stress (ref), or whether a more buoyant crust would resist subduction \citep{kite09}.''} \\
{\bf ``Since surface erosion on planets entirely covered with water is negligible, volcanism producing fresh uncarbonated crust is the only mechanism to maintain a carbon sink. Therefore, the weathering rate linearly depends on the rate of volcanism, and is therefore directly coupled to the carbon source.''} \\
{\bf ***``Neither emerged land (refs) nor plate tectonics (refs) are required for planets to remain habitable in the long-term. For both tectonic regimes, the CO2 concentration decreases during late evolution.''} \\
{\bf ``Since the stability of carbonated crust with depth is temperature-dependent, a positive feedback loop between the surface temperature, decarbonation rate, and atmospheric CO2 is established.''} \\

\subsection{White dwarf pollution by hydrated planetary remnants: hydrogen and metals in WDJ204713.76-125908.9 \citep{hoskin20}}

\subsection{Evryflare III: Temperature evolution and habitability impacts of dozens of superflares observed simultaneously by Evryscope and TESS \citep{howard20}}

\subsection{Survival of primordial planetary atmospheres: mass loss from temperate terrestrial planets \citep{howe20}}
{\bf ***Many equations for atmospheric loss are included throughout the paper, including Jeans escape, hydrodynamic escape, etc., with derivations in Appendix.***}
A toy model is used to form the Earth from the fast pebble accretion formation mechanism, in which it accretes a substantial H/He atmosphere before the gas disk dissipates. A variety of mass loss mechanisms are calculated and their overall impact on atmospheric removal are included. This study also includes photodissociation along with photoionization, proving that it is significant for early Earth, and should be included in future temperature planet atmospheric loss models. However, the authors do not find a reasonable way to completely evaporate Earth's primordial atmosphere. \\
{\it NOTES:} \\
{\bf ``The EUV dissociation of hydrogen molecules can also theoretically drive atmospheric evaporation on low-mass planets. For temperature planets such as the early Earth, impact erosion is expected to dominate in the traditional planetesimal accretion model, but it would be greatly reduced in pebble accretion scenarios, allowing other mass loss processes to be major contributors.''} \\
{***\bf ``Photodissociation is likely a subdominant, but significant component of mass loss (similar for super-Earths). This mechanism could also preferentially remove hydrogen from a planet's primordial atmosphere, thereby leaving a larger abundance of primordial water compared to standard dry formation models.''***} \\
{\bf ``If planets do accrete atmospheres during formation, their subsequent thermal evolution and mass loss are almost certainly dominated by giant impacts (Biersteker \& Schlichting 2019).''} \\
{\bf ***``Building up an Earth-mass planet through successive giant impacts would likely have removed any primordial volatiles, and any thin, primordial atmosphere surviving these impacts could be quickly lost to photoevaportation (Johnstone et al. 2019).''***} \\
{\bf ``In a pebble accretion scenario, a super-Earth-mass planet could form within 1 Myr and capture a much deeper gas-rich envelope. Estimates from Ginzburg et al. (2016) suggest that mass fractions up to 2\% could be realized in a hydrogen- and helium-rich atmosphere.''} \\
{\bf ***``If the nebula is wet (refs) due to migration of small icy bodies interior to the ice-line (and subsequent sublimation of volatiles into the gas phase), an Earth-mass planet could capture a water mass fraction approaching $2 \times 10^{-4}$, comparable to present-day Earth within a factor of 3 (refs).''***} \\
{\bf ***``Photodissociation by ultraviolet light should be effective for young exoplanets, because photodissociation of hydrogen occurs at lower energies than photoionization (refs), and thus would increase the ultraviolet flux available for upper atmosphere heating and escape.''***} \\
{\bf ``Related question: whether this mass loss of hydrogen and helium will leave behind any of the water accreted from the disk.''} \\
{\bf ``Majority of water in Earth's oceans must have originated from beyond the ice line, based on D/H ratio.''} \\
{\bf ***``Measurements of D/H ratios in deep mantle lava are closer to the solar ratio, indicating the possibility of a deep mantle reservoir for primordial water (refs), which could have been accreted from the primordial circumstellar disk.''} \\
{\bf ***``Pebble accretion has a natural endpoint at a super-Earth mass. To produce an Earth-mass planet, the accumulation phase must be interrupted by some other mechanism, and this process may also affect the volatile content of the planet.''***} \\
{\bf ``We find that other processes may be comparable to photoevaporation for planets with relatively low insolation like early Earth, but they probably do not dominate mass loss in most cases.''} \\
{\bf ``Watson et al. (1981) considered energy-limited mass loss due to heating by extreme ultraviolet (EUV) radiation from the star. This mechanism is dominant for planets that are highly irradiated.''} \\
{\bf ***``Most studies of atmospheric loss from exoplanets use directly or indirectly the estimates of Ribas et al. (2005) for the XUV flux as a function of time for solar-analog stars... but this model does not model the exact ionizing flux.''***} \\
{\bf ***``Lammer et al. (2014) considered both ionization and dissociation processes, but their model was based only on XUV fluxes and thus did not take photodissociation into account in practice.''***} \\
{\bf ``Photodissociation-induced atmospheric evaporation has received little attention in application to exoplanets... but the energy levels involved indicate that this process should be taken into account when modelling exoplanet evolution.''} \\
{\bf ``Photoevaporation processes occur in the collisional region of the atmospheres and therefore involve heating of the atmosphere rather than ejection of individual particles by photons.''} \\
{\bf ``Jeans escape depends on the temperature of the exosphere and is thus exponentially increased when the young planet is heated by rapid accretion.''} \\
{\bf ``Jeans escape is sufficient to evaporate $\sim 0.6$ bars of hydrogen from a Earth-like primordial atmosphere (which begins in this study at $\sim 23000$ bars).''} \\
{\bf ***``Hydrodynamic outflow rate can be estimated for an isothermal atmosphere, which is expected to be a good approximation to a planet's upper atmosphere, based on the requirement that the flow must pass smoothly through the sonic point.''***} \\
{\bf ``When the mass loss is driven by incoming UV photons, the base of the flow occurs where the incoming radiation becomes optically thick.''} \\
{\bf ``At higher temperatures (than those considered here), the flow does not transition smoothly through the sonic point and will be time-dependent, or will take on the form of a shock, while at low temperatures, it is exponentially suppressed.''} \\
{\bf ``Stellar winds can erode planetary atmospheres by directly imparting momentum to the upper atmosphere through the action of wind particles.''} \\
{\bf ***''Stellar wind ablation can remove only 10 bars of hydrogen from an Earth-like primordial atmosphere, much less than any other processes under consideration.''} \\
{\bf ``Giant impacts can cause mass loss from planets in two ways: 1) the direct mechanical ejection of a large fraction of the atmosphere; 2) through the thermal wind induced by the heating of the remaining atmosphere. Both of these can be major contributors to mass loss, with magnitudes comparable to or greater than the other mechanisms under consideration.''} \\
{\bf ``The minimum impactor size for mass ejection varies with the thickness of the atmosphere.''} \\
{\bf ``As the impactor size grows, the ejected cone widens until it reaches the horizon, thus ejecting the entire spherical cap above the tangent plane to the impact.''} \\
{\bf ``For truly giant impactors, regardless of atmospheric mass, direct transfer of momentum through the solid mass of the planet will eject significantly more of the atmosphere than the spherical cap.''} \\
{\bf ***``Ten Mars-sized impactors in sequence could plausibly remove virtually all of the primordial atmosphere from an Earth-mass planet over the course of planet formation. In this case, any remaining atmosphere would have to be produced via outgassing.''} \\
{\bf ``Upper bound for impact erosion (which can also be appied to pebble accretion): no more than 20\% of the mass accreted in the form of planetestimals will be ejected from the atmosphere.''} \\
{\bf ``An extended atmosphere, caused by heating from the impact, will thermally evaporate much more quickly than any of the other processes we study in this work.''} \\
{\bf ***``Planetesimal accretion models often postulate a ``late veneer'' scenario (ref) in which Earth accreted an additional 1\% of its mass from small bodies after the final assembly of the planets.''***} \\
{\bf ``Planetestimal accretion and pebble accretion can potentially coexist in comparable amounts during the gas disk lifetime (ref), but late-stage planetesimal accretion in a pebble accretion model (the latter of which can form planets in less than 1 Myr) is expected to be $\lesssim 1$\%.''} \\
{\bf ``In this toy model, impact erosion will remove 2300 bars of hydrogen from the planet, any Moon-forming impacts notwithstanding.''} \\
{\bf ***``The standard prescription for mass loss on super-Earths, due to Watson et al. (1981), is to assume an energy-limited approximation for ionization by XUV photons, using a specific efficiency factor, usually $\sim10$\%. This is an optimistic approximation for hydrodynamic escape, but it often applies for low stellar fluxes (refs), so we likewise use it as an optimistic approximation for our analysis.''} \\
{\bf ``A rough estimate of mass loss can be made by computing the total XUV radiation absorbed by the planet over its lifetime... With an efficiency of 10\%, the total mass of hydrogen lost to photoionization from Eqn. 13 is $\sim 750$ bars.''} \\
{\bf ``Most stars are born in clusters. A planet in a solar system forming in an especially favorable position in the birth cluster may experience up to twice as much photoevaporation as a planet orbiting an isolated star.''} \\
{\bf ***``If XUV flux saturates at an early time than Ribas et al. (2005) predict, the star's initial XUV flux will be higher, allowing for greater mass loss in the first 100 Myr of the planet's history.''***} \\
{\bf ***``In addition to ionization, longer-wavelength photons of 91-111 nm are sufficient to photodissociate hydrogen molecules (Draine \& Bertoldi 1996). This dissociation is a second pathway to input energy into the upper atmosphere and drive evaporation, analogous to the action of ionization, and suggests that the usual convention of 10\% efficiency of photoevaporation may be an underestimate.''***} \\
{\bf ***``We compute the flux of the individiaul wavelength bins of Ribas et al. (2005) for a more precise result, which can also be applied to photoionization.''} \\
{\bf ***Table 1 shows adopted XUV and EUV flux prescriptions for the 5 wavelength bins stated above.***} \\
{\bf ``With an efficiency factor of 10\%, the combined ionizing and dissociating flux impinging on a young, Earth-like planet could remove about 750 bars of hydrogen, or about 0.065\% of the mass of our model planet.''} \\
{\bf ***``We note that using the energy-limited approximation provides an upper limit to the expected evaporation rates.''***} \\
{\bf ***``The results of the models indicate that photodissociation is a non-negligible contribution to mass loss on a young, Earth-like planet formed by pebble accretion, and photoevaporation in general is dominant over all other mechanisms other than impact erosion even before accounting for the potential greater efficiency due to the dissociation contribution to upper atmosphere heating.''***} \\
{\bf ***``The difficuly, as noted above, is that planet formation models that seek to explain rapid terrestrial planet formation must strip any primary atmosphere early in the planet's history, and the total mass loss we find for an early Earth analog is not sufficient.''***} \\
{\bf ``The initial gas accretion must be significantly less efficient, leading to a less massive initial atmosphere.''} \\
{\bf Table 3 shows mass loss from all mechanisms, and total combined.} \\
{\bf ***``Energy-limited photodissociation could be a significant contributor to photoevaporation and should be incorporated into existing models of mass loss.''***} \\
{\bf ``For photoevaporation, its efficiency may be reduced due to energy loss from line cooling; both Lyman-$\alpha$ cooling and metal cooling have been considered in the case of photoionization (refs).''} \\
{\bf ***``For water dissociation, some of the photon energy will be lost to rotational and vibrational modes of the hydroxide radical, reducing the mass loss efficiency by $\sim 50$\% over much of the FUV range.''***} \\
{\bf ``Dissociation of molecular hydrogen does not have this concern because the dissociated atoms have no molecular lines.''} \\
{\bf ``If surface fields are on order of 1 gauss, then magnetic fields may be sufficient to suppress outflows from terrestrial planets.''} \\
{\bf ``Photodissociation of molecular species can be a significant source of mass loss in the early evolution of temperate planetary atmospheres in addition to photoionization.''} \\
{\bf ``Within the context of our model, where the planet forms rapidly in the presence of a gas-rich disk, the early Earth is expected to a develop an atmosphere of $\sim 23000$ bars. However, there is not a clear path to evaporating the bulk of this primordial hydrogen and helium.''} \\
{\bf ***``The remaining atmosphere would be enriched in water and perhaps other volatiles because of the preferential loss of hydrogen (and helium) in the outer atmosphere.''***} \\
{\bf ***``We further suggest that researchers studying mass loss in temperate planet atmospheres consider the impact of photodissociation in their models.''} \\

\subsection{New experimental data and semi-empirical parameterization of H2O-CO2 solubility in mafic melts \citep{iacono12}}
Experimental data is used to constrain semi-empirical H2O-CO2 solubility relations for mafic melts (see paper for equations). CO2 solubility is strongly dependent on the amount of non-bridging oxygen (NBO) in the melt, along with melt composition and structure; H2O dissolution in the melt enhances CO2 solubility, likely by triggering NBO formation, while H2O solubility is not affected by melt composition and structure. \\
{\it NOTES:} \\
{\bf ``Solubility laws of volatiles generally describe increasing amounts of dissolved components in volatile saturated melts as pressure and, therefore, depth increases (ref).''} \\
{\bf ***``The most abundant volatile component in magmatic systems is H2O generally followed by CO2 (ref), although some volcanic systems can intermittently degas more CO2 than H2O (refs).''} \\
{\bf ``Elevated CO2/H2O ratios in volcanic plumes are generally interpreted to indicate deep degassing (refs).''} \\
{\bf ``Due to the strong influence of melt composition on CO2 solubility, accurate solubility laws are essential.''} \\
{\bf ``It was proposed that at low water contents, water solubility in silicate melts depends on the square root of water pressure or fugacity. (ref) proposed a model for both H2O and OH species that applies at high-water contents.''} \\
{\bf ``In this paper, we present experimental data showing that melt SiO2 content is not sufficient to accurately predict CO2 solubility.''} \\
{\bf ``In mafic compositions, CO2 is observed to dissolve in the melt uniquely as CO$_3^{2-}$. The amount of available oxygens and the type of cation bonded to these oxygens are therefore key factors in controlling CO2 solubility (refs).''} \\
{\bf ``(ref) suggest CO2 solubility is a function of the degree of polymerization of the melt, o the basis of a strong decrease in CO2 solubility with increasing NBO/T (where T is a tetrahedron).''} \\
{\bf ``The data shows CO2 solubility generally increases with CO2 partial pressure for Etna composition.''} \\
{\bf ``In general, MORBs show the lowest CO2 solubilities, while foiditic and tephritic melts display the highest ones.''} \\
{\bf ***``A plot of the ratio of dissolved CO2 to CO2 partial pressure as a function of the bulk water content of the melt is consistent with an increase in CO2 solubility with increasing water content in the melt.''} \\
{\bf ***``The three datasets show that the dissolution of 7-9 wt\% H2O enhances CO2 solubility by a factor of 2-3.''} \\
{\bf ``For a given H2O content, CO2 solubility in three substantially different compositions has a clear linear relationship with CO2 partial pressure, up to 5000 bar total pressure.''} \\
{\bf ``CO2 fugacity is calculated to exponentially increase with pressure.''} \\
{\bf ``For a given melt composition, CO2 solubility increases with increasing H2O dissolved in the melt.''} \\
{\bf ``It seems difficult to assess any chemical control on water solubility in mafic melts, as underlined by earlier works (refs).''} \\
{\bf ***``Within this paper, NBO/O, non-bridging oxygen divided by oxygen, is considered as a measure of the activity of oxygen anions: it expresses the availability of oxygen in the melt to form carbonate groups.''} \\
{\bf ``Mixing properties in the H2O-CO2 fluid, which are ignored here, may well affect the solubility of H2O-CO2 in melt without requiring any interaction between dissolved water and dissolved CO2.''} \\
{\bf ***``NBO/O is the chief factor in controlling CO2 solubility, the type of the modifier cation bonded to NBO also having a crucial role.''} \\
{\bf ***``For this model, water incorporation increases NBO/O which then translates into an increase in CO2 solubility.''} \\
{\bf ***``H2O solubulity seems poorly sensitive to the melt chemical compositions as found in earlier studies (refs).''} \\
{\bf ``The temperature effect on H2O-CO2 solubility has not been evaluated, because the variations in experimental temperatures are very limited.''} \\
{\bf ``Crucial role of melt structure and chemistry on CO2 solubility: Adding modifier cations results in an increase in NBO/O and therefore to an increase in CO2 solubility, but the intensity of the effect depends on the type of cation added.''} \\
{\bf ``Alkalis have the strongest effect on CO2 solubility.''} \\
{\bf ***``Water solubility is reasonably well accounted for by existing models, whatever the melt composition.''} \\
{\bf ***``Water appears to enhance CO2 solubility, especially at high water contents.''} \\

\subsection{Constraints on the mass of a habitable planet with water of nebular origin \citep{ikoma06}}

\subsection{Subduction fluxes of water, carbon dioxide, chlorine, and potassium \citep{jarrard03}}

\subsection{Extreme hydrodynamic losses of Earth-like atmospheres in the habitable zone of very active stars \citep{johnstone19}}

\subsection{Lithium pollution of a white dwarf records the accretion of an extrasolar planetesimal \citep{kaiser20}}

\subsection{Transits of Earth-like planets \citep{kaltenegger09}}
A model of Earth's atmospheric spectrum, including clouds (aerosols, Rayleigh scattering) is used as a proxy for detecting biosignatures in an Earth-like planet's atmosphere using transit spectroscopy, where the Earth is orbiting either a Sun-like star or M-dwarf. Multiple transits are required to detect biosignatures, and an extended atmosphere is ideal, since most of the lower atmosphere cannot be probed in transit because of cloud effects. Features are easiest to detect in the IR around M-dwarfs, due to the peak stellar irradiation. M-dwarfs are the most favourable (specifically M8V and M9V) in the IR and near-IR.

\subsection{Finding signs of life on transiting Earth-like planets: high-resolution transmission spectra of Earth through time around FGKM host stars \citep{kaltenegger20}}

\subsection{Around which stars can TESS detect Earth-like planets? The revised TESS habitable zone catalog \citep{kaltenegger21}}

\subsection{Runaway and moist greenhouse atmospheres and the evolution of Earth \& Venus \citep{kasting88}}
The critical flux required to push a planet into either a moist or runaway greenhouse are calculated, using a sophisticated (at the time) 1D atmosphere model, for Earth and Venus today and in their pasts (when the Sun was fainter). \\
{\it NOTES:} \\
{\bf ``Clouds should tend to depress the surface temperature on a warm, moist planet.''} \\
{\bf ``The critical solar flux above which water is rapidly lost could be as low as 1.1 times the solar flux received at Earth today.''} \\
{\bf ***``(ref) showed that H2O would have remained a major constituent of a relatively thin ($\sim2$ bar) N2-H2O atmosphere with surface temperature near 100 degrees C up to very high altitudes, so that water could have been rapidly lost by photodissociation followed by hydrogen escape. (ref) termed this atmosphere the ``moist greenhouse''.''} \\
{\bf ***``Ozone would presumably be destroyed in a warm, moist atmosphere by the by-products of water vapour photolysis.''} \\
{\bf ``Clouds were excluded from most of my calculations because it is not known how much they would vary in an atmosphere much hotter (or colder) than our own. I assume surface albedo $=0.22$, and no cloud feedback whatsoever.''} \\
{\bf ***``The lapse rate in Fig. 1 is necessarily steeper than the moist convective region because of the absence of latent heat release from condensation.''} \\
{\bf ``The lower portion of the atmosphere in my model should therefore be convective, as indeed is has been assumed to be.''} \\
{\bf ***``The results are rather insensitive to the assumed stratospheric temperature profile. The reason is that the stratosphere is so tenuous at these high surface temperatures that its effect on the outgoing flux is negligible. This should be true for any atmosphere in which the stratosphere is water-dominated.''} \\
{\bf ***``For surface temperatures $<647.1$ K, the critical point for water, the H2O pressure follows the saturation vapour pressure curve. The atmosphere in this temperature regime is in the moist greenhouse state. Above the critical point, the H2O pressure is constant and the atmosphere is in the runaway greenhouse state.''} \\
{\bf ***``Above the critical point surface temperatures, the model atmosphere is composed of nearly 100\% water vapour, so the moist adiabats are all close to the saturation vapour curve.''} \\
{\bf ***``The reason that the net IR flux flattens out is that, as the moist convective layer thickens, the atmosphere becomes opaque to infrared radiation at all wavelengths. Above about 1400 K, the net IR flux once again begins to increase with surface temperature. The reason is that, at these extremely hot temperatures, the lower atmosphere and surface begin to radiate in the visible and near infrared, where the water vapour opacity is low.''} \\
{\bf ***``The initial rise in net solar fluxis caused by increasing absorption of solar radiation by atmospheric H2O. The subsequent decline at higher surface temperatures is caused by Rayleigh scattering.''} \\
{\bf ***``According to this model, the critical solar flux at which the greenhouse effect ``runs away'' is 1.4 times the solar flux received at Earth today.''} \\
{\bf ***``On the present Earth, the increase in albedo caused by clouds is partly compensated by their contribution to the greenhouse effect; hence, clouds can warm or cool the surface depending on their altitude and optical depth.''} \\
{\bf ***``Increases in CO2 have little effect on the solar flux required to trigger a runaway greenhouse.''} \\
{\bf ``(ref) predicted that the growing Earth would have had a 100-bar steam atmosphere with a surface temperature near 1500 K -- above the solidus for typical silicate rocks.''} \\
{\bf ***``The surface heat flux on a growing Earth would have been sufficient to vaporize any water that was present, creating a dense steam atmosphere.''} \\
{\bf ``For a fully saturated, cloud-free model, the water vapor content of the stratosphere increases dramatically for solar fluxes greater than about 1.1 times the current solar flux at Earth.''} \\
{\bf ``Calculations with a fully saturated, cloud-free, climate model indicate that the critical solar flux required to trigger a runaway greenhouse is 1.4 times the value at Earth's orbit.''} \\
{\bf ``A runaway greenhouse atmosphere was probably present on Earth throughout much of the process of accretion. This finding provides initial support for the idea that Earth's surface was molten at the time.''} \\
{\bf ``Earth's atmosphere is apparently stable with respect to solar flux increases of less than 10\%. The Earth might therefore be uninhabitable had it formed approximately 5\% closer to the Sun.''} \\

\subsection{Habitable zones around main sequence stars \citep{kasting93}}
An often-cited habitability paper, calculating the edges of both the habitable zone (runaway greenhouse limit and maximum greenhouse limit)  and continuously habitable zone around Sun-like stars (and other dwarf stars), based on the presence of liquid water at the surface, using a 1D model. Included are a description of carbonate-silicate weathering (critical to the model, including plate tectonics), atmospheric modelling calculations (e.g., the location of the tropopause, composition), hydrogen-limited escape. Clouds are (mostly) neglected in the model. Larger planets can hold onto their atmospheres more readily (Earth can only lose H and He; Mars can also lose C, N, O). Tidal locking of planets around M-dwarfs is briefly discussed.

\subsection{Evolution and spectral response of a steam atmosphere for early Earth with a coupled climate-interior model \citep{katyal19}}
A coupled atmosphere-interior model is used to model the magma ocean stage of early Earth, below a steam atmosphere {\bf (but the authors neglect atmospheric loss in this study)}. A pure H2O atmosphere from outgassed water extends the magma ocean phase by several Myr, and reduces the outgoing longwave radiation (OLR). H2O, H2, and O2 will appear at different levels in the atmosphere, which is observationally useful for constraining magma oceans. Hot surfaces will see a rapid cooling of the magma ocean, while cooler surfaces are blanketed by the steam atmosphere, which reduces the OLR. The magma ocean lasts $\sim 1$ Myr. Effective height of atmosphere decreases as magma ocean solidifies or surface temperature reduces.
{\bf Table 3 contains a list of atmosphere-interior models for magma ocean phase, including \citet{hamano13} and \citet{schaefer16}}. \\ 
{\it NOTES:} \\
{\bf ``During accretion phase and following giant impacts, Earth had a transient atmosphere due to impact devolatilization above magma ocean, but the planet did not acquire its initial water content through ingassing from such an atmosphere; only 1\% of the water entered the planet this way (Wu et al. 2018).''} \\
{\bf ``Current general consensus is that Earth acquired its water inventory largely via accretion of chondritic materials (Russell et al. 2017).''} \\
{\bf ``When the Earth's magma ocean started to solidify, it already had a significant amount of water to be outgassed, no matter how it was acquired.''}
{\bf ``It has been shown by Elkins-Tanton (2011) that a magma ocean with as low as 0.1 wt \%(1000 ppm) of water can potentially outgas hundreds of bars of water. Moreover, the formation of water oceans on rocky planets (e.g., due to the collapse of a steam atmosphere) suggests that steam (H2O) is the major volatile reservoir for most planetary surfaces and their mantles.}

\subsection{A new parameterization of hydrous mantle melting \citep{katz03}}

\subsection{The effect of varying atmospheric pressure upon habitability and biosignatures of Earth-like planets \citep{keles20}}
The competing effects of greenhouse warming (increases with increasing atmospheric mass) and Rayleigh scattering (which leads to cooling as atmospheric pressure increases, as much of the shortwave radiation is scattered back to space) are investigated against increasing surface pressure. Greenhouse warming dominates up to a point (4 bar in this study), increasing surface temperature; after this pressure, Rayleigh scattering dominates, decreasing surface temperature. The authors also note that higher surface pressures tend to enhance biosignatures, potentially making them easier to observe in higher-pressure atmospheres. \\
{\it NOTES:} \\
{\bf ***``Increasing the surface pressure up to 4 bar leads to an increase in the surface temperature due to increase greenhouse warming. Above this point, Rayleigh scattering dominates and the surface temperature decreases, reaching surface temperatures below 273 K (approximately at $\sim34$ bar surface pressure).''} \\
{\bf ``Nitrous oxide on Earth is mainly $geq 99$\%) produced by (de)nitrifying bacteria, and in this sense it is an excellent indication for life.''} \\
{\bf ***``A planet with a liquid water reservoir on the surface will show water vapour in its atmosphere.''} \\
{\bf ***``The simultaneous detection of ozone and the life-related compounds carbon dioxide and water could constitute evidence of life (ref). Also the combined detection of huge amounts of oxygen and methane, that is, oxidizing and reducing species, has been regarded as possible evidence for life (ref).''} \\
{\bf ***``For planetary habitability, it has been shown that increasing the atmospheric mass may increase the surface temperature via an increased greenhouse effect (refs). However, a larger atmosphere may also lead to enhanced cooling due to increased Rayleigh scattering.''} \\
{\bf ***``In our simulations, for higher surface pressure (of approximately $\sim34$ bar), the surface was no longer habitable ($T_0 < 273$ K).''} \\
{\bf ***``The 1 bar case represents the modern Earth case, the 4-bar scenario shows the highest increase in surface temperature, and the 30-bar scenario is the highest pressure scenario that enables global-mean habitable conditions.''} \\
{\bf ``In Figure 1, the surface temperature first increases (up to 4 bar surface pressure) up to a value of 295 K and then decreases for higher pressure scenarios.''} \\ 
{\bf ***``When considering Rayleigh scattering, the surface temperature decreases for higher pressure simulations. A similar effect, where increased pressure leads to tropospheric cooling induced by Rayleigh scattering for an Earth-like carbon dioxide increased atmosphere has been shown, for example, by (ref) and \citet{kasting93} for an Earth-like planet at the outer edge of the HZ.''} \\
{\bf ***``This has also been shown by (ref) for early Mars and by \citet{zsom13} for desert worlds at the inner edge of the HZ.''} \\
{\bf ``The increase/decrease in surface temperature results from the net effect of the greenhouse effect (difference between thermal radiation at surface and TOA) and planetary albedo (which steadily increases with surface pressure, leading to a cooling of the atmosphere for higher pressure scenarios).''} \\
{\bf ``Ozone amounts in the atmosphere are subject to a stabilizing ozone-, UV-feedback. There is a trade-off situation between the availability of UV-flux and molecular oxygen, both of which are needed to produce ozone, which changes with altitude. However, there is not a significantly stronger biosignature fingerprint in emission spectra for high pressure scenarios.''} \\
{\bf ``Due to increasing amounts of tropospheric nitrous oxide, its spectral features are enhanced for higher pressure scenarios as seen in the emission spectra.''} \\
{\bf ***``The total amount of water (see Table 2) shows an increase for scenarios with up to 3 bar surface pressure and decreases for higher pressure scenarios (Fig. 6a). The water fingerprint in emission spectra follows the surface temperature.''} \\
{\bf ``The methane amount increases for higher surface pressure scenarios, and similar to nitrous oxide, methane shows stronger signatures in the emission spectra for higher pressure scenarios.''} \\
{\bf ``A false positive detection could be possible for atmospheres where little ozone and yet sigificant amounts of carbon dioxide exist.''} \\
{\bf ***``An important aspect of this work is the opposing effects of warming by increasing greenhouse gas amounts and the cooling by Rayleigh scattering, which determine the resulting surface temperature.''} \\
{\bf ***``Our models confirm the cooling of the atmosphere by the scattering effect for high pressure atmospheres.''} \\
{\bf ***``The increase in surface temperature up to 4 bar surface pressure is induced by the increasing greenhouse effect. For further increases in surface pressure, the cooling induced by back-scattering of the incoming shortwave radiation to space due to Rayleigh scattering is stronger than the greenhouse heating and the surface temperature decreases.''} \\
{\bf ``All in all, the spectral features showed an increase in band width for higher surface pressure. The band depths of water are strongest for the 3-bar and 4-bar scenarios where water reaches the highest amount in the atmosphere.''} \\
{\bf ***``Nevertheless, biosignatures like ozone and nitrous oxide are enhanced and could be easier to identify for an Earth-like planet with higher surface pressure compared with that of modern Earth.''} \\

\subsection{Mutual detectability: a targeted SETI strategy that avoids the SETI paradox \citep{kerins20}}

\subsection{Formation of aqua planets with water of nebular origin: Effects of water enrichment on the structure and mass of captured atmospheres of terrestrial planets \citep{kimura20}}
The authors use a 1D atmospheric structure model to determine the water inventory of terrestrial planets orbiting a 0.3 solar mass M dwarf; water is produced by the oxidation of atmospheric hydrogen with oxygen buffers in the magma ocean. The results indicate that using a highly-enriched atmosphere instead of one based on solar-abundance leads to a more massive atmosphere, and that Mars-mass planets can obtain water comparable to the amount in Earth's oceans. These results suggest that water-rich sub-Earth and Earth-like planets are more common than previously thought. \\
{\it NOTES:} \\
{\bf ``A primordial atmosphere of nebular origin itself can produce water through oxidation of the atmospheric hydrogen with oxidizing materials from incoming planetesimals or the magma ocean.''} \\
{\bf ``Because of oxygen buffers, the primordial atmosphere would likely be highly enriched with water vapour.''} \\
{\bf ***``We find that the well-mixed, highly-enriched atmosphere is more massive by a few orders of magnitude than the solar-abundance atmosphere (usually assumed in studies), and that even a Mars-mass planet can obtain water comparable to the present Earth's oceans.''} \\
{\bf ***``These results suggest that there are much more sub-Earths with Earth-like water contents than previously predicted.''} \\
{\bf ``If Earth had oceans three times more massive than the present, all the continents would be submerged in the global ocean (e.g., ref).''} \\
{\bf ***``Recent theories predict that terrestrial planets covered completely with oceans have extremely hot or cold climates \citep{abbot12, alibert14, nakayama19}. Thus, stable temperature climates are possible in a relatively narrow range of ocean mass.''} \\
{\bf ``For Sun-like stars, direct n-body simulations for planetesimal accretion under the gravity of a Jupiter-mass planet at 5.2 AU (e.g., \citealt{raymond04}) demonstrate that rocky planets in HZ, in most cases, obtain more water by a factor of $\sim$ 3-100 than the Earth's oceans.''}
{\bf ***``Exoplanet surveys show that the occurrence rate of giant planets around M dwarfs is lower than those around Sun-like stars (refs).''} \\
{\bf ***``Since water delivery by giant planets is considered not common to occur around M dwarfs, rocky planets with Earth-like water contents are rare in the HZ around M dwarfs \citep{tian15b}.''} \\
{\bf ***``The primordial atmosphere that a protoplanet captures from the circumstellar disc (refs) is capable of producing water. The atmospheric hydrogen is oxidized to produce water by oxides in vaporizing materials from planetesimals passing through the atmosphere and those in the magma ocean covering the protoplanetary surface.''} \\
{\bf ***``Rocky planets can acquire water in situ even inside the snowline, provided they are embedded in  circumstellar disc.''} \\
{\bf ``Studies find that as long as the atmosphere is optically thick enough, the atmospheric mass is closely related to the thermal state of the atmosphere, which is controlled by the opacity and the energy flux.''} \\
{\bf ``It is predicted by previous studies that protoplanets with masses less than a few Mars masses are unable to have massive primordial atmospheres... but they only used solar element abundances and ignored the effects of water vapour enrichment on the structure and mass of the atmosphere.''} \\
{\bf ***``The water production through oxidation of hydrogen in the primordial atmosphere of terrestrial planets is expected to be effective in increasing the atmospheric mass and, thus, water mass significantly.''} \\
{\bf ***``This study is aimed at quantifying that effect, focusing on planets of $sim 0.1-1.0 M_{\mathrm{oplus}}$, which are expected to be abundant around M dwarfs (refs).''} \\
{\bf ``For atmospheric masses of $\lesssim 20-30$\% of the solid-body mass, the assumptions both of hydrostatic and thermal equilibria are valid, as verified by previous studies (ref).''} \\
{\bf ***``An overall trend in our results is that $M_{atm}$ increases with initial water mass fraction (see Fig. 1).''} \\
{\bf ***``The dependence of $M_{atm}$ on initial water mass fraction itself is greater at 1.5 AU than at 0.5 AU.''} \\
{\bf ***``Our results indicate that the homopause altitude has a large impact on the atmospheric mass, and thus, the water amount. ''} \\
{\bf ***``Results also indicate that for planets at $\lesssim 0.8$ AU, the obtained water amount is quite insensitive to the semi-major axis and, thus, the disk temperature.''} \\
{\bf ***``We have demonstrated that even a protoplanet of Mars-mass or small obtains water comparable in mass to the present Earth's oceans, as long as the atmosphere is highly enriched with water ($\gtrsim 50$\% by weight or $\gtrsim10$\% by mole number.''} \\
{\bf ***``Once enriched with water (or other volatiles) and thereby becoming hot enough for rocks to be molten in some way, the atmosphere would keep itself enriched through chemical reactions between the atmospheric hydrogen and oxidizing rocky materials (refs).''} \\
{\bf ``Volatile-rich planetesimals would readily enrich the atmosphere with water: even in shallow parts of the atmosphere, the temperature is high enough for ice to evaporate.''} \\
{\bf ***``A rocky protoplanet likely has molten areas of its surface and thereby produces water through oxidation of atmospheric hydrogen.''} \\
{\bf ``We assume uniform composition in atmosphere. Vigorous convection and thereby mixing occur up to a certain altitude during a phase of rapid planetesimal accretion (i.e., high luminosity.''} \\
{\bf ``Condensation has been shown to have a great impact on the atmospheric mass (see Section 3.5).''} \\
{\bf ***``Mixing in the magma ocean is also an important factor because it effects the efficiency of oxygen supply to the surface. During the planetesimal accretion stage, the magma ocean is considered to be strongly convective due to its low viscosity and the large amount of deposited accretion energy (ref).''} \\
{\bf ***``An increase in surface temperature due to the water production would lead to stratifying the upper layer of the magma ocean, which may limit oxygen supply and thus water production.''} \\
{\bf ``Mixing processes are key, for both atmosphere and magma ocean, to understanding how much water terrestrial planets can finally acquire from disk gas.''} \\
{\bf ***``Even in the low energy flux case, the droplets are likely to remain in the atmosphere during the protoplanetary disk lifetime (a few Myr; ref).''} \\
{\bf ***``We assume that the disc gas prevents stellar XUV radiation from reaching the planet before 5 Myr.''} \\
{\bf ***``Lower stellar luminosities result in more massive primordial atmospheres because of the cooler nebula gas lower outer boundary temperature, and probably in smaller escape rate because of weak UV emission from the less active star. Thus, the survived mass of water could be larger.''} \\
{\bf ***``We assumed water is always vapour; however, as the atmosphere cools, water condenses and rains down to the surface (i.e., ocean formation), surviving the atmospheric escape.''} \\
{\bf ``Volatiles such as water and hydrogen are known to dissolve well in molten silicate (or magma). Thus, there occurs ingassing of the captured disc gas and the produced water into the magma ocean.''} \\
{\bf ``Models suggested that for protoplanets that have a primordial atmosphere, the amount of ingassed hydrogen and water could be as much as several Earth ocean masses.''} \\
{\bf ***``M dwarfs may be able to harbour habitable planets, although sub-Earths are expected to be a majority around M dwarfs (refs).''} \\
{\bf ***``This water production process would have a great importance for habitable planet formation around such stars.''} \\

\subsection{Models of mantle viscosity \citep{king95}}

\subsection{The viscosity structure of the mantle \citep{king95}}

\subsection{The XUV irradiation and likely atmospheric escape of the super-Earth $\pi$ Men c \citep{king19}}
$\pi$ Men C is a super-Earth orbiting a Sun-like star, but with a much closer orbit than Earth; as such, it receives 2000 times the XUV-radiation, and is likely losing its atmosphere at a substantial rate. Detection of the atmospheric loss and composition could determine whether the planet is a rocky, water-rich world, or a dense, rocky core with a hydrogen \& helium-rich envelope.

\subsection{EUV irradiation of exoplanet atmospheres occurs on Gyr timescales \citep{king20}}
Contrary to previous atmospheric escape models which assumed a general decline in stellar XUV, the authors find that stellar EUV declines much slower than X-ray. As a result, the total EUV irradiation on a planet atmospheres is dominated by the post-saturation phase rather than the $\sim 100$ Myr saturation phase (see below for explanation) itself, with the total combined XUV also received mostly after the saturation phase, on Gyr timescales. There is a brief discussion of the longer-timescale of photoevaporation due to these results, for hot Jupiters to Neptune-sized planets. {\bf The authors include relations for EUV and X-ray radiation, as a function of bolometric luminosity, which may be applied to M-stars (although possibly not as robust since the empirical relations are for FGK stars).} \\
{\it NOTES:} \\
{\bf ``For the first 100 Myr or so, when the star is rapidly rotating, the X-ray emission appears independent of rotation period and is described by an approximately constant ratio of X-ray and bolometric luminosities, $L_X/L_{bol} \approx 10^{-3}$.''} \\
{\bf ``Common assumption in literature that the high-energy irradiation of a planetary atmosphere and the resulting atmospheric escape is dominated by the saturated phase (refs).''} \\
{\bf ``Typically it is assumed that the EUV irradiation of planetary atmospheres declines at the same rate as in X-rays (many refs).''} \\
{\bf ``(refs) have noted that solar EUV emission remains relatively strong as X-ray surface flux decreases.''} \\
{\bf ***``Understanding the time dependence of photoevaporation may allow its effect to be separated from competing processes such as core-powered mass loss (refs).''} \\
{\bf ``It is worth noting that the power-law indices of previous empirical relations are only marginally steeper than -1, implying that significant X-ray emission is to be expected beyond the saturated interval.''} \\
{\bf ***``While the X-ray energy is seen to gradually level off, the corresponding EUV energy continues to increase sharply, even at late times. Only around 10\% of the lifetime EUV energy, and 20\% of the total XUV energy, is emitted during the first 100 Myr.''} \\
{\bf ***``Significant EUV-driven atmospheric escape from exoplanets may persist for Gyr timescales, which is much longer than assumed in the theoretical studies from Section 1.''} \\
{\bf ***``Two-thirds of the lifetime X-ray emission occurs after 100 Myr.''} \\
{\bf ``FGK stars typically emit 3-6 times more EUV radiation between the age of 100 Myr and 1 Gyr than they do during their first 100 Myr.''} \\
{\bf ***``Photon-limited escape is even further weighted to late times, since the average energy of ionising photons decreases and hence the energetic efficiency of mass loss increase. This softening of EUV spectrum therefore acts to extend atmospheric escape to even later time for low mass planets.''} \\
{\bf ``Core-powered mass-loss will occur on timescales of Gyr (refs), whereas it has been expected that photoevaporation will be dominated by the first 100 Myr (refs). Our results suggest that photoevaporation may act on longer timescales than has generally been considered.''} \\
{\bf ``(ref) suggested that a range of initial stellar spin periods leads to some stars spending much longer in the saturated phase than others (a range of 20-500 Myr, instead of the canonical 100 Myr).''} \\
{\bf ``For stars that form with relatively slow initial rotation, the integrated EUV emission after the saturated phase will be an even greater proportion of the lifetime XUV emission.''} \\
{\bf ***``We also find that the EUV spectrum softens significantly during the decline, which increases the flux of ionising photons for a given energy flux.''} \\
{\bf ***``Our results strongly suggest that models of atmospheric escape need to account for EUV irradiation that is dominated by Gyr timescales, and not just the first 100 Myr of the life of the system.''} \\

\subsection{Stellar wind interaction and pick-up ion escape of the Kepler-11 ``super-Earths'' \citep{kislyakova14}}

\subsection{Stellar driven evolution of hydrogen-dominated atmospheres of terrestrial exoplanets \citep{kislyakova20}}

\subsection{Geodynamics and rate of volcanism on massive Earth-like planets \citep{kite09}}
Volcanism rates are estimated for a spread of masses of Earth-like planets using a thermal evolution model and two possible tectonic modes: plate tectonic and stagnant lid. Crustal formation is also accounted for in the model to determine if melting and/or subduction are possible at a given time during the planet's evolution. Plate tectonics may be unlikely on massive planets due to buoyant crust being hard to subduct; stagnant lid planets have higher rates of melting early on but melting ends earlier. Overburden pressure (due to a massive ocean) may make degassing difficult, but still allows regassing, suggesting a steady-state ocean mass over time. It is also noted that volatiles partioned into the early melts may simply sink to the bottom of the mantle. \\
{\bf XXXX NOTE: Tables 1 \& 2 show references for many of the parameters from my model! May be useful if I need to reference everything. XXXX} \\
{\it NOTES:} \\
{\bf ***``(1) Volcanism is likely to proceed on massive planets with plate tectonics over the MS lifetime of the parent star; (2) Crustal thickness (and melting rate normalized to planet mass) is weakly dependent on planet mass; (3) stagnant lid planets live fast (they have higher rates of melting than their plate tectonics counterparts early in evolution), but die young (melting shuts down after a few Gyr); (4) plate tectonics may not operate on high-mass planets because of the production of buoyant crust which is difficult to subduct; (5) melting is necessary but insufficient for efficient volcanic degassing -- volatiles partition into the earliest, deepest melts, which may be denser than the residue and sink to the base of the mantle on young, massive planets.''}
{\bf ``A planet's atmosphere will consist of gas (1) accreted from the nebula, (2) degassed during impact accretion, and (3) degassed during subsequent geologic activity... All will be modified by atmospheric escape.''} \\
{\bf ``Volcanism results from partial melting of upper mantle.''} \\
{\bf ``Stagnant-lid lithospheres cool conductively. Because mantle cannot rise far into the stagnant lid, melting can only occur if the temperature at the base of the stagnant lid exceeds the local solidus of mantle rock.''} \\
{\bf ``Io shows yet another style of rocky-planet mantle convection: magma pipe cooling.''} \\
{\bf ``There is disagreement over whether plate tectonics will operate on massive planets. One previous study uses scaling arguments to argue that higher gravity favors subduction (Valencia et al. 2007). Another shows that subduction might never begin if the yield stress of old plate exceeds the stressed imposed by mantle convection (O'Neill \& Lenardic 2007).''} \\
{\bf ``***We relate our results to atmospheric degassing and discuss the possible supression of degassing (and, perhaps, melting) by the higher ocean pressure expected on massive Earth-like planets.***''} \\ 
{\bf ``A useful rule of thumb is that doubling a planet's concentration of radiogenic elements makes it behave like a planet with double the radius (Stevenson 2003).''} \\
{\bf ``Hot starts are overwhelmingly likely for differentiated massive rocky planets.'' (The authors initiate the mantle temperature at 3273 K.)} \\
{\bf ``Stagnant lid convection is less efficient at transporting heat than plate tectonics.''} \\
{\bf ``Beneath mid-ocean ridges, the mantle undergoes corner flow. Melt is generated in a prism with triangular cross section, ascends buoyantly, and is focused to a narrow magma lens beneath the ridge.''} \\
{\bf ``Mantle temperature is greater for more massive planets because their decreased surface area/volume ratio requires higher heat fluxes (and more vigorous convection) to dispose of the same heat flux.''} \\
{\bf ``Because stagnant lid convection is less efficient at transferring heat than plate tectonics, a planet in which plate tectonics is suddenly halted will heat up.''} \\
{\bf ``For young ($<2-3$ Gyr) planets, an ascending column of mantle produces more melt in a stagnant lid mode than in plate tectonic mode -- the higher temperature matters more than the (small) lithospheric thickness... Someone older planets produce less melt in stagnant lid.''} \\
{\bf ``Weakening the lithosphere by hydration is thought to be a prerequisite for plate tectonics.''} \\
{\bf ``On Earth, heat lost by conduction through thin lithosphere near mid-ocean ridges greatly exceeds heat lost by advection of magma.''} \\
{\bf ``Subduction will cease if the relative buoyancy of crust, less dense than mantle, exceeds that of the colder and denser lithospheric mantle.''} \\
{\bf ``Hotter -- that is, bigger or younger -- planets must recycle plate faster, so a plate has less time to cool. In addition, higher potential temperatures produce a thicker crust. Both factors tend to produce positively buoyant plate, which is harder to subduct. This effect is more severe for massive planets because of their greater gravity.''} \\
{\bf ``We find that the rate of volcanism on stagnant lid planets initially exceeds that on their plate tectonic counterpart but this contrast soon reverses as the stagnant lid thickens.''} \\
{\bf ``Through a greenhouse effect, a volatile envelope can raise surface temperatures, and increase partial melting... In thermal equilibrium, the reduced mantle-surface temperature difference demands more vigorous convection to drive the same heat flux across the upper boundary layer.''} \\
{\bf ***``Through overburden pressure, a volatile envelope can suppress degassing (and melting for sufficiently thick volatile layers). Water degassing is readily suppressed by 0.1-0.2 GPa of overburden (Papale 1997).''***} \\
{\bf ``A planet with a massive ocean cannot degas, but it can regas... This suggests a steady state ocean mass over geodynamic time.''} \\
{\bf ***``Much more pressure is needed to shut down melting than is needed to shut down degassing.''***} \\
{\bf ``Note that even for an ocean of constant depth, land is unlikely on planets much more massive than Earth. Gravity defeats hypsometry.''} \\
{\bf ``Planet mass can determine which mode of mantle convection the planet is in, which in turn determines whether melting is possible at all.''} \\
{\bf ``Shutdown of plate tectonics could place a planet in stagnant lid mode.''} \\
{\bf ``Observed volcanism on rocky planets $>8$ Gya would provide some support for plate tectonics -- if and only if tidal heating could be shown to be small.''} \\
{\bf ``On Earth, mantle heat loss is largely by conduction aided by hydrothermal convection at mid-ocean ridges, but mantle heat loss on stagnant lid planets should be less sensitive to te details of surface deformation.''} \\

\subsection{Habitability of exoplanet waterworlds \citep{kite18}}
Waterworlds (i.e., rocky planets with deep oceans) orbiting Sun-like stars may maintain habitable surface water conditions for $>$1 Gyr without the need for volatile cycling between the surface and the mantle. This is strongly dependent on initial water content, cation/carbon ratios, and initial carbon/water ratios. These results show promise for waterworlds orbiting M-dwarfs (with extended lifetimes likely leading to extended habitability). 

\subsection{Superabundance of exoplanet sub-Neptunes explained by fugacity crisis \citep{kite19}}

\subsection{Exoplanet secondary atmosphere loss and revival \citep{kite20}}

\subsection{Water on hot rocky exoplanets 
\citep{kite21}}
Hot rocky exoplanets with $P < 100$ d, which formed as gas-rich sub-Neptunes, might today have substantial (10-2000 bar), long-lived, H2O-dominated atmospheres.

\subsection{Characterization of M dwarfs using optical mid-resolution spectra for exploration of small planets \citep{koizumi20}}

\subsection{Effect of surface-mantle water exchange parameterizations on exoplanet ocean depths \citep{komacek16}}
This study builds upon the seafloor pressure-dependent \citep{cowan14} and mantle temperature-dependent \citep{schaefer15} volatile cycling models of previous studies, while adding a more realistic (based on Earth's current steady state) hybrid model incorporating pressure-dependent degassing (since volcanism is less efficient with greater overburden pressure, i.e., more surface water) and temperature-dependent regassing (serpentinization cannot occur if the mantle temperature is too high). Plate tectonics are assumed, as with previous studies, for better comparison with Earth. Each model reaches an eventual steady state after $\sim$ 2 Gyr, implying that many terrestrial exoplanets may be at or near steady state as well. The former two models lead to the result that super-Earths are less likely to be waterworlds; however, the hybrid model of this paper determines that planets with a similar water mass fraction to Earth are actually more likely to become waterworlds, greatly affecting potential habitability due to the lack of an efficient silicate weathering feedback. {\bf I am using the time-dependent volatile cycling code for my thesis; I also have access to the steady-state code used in Section 4.}

\subsection{Habitable zones around main-sequence stars: new estimates \citep{kopparapu13}}

\subsection{Habitable zones around main-sequence stars: dependence on planetary mass \citep{kopparapu14}}

\subsection{Habitable moist atmospheres on terrestrial planets near the inner edge of the habitable zone around M dwarfs \citep{kopparapu17}}

\subsection{On the likelihood of plate tectonics on super-Earths: does size matter? \citep{korenaga10}}
This study suggests that the initiation of plate tectonics is more dependent on the presence of surface water than the size of the planet, as suggested by previous studies, based on plate-tectonic convection arguments. A large viscosity contrast is required between the thermal boundary layer and mantle (i.e., a low efficient friction coefficient) for plate tectonics to operate; this seems most likely possible by hydration of the boundary layer. The geodynamics of a planet may be more dependent on water in the mantle than on planetary mass.  {\bf There is a good brief comparison of plate-tectonic vs. stagnant-lid regimes, and equations to calculate the transition between the two are included.}

\subsection{Global water cycle and the coevolution of the Earth's interior and surface environment \citep{korenaga17}}

\subsection{The L 98-59 system: three transiting, terrestrial-sized planets orbiting a nearby M-dwarf \citep{kostov19}}
TESS discovery of three transiting Earth-sized planets orbiting a bright, nearby (10.6 pc) M3 dwarf star, which can provide a test of dynamical evolution of multi-planet systems. The planets are small (0.8 to 1.59 $R_{\oplus}$) and in tight orbits (2.25 to 7.45 days); they are dynamically stable for circular orbits.

\subsection{Subduction and atmospheric escape of Earth’s seawater constrained by hydrogen isotopes \citep{kurokawa18}}
The evolution of the water mass fraction throughout the mantle, surface, and atmosphere is modelled, along with D/H ratios (which reflect global water cycling) in the present-day oceans, oceanic crust, and continental crust, which show different values. The model suggests three mutually exclusive explanations for this: hydrogen escape from a reduced early atmosphere, secular net regassing, or faster plate tectonics on early Earth than predicted by models. These results stress the importance of proper constraints on the D/H ratio throughout Earth's history. 
Seafloor alteration, slab dehydration, and chemical alteration of continents lead to D-enrichment in liquid water. Neither secular regassing nor hydrogen escape is required to get the D/H ratio of present-day Earth in their fast plate tectonics model, but the former is required in the slow model. {\bf The authors argue that secular regassing, where water in present-day mantle entirely resulted from regassing throughout Earth's history may explain the initiation of plate tectonics; it is also consistent with constant continental freeboard, and with theoretical predictions of crystallization of magma oceans, where the majority of water was partitioned in atmosphere \& oceans \citep{hamano13}.} Korenaga (2013) argued the ideal water distribution to initiate plate tectonics was large oceans above a dry mantle.
{\bf ``The theoretical modles assuming equilibrium partitioning of water predicted that more than $sim90$\% of the water would be partitioned into the exosphere at the time of magma ocean solidification (\citealt{hamano13}; Elkins-Tanton, 2008)... more than half of the initial water budget should be partitioned into oceans.''}

\subsection{Origin and loss of nebula-captured hydrogen envelopes from 'sub-' to 'super-Earths' in the habitable zone of Sun-like stars \citep{lammer14}}

\subsection{Origin and evolution of the atmospheres of early Venus, Earth and Mars \citep{lammer18}}

\subsection{How to build a habitable planet \citep{langmuir12}}
A textbook, covering many processes that lead to the formation and maintaining of a habitable planet. Since it is a long textbook with many chapters, a summary will be provided elsewhere than this document.

\subsection{3D climate modeling of close-in land planets:
Circulation patterns, climate moist bistability, and habitability \citep{leconte13}}
A 3D global climate model is used to model the atmospheric evolution of two close-in land planets, which orbit interior to the defined land planet HZ. The authors find a ``moist'' bistability in final states, dependent on evolutionary pathway and initial water inventory: the classical runaway greenhouse state (where all water is vaporized), and another where all water is in permanent cold traps such as ice caps at the poles. This bistability arises due to competition between greenhouse effect of water vapour and its condensation on nightside/near poles, so the runaway greenhouse is dynamical. \\
{\it NOTES:} \\
{\bf ``Earth's climate can be considered an aqua planet (i.e., a planet whose surface is largely covered by connected oceans that have a planet-wide impact on climate. Not to be confused for an ocean planet which, in addition, water must represent a significant fraction of the bulk mass (Leger et al. 2004)).''} \\
{\bf ``It has been shown that above a critical absorbed stellar flux, the planet cannot reach radiative equilibrium balance and is heated until the surface can radiate at optical wavelength around a temperature of 1400 K (Kasting et al. 1984; Kasting 1988). All the water at the surface is vaporized. The planet is in a runaway greenhouse state. Roughly speaking, this radiation limit stems from the fact that the surface must increase its temp to radiate more, but the amount of water vapor in the troposphere, hence the opacity of the latter, increases exponentially with this surface temperature (because of the Clausius-Clapeyron relation).''} \\
{\bf ``The theory outlined above assumes that a sufficient (meaning that, if fully vaporized, the water reservoir must produce a surface pressure that is higher than the pressure of the critical point of water, 220 bars) reservoir of liquid water is available everywhere on the surface (Selsis et al. 2007).''} \\
{\bf ``Global climate models show that ``land planets'' can have a wider HZ \citep{abe11}. Near the inner edge, this property stems from the fact that atmospheric transport of water from warm to cold regions of the planets is not counteracted by large-scale surface runoff (Abe et al. 2005). Heavily irradiated regions are then drier and can emit more IR flux than the limit found for a saturated atmosphere.''}
{\bf ``In principle, it is this possible to find planets that absorb and re-emit more flux than the runaway greenhouse threshold and for which liquid or solid water can be thermodynamically stable at the surface, provided they have accreted limited water supplies (or lost most of them)... These planets lie even closer to the star than the inner edge of the limits set by \citet{abe11} for land planets.''} \\
{\bf ``For those close-in planets whose rotation states have been strongly affected by tidal dissipation, ...efficient and permanent cold traps present on the dark hemisphere or at the poles could irreversibly capture all the available water in a permanent ice cap.''} \\
{\bf ``For a wide range of parameters, two stable equilibrium climate regimes exist because of the very strong positive feedback of water vapor... determined not only by the incoming stellar flux and spectrum, or by the planet's atmosphere, but also by the amount of water available and its initial distribution.''} \\
{\bf ``The present climate of a given object may depend on the evolutionary path that it followed. In particular, the history of the water delivery and of the atmospheric escape may play a crucial role in determining whether these objects are dry rocks surrounded by hot steam atmosphere or else harbor vast ice deposits on the night side and near the poles.''} \\

\subsection{Continuous reorientation of synchronous terrestrial planets controlled by mantle convection \citep{leconte18}}
While many terrestrial exoplanets are expected to be synchronously rotating w.r.t. their host star, creating a distinct dayside/nightside profile, true polar wander (TPW) caused by even weak mantle convection, as seen on Earth, can reorient the planet throughout its evolution, changing the day- and night-sides of the planet. It is also noted that plate tectonics is not a requirement for TPW.

\subsection{Carbonate-silicate cycle predictions of Earth-like planetary climates and testing the habitable zone concept \citep{lehmer20}}

\subsection{A mechanism for crustal recycling on Venus \citep{lenardic93}}

\subsection{Climate-tectonic coupling: Variations in the mean, variations about the mean, and variations in mode \citep{lenardic16}}

\subsection{The diversity of tectonic modes and thoughts about transitions between them \citep{lenardic18a}}
Four tectonic modes are discussed -- active lid (plate tectonics, as we have on Earth), stagnant lid (as is seen on Venus), sluggish lid (proposed for the early Earth), and an ``episodic'' regime which can see episodes of a large subduction ``pulse''. Transitions between these regimes are fuzzy, but a variety of studies have sought to determine the transitions based on various parameters (e.g., lithospheric strength and a planet's internal energy); however, even with transition regime diagrams, much uncertainty still exists. It is difficult to answer the question of ``why and when'' Earth has/initiated plate tectonics, and there may be a significant stochastic component due to impacts. {\bf This is a great review of different tectonic regimes, and should be read again.}

\subsection{Volcanic-tectonic modes and planetary life potential \citep{lenardic18b}}
A review discussing the implications for life induced by active-lid (including plate tectonics and distributed deformation), transition regimes (sluggish lid and episodic), and stagnant lid (hot and cold) regimes, outlining their volcanic and tectonic modes. Plate tectonics (``active-lid'') means lithospheric motions drive the mantle motions below (i.e., plates are self-driven, mantle responds to plate motions). Sluggish lid refers to mantle velocities being higher than those of the lithosphere, driving the lithospheric motion. ({\bf See Figure 5 for a great explanation of four regimes and their submodes.}) Based on simulations and studies, it seems possible that life could exist on planets in any of these modes, within other considerations (e.g., weathering stabilizing climate, stellar insolation). There is still uncertainty whether plate tectonics initiated before or after life had taken hold. {\bf This is a wonderful reference, and should be read carefully more than once.}

\subsection{High-resolution spectral discriminants of ocean loss for M dwarf terrestrial exoplanets \citep{leung20}}
Ground-based high-resolution spectra are simulated for three possible Earth-like planet atmospheres: a completely desiccated planet, one where oceans can still be degassed (i.e., Earth without vegetation), and Earth itself. The results indicate that the biogenicity of atmospheric O2 can be discriminated by observing multiple O2 bands, since O2-O2 CIA strongly suppresses the 1.27 $\mu$m band in high-O2 atmospheres. This provides some context of atmospheric loss history of a terrestrial planet. \\
{\it NOTES:} \\
{\bf ``While Earth's abundant O2 is photosynthetic, early ocean loss may also produce high atmospheric O2 via water vapor photolysis and subsequent hydrogen escape.''} \\
{\bf ***``We find that 10 bar O2 post-ocean-loss atmospheres have strong suppression of oxygen bands, and especially the 1.27 $\mu$m band.''***} \\
{\bf ***``The presence of an ocean-loss high-O2 atmosphere could be inferred via detection of a strong 0.69 $\mu$m O2 band, and a weaker or undetected 1.27 $\mu$m band.''***} \\
{\bf ``One of the most promising techniques to search for O2 in exoplanet atmospheres is high-resolution spectroscopy with ground-based telescopes (refs).''} \\
{\bf ``O2 may be one of the most readily detectable biosignature gases to search for, because it is produced by photosynthesis, and can accumulate to high abundance in planetary atmospheres (refs).''} \\
{\bf ``As noted by (refs), high resolution ground based spectroscopy is particularly well suited to observing Earth-size planets orbiting M dwarfs due to the favorable planet-to-star flux contrast ratios.''} \\
{\bf ``The high-resolution technique can be used to observe planetary atmospheres in transmission, for transiting M dwarf planets, or in reflected light, for planets that do not transit.''} \\
{\bf ``Compared to G dwarf stars like our Sun, M dwarfs are highly active, with frequent flares, strong stellar winds and other stellar events that could potentially modify the composition or remove a planetary atmosphere (refs).''} \\
{\bf ``Early, pre-MS M dwarfs are super-luminous as they contract slowly down to their MS size and luminosity (refs).''} \\
{\bf ``Photolysis of water in upper atmosphere during runaway greenhouse, and subsequent H escape to space, could leave behind high amounts (1000 bar) of atmospheric oxygen (refs). However, O2 atmospheric loss processes and surface sinks may reduce this ocean-loss O2 atmosphere to 10s of bars or less.''} \\
{\bf ***``It may be that O2 in M dwarf planetary atmospheres may be a common outcome of their early evolution, even for planets in what becomes the MS HZ (refs), and the challenge will be to determine whether the O2 detected is due to ocean loss, or a true photosynthetic biosphere.''***} \\
{\bf ***``In the ocean loss case, the high-O2 atmosphere that remains may be of sufficiently high pressure, potentially several bars or more (refs), that it produces O2-O2 collisionally-induced absorption (CIA). This may be detectable with JWST, and would be a strong indicator of past ocean loss (refs).''***} \\
{\bf ***``Near-term observations of O2 on extrasolar planets in the HZ may also be possible from the ground.''***} \\
{\bf ``To quantify the detectability of individual spectral bands we use an integration scheme similar to the methods used to extract exoplanet spectra at high resolution. This method is based off of that used in Snellen et al. (2017).''} \\
{\bf ***``For the band centered at 1.27 $\mu$m, the high O2 atmosphere has greatly reduced reflectance, agian due to O2-O2 CIA, to the degree that it appears nearly as a flat line with barely perceptible O2 absorption features on the bottom side. This is in sharp contrast to the deep, frequent lines that make up the 1.27 $\mu$m O2 band of the Earth-like atmosphere.''***} \\
{\bf ``It is apparent from Figure 8 that O2-O2 CIA is overpowering the structure of the oxygen absorption, effectively lowering the continuum, ad leaving the band much shallower.''} \\
{\bf ***``The 1.27 $\mu$m band has the highest relative detectability of the four bands for the Earth-like case... but the detectability of the 1.27 $\mu$m band is suppressed by a factor of $\sim200$ for both of the high oxygen atmosphere cases compared to the Earth-like case, due to the stronger O2-O2 CIA in this band.''} \\
{\bf ***``We found that discriminating the more Earth-like atmospheres that might be produced by a photosynthetic biosphere from ocean loss scenarios with remnant high-O2 atmospheres may be possible by examining the relative strength of the 1.27 $\mu$m and 0.69 $\mu$m (or 0.76 $\mu$m) oxygen bands.''***} \\
{\bf ``In our simulations, the high-O2 post-ocean-loss atmospheres tended to have weaker O2 detectability than the Earth-like case at all wavelengths due to strong O2-O2 CIA over a similar wavelength range to the O2 bands.''} \\
{\bf ***``The different impact of O2-O2 CIA on the O2 bands affords a potential, and critical test for the biogenicity of the observed O2.''***} \\
{\bf ***``If the 1.27 $\mu$m band is detected, the atmospheric O2 is more likely to be due to life, and it is unlikely that a massive O2 atmosphere is present.''***} \\
{\bf ***``In comparison, if the 0.69 $\mu$m O2 B-band (or 0.76 $\mu$m A-band) is detected, but the (normally much stronger) 1.27 $\mu$m band is not, this would point towards a more massive atmosphere, with ocean loss as the more likely source of any O2 detected.''} \\
{\bf ***``In the near-term, high-resolution ground-based observations are therefore likely the best, if not the only, way to detect biogenic levels of O2.''***} \\
{\bf ``JWST may b able to use transmission observations to detect strong absorption from O2-O2 CIA in as few as 4-20 transits of the planets TRAPPIST-1 b-e (refs).''} \\
{\bf ***``Suppression of the continuum near oxygen bands as a result of O2-O2 CIA is a key feature of high-O2 atmospheres.''} \\
{\bf ***``Detection of the A- or B-band and a non-detection or strong suppression of the 1.27 $\mu$m band would suggest ocean loss as the likely source of the observed O2.''} \\

\subsection{Water contents in mantle xenoliths from the Colorado Plateau and vicinity: implications for the mantle rheology and hydration-induced thinning of continental lithosphere \citep{li08}}

\subsection{A water budget dichotomy of rocky protoplanets from \textsuperscript{26}Al-heating \citep{lichtenberg19}}
Internal radiogenic heating of the short-lived isotope \textsuperscript{26}Al leads to a decrease in water mass fraction with time. The simulations of this paper claim that high concentrations this isotope can explain the water-poor terrestrial planets in our own Solar System, which would have formed mainly from rapidly dehydrated planetesimals, while suggesting that a lower content of \textsuperscript{26}Al leads to the formation of ocean worlds. The results help explain the system-wide low water mass fractions and lack of orbital trend in water content of the TRAPPIST-1 planets. 

\subsection{Vertically resolved magma ocean-protoatmosphere evolution: H2, H2O, CO2, CH4, CO, O2, and N2 as primary absorbers \citep{lichtenberg21}}

\subsection{Brown dwarf atmosphere as the potentially most detectable and abundant sites for life \citep{lingam19}}

\subsection{Implications of abiotic oxygen buildup for Earth-like complex life \citep{lingam20}}
{\it NOTES:} \\
{\bf ***``From our simple model, it is suggested that worlds that receive time-averaged X-ray and extreme UV fluxes that are $\gtrsim10$ times higher than Earth might not be capable of hosting complex lifeforms because the photolysis of molecules such as water may lead to significant O2 buildup.''} \\
{\bf ***``A number of models suggest that high XUV fluxes play a vital role in the buildup of massive O2 atmospheres by way of desiccation and photolysis (refs); however, other models indicate that this buildup may be mitigated by a number of mechanisms that may result in modest O2 accumulation and H2O retention (refs).''} \\
{\bf ***``The buildup of dense abiotic O2 atmospheres is often (but not always) accompanied by the depletion of surface water, because it serves as the source for oxygen by undergoing evaporation and photolysis by XUV radiation.''} \\
{\bf ``By the 1960s, it was already well established that high concentrations of O2 were responsible for the inhibition of metabolic processes in eukaryotes as well as prokaryotes due to the intricate interactions between O2 and a number of proteins such as flavoproteins and ferredoxins.''} \\
{\bf ***Section 3, Oxygen Sources and Sinks on Earth-like planets, contains an outline of sources and sinks contributing to abiotic O2 buildup} \\
{\bf ***``We wish to reiterate that the most dominant source of O2, corresponding to Equation 2, is concomitantly accompanied by the loss of H2O. The depletion of $\sim270$ bars of H2O -- which is equivalent to the production of $\sim135$ bars of O2 assuming that the stoichiometric ratio is preserved -- would result in the complete loss of Earth's oceans.''} \\
{\bf ***``The synthesis of prebiotic compounds, and potentially the origin of life, is believed to have required a reducing atmosphere (refs); the synthesis of biomolecules may be suppressed in massive O2 atmospheres as they could undergo rapid oxidation.''} \\
{\bf ***``At sufficiently high XUV fluxes, the sustenance of Earth-like complex life is rendered very difficult on most planets.''} \\
{\bf ***``Figure 2 shows that larger worlds exhibit lower O2 buildup at smaller values of the XUV flux, but eventually increase rapidly and undergo near-saturation.''} \\
{\bf ``As the atmospheric O2 inventory increases, so too does the maximal size of organisms that are reliant on `simple' mechanisms of O2 capture and transport, with diffusion and blood circulation constituting two common examples.''} \\
{\bf ``Based on Figure 2, it is conceivable that atmospheres with 100-1000 bar of O2 could exist on certain worlds \citep{luger15}.''} \\
{\bf ``Worlds with high oxygen inventories may prove to be amenable to the development of large brains and concomitant high intelligence.''} \\
{\bf ``Worlds with high O2 inventories (at least up to a certain limit) might provide conducive environments for the advent of organisms with sophisticated cognitive abilities.''} \\
{\bf ***``One of the major results from our analysis was that worlds that receive temporally averaged XUV fluxes that are an order of magnitude more than Earth are potentially likely to exceeds the limits of oxygen toxicity for Earth-based life.''} \\
{\bf ***``In a nutshell, CO2, CO, and O2 toxicity might jointly conspire to preclude complex Earth-like life on many M-dwarf exoplanets.''} \\

\subsection{The intrinsic extreme ultraviolet fluxes of F5V to M5V stars \citep{linsky14}}

\subsection{A geologically robust procedure for observing rocky exoplanets to ensure that detection of atmospheric oxygen is an Earth-like biosignature \citep{lisse20}}
A scheme is devised to prioritize targets most likely to contain Earth-like life, based on a flowchart of triages (Fig. 1); this includes properties such as activity of host star (to retain secondary atmosphere) and reflectance spectra of the planet (to find a blue-tinted hue, similar to Earth). The authors claim that atmospheric oxygen is only a useful biosignature for planets with both exposed land and surface water, based on a previous paper (Glaser et al. 2020). \\
{\it NOTES:} \\
{\bf ``Planets with water contents much greater than Earth appear more prone to false positives for oxygen (e.g., \citealt{luger15}), but desiccated planets may also be susceptible to O2 false positives (Meadows et al. 2018).''} \\
{\bf ``Atmospheric oxygen is a useful biosignature on rocky, Earth-like (radius less than 1.5 Earth radii) exoplanets only if they have both surface water and land, or exposed continental rock.''} \\
{\bf ***``In using oxygen as a biosignature, it is implicitly assumed that the atmosphere is in steady state and destruction balances production.''} \\
{\bf ``The export of O2 to the atmosphere on Earth is directly proportional to the flux of bioavailable phosphate from weathering of apatite on continents.''} \\
{\bf ``An exoplanet with as little as 0.1 wt\% water on its surface -- what we term a 'pelagic planet' -- would have no exposed land and no subaerial weathering of continents \citep{cowan14}.''} \\
{\bf ``The geochemical cycles on a planet with $>0.1$ wt\% water preclude using O2 as a biosignature.''} \\
{\bf ``It is necessary to know now that confident assignment of O2 to life requires confirmation of both surface land and water on an exoplanet, so that missions can be designed and yield needed data by the time they launch.''} \\
{\bf ***``Ultimately, the goal is to maximize the likelihood that a planet could eventually be demonstrated to not only have O2 in its atmosphere, but could also be demonstrated to have land and water on its surface, so that the O2 would be a reliable and defendable biosignature.''} \\
{\bf ***``As Glaser et al.(2020) conclude, searching for life becomes increasingly more difficult as the water fraction exceeds 0.1 wt\%. Just 5 oceans worth of water on the surface of a $1 M_E$, $1 R_E$ planet is sufficient to submerge all continents, assuming standard topopgraphy \citep{cowan14}.''} \\
{\bf ***``On an Earth-like planet with 50 oceans (1 wt\% bulk H2O) any continents and geochemical cycles would take place under a thick ($\sim100$ km) high-pressure ice mantle that would cut off chemical communications between the rocky planet and the oceans (refs).''} \\
{\bf ***``On an Earth-like planet with just 2 wt\% bulk H2O, silicate melting and outgassing would be suppressed by the high pressure, effectively shutting off geochemisty altogether \citep{kite09}''.} \\
{\bf ``Only if a planet had $<0.1$ wt\% H2O could we be sure that the biogeochemical cycles were sufficiently like Earth's to use O2 as a biosignature gas.''} \\
{\bf ``Not only must an abundance of O2 be found in an exoplanet's atmosphere, surface land and water also must be confirmed, to infer production rates and use O2 to detect life.''} \\
{\bf ***``Although life as we know it requires water, we should actually search for life on planets with less water than is currently detectable.''} \\
{\bf ``Planets with radii $>1.5 R_E$ can be deprioritized, because they are very likely to have thick H2/He atmospheres (refs), implying radically different geochemistry.''} \\
{\bf ``Planets with radii below some lower limit of $\sim 0.6 R_E$ may not be able to retain atmospheres, although the lower bound is sensitive to many factors.''} \\
{\bf ***``Presuming that exoplanets typically form with thick H2/He atmospheres accreted from their protoplanetary disks (ref), it is necessary that past stellar activity has been sufficient to strip the planets of these primary atmospheres.''} \\
{\bf ``Earth probably formed from a merger of two embryos, $0.4 M_E$ and $0.6 M_E$ (refs).''} \\
{\bf ``The criterion is that the integrated XUV heating over time exceed roughly 40\% of the planet's gravitational binding energy (ref), or about $10^{39}$ erg for a 1 $M_E$ planet.''} \\
{\bf ``It is also critical that the total integrated XUV flux on a planet from its birth through the present day not exceed the threshold necessary for retention of a secondary atmosphere.''} \\
{\bf ***``The Si/Mg ratio determines water storage in a planetary mantle and partioning of water between surface and mantle.''} \\
{\bf ``The rheology of the mantle, to which convection is sensitive, could also depend on sensitively on this ratio. Melting curves, and therefore the thickness of a lithosphere or the pressures at which degassing occur, depend on Si/Mg but also Na/Mg and Al/Mg.''} \\
{\bf ``Importance of measuring U/Mg, Th/Mg, and especially K/Mg. Even presuming a stagnant lid regime, the lifetime of degassing is sensitive to these ratios. In most respects, the closer these ratios are to Earth-like, the more likely the planet is to degas volatiles into their atmospheres and to emplace minerals with bioessential elements (e.g., P) at Earth-like rates.''} \\
{\bf ``Simple radius-mass scaling models (refs) do not accurately predict the interior structure for planets, including the Earth (ref).''} \\
{\bf ***``Simultaneous detection of reduced CH4 and oxidizing O2, which are in chemical disequilibrium with each other, would be a highly robust indicator of complex, established life producing steady state metabolic products (ref).''} \\
{\bf ***``Approximately 0.1\% H2O atmospheric vapor should be found to signal the presence of liquid water on the surface (ref), but not more (Glaser et al. 2020), and the water vapor must be restricted to the troposphere, as planets in runaway moist greenhouse stages will have significant amounts of stratospheric water (refs).''} \\
{\bf ***``Principal component analysis of the time-varying reflected light in various filters has allowed detection of land and oceans on Earth from space, and could be used to identify patches of land, ocean, and vegetation on an exoplanet \citep{cowan09}.''} \\
{\bf ***``Elemental abundances of the later M dwarfs are also difficult to obtain due to the increasing predominance of molecular vs. atomic absorption features as the photosphere becomes cooler.''} \\

\subsection{High-resolution lithosphere viscosity and dynamics revealed by magnetotelluric imaging \citep{liu16}}

\subsection{Extreme water loss and abiotic O\textsubscript{2} buildup on planets throughout the habitable zones of M dwarfs \citep{luger15}}
Atmospheric escape, particularly of hydrogen and oxygen (through hydrodynamic escape) is modelled for planets in the habitable zones of M dwarfs. During the early evolution of the star, the planet may enter an extended runaway greenhouse phase, if the initial water content is high enough; the planet may then lose tens of terrestrial oceans (TO) worth of water and potentially become desiccated. Planets with inefficient oxygen sinks will retain much of the oxygen remaining from the photolysis of water in their atmospheres, which may be both detrimental to habitability and a false positive biosignature. Both the amount of water lost and the final oxygen pressure scale with planetary mass.

\subsection{Detecting ocean glint on exoplanets using multiphase mapping \citep{lustig18}}

\subsection{The interior and atmosphere of the habitable-zone exoplanet K2-18b \citep{madhusudhan20}}

\subsection{Analyzing atmospheric temperature profiles and spectra of M dwarf rocky planets \citep{malik19}}

\subsection{Thermal equilibrium of the atmosphere with a convective adjustment \citep{manabe64}}

\subsection{Thermal equilibrium of the atmosphere with a given distribution of relative humidity \citep{manabe67}}

\subsection{The origins and concentrations of water, carbon, nitrogen, and noble gases on Earth \citep{marty12}}

\subsection{Thermal evolution of the Earth: effects of volatile exchange between atmosphere and interior \citep{mcgovern89}}
A volatile cycling model which includes cycling between the surface and mantle, and the lowering of mantle viscosity through regassing of water. Degassing is taken as seafloor pressure-dependent, while regassing also depends on overlying atmosphere-hydrosphere mass and efficiency of recycling at arc volcanoes. Volatile loss from the mantle impacts the mantle temperature: higher mantle temperatures lead to a faster cooling. Degassing is compensated by an increase in temperature, regassing by a decrease. {\bf Important early reference for my project.}

\subsection{The habitability of Proxima Centauri b: Environmental states and observational discrimants \citep{meadows18a}}

\subsection{Factors affecting exoplanet habitability \citep{meadows18b}}

\subsection{Origin of Earth's water: sources and constraints \citep{meech19}}

\subsection{Hemispheric tectonics on super-Earth LHS 3844b \citep{meier21}}
{\bf XXXX Probably not relevant to the project yet, but interesting to mention in a Discussion section at the very least. XXXX}

\subsection{Estimating the ultraviolet emission of M dwarfs with exoplanets from Ca II and H$\alpha$ \citep{melbourne20}}

\subsection{Climate stability of habitable Earth-like planets \citep{menou15}}

\subsection{Planet formation around M dwarfs via disk instability \citep{mercer20}}

\subsection{Why must a solar forcing be larger than CO2 forcing to cause the same global mean surface temperature change? \citep{modak16}}

\subsection{The thermal state of Io \citep{moore01}}

\subsection{Tidal heating and convection in Io \citep{moore03}}

\subsection{Mantle convection with a brittle lithosphere: thoughts on the global tectonic styles of the Earth and Venus \citep{moresi98}}

\subsection{Observing the atmospheres of known temperate Earth-sized planets with JWST \citep{morley17}}

\subsection{Origin of water in the inner Solar System: a kinetic Monte Carlo study of water adsorption on forsterite \citep{muralidharan08}}

\subsection{Water circulation and global mantle dynamics: insight from numerical modelling \citep{nakagawa15}}

\subsection{A study on the ``runaway greenhouse effect'' with a one-dimensional radiative-convective equilibrium model \citep{nakajima92}}

\subsection{Runaway climate cooling of ocean planets in the habitable zone: a consequence of seafloor weathering enhanced by melting of high-pressure ice \citep{nakayama19}}
A coupled model of climate (assuming plate tectonics), including high-pressure ice melting, seafloor weathering, and carbon cycling, is used to study HZ ocean-covered planets. Previous studied assumed would be hot due to the high-pressure ices at the bottom of the ocean preventing chemical weathering. However, this study finds that mid-ocean ridge heating can create a ``sorbet'' region (water-ice) which allows seafloor weathering to operate. Planets with large oceans and CO\textsubscript{2}-poor atmospheres are likely to have non-temperate climates, possibly lapsing into a snowball state, even if orbiting within the HZ. The critical ocean mass for this snowball state is weakly sensitive to planetary mass, degassing flux, and temperature-dependent seafloor weathering, i.e., ocean mass plays a critical role on the climate of large ocean planets. {\bf Introduction includes an explanation of the effects of high-pressure ices.}

\subsection{Radiogenic heating and its influence on rocky planet dynamos and habitability \citep{nimmo20}}

\subsection{Coupling the atmosphere with interior dynamics: Implications for the resurfacing of Venus \citep{noack12}}
A coupled interior-atmosphere model (including H loss to space) is used to investigate the tectonic evolution and resurfacing of Venus. When the surface temp reaches a critical value, it transitions from stagnant to mobile, which increases the heat removal from mantle \& interior cooling rate, which in turn reduces partial melting and H2O outgassing. The high surface temperature of Venus allowed crust mobilization (but NOT plate tectonics, which requires cool temps), and allowed episodic resurfacing events throughout Venus' history. \\
{\it NOTES:} \\
{\bf ``High surface temperature may reduce the plate-like behaviour of the lithosphere -- by making it more ductile -- as was suggested for Venus by Lenardic et al. (2008), which in turn influences the climate since plate tectonics can recycle atmospheric CO2 into the mantle.''} \\
{\bf ``(Models indicate) an increase in surface temperature leads to an increase in partial melting (and more outgassing) and hence an increase in atmospheric density and surface temperature. This triggers a positive runaway effect, which destabilizes the climate of the planet.''} \\
{\bf ``Parameterized models are limited by their inability to adapt the convectice regime (e.g., stagnant lid, mobile lid, plate tectonics) to changing boundary conditions... Transitions from stagnant lid to mobile lid cannot be simulated. This limitation does not exist in 2D or 3D interior models...''} \\
{\bf ``(In our model) the surface temperature is a function of water vapour concentration in the atmosphere, the latter of which is calculated from the mantle outgassing rate while employing observed exospheric loss rates.''} \\
{\bf ``Venus has a experienced (possibly episodic) catastrophic resurfacing event, with several explanations (SEE PAPER!)''} \\
{\bf ``(Studies estimate) surface temperatures must exceeds a critical value of about 1000 K to allow a transformation from stagnant-lid to global mobile regime.''} \\
{\bf ``***The model in this paper shows that atmospheric changes caused by outgassing and exospheric loss processes are key elements to consider when modelling the interior of planets like Venus.''***} \\

\subsection{Plate tectonics on rocky exoplanets: Influence of initial
conditions and mantle rheology \citep{noack14}}
A 2D model is used to study a pseudo-plastic mantle rheology to determine the factors influencing the transition from stagnant lid to plate tectonics. Initial mantle temperature appears to have the strongest effect, while wet/dry rheology and planet mass also contribute. Plate tectonics is most likely planets with a cool surface or cool interior, while a hot mantle may lead to the formation of a stagnant lid. Mantle pressure, and as a result mantle convection, is a strong contributing factor to the initiation of plate tectonics. A ``mobility'' factor is used in Fig. 6 to determine whether the planet is in a stagnant lid or mobile lid regime. {\bf A wet rheology/high Rayleigh number may lead to a stagnant lid over a dry rheology.}

\subsection{Water-rich planets: how habitable is a water layer deeper than on Earth? \cite{noack16}}

\subsection{Water in extrasolar planets and implications for habitability \citep{noack17}}

\subsection{The distribution of H2O between silicate melt and nominally anhydrous peridotite and the onset of hydrous melting in the deep upper mantle \citep{novella14}}

\subsection{Effects of a binary companion star on habitability of tidally locked planets around an M-type host star \citep{okuya19}}

\subsection{Geological consequences of super-sized Earths \citep{oneill07}}
Super-Earths are more likely to be in a stagnant-lid or episodic regime than Earth-sized planets (which can be in a plate tectonic or episodic regime), based on a model which determines the transition between regimes using the relationship between yield stresses and convective driving forces. This relationship appears to scale with mass. {\bf A good description of the geological processes on a stagnant lid planet, including atmospheric loss, is in Section 4.}

\subsection{A window for plate tectonics in terrestrial planet evolution? \citep{oneill16}}
The tectonic regime of a planet critically depends on the initial conditions, post-magma ocean, particularly internal and basal heating; planets with similar physical parameters may evolve with different tectonics depending on the initial conditions. The simulations indicate that plate tectonics may in fact be a phase in planetary evolution, between hot and cold stagnant lid states, instead of an end-member. \\
{\it NOTES:} \\
{\bf ``The tectonic regime of a planet depends critically on the contributions of basal and internal heating to the planetary mantle, and how these evolve through time.''} \\
{\bf ***``Hot mantle conditions, coming out of a magma ocean phase of evolution, can produce a `hot' stagnant lid regime, whilst a cooler post magma ocean mantle may begin in a plate tectonic regime.''} \\
{\bf ***``Planets may evolve from an initial hot stagnant-lid condition, through an episodic regime lasting 103 Gyr, into a plate-tectonic regime, and finally into a cold, senescent stagant-lid regime after $\sim10$ Gyr of evolution, as heat production and basal temperatures wane.''} \\
{\bf ***``The thermal state of the post magma ocean mantle, which effectively sets the initial conditions for the sub-solidus mantle convection phase of planetary evolution, is one of the most sensitive parameters affecting planetary evolution -- systems with exactly the same parameters may exhibit completely different tectonics depending on the initial state employed.''} \\
{\bf ***``Plate tectonics may be a phase in planetary evolution between hot and cold stagnant states, rather than an end-member.''} \\
{\bf ***``The ultimate tectonic state of a planet is a result of a balance between the coupling of the plates and mantle beneath, and also the buoyancy forces driving convective motion. These two factors are critically sensitive to how a planet's thermal state evolves through time; buoyancy forces are strongly coupled to the temperature drop across the convecting mantle, and induced lithospheric stresses to the internal velocity, and thus mantle temperature.''} \\
{\bf ***``Increases in internal heating rate for a given planet's size has also been shown to be able to cause a transition from mobile lid convection, into an episodic regime, and eventually into stagnant lid convection (refs).''} \\
{\bf ``The pioneering work of Tozer (1972) suggested that the negative feedback effects of temperature-dependent viscosity buffer the system so that initial conditions quickly decay, and the system reaches equilibrium with its internal heat generation.''} \\
{\bf ``Near the critical transition zone, non-linearities in the physical system lead to an inherent hysteresis, and the stable tectonic regime can depend strongly on starting state (refs).''} \\
{\bf ***``To obtain kinematic similarity, the system was scaled using a convective overturn time (ref) of 50 Myr (this defines the time and velocity scalings).''} \\
{\bf ***``A planet's tectonic lifetime from its initial starting conditions of a post-accretion planet to the end of its tectonic lifetime can last as long as 10-20 Gyr.''} \\
{\bf ``We are tracking a planet's evolution from post-magma ocean phase of its evolution onwards in our model; this means the initial conditions represent the state of the Earth coming out of a magma ocean phase of evolution.''} \\
{\bf ***``Recent work suggests that differences in accretion history of terrestrial planets can lead to variations in radiogenic heat content even for planets of the same bulk size and mass (ref).''} \\
{\bf ***``For each of the evolutionary paths of Figure 4, different outcomes are possible for exactly the same system parameters, depending on initial conditions.''} \\
{\bf ***``The assumption that the mantle evolves through a series of quasi-equilibrium states (ref), which is an underlying assumption behind parameterized convection and the use of statistically steady state calculations to determine scaling relations for mantle cooling, can break down for systems that display hysteresis effects (ref).''} \\
{\bf ``The heat lost during subduction events is far above the background level, which demonstrates that the episodic regime is an extremely efficient way of cooling the mantle on the early Earth.''} \\
{\bf ***``Our results indicate that plate tectonics is only a phase in the evolution of this model -- the transition to plate tectonics occurs only once th system has sufficiently cooled; and plate tectonics wanes once the CMB temperatures have cooled too much.''} \\
{\bf ``The evolutionary history of these models hinges critically on the initial model state.''} \\
{\bf ***``If Earth came out of a magma ocean phase ''cold'' -- i.e., it thermal state was roughly in equilibrium with the radioactive heat production at the time -- Earth may have had plate tectonics over its entire history to date.''} \\
{\bf ***``If, however, the formation of the Earth was accompanied by significant addition of primordial heat (e.g., through either core formation (ref) or impacting (ref)), then the possibility arises that Earth was episodic throughout much of its history, transitioning to a plate tectonic regime much later (ref).''} \\
{\bf ``If the accretion history stripped a planet of radiogenics then it could be more likely to enter a plate tectonic phase (ref).''} \\
{\bf ***``(ref) note that, for equivalent systems, starting in a plate-mode is more likely to maintain plate dynamics than starting with an immobile lithosphere.''} \\
{\bf ``Plate tectonics is often assumed to be the default tectonic state of a planet Earth's size, or larger (ref). Our simulations show this need not be the case.''} \\
{\bf ***``It has also been previously suggested, from scaling theory and steady-state simulations, that hysteresis may impact tectonic evolution (refs), and we have demonstrated this behaviour is applicable in evolutionary models as well.''} \\
{\bf ***``Consequently, it is apparent that the specific tectonic evolution of a planet is not solely governed by thermal controls, but the path is also heavily influenced by the initial state of the planet.''} \\
{\bf ``We have opted for a `minimum-artefacts' approach -- allowing a system to come to equilibrium at the initial conditions, and running the system forward from there.''} \\
{\bf ``What these models do not incorporate is depth-dependent structure; such as mantle phase changes or layered viscosity structures, such as an asthenosphere -- which also affect the wavelength of convection and thus stresses (ref).''} \\
{\bf ``It has been demonstrated that perturbations in either surface conditions and/or rheology during a planet's evolutionary path can significantly affect the outcome (ref). This may come about due to extreme climate evolution (e.g., Venus), impacts, or volatile loss \citep{sandu11}, and require a sophisticated approach to parameterize the coupling between the solid and surface systems.''} \\
{\bf ``Essentially, heat pipe volcanism modulates the dynamics of a stagnant lid. Much of the heat loss due to volcanism in stagnant interludes is, in our models, simply stored internally, then lost during episodic overturn events.''} \\
{\bf ***``We have demonstrated in this work that variations in the magnitude, and ratio, of basal to internal heating, can fundamentally affect the tectonic regime of a planet. We have also demonstrated that the initial conditions of a simulation can play a critical role, particularly near regime boundaries, due to inherent hysteresis in such systems (ref).''} \\
{\bf ***``A planet may begin, and end, its evolution in a stagnant state, and episodic plate tectonics may be the norm for planets beginning their geologic evolution from an initially hot thermal state. Consequently, plate tectonics may in fact be a tectonic phase evolving planets pass through, and there appears to exist a discrete window of opportunity for plate tectonic activity in the evolutionary arc of a planet.''} \\

\subsection{Planetary thermal evolution models with tectonic transitions \citep{oneill20}}
{\it NOTES:} \\
{\bf ***``We find an exponent $\beta$ of $\sim 0.26$ best describes the Nu-Ra relationship for evolving mobile lid systems, and $\beta \sim 0.12$ for stagnant-lid systems.''} \\
{\bf ***``We also find a time-lag between Ra (which primarily depends on mantle temperature) and Nu (normalized surface heat flow) of around 200-300 Myr, suggesting a significant delay between mantle thermal configuration and its surface manifestation.''} \\
{\bf ``Typically thermal evolution models adopt a parameterized apporach to mantle convection, utilising a well-known relationship between convective vigor (the Rayleigh number - Ra), and normalized heat flow, generally cast as the ratio of convective to conductive heat, a term called the Nusselt number (Nu; ref).''} \\
{\bf ``The possibility of tectonic transitions within Earth's history (ref) may significanty perturb Earth's thermal history (refs).''} \\
{\bf ***``Parameterized models integrating backwards, and assuming a value of $\beta \sim 1/3$ typically predict mantle temperatures of $>2000$K within the last 1-3 Gyr. This would imply a primarily molten, very-high temperature mantle throughout the Archean which is not born out by the volcanic record (refs).''} \\
{\bf ``Dubbed 'the Archean thermal catastrophe', this behaviour is primarily due to the efficiency of heat loss under the hard $\beta \sim 1/3$ mobile scaling.''} \\
{\bf Table shows ''heat producing isotopes and their properties, based on Turcotte \& Schubert (2002).''} \\
{\bf ``Our model may evolve into an episodic state, with intermittent bursts of tectonic activity interspersed with periods of relative stasis. The latter state is not a true stagnant lid - the lid in these cases exhibits significant internal deformation, but is not strictly mobile, nor does it exhibit wholesale recycling, except during episodic subduction episodes.''} \\
{\bf ***``The exponent in the effectively ``driven'' regime is quite high (0.41, similar in mobile and ``quasi-stagnant'' models), indicating a level of heat loss well above that expected from boundary layer theory, due to external drivers. It also highlights the signficant affect large impacts may have on planetary cooling histories.''} \\
{\bf ***``Figure 2 results imply a sizeable delay between surface heat loss and mantle temperature equilibration. This lag time is of the order of the internal heating timescale of the mantle (ref).''} \\
{\bf ***``Variable $\beta$ may mitigate some of the issues faced by thermal evolution models assuming a single value. One particular case is the Archean thermal catastrophe, which (refs) is largely an artefact of assuming current plate tectonics are appropriate over all of Earth's history. If we relax this constraint and permit tectonic transitions over time (ref), this catastrophe is largely alleviated.''} \\
{\bf ``Our hybrid model has imposed transitions at major geological boundaries in Earth's history.''} \\
{\bf ``Moore and Webb (2013) showed that for volcanic heat pipe regimes with eruptive efficiencies of $\sim100$\% (i.e., 100\% of melt formed in the mantle is erupted extrusively), volcanic heat transport can dominate conduction through the lithosphere.''} \\
{\bf ``For moderate eruption efficiencies of 20\%, which is appropriate for many intraplate settings (ref), then most volcanic heat goes into heating the crust. Under those circumstances, heat flows will approach traditonal convective scalings.''} \\
{\bf ``We are not explicitly calculating core heat flux here; this implicitly assumes the core is equilibrating to mantle temperatures - a behaviour seen in numerical simulations - but which may lead to issues when the coupling does not hold.''} \\
{\bf ***``We hope to be able to predict diagnostic atmospheric compositons that are a function of tectonic regime. The classic example of this is plate tectonics on Earth, where the CO2-silicate weathering cycle (a function of continual plate-boundary driven topography) removes atmospheric CO2 (ref). Plate subduction is also able to recycle atmospheric CO2 into the interior via buried carbonates. As a result, an Earth-like terrestria exoplanet with an ocean-continent topographic dichotomy, may be expected to be atmospheric CO2 poor.''} \\
{\bf ***``A simple prediction of a young volcanically active planet may be, paradoxically, low atmospheric O2. The counterpoint - a young planet with a rich O2 atmosphere, may be in a stagnant-lid regime.''} \\
{\bf ***``An additional challenge to understanding the tectonics of exoplanets is that these tectonic transitions are sensitive to the thermal evolution of the planet (refs), and thus it is not sufficient to solely consider an exoplanet's physical state (size, density, etc.), its history is critical too.''} \\

\subsection{Mantle redox state drives outgassing chemistry and atmospheric composition of rocky planets \citep{ortenzi20}}
The oxidation state of the mantle is critical for determining the atmospheric composition and thickness, and also affects volatile partitioning during mantle melting. This study focuses on {\bf stagnant-lid planets using an thermal evolution, melt, and volatile outgassing model}. Other factors (planetary mass, thermal state, age) affect the degassing rate, with the most efficient outgassing found within for planets between 2 and 4 $M_{\oplus}$. The authors also show that a reducing scenario leads to a lower atmospheric surface pressure and larger atmospheric thickness, compared to oxidizing. The results have important implications for observations of rocky exoplanet atmospheres, which could be used as a proxy for interior redox state. \\
{\it NOTES:} \\
{\bf ``Gaseous species released from present-day volcanoes on Earth are dominated by H2O and CO2 which are strong greenhouse gases.''} \\
{\bf ``The oxidation state at a given point in time depends on numerous factors such as primordial disk composition, core-mantle processes and loss of atmospheric hydrogen, to name but a few.''} \\
{\bf ***``The Earth's upper mantle is oxidized at present.''} \\
{\bf ``The efficiency and the composition of volcanic outgassing depend on properties such as mass, thermal state, age, tectonic style, and planetary bulk composition.''} \\
{\bf ``We analyse how a planet's mantle redox state affects its outgassed atmospheric composition through the double influences of mantle-melt volatile partitioning and gas chemical speciation.''} \\
{\bf ***``Whether outgassing is dominated by reduced or oxidized species can regulate the atmospheric scale height via the mean molecular mass.''} \\
{\bf ``We consider the influence of the initial volatile content in the magma nad the melt redox state.''} \\
{\bf ``Given the initial volatile contents in the mantle, the corresponding concentrations in the melt are regulated by the partitioning between the mantle and the melt produced.''} \\
{\bf ``For reducing conditions, the dissolution of carbonates from the mantle rocks into the melt is suppressed, and therefore the volatile content in the magma will be dominated by water.''} \\
{\bf ``An oxidizing scenario favours an enrichment in carbonates in the melt.''} \\
{\bf ***``H2 and CO are the most abundant outgassed species in the more reducing states, while in the more oxidized scenarios, the principal gas phases are H2O and CO2.''} \\
{\bf ***``The volatile concentrations in the melt depend on redox state as well. If both these effects are coupled, now in reducing scenarios, H2 is the most outgassing species, while zero-to-little CO is outgassed because the carbonate partitioning in the melt is inhibited. Upon increasing the oxidation state, carbonate starts to become present and the outgassing is dominated by CO2 and H2O.''} \\
{\bf ``Fig. 1: Composition of outgassed volatiles as a function of oxygen fugacity which is show in log values relative to Iron-Wuestite buffer.''} \\
{\bf ***``The atmospheric evolution scenarios considered here are affected by the melt production, temperature, pressure, volatile content,and oxidation state of the system.''} \\
{\bf ``Results suggest a general decrease in atmospheric thickness with increasing planetary mass, hence increased surface gravity in the scale height definition, and larger radial atmospheric extent for reducing conditions.''} \\
{\bf ``Results show highest outgassing values occur for planets of several Earth masses, and oxidized mantles lead to higher atmospheric surface pressure as compared to reduced mantles since the oxidized volatile species have a higher molecular weight than reduced gases (e.g., H2O vs. H2, weight varies by factor of 9).''} \\
{\bf ``In general, the more massive the planet, the later degassing occurs. The outgassing lag for increasing planetary mass is due to the larger internal pressure which produces a higher mantle viscosity. The rheology variation reduces the vigor of mantle convection so that outgassing starts later.''} \\
{\bf ``An increasing planet mass leads to higher initial mantle temperatures and an increased inventory of volatiles and radioactive heat sources. This is reflected in a larger amount of volcanic activity and outgassing for planets with masses up to $2-4 M_{\oplus}$. For planets of masses above this threshold, we can observe a negative trend in volcanic activity.''} \\
{\bf ``Fig. 3: Melt production over time (Gyr) for different planet sizes considering an Earth-like interior structure and composition.''} \\
{\bf ``For higher-mass planets, the pressure at the bottom of the lithosphere leads to a higher melting temperature than for Earth-mass planets, causing reduced or even no outgassing.''} \\
{\bf ***``The degassing trend still reflects that under oxidizing conditions the atmospheric pressure will be larger and mainly composed of H2O and CO2. Under reducing conditions the atmospheric pressure is lower because a part of the H2O content is replaced by H2 and CO outgassing is favoured over CO2 though strongly limited due to the smaller carbonate content in the reduced melt.''} \\
{\bf ***``For both reducing and oxidizing cases, maximum degassing occurs between 2 and 4 Earth masses.''} \\
{\bf ``A higher molecular weight of the atmosphere leads to a shallower atmospheric vertical extent.''} \\
{\bf ``For all the volatile scenarios, the oxidation level leads to strongest differences in atmospheric thicknesses for lower-mass planets.''} \\
{\bf ``In the case of a more reduced mantle, the atmospheric thickness is generally larger compared to the oxidized cases. However, this difference decreases as planetary mass increases. Results also suggest either low or virtually zero outgassing rates for the more massive planets considered here, consistent with the literature.''} \\
{\bf ***``These results (Fig. 5) suggest that evolution of atmospheric composition is strongly linked/coupled with the surface pressure.''} \\
{\bf ***``The chemistry of the atmosphere reflects the reducing or oxidizing nature of the mantle in terms of a CO- versus a CO2-dominated atmosphere.''} \\
{\bf ``In all cases analyzed, the reduced states have larger atmospheric thicknesses than the oxidized scenarios.''} \\
{\bf ***``It is interesting to note that, even though the reducing scenario atmospheres have smaller masses and densities, in our simulations their atmospheric thicknesses are larger than in the oxidized cases. This is due to the different molecular weights of the outgassed species. In a reducing scenario, the volatiles have a smaller molecular weight and this results in a larger atmospheric thickness.''} \\
{\bf ``Our numerical simulations show that the mantle's redox state influences the volatile content of the melt as well as the volatile chemistry during degassing.''} \\
{\bf ``Planetary mass affects the total volume of melt that is produced and hence the volatile depletion of the mantle.''} \\
{\bf ``CO$_3^{2-}$ partitioning in the melt is suppressed in the reducing scenarios causing an enrichment in H2O in the rising melt and favouring outgassing of H2O or H2 over carbon-containing species.''} \\
{\bf ***``Depending on the initial volatile composition, CO, H2, and H2O dominate the outgassing for the reduced mantle case whereas H2O and CO2 dominate for the oxidized case.''} \\
{\bf ``Masses from 2 to 4 Earth masses are the most efficient in depleting the mantle of volatiels and outgassing large volumes of gas.''} \\
{\bf ***``Regarding the loss of hydrogen from early Earth's atmosphere, a rapid steam collapse leading to the formation of a water ocean would likely result in enhanced hydrogen escape from the atmosphere due to the strong EUV radiation of the young Sun.''}  \\
{\bf ***``A reduced interior with strong, long-lasting volcanic activity would be expected to replenish reduced gases to the atmosphere, which might lead to detectable signatures.''} \\
{\bf ***``We showed that the redox state of the mantle is one of the main factors influencing the thickness of the atmosphere and its evolution, translating into a few percent change (all other things being equal) in the observed planetary radius, comparable to detection accuracy of PLATO.''} \\
{\bf ***``The interior redox state of rocky planets could therefore be constrained with PLATO data especially for the thicker, Venus-like terrestrial atmospheres orbiting quieter stars where atmospheric escape is kept to a model level and where condensation of water is unlikely.''} \\

\subsection{Habitability of terrestrial-mass planets in the HZ of M dwarfs - I. H/He-dominated atmospheres \citep{owen16}}
A stellar model (for XUV) and a coupled atmosphere-thermal evolution model is used (which is includes the often-neglected transition from hydrodynamic escape to Jeans escape) to determine whether terrestrial planets around M-dwarfs, which form with a H/He envelope of mass fraction $\sim1$\%, can become habitable. Planets with low core masses at inner edge of HZ can lose enough of the envelope to become habitable, but higher mass core at the outer edge are unlikely to become habitable. The model in this paper better encapsulates atmospheric loss, as opposed to the energy-limited prescription of \citet{luger15}, the latter of which leads to overestimation of mass-loss rates. \\
{\it NOTES:} \\
{\bf ``A large fraction of low-mass, close-in \textit{Kepler} planets are enshrouded by voluminous H/He envelopes, which contain a non-negligible fraction of the planet's mass (*references*), very unlike the low-mass planets in our own Solar system.''} \\
{\bf ``***(Studies indicate) that the dominant structure, instead of completely rocky composition, is a solid core overlaid with a $sim1$\% (by mass) H/He envelope.''***} \\
{\bf ``(Modelling suggests) that the majority of close-in exoplanets were born with H/He envelopes, but about half have subsequently lost these through evaporation (Lopez \& Fortney 2013; Owen \& Wu 2013).''} \\
{\bf ``For a terrestrial mass planet, orbiting either a solar-type star or an M dwarf and located within the classical HZ, an H/He envelope with a mass-fraction of order a percent would preclude habitability, by yielding very high surface temperatures and pressures incompatible with liquid water.''} \\
{\bf ``These planets can become habitable by 1) reducing the envelope to mass fraction $\ll 10^{-3}$ (refs); 2) strip away envelope and replace with tenuous secondary atmosphere (like terrestrial Solar System planets) (refs); or 3) separate H \& He in planets with low initial H/He fraction to leave a habitable He atmosphere (refs).''} \\
{\bf ***``Evaporation may turn a large population of uninhabitable low-mass planets with voluminous H/He envelopes into habitable ones... Efficient evaporation could lead to a plethora of habitable planets in M dwarf systems.''***} \\
{\bf ``The terrestrial planets in our own Solar System evince signatures of having coalesced after the gas disc dispersed (as the radiometrically determined age of the Earth (Manhes et al. 1980) is estimated to be 50-100 Myr younger than that of the Solar System (e.g., Bouvier \& Wadhwa 2010), well after the gas disc would have dispersed... Solid-core planets around M-dwarfs may form in a similar fashion, and be in the classical HZ.''} \\
{\bf ``The inner boundary of the HZ (set by either moist or runaway greenhouse) for such super-Earth planets will be closer to the star than for Earth-mass ones.''} \\
{\bf ***``Rigorous treatment of evaporation, using hydrodynamic models that account for radiation cooling as well as transition to Jeans escape, and including thermal evolution of planet, required to determine actual M-dwarf habitability frequency. A simplistic energy-limited formalism (which assumes escape always hydrodynamic; \citet{luger15}) leads to gross overestimation of evaporative mass-loss rates and an overestimation of planets that can become habitable. OUR RESULTS ARE IN DIRECT CONFLICT WITH THOSE OF \citet{luger15}.''***} \\
{\bf ***``At late times, when the XUV flux declines and the envelope mass fraction drops below $\sim$ 1 percent (i.e., when the planet's radius shrinks rapidly with decreasing envelope mass), the efficiency of hydrodynamic mass-loss decreases significantly (cf. Owen \& Wu 2013); furthermore, the flow transitions to non-hydrodynamic (Jeans escape) at late times. \citet{luger15} neglect these effects, using constant efficiency energy-limited mass-loss prescription.''} \\
{\bf ***``There is no single efficiency value that can mimic the trends in evaporation in our results, especially the hydrodynamic cut-off, which turns out to be extremely important for evaporation and the question of potential habitabilty.''***} \\

\subsection{Atmospheric escape and the evolution of close-in exoplanets \citep{owen19}}
A review of atmospheric escape emphasizing hot Jupiters and super-Earths/mini-Neptunes. Current modelling techniques are discussed (particularly hydrodynamic simulations) and applied to potential observables (although atmospheric escape has already been detected from hot Jupiters). Equations for thermal \& non-thermal escape are included, along with a discussion of ``energy-limited'' escape, which is disfavoured due to its negligent application to even rocky planets, whether or not it is applicable. \\
{\it NOTES:} \\
{\bf ``For lower mass (rocky) planets atmospheric escape drives and controls their evolution, sculpting the exoplanet population we see today.''} \\
{\bf ``Calculations indicate at least one super-Earth/mini-Neptune per FGK star in orbits less than 100 days, and more common around M stars.''} \\
{\bf ``It is still unclear what constitutes the atmospheric composition of super-Earths and mini-Neptunes.''} \\
{\bf ``Density calculations reveal planets consistent with being entirely solid, or low density with an extensive H/He envelope... Broad range of atmospheric compositions possible, including water-rich and H-He rich.''} \\
{\bf ``For H dominated atmospheres, high UV fluxes close to the star dissociates the molecular H2, resulting in upper regions dominated by atomic H... At such high temperatures, upper atmospheres of close-in exoplanets are weakly bound, and the gas can escape the planet's gravity.''} \\
{\bf ***``In the context of close-in exoplanets, atmospheric escape can affect the evolution of a planet's bulk composition.''***} \\
{\bf ``To confirm the presence of atmospheric escape observationally, one needs to show that any escaping atmosphere extends beyond the Roche lobe radius of the planet.''} \\
{\bf ``The extreme heating that close-in planets receive implies thermal escape processes dominate.''} \\
{\bf ``The role of the sonic point is critical (equation defined in paper). As the flow beyond the sonic point is travelling super-sonically, it is unable to propagate any information upstream.''} \\
{\bf ***``In Watson et al. (1981)'s parlance, the mass-flux is `energy-limited' by the upstream conductive heat-flux to balance the adiabatic cooling of the expanding gas as it approaches the radius where radiation is absorbed.''***} \\
{\bf ***``Equation for energy-limited loss contains an efficiency factor included to incorporate (lack of) knowledge of energy gain \& loss processes; however, this presents several issues. 1) Wavelength range of incoming flux unclear; 2) Unclear what efficiency factor to use; 3) No way to determine if hydrodynamic escape or Jeans escape is appropriate.''***} \\
{\bf ``One of the least explored aspects of atmospheric escape from close-in exoplanets is the importance of magnetic fields. The extreme high-energy irradiation of the outflows means it is highly ionized, and hence coupled to any planetary magnetic field.''}
{\bf ***``The effect of planetary magnetic fields around lower mass super-Earths/mini-Neptunes remains unexplored.''***} \\
{\bf ``A planet's atmosphere starts off with high entropy and is extended and then cools by contracting under gravity, becoming denser and smaller. This cooling means that planets are largest when they are young, and therefore able to absorb a more significant fraction of the star's high-energy luminosity resulting in higher mass-loss rates.''} \\
{\bf ``Stars emit a higher fraction of their total luminosity non-thermally at high energies when they are young.''} \\
{\bf ***``At young ages, X-ray luminosity is roughly constant fraction of bolometric luminosity of the star and is considered ``saturated'' at some value. This ``saturation timescale'' typically lasts 100 Myr, and after the X-ray luminosity decays with time... Atmospheric escape is most important when the high-energy flux is saturated.''**} \\
{\bf ``For lower mass stars the saturation timescale was longer and the saturation value was typically higher... At fixed equilibrium temperature, atmospheric escape is significantly stronger around lower mass stars.''} \\
{\bf ***''The loss of water is critical to studies about habitability and important around lower-mas stars which have long pre-MS lifetimes resulting in much higher fluxes for young planets. Much of the work on this topic performed to date has typically invoked the energy-limited models, with limited consideration as to whether it is actually applicable.''***} \\
{\bf ``The role of heavy elements and outgassed secondary atmospheres need to be folded into our current understanding of exoplanet evolution.''} \\

\subsection{Hydrogen dominated atmospheres on terrestrial mass planets: evidence, origin and evolution \citep{owen20}}

\subsection{Modeling of the solubility of a one-component H2O or CO2 fluid in silicate liquids \citep{papale97}}
The solubility of either H2O or CO2 in 10-component natural silicate liquid (in equilibrium between liquid magma and gas phase) is modelled and compared with experimental solubilities, with which it reasonably agrees. This is important, since even a small amount of volatiles can significantly alter the chemical evolution of silicate liquids, and can lead to magmatic eruptions. \\
{\it NOTES:} \\
{\bf ``The model correctly predicts: (1) the progressive decrease of water solubility with decreasing Na2O (and increasing K2O) content (alkali effect, ref); (2) a minor effect of the variations of SiO2 content with respect to the varitations of the alkali species; (3) a larger compositional dependence of water solubility at larger pressure (see Fig. 4 for dissolved H2O vs. pressure plot).''} \\
{\bf ``According to the reported data, the CO2 solubility increases with temperature at all conditions, and the rate of increase is larger at larger pressure and temperature.''} \\
{\bf ``In this model, the water solubility is expessed by means of a regular and isometric model (symmetric binary interaction coefficients independent of pressure and temperature), whereas the carbon dioxide solubility is expressed by means of a regular and non-isometric mixing model (symmetric binary coefficients depend on pressure but not temperature).''} \\
{\bf ``The ratio of hydroxyl groups to molecular water is principally a function of the total dissolved water content.''} \\
{\bf ``Since the total dissolved H2O content is in itself a function of the liquid composition, the water speciation depends also on composition at constant temperature and pressure.''} \\
{\bf ``The model results suggest a Henrian behaviour within a few tens of MPa for CO2 (ref), and a linearity between fugacity and square of the mole fraction of dissolved volatile up to several tens or hundreds of MPa, depending on the considered volatile and on the temperature and liquid composition (refs).''} \\

\subsection{Modeling of the solubility of a two-component H2O + CO2 fluid in silicate liquids \citep{papale99}}
An expansion of the solubility model of \citep{papale97}, now including {\it both} H2O + CO2 in the silicate fluid. The results indicate that the multicomponent natures of fluid should not be neglected, since a small amount of CO2 can significantly increase volatile saturation pressure. The results also indicate a transition from Henrian to non-Henrian behaviour as pressure increases, and more efficient H2O enrichment in the fluid phase as pressure decreases in an open system (compared to a closed system). The model results again agree with experimental results for the mutual effects of the presence of both H2O and CO2. \\
{\it NOTES:} \\
{\bf ``Compared to the single-component model, the amount of each volatile dissolved in the liquid is not determined uniquely by specifying the P-T composition conditions (as in the case of a pure H2O or pure CO2 fluid phase), but also depends on the total amounts of H2O and CO2 in the system (Eqn. 8).''} \\
{\bf ``Both the present model and the simpler model of (ref) suggest that most of the available H2O is retained in the liquid magma up to a very shallow depth (Fig), and that a significant increase in the amount of the released gas only occurs when the magma approaches the surface (ref).''} \\
{\bf ``Note that the progressive addition of one volatile to the system results in a decrease in the dissolved amount of the other one.''} \\
{\bf ``Larger quantities of CO2 bring about larger increases in the saturation pressure.''} \\
{\bf ``The addition of 0.1 wt\% H2O has a very small effect on the amount of dissolved CO2, whereas larger quantities significantly reduce it (Fig).''} \\
{\bf ``The predicted amount of H2O in the liquid is enhanced and that of CO2 is reducd under open- comapred with closed-system conditions.''} \\
{\bf ``In both closed- and open-system conditions, the fluid is predicted to become progressively enriched in H2O as the pressure decreases. However, this H2O enrichment is more prominent under open-system conditions.''} \\
{\bf ``Experimental data on H2O+CO2 dissolution in silicate liquids indicates that with increasing pressure, the equilibrium condition progressively moves from Henrian to strongly non-Henrian.''} \\
{\bf ``The predicted behaviour of H2O and CO2 is not strictly Henrian at any considered pressure, but at low pressures or from some given range of dissolved H2O and CO2, it may effectively approximate as Henrian behaviour.''} \\
{\bf ``When two volatile components are present, a dilution effect similar to that occurring in the fluid phase also occurs in the liquid phase, and its importance increases with increasing pressure (i.e., with increasing amount of dissolved volatiles).''} \\
{\bf ``The predicted effect of the addition of water to a pure CO2 fluid at high pressure is not simply that of reducing the dissolved amount of CO2, at some pressures actually leading to an increase in the amount of dissolved CO2.''} \\

\subsection{HAZMAT. IV. Flares and superflares on young M stars in the far ultraviolet \citep{parke18}}

\subsection{Predicting the extreme ultraviolet radiation environment of exoplanets around low-mass stars: the TRAPPIST-1 system \citep{peacock19a}}

\subsection{Predicting the extreme ultraviolet radiation environment of exoplanets around low-mass stars: GJ 832, GJ 176, and GJ 436 \citep{peacock19b}}

\subsection{Hydrous mantle transition zone indicated by ringwoodite included within diamond \citep{pearson14}}
Mantle transition zone (MTZ) likely made up of wadsleyite and ringwoodite, which are expected to have a high water storage capacity ($\sim$ 2.5\% water by weight). This study looks at ringwoodite in a diamond using various techniques and provides the first direct evidence that the MTZ is hydrous, as the ringwoodite is roughly 1 wt \% water.

\subsection{A regime diagram of mobile lid convection with plate-like behaviour \citep{petersen15}}

\subsection{Principles of planetary climate (textbook) \citep{pierrehumbert10}}

\subsection{The most common habitable planets II - salty oceans in low mass habitable planets and global climate evolution \citep{pinotti20}}

\subsection{Isotope composition and volume of Earth's early oceans \citep{pope12}}

\subsection{The frequency of giant impacts on Earth-like worlds \citep{quintana16}}
{\bf Suggested by Billy Quarles at AAS: "If giant planets are indeed more rare for M-dwarfs, then the bombardment phase for terrestrial planets could extend to $>$1 Gyr.  I would expect that you could include it as a random component on top of the XUV.  But the impacts may also disrupt the ability of the mantle to sequester the water.  Hopefully that clarifies my question."}

\subsection{A volcanic hydrogen habitable zone \citep{ramirez17}}

\subsection{The surface UV environment on planets orbiting M dwarfs: implications for prebiotic chemistry and the need for experimental follow-up \citep{ranjan17}}
The top-of-atmosphere UV radiation from a variety of M dwarfs is calculated to determine whether UV-controlled prebiotic abiogenesis reactions could originate life (specifically RNA) on M-dwarf planets. The authors find that these planets have access to 100-1000 times less UV than the young Earth, but it is unclear whether UV-controlled abiogenesis could occur on M dwarf planets. An alternative solution to the UV-paucity is transient flares, which could provide the necessary UV radiation for abiogenesis. \\
{\it NOTES:} \\
{\bf ***``Potentially habitable planets orbiting M dwarfs are of intense astrobiological interest because they are the only rocky worlds accessible to biosignature search over the next 10+ years because of a confluence of observational effects.''***} \\
{\bf ***``Recent experimental and theoretical work suggests that UV light may have played a key role in the origin of life on Earth, especially the origin of RNA.''***} \\
{\bf ``We couple radiative transfer models to observe M-dwarf spectra to determine the UV environment on prebiotic Earth-analog planets orbiting M dwarfs.''} \\
{\bf ***``We find that M-dwarf planets have access to 100-1000 times less bioactive UV fluence than the young Earth.''***} \\
{\bf ***`` It is unclear whether UV-sensitive prebiotic chemistry that may have been important to abiogenesis, such as the only known prebiotically plausible pathways for pyrimidine ribonucleotide synthesis, could function on M-dwarf planets.''***} \\
{\bf ***``If steady-state M-dwarf UV output is insufficient to power these pathways, transient elevated UV irradiation due to flares may suffice.''***} \\
{\bf ``Perhaps most importantly, due to a confluence of observational effects, M-dwarf terrestrial planets (and \textit{only} M-dwarf terrestrial planets) will be accessible to atmospheric characterization and hence biosignature search with the flagship telescopes that are due to come online in the next decade (refs).''} \\
{\bf ``Many studies on M-dwarf planet habitability, but far fewer investigations have been conducted as to the favorability of M-dwarf planets for abiogenesis (the origin of life) i.e., whether life as we know it could emerge on these worlds.''} \\
{\bf ``The paucity of work on this question is due to limitations in our understanding of the origin of life. For example, much of the work on exoplanet habitability has been motivated by the biological fact that life as we know it today requries liquid water, and has focused on understanding the availability of this key requirement for life in different planetary environments (e.g., \citealt{kasting93}).''} \\
{\bf ***``UV light plays a key role in recently proposed prebiotic pathways. While UV light can destroy nascent biomolecules (ref), it can also power synthetic prebiotic photochemistry.''***} \\
{\bf ``Measurements of nucleobase photostability suggest that the biogenic nucleobases (the informational components of the RNA and DNA monomers) are exceptionally stable to UV irradiation compared to structurally similar molecules with comparable thermal properties, suggesting that they evolved in a UV-rich environment (refs).''} \\
{\bf ``This scenario is consistent with our understanding of conditions on prebiotic Earth: UV light is thought to have been abundant on young Earth as a result of the absence of a biogenic ozone layer (refs).''} \\
{\bf ***``It is important to constrain the UV environment on the surface of planets orbiting M dwarfs to understand if UV-sensitive prebiotic chemistry pathways that could have lead to the origin of life on Earth could function on such worlds. This represents a new criterion for planetary inhabitability.''***} \\
{\bf ``Our model includes the effects of absorption and multiple scattering from the surface and from atmospheric gases, and uses recently measured high-quality UV observations of M-dwarf stars (ref) to provide realistic top-of-atmosphere (TOA) stellar irradiation spectra.''} \\
{\bf ``Our work suggests the need for specific experimental tests that must be done to determine whether the UV-sensitive prebiotic chemistry that may have powered the origin of life on Earth could function on planets orbiting M-dwarf stars.''} \\
{\bf ``Most habitability studies of M dwarfs treat UV radiation as a negative for habitability (refs), motivated by the observation that UV radiation has deleterious effects on modern life (ref).''} \\
{\bf ``Works focusing on surface life protection against UV radiation show that UV irradiance at the surface of modern-Earth-analog M-dwarf planets should be suppressed to levels below those of modern Earth itself, because of lower M-dwarf near-UV (NUV) output and because of favorable ozone-generating atmospheric photochemistry.''} \\
{\bf ``The UV levels on M-dwarf planets analogous to young Earth (i.e., with anoxic atmospheres) should be lower than on young Earth itself, because of the lower M-dwarf NUV emission.''} \\
{\bf ***``These works in aggregate conclude that UV radiation environment on terrestrial planets orbitng M-dwarf stars should be clement for life, with potential caveats for very active stars.''***} \\
{\bf ***``The above presumes planetary atmospheres can be retained despite processes like XUV-powered atmospheric escape. Results remain unclear whether the atmospheres of Earth-mass planets orbiting M dwarfs should be stable to escape, particularly in light of the M-dwarf enhanced fractional XUV emission relative to solar-type stars.''***} \\
{\bf ``We consider in addition the possible positive roles of UV irradiation, motivated by recent experimental advances in prebiotic chemistry that require UV light.''} \\
{\bf ``Scalo et al.(2007) hypothesize that highly variable M-dwarf UV emission could drive variations in mutation rates on orbiting planets, which might enhance the rate of evolution.''} \\
{\bf ``Buccino et al.(2007) argue based on the ``Principle of Mediocrity'' (that Earth's properties should be typical of inhabited planets) that habitable planets should receive stellar irradiance similar to Archean Earth in order to power potential prebiotic chemistry, and use it to suggest that life cannot arise on planets orbiting inactive low-UV M dwarfs because in order to receive Earth-like UV instellation, planets will need to orbit within the inner edge of the HZ.''} \\
{\bf ***``Buccino et al.(2007) suggest that moderately active M dwarfs may consequently be better candidates for habitability because of enhanced UV output during flares.''***} \\
{\bf ``Our work is distinct from these works in coupling the M-dwarf UV environment to specific prebiotic photoreactions through their action spectra.''} \\
{\bf ``In brief, for our model, we calculated the attentuation of empirically measured M-dwarf UV emission by the atmosphere to compute the spectral surface radiance, and coupled the spectral radiance to prebiotically relevant action spectra to evaluate the implications of prebiotic chemistry.''} \\
{\bf ***``We take the surface albedo to be 0.2, a representative value for rocky planets and consistent with past 1D modelling for Earth and Mars (refs).''***} \\
{\bf ``Following (ref), we take the planetary atmosphere to be cloud-free, with a surface pressure of 1 bar composed to 0.9 bar N2 and 0.1 bar CO2. Such high levels of CO2 are expected for abiotic Earth-analogs because CO2 is emitted in bulk from volcanos for planets with Earth-like mantle oxidation states.''} \\
{\bf ``The absorption of most major atmospheric gases (e.g., N2, H2O, CH4) is degenerate with this CO2 absorption, meaning that surface UV is insensitive to their abundances (ref).''} \\
{\bf ``The absorptions of both CO2 and H2O clouds are degenerate with gaseous CO2 absorption, and for Earth-like global mean cloud optical depths of 4-10 (ref), the surface UV environment is insensitive to the presence of clouds (ref).''} \\
{\bf ***``Because the thermal emission of the atmosphere is negligible at UV wavelengths, the UV surface radiance is insensitive to the atmospheric temperature/pressure profile.''***} \\
{\bf ***``M dwarfs are known for their frequent and energetic flares (ref).''***} \\
{\bf ``Results derived our use of the largest known M dwarf flare may be interpreted as a limiting case on the impact of M-dwarf flares on the surface UV environment.''} \\
{\bf ***``Table 1 provides scaled distances and also summarizes other relevant properties of the M dwarfs in our sample.''***} \\
{\bf ``We use action spectra corresponding to simple prebiotically relevant photoprocesses to measure the impact of UV light on nascent life.''} \\
{\bf ``Focus on DNA damage is inappropriate for prebiotic chemistry becasue (1) DNA is not thought to have been the primordial biomolecule, (2) modern organisms have evolved sophisticated methods to deal with environmental stress, including UV exposure, that would not have been available to the first life, and (3) this approach ignores the role of UV light a beneficial for abiogenesis.''} \\
{\bf ``If a UV environment is destructive for UMP (an RNA monomer), it should be destructive for the other RNA monomers, and hence for abiogenesis in the RNA world hypothesis, as well.''} \\
{\bf ***``Young-Earth-analog planets orbiting M-dwarfs are exposed to far less UV radiation than those orbiting Sun-like stars because of the cooler photosphere and hence lower NUV emission of M dwarfs.''***} \\
{\bf ***``At short wavelengths, M dwarfs emit proportionately more radiation than the young Sun, but fluence at these wavelengths is robustly blocked by a range of atmospheric absorbers, including CO2 and H2O, which shield out $<200$ nm radiation.''***} \\
{\bf ***``M-dwarf planets, so long as they can retain their atmospheres, are low-UV environments.''} \\
{\bf ***``Except for the most active M dwarf, all dose rates are suppressed by $>2$ orders of magnitude relative to the young Earth. Putative UV-dependent prebiotic chemistry will proceed at rates 2-4 orders of magnitude slower on planets orbiting non-active M dwarfs compared to the young Earth. Active stars like AD Leo emit more UV radiation, so bioactive fluence on planets orbiting such active stars will only be suppressed by 1-2 orders of magnitude.''***} \\
{\bf ***``We conclude that M dwarf planets are comparable to early Earth in terms of how much their UV environments favor stressor processes over eustressor (beneficial) processes.''***} \\
{\bf ***``While in quiescence prebiotically relevant reaction rates are suppressed by 1-2 orders of magnitude relative to the young Earth, in flare the reaction rates are enhanced, by up to 1 order of magnitude relative to the young Earth.''} \\
{\bf ``Prebiotic chemistry on planets orbiting M dwarfs has access to orders of magnitude less UV radiation than prebiotic chemistry on planets orbiting solar-type stars because the NUV emission by the cooler M dwarfs is lower.''} \\
{\bf ***``UV light is a key requirement of several proposed prebiotic pathways, including the only known pathways for prebiotic syntheses of ribonucleotides (refs).''***} \\
{\bf ***``It is unclear whether UV-dependent prebiotic pathways that may have been important to the origin of life on Earth can proceed in the low-UV environment on planets orbiting M dwarfs.''***} \\
{\bf ``If UV-dependent pathways like the RNA monomer synthesis pathways of (ref) were a rate-limiting step in the origin of life on Earth, then lower M-dwarf NUV radiation could delay the origin of life on planets orbiting M stars by orders of magnitude.''} \\
{\bf ``In the worst-case scenario, the Universe might be too young for M-dwarf-orbiting life to have evolved at all.''} \\
{\bf ***``\citet{luger15} show that terrestrial planets with significant initial water inventories orbiting in the current HZs of MS M dwarfs may have undergone a runaway greenhouse and suffered massive loss of water due to enhanced M-dwarf luminosity in their extended pre-MS phase, which could have generated high levels of UV-shielding O2 and O3 if not balanced by loss to the mantle and crust.''***} 
{\bf ***``The greater NUV/FUV ratio of M dwarfs means that O3 production rates are higher and dissociation rates are lower.''***} \\
{\bf ***``\citet{wordsworth14} showed that water-rich planets with low inventories of non-condensible gases like N2 may also abiotically oxygenate.''***} \\
{\bf ***``Planets orbiting M dwarfs with inefficient sinks of O2 and CO, with desiccated CO2-rich atmospheres, or lacking non-condensible background gases, may photochemically produce atmospheric UV shields like O3 that can further block the already low M-dwarf UV from the planet surface.''***} \\
{\bf ``The origin of life may not depend on UV at all. Indeed, UV light would be altogether absent in the deep-sea hydrothermal vent hypothesis for the origin of life (ref).''} \\
{\bf ``The energetic nature of UV light means that it is capable of directly affecting or altering molecular electronic structure, permitting it to effect irreversible changes in the entropic states of molecular systems.''} \\
{\bf ***``It is difficult to avoid the conclusion that a paucity of UV light might pose a problem for abiogenesis scenarios on M-dwarf planets. Conversely, if life is found on an M-dwarf planet, it might imply a pathway to the origin of life that is very different from what seems to have played out on Earth.''***} \\
{\bf ***``We conclude that atmospheric stripping cannot solve the UV-paucity problem for prebiotic chemistry.''***} \\
{\bf ``Young M dwarfs, especially pre-MS M dwarfs, emit a larger fraction of their bolometric luminosity as NUV radiation than in their MS phase \citep{shkolnik14}.''} \\
{\bf ``Planets in the MS HZs of such stars are liable to be in a runaway greenhouse state during their pre-MS evolution, with temperatures globally above the boiling point of water and possibly as high as $\gtrsim 1000$ K (refs).''} \\
{\bf ``Planets orbiting in the HZs of late-type (low mass) pre-MS M dwarfs remain of special interest from the perspective of UV-sensitive prebiotic chemistry.''} \\
{\bf ``One might imagine a scenario whereby photosensitive prebiotic chemistry proceeds during the high-UV flares and ceases during stellar quiescence, providing an activity-powered analog to the terrestrial day/night cycle that would be absent on tidally-locked HZ M-dwarf planets.''} \\
{\bf ``Our work suggests a critical need for laboratory studies to determine whether putative UV-dependent prebiotic pathways (refs) could function at the $10-1000 \times$ lower-UV irradiation accessible on planets orbiting M dwarfs compared to the young Earth.''} \\
{\bf ``If flare frequencies $\sim 5 \times$ lower than that of AD Leo are found to be sufficient to power UV-dependent prebiotic chemistry, then flare instellation could solve the potential UV-paucity problem for Proxima Centauri b.''} \\
{\bf ***``We find that the UV surface environment on M-dwarf planets is chiefly different from planets orbiting Sun-like stars in that they have access to orders of magnitude less prebiotically useful NUV radiation because of the lower emission of M dwarfs at these wavelengths.''***} \\
{\bf ***``Planets orbiting in the transient HZs of pre-MS late-type M dwarfs should experience more NUV irradiation, but only transiently, and not enough to close the deficit with planets orbiting Sun-like stars.***} \\
{\bf ***``This raises uncertainty over whether the UV-dependent prebiotic pathways that may have led to the origin of life on Earth could function on planets orbiting M dwarfs.''***} \\
{\bf ``Even if the pathways proceed, their reaction rates will likely be orders of magnitude lower than for planets around Sun-like stars, potentially slowing abiogenesis.''} \\
{\bf ``If laboratory studies reveal that near-solar levels of NUV fluence are required to move forward UV-sensitive prebiotic chemistry, they will raise questions regarding the prospects for the emergence of life on M-dwarf planets.''} \\
{\bf ``Interestingly, in this case planets orbiting active M dwarfs may be more compelling candidates for abiogenesis scenarios as a result of both the higher quiescent emission of such stars and the frequent flares from such stars, which will periodically illuminate the planet with elevated levels of UV that may power prebiotic photochemistry.''} \\
{\bf ``Theoretical studies are required to ensure that enhanced UV and particle fluxes from such flares would not also strip the planetary atmosphere, obviating this solution.''} \\

\subsection{Making other earths: dynamical simulations of terrestrial planet formation and water delivery \citep{raymond04}}
Terrestrial planet formation, within the HZ of their parent stars, is simulated, with a focus on water delivery. The model begins with a relatively low density protoplanetary disk, and forms planets with varying parameters such as Jupiter's mass, eccentricity, semi-major axis, and time of formation, along with protoplanetary disk density and the location of the snow line. A range of masses of planets form, from smaller than Earth to super-Earths, with dry to water world conditions. \\
{\it NOTES:} \\
{\bf ***``(Simulated planets) range from dry worlds to `water worlds' with $100$ oceans of water, and vary in mass between $0.23 M_{\oplus}$ and $3.85 M_{\oplus}$.''***} \\
{\bf ``Until recently, a leading hypothesis for the origin of Earth's water was the `late veneer' scenario, in which the Earth formed primarily from local material, and acquired its water at later times from a large number of cometary impacts. These impacts resulted in a hot water vapor atmosphere which condensed into oceans (Matsui \& Abe, 1986).''} \\
{\bf ``(D/H ratios) imply that at most 10\% of Earth's water came from a cometary source (Morbidelli et al. 2000)... proposed instead that the bulk of Earth's water may have come from the asteroid belt in the form of planetary embryos.''} \\
{\bf ``In the Morbidelli model, the Earth accreted water since its formation, in the form of an early bombardment of asteroids and comets, a few large `wet' planetary embryos, and continual impacts of small bodies over long timescales.''} \\
{\bf ``Our results imply the existence of a huge variety of planetary systems in our galaxy, as planets form from disks around stars with a variety of masses and metallicities.''} \\
{\bf ``Figure 9: Histogram of the water content of 45 planets with 0.8 AU $< a <$ 1.5 AU which formed in 44 simulations: Mars-like, Earth-like, water-rich, and water worlds.''} \\
{\bf ***``With the uncertainty in both volatile retention during the formation process as well as the snow line's position, we conclude that Earth-like planets are likely to be relatively common in the galaxy.''***} \\
{\bf ``The spectroscopic signature of water vapor in an atmosphere is such as to be an insensitive diagnostic of water abundance -- other than indicating that indeed surface water is present and surface temperatures warm enough for significant water vapor.''} \\
{\bf ``Super embryos serve as a small dynamical barrier for inward-diffusing, volatile-rich planetesimals, but can also serve as the accretion seed for massive terrestrial planets with high water contents.''} \\

\subsection{Building the terrestrial planets: constrained accretion in the inner Solar System \citep{raymond09}}
A suite of high-resolution simulations are used to model the formation of the terrestrial planets in the Solar System, although neglecting Mercury. The only varied parameter is the configuration of Jupiter and Saturn at early times (both semi-major axis and eccentricity), some of which are consistent with the Nice model. No single simulation achieves all constraints, such as sufficient water delivery to Earth; simulations consistent with the Nice model cannot produce Mars' small size. \\
{\it NOTES:} \\
{\bf ``Several groups have succeeded in delivering water to Earth from hydrated asteroidal material, following the model of Morbidelli et al. (2000), see also Raymond et al. (2004, 2006, 2007) and O'Brien et al. (2006).''} \\
{\bf ``Terrestrial accretion lasts for $\sim10^8$ years, far longer than the few Myr lifetimes of the gaseous component of protoplanetary disks (refs).''} \\
{\bf ``Observations suggest that inner, dust-free cavities exist in many disks around young stars with varying radii, from $<0.1$ to $\sim 1$ AU (refs). If the Solar Nebula had a large inner cavity then the inner boundary for the planetisimal disk could have been at roughly Venus' orbital distance such that the radial compression of MVEM is a result of accretion in a radially compressed planetesimal disk.''} \\
{\bf ``Four of our cases based on the Nice model assume that delayed, planetesimal scattering-driven migration spread out the giant planet system, and that the 2:1 resonance crossing of Jupiter and Saturn triggered the late heavy bombardment (Gomes et al. 2005; see also Strom et al. 2005).''} \\
{\bf ``Our results clearly favour simulations that assume the Nice model is incorrect, mainly because of their ability to form Mars analogs.''} \\
{\bf ***``In the favoured model, Earth's water was delivered in part from hydrated asteroidal material, but mainly from adsorption of small silicate grains (Muralidharan et al. 2008), cometary impacts (Owen \& Bar-Nun, 1995) or oxidation of a H-rich primitive atmosphere (Ikoma \& Genda 2006). The late heavy bombardment of the terrestrial planets can then be explained by the delayed destabilization of planet V (Chambers 2007).''***} \\
{\bf ``Despite the limited resolutionand the difficulty with dynamical friction, all of our constraints were met in some simulations, including cases with AMD (angular momentum deficit) values lower than the MVEM values.''} \\

\subsection{Heart transport efficiency for stagnant lid convection with dislocation viscosity: application to Mars and Venus \citep{reese98}}

\subsection{Evolution of the solar activity over time and effects on planetary atmospheres I. high-energy irradiances (1-1700 Angstroms) \citep{ribas05}}
Observations of the Sun and solar analogs are used to model the XUV flux from solar-type stars (across different bins; see Table 5), including saturation timescales. Results indicate the young MS Sun may have been 100-1000 times more active in the XUV than today. {\bf Models of solar XUV are included within, which were later used by \citet{luger15}.} \\
{\it NOTES:} \\
{\bf ***``For Earth, most XUV radiation absorbed in thermosphere, at an altitude around 90 km...XUV ionizing radiation raises the temperatures of planetary thermospheres and affects their vertical temperature profiles and energy transport mechanisms.''***} \\

\subsection{Evolution of the solar activity over time and effects on planetary atmospheres ii. Kappa Ceti, an analog of the Sun when life arose on Earth \citep{ribas10}}

\subsection{The habitability of Proxima Centauri b I. Irradiation, rotation and volatile inventory from formation to the present \citep{ribas16}}
The potential habitability of Proxima Centauri b is assessed, from its formation to present. Many factors are investigated, including initial water inventory, irradiation \& XUV history of the host star leading to atmospheric loss, and rotation of the planet. The authors conclude that Proxima b is a viable habitable candidate planet, as it was unlikely to have lost more than 1 Earth ocean worth of hydrogen before reaching the HZ. \\
{\it NOTES:} \\
{\bf ***``We estimate the current high-energy irradiance on Proxima b and show that the planet currently receives 30 times more extreme-UV radiation than Earth and 250 times more X-rays.''***} \\
{\bf ``We use our spectral energy distribution to compute the hydrogen loss from the planet with an improved energy-limited escape formalism.''} \\
{\bf ***``Despite the high level of stellar activity we find that Proxima b is likely to have lost less than Earth's ocean's worth of hydrogen (EO$_H$) before it reached the HZ 100-200 Myr after its formation. (biggest uncertainty initial water budget)''***} \\
{\bf ``We conclude that Proxima b is a viable candidate habitable planet.''} \\
{\bf ``The phase of strong irradiation early in its life, when the HZ swept inward, has the potential to induce water loss, with the potential for Proximab entering the HZ as dry as present-day Venus.''} \\
{\bf ``While Earth's spin period is much shorter than its orbital period, Proxima b's rotation has been affected by tidal interactions with its host star. The planet is likely in one of two resonant spin states.''} \\
{\bf ***``To estimate the atmospheric loss rates, we need to know the spectrum of Proxima at wavelengths that photolyse water (FUV, H Ly$\alpha$) and heat the upper atmosphere, powering the escape (soft X-ray and extreme-UV, hereafter EUV), as well as its stellar wind properties.''***} \\
{\bf ***``To compute the wate rloss, we use an improved energy limited escape formalism (refs) based on hydrodynamical simulations (ref). This model was used by \citep{bolmont16} to estimate the water loss from planets around brown dwarfs and the planets of TRAPPIST-1 (ref).''***} \\
{\bf ``The water abundance of Earth's interior is not well known. Estimates for the amount of water locked in the mantle range between $\lesssim 0.3$ and 10 oceans (refs). The core is not thought to contain a significant amount of hydrogen (ref).''} \\
{\bf ``In the Solar System, the division between dry inner material and more distant hydrated bodies is located in the asteroid belt, at $\sim 2.7$ AU, which roughly divides S-types and C-types (refs).''} \\
{\bf ***``Earth's D/H and $^15$N/$^14$N ratios are a match to carbonaceous chondrite meteorites (ref) associated with C-type asteroids in the main outer belt. Primordial C-type bodies are the leading candidate for Earth's water supply.''***} \\
{\bf ***``In the newer Grand Tack model, water was delivered to Earth by C-type material, but those bodies actually condensed much farther from the Sun and were both implanted into the asteroid belt and scattered to the terrestrial planet-forming region during Jupiter's orbital migration (refs).''***} \\
{\bf ``While Proxima's HZ is much closer-in than the Sun's, its snow line was likely located at a similar distance. Water-rich material thus had a far greater dynamical path to travel to reach Proxima b and, as expected, water delivery is less efficient at large dynamical separations \citep{raymond04}.''} \\
{\bf ``But the snow line moves inward in time (refs).''} \\
{\bf ``The Solar System's snow line may have spent time as close in as 1 AU (refs). Yet the solar system interior to 2.7 AU is extremely dry. One explanation for this apparent contradiction is that the inward drift of water-rich bodies was blocked when Jupiter formed (ref).''} \\
{\bf ***``One can imagine that in systems without a Jupiter the situation might be quite different. In principle, if the snow line swept all the way into the HZ, it may have snowed on the planet late in the disk's lifetime.''***} \\
{\bf ***``The impacts involved in building Proxima b were more energetic than those that built the Earth. The collision speed between two objects in orbit scales with the local velocity dispersion (as well as the two bodies' mutual escape speed.''***} \\
{\bf ``For Proxima, the impacts that built planets in the HZ would have been a few times more energetic on average than those that built the Earth. This may have led to significant loss of the planet's atmosphere and putative oceans (ref).''} \\
{\bf ***``Proxima b likely took less time than Earth to grow. Simple scaling laws show that planets in the HZs of $\sim 0.1 M_{\odot}$ stars form in 0.1 to a few Myr (refs).''***} \\
{\bf ***``The dissipation of the gas disk after a few Myr (refs) may have triggered a final but short-lived phase of giant collisions. Although it remains to be demonstrated quantitatively, the concentration of impact energy in a much shorter time than Earth may have contributed to increased water loss.''***} \\
{\bf ***``Simulations have shown that in-situ growth can indeed deliver water-rich material into the HZs of low-mass stars (refs).''***} \\
{\bf ``The water-depleting effects discussed above are expected to increase in importance for the lowest-mass stars, so the retention of water remains in question.''} \\
{\bf ``Bodies more massive than $\sim 0.1-1 M_{\oplus}$ are subject to migration from tidal interactions with the protoplanetary disk (refs); migration is a plausible origin for Proxima b.''} \\
{\bf ***``If Proxima b or its building blocks formed much farther out and migrated inward, then their compositions may not reflect the local conditions of the disk. Rather, they would be extremely water-rich (refs).''***} \\
{\bf ``Migration would not have affected the planet's irradiation or tidal evolution, just its initial water budget.''} \\
{\bf (Several other possible mechanisms are mentioned in this section, including if the protoplanetary disk underwent photoevaporation; the impact of $^26$Al; and late bombardment of water-rich material.)} \\
{\bf ``We consider a broad range of initial water contents for Proxima b.''} \\
{\bf ***``The XUV flux rnage includes emissions from the X-rays (starting at $\sim0.5$ nm-2.5 keV) out to the far-UV (FUV) just short of the H Ly$\alpha$ line.''***} \\
{\bf ``For a nearby planet, both the so-called quiescent activity and flare rate of Proxima are relevant.''} \\
{\bf ***``It is interesting to note that Proxima's X-ray flux is quite similar to the solar one, which is between $\log L_X = 26.4$ and 27.7 erg/s, corresponding to solar minimum and maximum, respectively.''***} \\
{\bf ``Our analysis is based on the assumption of a linear response of the atmosphere to different amounts of XUV radiation, which is certainly an oversimplification, but should be adequate for an approximate evaluation of volatile loss processes.''} \\
{\bf ***There is an extensive discussion of how the XUV flux is estimated for Proxima Cen, including estimation of the contribution from flares from a) observations and b) using similarity with the Sun.***} \\
{\bf ``The total integrated flux today that is representative of the time-averaged high-energy radiation on the atmosphere of Proxima b is of 307 erg s$^{-1}$ cm$^{-2}$ between 0.6 and 118 nm.''} \\
{\bf ***``Integration in the relevant wavelength interval yields a total XUV flux at Earth of 5.1 erg s$^{-1}$ cm$^{-2}$. Thus, Proxima b receives 60 times more XUV flux than the current Earth, which we refer to as XUV$_{\oplus}$. Also, the far-UV flux on Proxima b between 118 nm and 170 nm is 147 erg s$^{-1}$ cm$^{-2}$, which is about 10 times higher than the flux received by Earth, FUV$_{\oplus}$. The H Ly$\alpha$ flux alone received by Proxima b is 15 times stronger than Earth's.''***} \\
{\bf ***``Planets orbiting very low-mass stars could be desiccated by this hot early phase and enter the HZ as dry worlds (as shown by ref, \citet{luger15}). In contrast, the detailed analysis of the TRAPPIST-1 system by \citet{bolmont16}, using a mixture of energy-limited escape formalism together with hydrodynamical simulations (ref), shows that the planets could have retained their water during the runaway phase.''***} \\
{\bf Figure 2 shows the evolution of the HZ inner edge, bolometric luminosity and XUV luminosity for Proxima.} \\
{\bf ``While the variation of XUV emissions with time is relatively well constrained for Sun-like stars \citep{ribas05, claire12}, the situation for M dwarfs (and especially mid-late M dwarfs) is far from understood.''} \\
{\bf ***``(Recent works) present a picture of a time-evolution in which $\log L_X/L_{bol}$ shows a flat regime starting at the ZAMS and extending out to about 1 or a few Gyr, and known as saturation (ref), followed by a regime in which the decrease shows a power law form. The timescales in this approximation are notoriously uncertain.''***} \\
{\bf ***``Even with significant uncertainties, all models and observations of rotational braking tend to agree that stars like Proxima exhibit saturated activity from early ages until an age of several Gyr, perhaps even until today.''***} \\
{\bf ***The rotational history and XUV flux of Proxima Cen are approximated by Equation (1), based on \citet{ribas05}***} \\
{\bf ``Our calculations show that Proxima b has received, in total, between 7 and 16 times more XUV radiation than Earth, with this range corresponding to the two XUV evolution scenarios described in the text.''} \\
{\bf ``We estimate that Proxima b is receiving a particle flux that could be within a factor of $\approx 4$ and $\approx 80$ of today's Earth value.''} \\
{\bf ***``We note that for a star such as Proxima in the low-wind flux regime, even in those cases where the ram pressure is non-negligible, the magnetic pressure of the stellar wind remains the dominant term, in contrast with he case of the Earth.''***} \\
{\bf ``The Earth is thought to be very tidally dissipative due to the shallow water reservoirs (as in the bay of Biscay, ref). In the absence of surface liquid layers, that is, before reaching the HZ, the dissipation of Proxima b would therefore be smaller than that of the Earth.''} \\
{\bf ``In agreement with (ref), we find that, even when assuming a high dissipation in the star, no orbital evolution is induced by the stellar tide.''} \\
{\bf ``The planetary tide leads mainly to an evolution of the planet's rotation period and obliquity. The eccentricity evolves on much longer timescales so that it does not decrease significantly over the 4.8 Gyr of evolution.''} \\
{\bf The spin-state of Proxima b and its evolution are thoroughly outlined and discussed in Section 4.6, including implications for the planet's climate} \\
{\bf ***``Before reaching the HZ, Proxima b could have spent 100-200 Myr in a region too hot for surface liquid water to exist. This can be compared to the Earth, which is thought to have spent a few Myr in runaway after the largest giant impact(s) \citep{hamano13}. During this stage all the water is in gaseous form in the atmosphere, and therefore it can photo-dissociate and the hydrogen atoms can escape.''***} \\
{\bf ``In order to estimate the amount of water lost, we use the method of \citet{bolmont16} which is an improved energy-limited escape formalism.''} \\
{\bf ***``The energy-limited escape mechanism requires two types of spectral radiation: FUV (100-200 nm) to photodissociate water molecules and XUV (0.1-100 nm) to heat up the exosphere.''***} \\
{\bf ***Equation (9) gives the mass loss equation, where the efficiency is estimated using hydrodynamic mass loss simulations***} \\
{\bf ``We assume that protoplanetary disks around dwarfs such as Proxima dissipate after between 3 Myr and 10 Myr (refs).''} \\
{\bf ***``To calculate the flux of hydrogen atoms, we need an estimation of the XUV luminosity of the star considered, as well as an estimation of the temperature T. We adopt an exosphere temperature of 3000 K (given by hydrodynamical simulations, e.g., \citealt{bolmont16}.''***} \\
{\bf ***``Figure 6 shows the evolution of the hydrogen loss with time for an initial time of protoplanetary disk dispersion of 3 Myr assuming different initial water reservoirs and with the different methods. Table 4 summmarizes the results.''***} \\
{\bf ***``Our calculations thus suggest that the planet does not lose a very high amount of water during the runaway phase.''***} \\
{\bf ``We find that for efficiencies $<0.2$, photo-dissociation becomes the limiting process for the stronger hypothesis on the FUV incoming flux (i..e, when we consider all that is emitted between 10 and 170 nm.''} \\
{\bf ``Assuming the planet initially has a water content equal to 1 Earth ocean, the O2 pressure starts decreasing as hydrogen becomes scarce and oxygen becomes the only species to escape.''} \\
{\bf ***``The presence of a background atmosphere would probably slow down the escape of hydrogen, which means our calculations are an upper value on the hydrogen loss.''***} \\
{\bf ``Compared with other volatile elements, and assuming a carbonaceous chondrite origin, nitrogen is depleted on Earth by one order of magnitude. There is about as much nitrogen in the atmosphere and in the mantle of Earth (ref), the missing part being possibly trapped in the core (ref) since the differentiation of the planet.''} \\
{\bf ``The atmospheric escape of nitrogen -- an element essential to life as we know it -- represents a major threat to its habitability.''} \\
{\bf ***``Two effects should however protect the nitrogen reservoir. First, during the runaway phase, the atmospheric content is in equilibrium with the magma ocean and a fraction of the nitrogen reservoir is therefore in solution in the interior.''***} \\
{\bf ***``This partition is not sufficient in itself to protect nitrogen from escaping. Indeed, equilibrium between the silicates and the atmosphere implies that mantle outgassing compensates for the loss to space, which therefore exhausts both reservoirs.''***} \\
{\bf ***``Vertical transport and fractionation within the mantle may, however, deplete the upper mantle that exchanges with the atmosphere and bury nitrogen at depth by the same process that may have enriched the core of the Earth in nitrogen. If part of this nitrogen is lost into the core another part is outgassed when the mantle solidifies after water condensation at the end of the runaway (ref).''***} \\
{\bf ***``As the solubility of nitrogen into silicates depends strongly on the redox state of the mantle and its water content (ref), it is difficult to draw quantitative conclusions but current geochemistry tells us that part of the nitrogen should indeed be released only once a planet enters the HZ.''***} \\
{\bf ``Another strong limitation of nitrogen escape is the photolysis of N2. Due to its triple bond, N2 is photodissociated only in the 40-100 nm wavelength range. But if hydrogen is escaping due to H2O photolysis, then it is absorbing most of the incoming radiation if both species have similar mixing ratio.''} \\
{\bf ***``Accounting for various factors, the integrated loss rate of nitrogen falls well below a bar of N2 in 100 Myr.''***} \\
{\bf ***``In the absence of a dense enough background atmosphere, the escape of oxygen and hydrogen could continue on much longer timescales without being limited by the diffusion of water vapour while the planet is in the HZ.''***} \\
{\bf ***``We find that the planet loses as much as $\sim 21$ Earth oceans of hydrogen by the age of the star. A situation with Proxima b in the HZ with a global ocean but a surface pressure lower than a few tens of mbars is likely to result in the rapid ($<1$ Gyr) total loss of water and volatiles.''***} \\
{\bf ``If an initial reservoir of heavy volatile elements comparable to that of Earth or Venus was present on Proxima b at the end of its accretion, the majority of it should not have been lost through thermal escape alone.''} \\
{\bf ***``The loss process that represents the main threat for the survival of a dense atmosphere on Proxima b is the joint effect of EUV-driven atmospheric thermal expansion and non-thermal losses, mainly ion pick-up by particles from the stellar wind (ref) and coronal mass ejections (refS).''***} \\
{\bf There is a good explanation of the exobase/magnetopause relation on Earth during times of low/high solar activity, and the possibility of ion pick-up escape processes} \\
{\bf ``Non-thermal losses on Proxima b depend more on the atmospheric expansion driven by XUV heating relative to the magnetopause distance maintained by the magnetic field of the planet.''} \\
{\bf ***``\citep{tian08} found that Earth exobase would rise to 4.8 and 12.7 planetary radii for 10 and 20 EUV$_{\oplus}$, respectively.''***} \\
{\bf ''(Authors) estimated that CMEs could prevent terrestrial planets in the HZ of low-mass stars like Proxima from having a dense atmosphere. This conclusion is based on the combination of enhanced stellar wind pressures, high XUV heating and inefficient magnetic protection due to the slow rotation of tidally-evolved planets and the weak resulting dynamo.''} \\
{\bf ``If we consider the present XUV flux at Proxima b (60 XUV$_\oplus$) and a magnetopause at 2 $R_{\oplus}$, less than 1 bar of atmosphere would be lost per Gyr. A less magnetized planet with a magnetopause at 1.5 $R_{\oplus}$ would lose  few bars and a non-magnetized planet would be stripped of its atmosphere at a rate of more than 100 bars per Gyr.''} \\
{\bf ``If we assume that the XUV flux was higher in the past, up to 150 XUV$_{\oplus}$, then even a magnetopause at 2 $R_{\oplus}$ could not prevent the planet from losing hundreds of bars per Gyr.''} \\
{\bf ``Theoretical studies concluded that a magnetic field of 0.03-0.3 G is induced in the ionosphere even when Earth's magnetic moment disappears during magnetic reversals (ref).''} \\
{\bf ***``Atmospheres exposed to a strong ionizing environment as well as a strongstellar wind or CMEs are expected to develop induced magnetic moments much stronger than those found in the Solar System.''***} \\
{\bf ***``The erosion of the volatile inventory critically depends on the time-dependent distribution of the volatiles between the atmosphere and the different layers of the interior, which is currently beyond predictive modelling.''***} \\
{\bf ***``It took about 100 Myr to build the Earth by colliding tens of Moon- to Mars-sized protoplanets (refS). Each of these giant collisions resulted in a runaway and magma ocean phase that lasted 1 to a few Myr \citep{hamano13}. Therefore, protoplanets that eventually became the Earth spent a large fraction of these 100 Myr in such runaway phase when exposed to the early Sun high-energy radiation. At that time, the Sun was likely to be i its saturated phase, characterized by a maximum $L_X/L_{bol}$ ratio. The XUV irradiation and stellar wind on the proto-Earth was therefore comparable, and possibly higher, than that of Proxima b.''***} \\
{\bf ***``Models predict that early Earth suffered massive volatile losses: hydrodynamic escape of hydrogen dragging away heavier species and non thermal losses under strong stellar wind exposure and CMEs (ref). Nonetheless, no clear imprint of these losses is found in the present volatile inventory.''***} \\
{\bf ***``From geochemical evidence alone there is no indication of a significant erosion of the Earth volatile inventory and one possible conclusion is that volatile loss from Earth was very limited.''***} \\
{\bf ``Different aspects including the initial water inventory, history of the stellar XUV and particle emissions, orbital evolution, volatile losses, including water and the background atmosphere before and after entering the HZ, have been evaluated.''} \\
{\bf ``We can envision a range of plausible formation scenarios that cover a broad range of volatile contents, from nearly dry planets to Earth-like water contents to waterworlds.''} \\
{\bf ``We note that, on a short timescale, the high-energy flux is composed of quiescent emission plus short-term flare events covering a variety of energy levels.''} \\
{\bf ``Proxima b receives 60 times more XUV flux than the current Earth, and ten times more far-UV flux. We note that the energy emission spectrum of Proxima is significantly harder than that of the Sun today, i.e., the relative contribution of X-rays compared to EUV is much higher.''} \\
{\bf ``We find that, at the age of the system, Proxima b may have retained most of its initial eccentricity (which could have been as high as 0.1).''} \\
{\bf ***``Proxima b experiences two distinct major phases in its evolution: (1) during the first few Myr after the protoplanetary disk dispersion the planet is too hot for surface liquid water to exist; and (2) after the first few hundred Myr the planet enters the HZ. During these two phases Proxima b experiences atmospheric loss.''***} \\
{\bf ``We find that during the initial runaway phase the planet loses less than 1 Earth ocean of hydrogen. Once in the HZ most water would be in the high atmosphere and susceptible to escape. In this case, Proxima b could have lost 16-21 Earth oceans of hydrogen by the age of the system. It is also possible that the planet has been able to keep its background atmosphere over the 4.8 Gyr evolution. In this case, the total water loss would be far lower.''} \\
{\bf ***``During the runaway phase the erosion of a significant water reservoir would have left large amounts of residual oxygen, possibly in the form of atmospheric O2, of up to 100 bar.''***} \\
{\bf ``Due to XUV-driven atmospheric erosion and compression of the magnetopause by stellar wind, the planet could have lost a significant fraction of its volatile content by non-thermal losses.''} \\
{\bf ***``Our results highlight the difficulty of assessing the habitability of a planet with an environment so different to Earth's and with only a rough estimate of past irradiation and orbital history.''***} \\
{\bf ``In spite of the strong solar emissions in the past, both Venus and Earth have managed to keep dense atmospheres and no conclusive evidence of massive volatile loss has been found, even though Venus doesn't have an intrinsic magnetic field.''} \\
{\bf ``The main general conclusion from our study is that Proxima b could have liquid water on its surface today and thus can be considered a viable candidate habitable planet.''***} \\

\subsection{The full spectral radiative properties of Proxima Centauri \citep{ribas17}}
This paper is an extension of \citet{ribas16}, accounting for the full spectrum (X-ray to mid-IR) of Proxima Centauri. The spectral energy distribution of Proxima Centauri, from the X-ray to the mid-IR, is determined using a combination of observational data and theoretical models. The authors do so to better constrain the top-of-atmosphere flux on Proxima b, important for calculating volatile losses from the planet as the system evolves. \\
{\it NOTES:} \\
{\bf ``We aim to characterize the X-rays to mid-IR radiative properties of Proxima with the goal of providing the top-of-atmosphere fluxes on the planet. We also aim at constraining the fundamental properties of the star, namely its mass, radius, effective temperature and luminosity.''} \\
{\bf ***``The data shows the top-of-atmosphere average XUV irradiance on Proxima b is 0.293 W m$^{-2}$, that is, nearly 60 times higher than Earth, and that the total irradiance is $877 \pm 44$ W m$^{-2}$, or $64 \pm 3$\% of the solar constant but with a significantly redder spectrum.''***} \\
{\bf ***``We also provide laws for the XUV evolution of Proxima corresponding to two scenarios, one with a constant XUV-to-bolometric luminosity value throughout its history and another one in which Proxima left the saturation phase at an age of about 1.6 Gyr and is now in a power-law regime.''***} \\
{\bf ***``The detailed studies of \citet{ribas16} and (ref) show that Proxima b is likely to have undergone substantial loss of volatiles, including water, in particular during the first $\sim100-200$ Myr, when it could have been in a runaway phase prior to enter the HZ. Volatile loss processes once inside the HZ could have also been at work.''***} \\
{\bf ***``There are numerous examples in the solar system that would contradict the hypothesis of substantial volatile losses in the early stages of its evolution in spite of the Sun being a strong source of high-energy radiation \citet{marty12}.''***} \\
{\bf ``Today, the flux that Proxima b receives in the XUV domain (X-rays to extreme-UV, EUV) is stronger than that received by the Earth by over an order of magnitude and the level of irradiation was probably even stronger in the past.''} \\
{\bf ***``The situation is likely to be quite different in the UV range as Proxima has a significantly lower photospheric temperature than the Sun and therefore a redder emission distribution.''***} \\
{\bf ***``UV irradiation has an impact on photolysis processes, as photoabsorption cross-sections of abundant molecules peak in the 100-300 nm range (ref) and is also of biological interest (ref).''***} \\
{\bf ***``The optical and IR irradiation is the main contributor to the overall energy budget, thus determining the surface temperature of the planet and, ultimately, its habitability.''***} \\
{\bf ``\citet{ribas16} obtained a rough XUV spectrum of Proxima and also discussed possible XUV evolution laws.''} \\
{\bf ``As a consequence of our SED analysis, we identify a conspicuous IR excess, possibly due to dust in the Proxima system.''} \\
{\bf ``Our methodology leads to a total integrated XUV flux value that differs by a few percent from that presented in \citet{ribas16}.''} \\
{\bf ``Flare events can significantly increase the flux with a relative contribution that is stronger at shorter wavelengths.''} \\
{\bf ``We note that the flux between 40 and 92 nm had been underestimated by about a factor of two in our previous calculations in \citet{ribas16}.''} \\
{\bf ***``The flux values for Proxima that we provide should be representative of the average flux to better than 5\%.''***} \\
{\bf ***``It is interesting to point out that the variability of the total irradiance of Proxima is about 25 times higher than the solar value (0.02\%; ref) and this could have an impact on the climate forcing.''***} \\
{\bf ``A possible physical explanation of the mid-to-near IR excess in Figure 3 is the presence of dust grains, in what could be a warm ring close to the star, scattering the light from Proxima. The presence of such a dust reservoir could be a leftover from the formation process of the planetary system aroud Proxima.''} \\
{\bf ***``At 10 Myr, the time when the protoplanetary disk may have dissipated (ref) and Proxima b became vulnerable to XUV radiation, the stellar luminosity was a factor of 10 larger than today. Thus, Proxima b spent some 90-200 Myr in an orbit interior to the stellar HZ and possibly in a runaway greenhouse state.''***} \\
{\bf ***``The XUV evolution of young M dwarfs is poorly constrained but some tantalizing evidence exists indicating that the saturation limit of $\log(L_X/L_{bol}$ also applies to the pre-MS (ref).''***} \\
{\bf ***``The bolometric luminosity should have experienced significant changes over the first few hundred Myr in the history of Proxima and therefore needs to be taken into account in calcuations (e.g., \citealt{luger15}).''***} \\
{\bf ``Different XUV evolution laws are discussed in \citet{ribas16}. One considers a saturated emission state up to a certain age followed by a power law decrease to today's XUV flux. The other one considers that Proxima has shown saturated behaviour since its birth and until today. Observations are inconclusive as to which is correct, so we consider both here.''} \\
{\bf ``We assume that the XUV flux scales in the same way as the X-rays. This is an approximation because the hardness ratio of the XUV spectrum may have softened as the star spun down.''} \\
{\bf ***Equation 2 parameterizes the evolution of the XUV flux over time, assuming $\log(L_{XUV}/L_{bol}$ has remained constant for Proxima's entire lifetime***} \\
{\bf ***``\citet{wright16} suggest that the X-ray evolution of fully convective stars is analogous to that of more massive Sun-like stars.''***} \\
{\bf ***Equation 3 parameterizes the evolution of the XUV flux over time, assuming that Proxima has evolved out of the saturation regime and is currently in the power law regime***} \\
{\bf ***Table 10 shows the hydrogen loss from Proxima b for the two XUV evolution prescriptions***} \\
{\bf ***``Equation (2) has two consequences on water loss compared to \citet{ribas16}: (1) during the runaway phase, and more especially during the first 100 Myr, the loss is more intense than in \citet{ribas16}; and (2) on the long term the total loss is lower.''***} \\
{\bf ***``Equation (3) leads to higher XUV fluxes throughout the entire lifetime of Proxima b when compared with \citet{ribas16}.''***} \\
{\bf ***``In spite of the strong volatile losses ($\sim 0.5-2$ Earth oceans of hydrogen), the planet could still have a significant amount of water reservoir when it entered the HZ depending on the initial content.''***} \\
{\bf ***``If we assume that the water loss processes were still active upon entering the HZ, we find that Proxima b could have lost up to 15-25 Earth oceans worth fo hydrogen during its lifetime. However, this needs to be considered an extreme upper limit because the volatile loss mechanisms would probably be significantly less efficient under such conditions.''***} \\
{\bf ``The detailed spectral energy distribution for Proxima presented here and the newly proposed XUV flux time evolution laws should help to provide the necessary constraints to model and interpret future observations of the nearest potentially habitable planet outside the Solar System.''} \\

\subsection{The origin of RNA precursors on exoplanets \citep{rimmer18}}
{\it NOTES:} \\
{\bf ``Given that he the macromolecular building blocks of life were likely produced photochemically in the presence of ultraviolet (UV) light, we identify some general constraints on which stars produce sufficient UV light for this photochemistry.''} \\
{\bf ``We estimate how much light is needed for the UV photochemistry by experimentally measuring the rate constant for UV chemistry (``light chemistry'', needed for prebiotic synthesis) versus the rate constants for the bimolecular reactions that happen in the absence of UV light (``dark chemistry'').''} \\
{\bf ``By balancing the light and dark chemistry rates, we delineate the ``abiogenesis zones'' around stars of different stellar types based on whether their UV fluxes are sufficient for building up this macromolecular prebiotic inventory.''} \\
{\bf ***``We find that the SO$_3^{2-}$ light chemistry is rapid enough to build up the prebiotic inventory for stars hotter than K5 (4400 K).''***} \\
{\bf ***``We show how the abiogenesis zone overlaps with the liquid water HZ.''***} \\
{\bf ***''Stars cooler than K5 may also drive the formation of these building blocks if they are very active.''} \\
{\bf ``We can delineate an abiogenesis zone, outside of which life is unlikely to have originated in the way the given scenario describes.''} \\
{\bf ``The photochemical scenarios in Patel et al. and Xu et al. are compelling because these are the only known prebiotically plausible chemical networks to selectively achieve high yields of nucleosides, amino acids, and lipid precursors from chemical initial conditions that are realistic for early Earth.''} \\
{\bf ``We can connect the prebiotic chemistry to the stellar ultraviolet (UV) spectrum to determine whether these reactions can happen on rocky planets around other stars.''} \\
{\bf ***``Experiments have shown that the critical wavelength range for these photochemistry reactions is between 200 nm and 280 nm (ref). A comparitively small amount of UV light in this range is produced by ultracool stars, raising the question of whether the prebiotic inventory could ever arise on planets around these stars without the help of flares (refs).''} \\
{\bf ``We focus on key reactions along the seven-step pathway to form the pyrimidine nucleotide RNA precursors. This formation is driven by UV detachment of electrons from anions in solution, such as HS$^-$ from H$_2$S (Patel) and SO$_3^{2-}$ from SO$_2$ (Xu), in the presence of HCN.''} \\
{\bf ``Each of these photochemical reactions take approximately the same amount of time under the exposure of the UV lamp used in both studies.''} \\
{\bf ***``We define the abiogenesis zone as the zone in which a yield of 50\% for the photochemical products is obtained, adopting the current UV activity as representative of the UV activity during the stellar lifetime and assuming a young Earth atmosphere. Only the 50\% yield per step is high enough for a robust prebiotic photochemistry (more details in text).''***} \\
{\bf ***``We found that the SO$_3^{2-}$ dark chemistry was somewhat slower than the HS$^-$ dark chemistry at low temperatures. In addition, the light chemistry is far more rapid for SO$_3^{2-}$. Liquid water is a requirement for both the light and dark chemistry to proceed.''} \\
{\bf ``Planets with 80\% N2, 20\% CO2, and 0.1\% H2O atmosphere with a surface pressure of 1 bar (a plausible abiotic atmosphere similar to what was likely the atmosphere of Earth 3.5 billion years ago (ref).''} \\
{\bf ***``The surface actinic flux (i.e., light able to cause photochemical reactions) of Earth today in inclued in Figure 1, along with 3 M dwarfs, 1 K dwarf, and young Earth. Earth's 200- to 280-nm flux is strongly attenuated by atmospheric ozone.''***} \\
{\bf ***``The yield for the light chemistry, for each step, exceeds 50\%, and therefore, early Earth meets our criterion for lying within the abiogenesis zone.''} \\
{\bf ``With a clear understanding of the temperature and UV flux where the SO$_3^{2-}$ chemistry can occur, we can now predict the planets on which this chemistry can occur.''} \\
{\bf ***``Figure 4 shows which of the potentially known liquid water habitable zone rocky planets also lie within the abiogenesis zone, plotting the stellar effective temperature against the orbital period. Only Earth and Kepler-452 b fall confidently within both.''***} \\
{\bf ``Stars are not quiet, and several researchers have wondered whether the frequent flaring of young ultracool dwarfs may be sufficient to initiate the photochemical production of the prebiotic inventory.''} \\
{\bf ``If a sufficient number of flares occur such that a 50\% yield of the photochemical products is expected per step, then these planets meet the criterion for lying within their star's abiogenesis zone due to the stellar activity.''} \\
{\bf ***``From Figure 5, we find that $\sim30$\% of stars cooler than K5, and $\sim20$\% of the early M dwarfs, are active enough for the planets they host to be within their abiogenesis zones. Alternatively, the flux of photons from coronal mass ejections impinging on the upper atmosphere, however, would still be an order of magnitude too small to overcome the dark chemistry. ''***} \\
{\bf ***``Using a known reliable pathway for photochemically building up the prebiotic inventory for large yields, we show that hotter stars serve as better engines for prebiotic chemistry.''***} \\
{\bf ***``We find, based on our requirement for $>50$\% yields, that, even for early Earth, the prebiotic inventory would need to be built up in places where the surface temperature is below $\sim20$ degrees Celsius.''***} \\
{\bf ``In the absence of UV light, HS$^-$ contributes toward a more efficient pathway to form pyrimidine nucleosides.''} \\
{\bf ``With both HS$^-$ and bisulfate present, the young Sun provides both varieties of chemistry (bisulfite light chemistry and HS$^-$ dark chemistry) simultaneously: the best of both worlds.''} \\
{\bf ***``Because of the efficiency of bisulfite photochemistry, rocky planets within the liquid water HZ of K dwarfs can also lie within the abiogenesis zone, so long as the temperature is very close to 0 degrees Celsius.''***} \\
{\bf ***``The abiogenesis zone we define need not overlap the liquid water HZ. In the scenario we consider, the building blocks of life could have been accumulated very rapidly compared to geological timescales, in a local transient environment, for which liquid water could be present outside the liquid water HZ.''***} \\
{\bf ***``For MS stars cooler than K5 dwarfs, the quiescent stellar flux is too low for the planets within the HZ to also lie within their abiogenesis zones. Planets within the HZs of quiet ultracool dwarfs may be able to house life, but life could not presently originate as a result of photochemistry on these worlds, although it possibly could have done in the past, if these stars emitted much more strongly in the UV before they entered into the MS or if they had been much more active in the past.***} \\
{\bf ***``There is a decent chance that, for the most active M dwarfs, flares could be sequentially timed during intermediate reactions alonng the chemical pathway to build up the prebiotic inventory.''***} \\
{\bf ***``It turns out that stellar activity is not always bad for life but may, in fact, be the only pathway to starting life on planets around ultracool stars. If the activity of an ultracool dwarf decreases as it ages, there might be sufficient 200- to 280-nm light from flares to initiate life on these planets soon after their star's formation; as the star get older, the 200- to 280-nm light could decrease to a low enough level so that the resulting RNA would not be damaged as frequently, although flares may still pose a problem for the stability of RNA strands.''***} \\
{\bf ``What do our results say about looking for biosignatures on rocky planets around ultracool dwarfs? This is an important consideration, given that ultracool dwarfs are likely the only possible candidates for this search within the next decade or so.''} \\
{\bf ***``Present flare statistics for cool stars suggest that as many as 20\% of M dwarfs may flare often enough to drive the prebiotic photochemistry on rocky planets within their HZs, assuming that the atmospheres of these planets can survive this extreme activity.''***} \\
{\bf ***``If atmospheres are observed for a large fraction of rocky planets around active ultracool stars, then it may well be the case that life is more likely to have originated in systems with the most active stars, all else being equal.''***} \\
{\bf ``If definitive biosignatures were discovered within the atmospheres of multiple rocky planets around quiet ultracool dwarfs, at the very least, then such findings would suggest that the mechanism by which Earth-like life could originate is not universal.''} \\
{\bf ***``For early Earth, we took a solar analog spectrum of $\kappa 1$ Ceti (ref).''***} \\
{\bf ``Given that volatile acquisition is unsolved for our own solar system and that the secondary atmosphere is determined primarily by the outgassing pressure and C/O ratio, CO2 is expected to be a dominant component of the atmosphere of a rocky planet of approximately Earth mass and with a crustal and upper mantle C/O ratio of $\lesssim 0.8$ (refs).''} \\
{\bf ``Why Earth has its present quantity of atmospheric nitrogen is a mystery, and at present, we have no alternative recourse but to assume similar amounts of N2 in the atmospheres of rocky planets. Note N2 does not absorb UV between 200 and 280 nm (ref).''} \\
{\bf ***``We note that previous studies found that flares are more frequent on mid-M dwarfs (M3 to M5) than on earlier-M dwarfs (M0 to M2) but typically have lower energies.''***} \\

\subsection{Detecing oceans on extrasolar planets using the glint effect \citep{robinson10}}

\subsection{An analytic radiative-convective model for planetary atmospheres \citep{robinson12}}

\subsection{Near-infrared variability in dusty white dwarfs: tracing the accretion of planetary material \citep{rogers20}}
A study involving near-IR observations of 34 white dwarfs, either previously observed to have IR excess, or candidates of such. This is motivated by polluted white dwarfs, which should show variability on human timescales (i.e., days to years), caused by close-in planetesimals (which have been scattered inwards) being tidally-disrupted (creating a ``dusty'' infrared excess) and potentially accreting onto the WD, polluting its atmosphere. However, the authors find no statistically significant near-IR excess outside of the observational uncertainties. This is interpreted as these ``polluting events'' being rare and occurring on short timescales, disappearing from observations in only 2-3 years. \\
{\it NOTES:} \\
{\bf ``Observations suggest that between 25-50\% of white dwarfs display elements heavier than He in their atmospheres (refs). The rapid gravitational settling times in comparison to the white dwarfs' cooling age implies ongoing accretion from a reservoir (ref).''} \\
{\bf ``The abundances in polluted WDs resemble solar system analogues, with most pollutants resembling bulk Earth to zeroth order. Some bodies also show evidence for differentiation, posing questions about collisional processing (ref).''} \\
{\bf ``Additional support for the asteroid tidal disruption model is the discovery of disintegrating planetesimals transiting the polluted and dusty white dwarf WD 1145+017 (ref, 2015).''} \\
{\bf ``There is previous evidence that large-scale flux changes can occur on time-scales of less than a year (ref).''} \\
{\bf ``The leading explanation for white dwarf pollution cites asteroids as the source which are scattered inwards stochastically.''} \\
 
\subsection{UV surface environment of Earth-like planets orbiting FGKM stars through geological evolution \citep{rugheimer15}}

\subsection{The effect of land albedo on the climate of land-dominated planets in the TRAPPIST-1 system \citep{rushby20}}

\subsection{A readily implemented atmosphere sustainability constraint for terrestrial exoplanets orbiting magnetically active stars \citep{samara21}}

\subsection{The effects of deep water cycling on planetary thermal evolution \citep{sandu11}}
This model provides the basis for the model later used by \citep{schaefer15}. It is a two-box deep-water cycling model of the surface and mantle as water reservoirs, and uses mantle temperature-dependent degassing and regassing, which in turn are also affected by mantle viscosity (dependent on water content, i.e. wet mantle = low viscosity). The water content in the mantle creates a negative feedback, regulating the amount of surface water so it is only weakly dependent on initial water content of the planet; this feedback can be used to estimate mantle water content based on initial conditions. The amount of surface water should be strongly dependent on the thermal evolution of the mantle. {\bf Come back to re-read the details of the model later.}

\subsection{M stars as targets for terrestrial exoplanet searches and biosignature detection \citep{scalo07}}
{\bf Summarize this.}

\subsection{The persistence of oceans on Earth-like planets: insights from the deep-water cycle \citep{schaefer15}}
A parameterized model of deep-water cycling between the surface and mantle reservoirs; this model is mantle temperature-dependent, and thus also relies on mantle viscosity. Two potential convection states are used for super-Earths: single layer convection and boundary layer convection. The former leads to smaller, but longer-lived, oceans, while the latter can lead to potential waterworlds which can be desiccated by total regassing of the water into the mantle. Super-Earths may be less habitable early in their life, but if they attain habitable conditions, they will exist much longer than for Earth-like planets. This model is expanded upon in \citep{komacek16}.

\subsection{Predictions of the atmospheric composition of GJ 1132b \citep{schaefer16}}
GJ 1132b is a nearby, Earth-like planet (in mass and size) orbiting an M-dwarf interior to its HZ. This study models both the initial magma ocean stage of the planet plus its solidified state, and looks at water cycling between the atmosphere and interior to estimate the current atmospheric composition, including potential atmospheric loss. For a large initial water content, the atmosphere may build up several thousands of bars of oxygen; the magma ocean only absorbs up to 10\% of the photodissociated oxygen, while 90\% can be lost to space. The planet would require an initial 5 wt \% water to maintain a steam atmosphere, and the atmospheric water content depends strongly on the XUV flux from the host star. 

\subsection{Magma oceans as a critical stage in the tectonic development of rocky planets \citep{schaefer18}}
A review of the important (and common) magma ocean phase for rocky planets through models of Earth, Mars, and the Moon, and comparisons with present-day observations (to which the models match quite well, including the Earth's mantle oxidation state). The magma ocean phase could potentially delay or completely inhibit the onset of plate tectonics on a rocky planet. {\bf Currently not concerned with magma ocean phase, only its end result.}

\subsection{Experimentally based water budgets for dehydrating slabs and consequences for arc magma generation \citep{schmidt98}}

\subsection{HAZMAT III. the UV evolution of mid- to late-M stars with GALEX \citep{schneider18}}
Another paper in the HAZMAT series, this time looking at GALEX FUV and NUV observations of mid- to late-M stars. The authors find that the UV evolution of mid- to late-Ms is qualitatively different than that of early Ms, with lower-mass M stars remaining active for longer. The FUV to NUV ratio, which varies with mass and age for early-Ms, remains relatively constant in the studied sample of mid- to late-Ms. \\
{\it NOTES:} \\
{\bf ***``We find clear evidence that mid- and late-type M stars do not follow the same UV evolutionary trend as early-Ms. Lower-mass M stars retain high levels of UV activity up to field ages, with only a factor of 4 decrease on average in GALEX NUV and FUV flux density between young ($<50$ Myr) and old ($\sim 5$ Gyr) stars, compared to a factor of 11 and 31 for early-Ms in NUV and FUV, respectively.''***} \\
{\bf ***``We also find that the FUV/NUV flux density ratio, which can affect the photochemistry of important planetary biosignatures, is mass- and age-dependent for early-Ms, but remains relatively constant for the mid- to late-type Ms in our sample.''***} \\
{\bf ``The stellar UV flux incident on a planet in an M star's HZ may impact the development of extrasolar biology (ref), destroy the very biosignatures we hope to detect, or even lead to the formation of abiotic oxygen and ozone, producing false-positive biosignatures (refs).''} \\
{\bf ***``Planet occurrence rates have been shown to increase with decreasing stellar mass (ref), and the planets around low-mass stars are also typically smaller than those around higher-mass stars (ref).''***} \\
{\bf ``As in \citet{shkolnik14}, we see a significant amount of scatter in NUV and FUV flx density ratios for each age, with ratios typically spanning one or two orders of magnitude within each age group.''} \\
{\bf ***``The values in Table 4 and Figure 7 can be used to predict NUV and FUV flux densities for other low-mass stars, or incident flux densities on exoplanetary atmospheres for hypothetical stars.''***} \\
{\bf ``At the age of the Hyades (625 Myr), very few mid- to late-Ms have spun down while a significant fraction of mid- to late-Ms have spun down at field ages. However, there are still a substantial amount of late-M rapid rotators at field ages, with $\sim 33$\% having periods less than 1 day.''} \\
{\bf ``Both NUV and FUV flux density measurements indicate that mid- to late-Ms continue to remain more active with increased UV flux density levels up until field ages compared to early-Ms.''} \\
{\bf ``Numerous studies of the activity-rotation relationships for low-mass stars have shown that more rapidly rotating stars also show higher levels of activity (refs).''} \\
{\bf ``If the mechanism that causes M stars to spin down as they age, typically ascribed to a magnetized stellar wind, does not function efficiently for fully convective stars, then we would expect them to remain in states of rapid rotation for much longer portions of their lifetimes than early Ms. Because activity is directly related to rotation, this implies that mid- to late-Ms should remain active for much of their lifetimes.''} \\
{\bf ``The FUV to NUV ratio has been shown to be more than two orders of magnitude from solar-type dwarfs to M dwarfs (ref).''} \\
{\bf ***``FUV radiation will photolyze CO2, producing atomic oxygen which can then combine to form O2 and eventually O3.''***} \\
{\bf ***``NUV radiation on the other hand will dissociate O3, and thus O3 abundances are critically dependent on the FUV/NUV ratio values.''***} \\
{\bf ***``The GALEX FUV to NUV flux density ratio for mid- to late-type M stars remians relatively constant over their lifetimes, while this ratio clearly evolves to lower values with age for early-type Ms.''***} \\
{\bf ***``Mid- to late-Ms remain relatively active throughout their lifetimes, with only a small UV flux density decrease from young to old ages, whereas early-Ms have a much more significant decrease over the same age range.''***} \\
{\bf ***``Strongest FUV emission line, Lyman $\alpha$, dominates the UV spectrum of low-mass stars (refs), and controls the photodissociation of molecules such as O2, H2O, CH4 (ref).''***} \\
{\bf ***``FUV flux has been shown to be strongly correlated to X-ray and EUV flux (refs), which controls much of the upper atmospheric heating of orbiting planets (refs).''***} \\
{\bf ``We attribute the prolonged UV activity seen for low-mass Ms to different rotation rates for early-Ms and fully convective mid- to late-Ms.''} \\

\subsection{The probability that a rocky planet's composition reflects its host star \citep{schulze20}}

\subsection{Exoplanet biosignatures: a review of remotely detectable signs of life \citep{schwieterman18}}

\subsection{Chemical diversity of super-Earths as a consequence of formation \citep{scoro20}}

\subsection{A note on planet size and cooling rate \citep{seales21}}

\subsection{Ozone concentrations and ultraviolet fluxes on Earth-like planets around other stars \citep{segura03}}

\subsection{The effect of a strong stellar flare on the atmospheric chemistry of an Earth-like planet orbiting an M dwarf \citep{segura10}}
{\bf Summarize this.} 94\% of atmosphere eroded by M-dwarf flares in 10 years??? (Confirm this in actual paper, because it is unbelievable.)

\subsection{Internal water storage capacity of terrestrial planets and the effect of hydration on the M-R relation \citep{shah20}}
{\bf This one will be great to come back to when setting constraints on initial water content.}

\subsection{What makes a planet habitable? \citep{shahar19}}

\subsection{The habitability of planets orbiting M-dwarf stars \citep{shields16}}
A review of the factors impacting the habitability of M-dwarf planets; this is important as $\sim$70\% of stars in the galaxy are M-dwarfs, and are thus prime targets for observations of planets in their HZs. Observations are advantageous due to the increased star-planet contrast ratio, but also due to the increased frequency of transit because of the close-in orbits of the planets. The review outlines radiative effects on the planet (e.g., the longer pre-main sequence phase of the host stars leading to higher activity, and the bombardment from stellar flares due to the close-in orbits), as well as gravitational effects (e.g., tidal-locking creating a distinct dayside/nightside). Some of these effects may be counteracted by the planets themselves, depending on e.g. atmospheric composition, instead increasing their habitability.

\subsection{HAZMAT. I. The evolution of far-UV and near-UV emission from early M stars \citep{shkolnik14}}
Observations of M dwarfs of a variety of ages from different clusters are used to measure the evolution of GALEX NUV and FUV flux with age. UV flux appears ``saturated'' early on, similar to what is observed for X-ray emission, but the activity eventually decreases with age. However, there is a spread in activity levels at every age. The results of this first HAZMAT paper should guide M dwarf atmosphere models to correctly model UV flux of the upper atmosphere, something lacking in previous modelling. \\
{\it NOTES:} \\
{\bf ``The median UV flux for our sample remains at a ``saturated'' level for a few hundred Myr, analogous to that observed for X-ray emission.''} \\
{\bf ``This decline in activity beyond 300 Myr follows roughly $t^{-1}$.''} \\
{\bf ***``Despite this clear evolution, there remains a wide range, of 1-2 orders of magnitude, in observed emission levels at every age.''***} \\
{\bf ``$\sim50$\% small planet occurrence rate around M-dwarfs in the habitable zone (refs)... This implies that most of the planets in our galaxy, including those in the HZ, orbit M dwarfs as these low-mass stars make up 75\% of the total stellar population (ref).''} \\
{\bf ``The UV increases the generation of surface-shielding hazes in reducing atmospheres (ref) and ozone in oxidizing atmospheres (refs), both of which can strongly affected the observed spectrum.''} \\
{\bf ``Observations of exoplanet transit transmission spectroscopy (refs) are showing that hazes in planets around M dwarfs might in fact be quite common, probably generated by the star's UV.''} \\
{\bf ***``The EUV stellar radiation (100-900 Angstroms) can be particularly damaging as it photoionizes, heats, and inflates a planet's upper atmosphere, making it vulnerable to mass loss (ref).''***} \\
{\bf ***``Lammer et al.(2007) conclude that the atmosphere of an unmagnetized telluric planet can be completely eroded in its first Gyr by high M dwarf activity levels and corresponding coronal mass ejections.''***} \\
{\bf ***``Tian (2009) found that the atmosphere of a super-Earth is stable even around very active M dwarfs.''***} \\
{\bf ``The rate of decline in stellar X-ray emission was shown to vary with stellar mass, with M dwarfs remaining X-ray active (and rotating faster) much longer than FGK stars (refs).''} \\
{\bf ``X-ray emission is ubiquitous among low-mass stars and is indicative of active stellar upper atmospheres throughout their lifetimes.''} \\
{\bf **``Fractional X-ray luminosities have also been shown to be ``saturated'' across a wide range of spectral types, H$\alpha$ equivalent widths, and ages at the value of $\log(L_X/L_{bol} \sim -3$, with the bulk of the dispersion in both field and cluster samples between $\log(L_X/L_{bol}$ of -2 and -4, primarily due to variations in stellar rotation (refs).''***} \\
{\bf ``For young Ms, larger and more frequent flares are expected.''} \\
{\bf ``It has been well established that the chromospheric activity and coronal emission of FGKM stars steadily decreases with age due to reduced dynamo production of magnetic fields as the star spins down.''} \\
{\bf ***``Unlike the spin-down timescale for higher-mass stars ($<1$ Gyr; ref), the spin-down timescales for field M dwarfs range from 1 to 10 Gyr, taking longer with decreasing stellar mass (refs).''***} \\
{\bf ***``The consistency between the X-ray and FUV implies that, at least qualitatively, we can draw similar conclusions for the EUV -- i.e., a saturation level of emission until a few hundred millions of years and a reduction in flux with age afterward following roughly $t^{-1}$.''***} \\
{\bf ***``Our results indicate that the X-ray and UV fluxes correlate over a broad range of activity levels, defined by two groups: high and low emitters.''***} \\
{\bf ***``Qualitatively, the FUV and NUV excess flux densities decay in a similar fashion to X-ray results, wth a high saturation level from 10 Myr until a few hundred Myr.''***} \\

\subsection{Carbon dioxide cycling and implications for climate on ancient Earth \citep{sleep01}}
Carbon cycling on the modern Earth is used as a proxy to model and study the carbon cycling during the Hadean (which ended $\sim$ 4 billion years ago). Silicate weathering and metamorphism aid in regulating the climate -- warmer temperatures increase weathering and decrease atmospheric CO\textsubscript{2} levels, providing a stabilizing feedback. CO\textsubscript{2} is also degassed from the mantle at mid-ocean ridges and by arc volcanism, which may have dominated on the early Earth. Previous studies indicated a warm Hadean climate; however, through inclusion of ejecta (as efficient  CO\textsubscript{2} sinks, so low  CO\textsubscript{2} levels in the Hadean atmosphere), the model in this paper predicts cold surface temperatures during the Hadean, with possible ocean ice coverage if no other greenhouse gases were present; an ice-covered Earth has been suggested as the origin of life. {\bf The details of the modelling may be helpful when adding complexity to my water-cycling model.}

\subsection{Scaling of temperature- and stress-dependent viscosity convection \citep{solomatov95}}

\subsection{The influence of bulk composition on long-term interior-atmosphere evolution of terrestrial exoplanets \citep{spaargaren20}}
The thermal evolution of terrestrial exoplanet interiors and atmospheres is studied using a 1D model, beginning with magma ocean solidification, allowing study of the near-surface melt production and volatile outgassing (i.e., the interior-atmosphere interplay). The results indicate that half of rocky exoplanets would have single-layer mantle convection (i.e., the whole mantle convects as one). The other half experience double-layered convection due to a mid-mantle compositional boundary, being more likely for planets with high bulk Fe-content and low Mg/Si ratio. Low Mg/Si also leads to slower cooling and extensive melting, meaning mantle volatiles are lost quickly -- mantle viscosity increases with decreasing Mg/Si The tectonic regime (plate tectonics vs. stagnant lid) also influences the thermal evolution and volatile cycling to first-order, with stagnant-lid planets cooling slower due to the thick lid. Overall, the results indicate that atmosphere observations may provide constraints on the tectonic regime along with the thermal evolution of the interior. {\bf This is a great reference to come back to!} \\
{\it NOTES:} \\
{\bf ``Complementary constraints on interior properties might be provided by the atmospheres of terrestrial exoplanets, since these form and evolve as a consequence of interaction with the interior (ref).''} \\
{\bf ``Stellar compositions are a good indicator for planetary compositions, since the planet, disk, and star form from the same cloud of gas and dust.''} \\
{\bf ``The primordial atmosphere (captured from the disk, and replaced by primary atmosphere during magma ocean crystallization) escapes almost entirely for planets up to super-Earth size (refs).''} \\
{\bf ``The magma ocean phase also determines the mineralogy and compositional structure of the silicate mantle that resides above the core.''} \\
{\bf ``Single-layer (whole-mantle) convection occurs through most of Earth's history to the present day.''} \\
{\bf ``Estimates for the timescale of magma-ocean crystallization on Earth range from thousands (ref) to millions (ref) of years, so a rocky planet spends most of its life post-magma ocean.''} \\
{\bf ***``Thermal evolution of planets with solid interiors is dictated by the viscous creep of solid materials, which manifests as mantle convection.''***} \\
{\bf ``Ingassing likely occurs at a higher rate for planets in the plate tectonics regime \citep{tosi17}, where ingassing occurs by transport of crustal material into the mantle by subduction (ref).''} \\
{\bf ``Outgassing occurs by melt production in the interior and the escape of volatiles by volcanism (ref).''} \\
{\bf ***``The chemistry of outgassing volatiles depends on the interior state, mainly the oxygen fugacity (ref), the planetary volatile budget, and atmospheric conditions (ref).''} \\
{\bf ``We consider water as the representative volatile, and limit consideration to planets with the same mass, radius, and core size as Earth.''} \\
{\bf ***``Terrestrial planets separated into two major reservoirs: silicate mantle and iron core. Most Si and Mg is present in the mantle, while Fe is partitioned between the mantle and core: hence stellar Fe/Mg cannot be used directly as a proxy for bulk mantle Fe/Mg.''***} \\
{\bf ``For Earth-sized planets, an MO (magma ocean) may crystallize from the bottom-up (refs) or middle-out (refs).''}
{\bf ***``For fractional crystallization (considered in this model), crystals form in the MO due to cooling, and sink to settle at the crystallization front (ref). This allows distinct compositional evolution (i.e., fractionation) of MO liquid and forming crystals (ref).''} \\
{\bf ***``Overturning instigates decompressional melting and re-melting, thereby homogenizing the upper mantle material, and thus justifying our assumption that the solid and melt will have the same composition in the upper mantle.''} \\
{\bf ``The two end-member cooling regimes of mantle evolution involve a mobile lithosphere which participates directly in mantle convection (plate tectonics, PT), and an immobile lithosphere that hinders efficient cooling (stagnant lid, SL).''} \\
{\bf ``Tectonic regime changes are predicted by geodynamic simulations, and consistent with the geologic record for Earth and Venus (refs).''} \\
{\bf ``Water partitions similarly to most other volatiles (except carbon), and depresses the solidus temperature for melting. Iron also depresses the solidus.''} \\
{\bf ``Water behaves like an incompatible element and thus preferentially partitions into the melt. The melt fraction and water concentration in the melt are mutually dependent.''} \\
{\bf ``Transfer of volatiles from interior to atmosphere is only possible when the melt extrudes (i.e., reaches the surface and is not trapped in the lithosphere.''} \\
{\bf ``In the SL regime, volatiles trapped within intrusive magmas are stored in the lid reservoir, and have lower melt extrusion rates due to the large thickness of the lithosphere.''} \\
{\bf ***``On planets in the SL regime, ingassing can occur by lithospheric delamination, where blobs of crustal material sink into the mantle from the base of the stagnant lid, transporting volatiles from lid to mantle reservoir.''} \\
{\bf ``Hydrous minerals are not stable at the hot lid base (ref), such that water is exclusively transported by nominally anhydrous minerals.''} \\
{\bf ``Volatile transport from the atmosphere to the mantle is inefficient for planets with an immobile and sufficiently thick lid (Foley \& Smye 2018).''} \\
{\bf ``Water retention, and therefore ingassing rate, depends on the stability of hydrous minerals, which is affected by composition and mantle temperature.''} \\
{\bf ``The predicted compositional profile of solid material that is crystallized from the MO depends mainly on the bulk Mg/Si-ratio.''} \\
{\bf ``In double-layer stratified mantles, the two layers typically have distinct Mg/Si-ratios, with the upper layer generally having a higher Mg/Si ratio than the lower layer. Mg/Si is the stronger control on the evolution of planetary mantles.''} \\
{\bf ***``For PT planets, after about 1 Gyr, net ingassing of the mantle becomes dominant for planets with high Mg/Si. In contrast, planets with low Mg/Si display net outgassing for their entire lifetime.''} \\
{\bf ``Higher average temperatures of planets with low Mg/Si leads to greater melt production over the lifetime of the planet.''} \\
{\bf ***``For SL planets (with no ingassing), outgassing continues for at least 4 Gyr, with low Mg/Si cases up to 8 Gyr. The volatile reservoirs approach a steady state, in which most volatiles reside in the atmosphere.''} \\
{\bf ``The lid reservoir does not contain more than a few percent of the bulk volatile budget because of high delamination fluxes as long as outgassing is efficient.''} \\
{\bf ``For planets with stratifed mantles, the upper mantle composition is different than the bulk mantle; hence, the bulk mantle composition is no longer representative of the composition of the volatile-carrying reservoir.''} \\
{\bf ``Stable mantle stratification is more likely for planets with a low bulk Mg/Si ratio, and with high mantle iron content. Thus, the presence or absence of mantle layering can help constrain core size.''} \\
{\bf ``The average mantle temperatures for SL planets are always higher than for PT planets at a given age, regardless of composition or presence of stable stratification.''} \\
{\bf ``However, in SL planets, (latent) heat may also efficiently be removed by melting and related extrusive volcanism (refs), which would cool the interior more efficiently than modelled here.''} \\
{\bf ``SL planets are assumed to have a high intrusion-to-extrusion ratio, which delays atmospheric build-up compared to PT. The lack of a transport mechanism from the atmosphere to the interior allows the atmospheric reservoir to grow until melting in the interior, and therefore outgassing, ceases.''} \\
{\bf ``Parameters that can influence the propensity of PT are mantle temp (refs), rock hydration (ref), planet size (refs), complex rheologies and depth-dependent parameters (ref), and mantle viscosity (which is strongly affected by bulk composition).''} \\
{\bf ***``Magma ocean models predict that 70-90\% of the volatiel budget outgasses to form a primary atmosphere, depending on the bulk volatile budget (Elkins-Tanton \& Seager 2008).''} \\
{\bf ***``Our modelled mantle is inefficient at retaining water, which could be because we do not treat the mantle transition zone as a water filter (ref).''} \\
{\bf ``Secondary atmosphere masses are consistently larger for planets with low Mg/Si.''} \\
{\bf ``The presence of carbon volatile species can suppress the mantle solidus, but in a very distinct way compared to water (refs).''} \\
{\bf ``Our results demonstrate that the bulk planetary composition affects interior evolution and produces a diversity of atmosphere sizes.''} \\
{\bf ``SL planets tend to have hotter interiors, and more massive atmospheres later in the planet's life. Meanwhile, PT planets have more efficient outgassing, and therefore more massive atmospheres early in a planet's life.''} \\
 
\subsection{The influence of pressure-dependent viscosity on the thermal evolution of super-Earths \citep{stamenkovic12}}
A 1D parameterized convection model is used to study the thermal evolution of Earth-like planets and super-Earths, for pressure- and temperature-dependent regimes. {\bf Note: This model may be a good reference when improving my mantle model.} Depending on initial conditions, super-Earths can develop a ``stagnant lid'' at the core-mantle boundary (a CMB-lid) in the lower mantle, the likelihood of which increases with mass. The so-called ``thermostat effect'' {\bf (described well in the paper)} does not operate for larger planets. Although the thermal evolution of super-Earths depends strongly on their initial thermal state, plate tectonics and magnetic field generation appear to be less likely for super-Earths.

\subsection{The effect of rheological parameters on plate behaviour in a self-consistent model of mantle convection \citep{stein04}}

\subsection{Is plate tectonics needed to evolve technological species on exoplanets? \citep{stern16}}

\subsection{Detection of an atmosphere on a rocky exoplanet \citep{swain21}}
Detection of a low mean molecular weight atmosphere on GJ 1132b, a rocky planet similar in size and density to Earth.

\subsection{Hydrodynamic planetary thermosphere model: 1. Response of the Earth's thermosphere to extreme solar EUV conditions and the significance of adiabatic cooling \citep{tian08}}
A 1-D model is used to study the effects of extreme solar EUV radiation on the Earth's thermosphere. The results indicate that for hydrostatic equilibrium, above a critical EUV flux ($\sim 5 \times$ present solar EUV), the atmosphere can enter an atmospheric blowoff state; however, this atmospheric blowoff is prevented for hydrodynamic flow and with adiabatic cooling included in the thermosphere. \\
{\it NOTES:} \\
{\bf ***``Our simulations show that if forced in hydrostatic equilibrium and maintaining the current composition, the Earth's thermosphere could experience a fast transition to an atmospheric blowoff state when exposed to solar EUV radiation stronger than a certain critical flux (5 times present EUV flux).''***} \\
{\bf ***``When hydrodynamic flow and its associated adiabatic cooling are included, atmospheric blowoff is prevented and Earth's exobase temperature decreases with increasing solar EUV beyond the critical solar EUV flux.''***} \\
{\bf ***``We propose that hydrodynamic flow and associated adiabatic cooling should exist in the atmospheres of a broad range of early and/or close-in terrestrial type planets and that the adiabatic cooling effect must be included in the energy balance in order to correctly estimate their thermospheric structures and their evolutionary paths.''***} \\
{\bf ``In thermal escape, particles at the exobase are assumed to have an approximately Maxwellian velocity distribution that is determined by the exobase temperature, and those with outgoing velocities exceeding the escape velocity of the planet can escape (refs)... controlled by Jeans parameter, $\lambda_c$.''} \\
{\bf ***``An extreme case of thermal escape is atmospheric ``blowoff'', which occurs when the mean thermal energy of the major gases at the exobase level (where the mean free path of the gas particles is comparable to the scale height of the atmosphere) exceeds their gravitational potential energy (equivalent to $\lambda_c < 1.5$)... can be indiscriminate as to species (ref).''***} \\
{\bf ``In this definition, atmospheric blowoff is essentially an uncontrolled process in the sense that the loss of gas (at an arbitrarily high rate) does not affect the exobase temperature.''} \\
{\bf ``Realistically, gases should escape planetary atmospheres in a more controlled manner (limited by energy), but still faster than predicted by the Jeans' formula, because pressure forces contributes to the escape process.''} \\
{\bf ``In non-thermal escape processes, such as H/H$^+$ charge exchange, the escaping atoms acquire energy from nonthermal sources, in this case hot H$^+$ ions.''} \\
{\bf ``The above classification of escape mechanisms concerns only the region near and above the exobase. To understand how the underlying planetary atmosphere responds to different escape scenarios requires a thermosphere model that includes both energetics and dynamics.''} \\
{\bf ``We propose that planetary atmospheres can be classified into two regimes: regime I is the hydrostatic equilibrium regime, in which the bulk atmosphere below the exobase can be considered as static; and regime II is the hydrodynamic flow regime, in which the major gases in the thermosphere can escape efficiently as a result of large energy input and/or a weak planetary gravitational field.''} \\
{\bf ***``A planetary atmosphere can be in the hydrodynamic flow regime without satisfying the blowoff criteria ($\lambda_c < 1.5$); indeed, the proper treatment of hydrodynamic flow prevents the occurrence of atmospheric blowoff.''***} \\
{\bf ***``The upper atmospheres of hydrogen-rich and water-rich early terrestrial planets (refs) and close-in exoplanets (refs) are in the hydrodynamic flow regime.''***} \\
{\bf ***``Because of the adiabatic cooling effect of the hydrodynamic flow, the atmospheric temperature can be reduced significantly if the bulk motion velocity is substantially high, thereby preventing the occurrence of atmospheric blowoff.''***} \\
{\bf ``In the simplest case, expansion of a dense (and, hence, collisional) fluid into a vacuum, the flow should be transonic. This solution, and no other, satisfies the condition that the mass of the atmosphere is finite if one integrates out to infinity.''} \\
{\bf ``Planetary winds are more complex in the sense that they are only partially ionized, and so the escaping particles are less interconnected by long-range, electromagnetic forces. Such a flow becomes collisionless at some point, and the fluid dynamic approximation is no longer valid.''} \\
{\bf ``If this point (the exobase) is reached beyond the critical point where the flow becomes supersonic, then the transonic solution should still be valid. But there are other possibilities. The flow may become collisionless before it becomes supersonic, so that the entire hydrodynamic flow regime is subsonic. This approach may underestimate the actual escape rate because it ignores the nonthermal escape processes above the exobase.''} \\
{\bf ***``It has been suggested that the young Sun emitted stronger EUV radiation: approximately 3, 6, and 10 times that of today at $\sim3$, 3.5, and 3.8 Ga ago, respectively \citep{ribas05}. A few hundred Myr after its formation, the Sun's EUV radiation level could have been 100 times that of today.''***} \\
{\bf ***``For early Earth with the same composition as that of today, the model of Kulikov et al. (2007) predicted a thermospheric temperature of $\sim 10000$ K for $\sim 12$ times present EUV condition. The extremely high thermospheric temperatures of this model suggest that the thermosphere cannot be considered hydrostatic in extreme solar EUV conditions because atomic oxygen (the dominant gas in the upper thermosphere of present Earth) should be escaping at a significant rate.''***} \\
{\bf ``In this paper, we describe a 1-D, multi-component hydrodynamic thermospheric model, and we use it to explore the response of Earth's thermosphere under extreme solar EUV radiation conditions.''} \\
{\bf ``Our calculations confirm the findings of (ref) and shows that atomic nitrogen indeed becomes a species competing with atomic oxygen for dominancy at the exobase under extreme solar EUV conditions.''} \\
{\bf ***``As found in previous works (refs), O$^+$ is the dominant ion species at high altitudes, and both O$_2^+$ and NO$^+$ dominate at low altitudes.''***} \\
{\bf ``At solar maximum the most apparent change in the heating/cooling profiles, other than the overall increase of total heating/cooling, is the increased cooling contribution of NO in the lower thermosphere, which reaches $\sim80$\% at $\sim140$ km.''} \\
{\bf ***``Our model confirms the slower increase of the O$^+$ peak density when the solar EUV flux increases beyond $\sim10$ mW/m$^2$.''***} \\
{\bf ``(Previous authors) observed that the concentration of atomic nitrogen increases by a factor of 4 when the solar EUV flux is increased by a factor of 2 from that at solar maximum and attributed this change to the increased photodissociation of N2 and accelerated production of N through ion-neutral chemical reactions. Our calculations confirm this finding.''} \\
{\bf ***``For solar EUV fluxes smaller than $\sim5$ times that of today ($\sim 25$ mW/m$^2$), the peak temperature in the thermosphere occurs at the exobase level.''***} \\
{\bf ***``For solar EUV fluxes greater than $\sim5 \times$ EUV, the peak temperature still increases with energy flux, but the upper part of the thermosphere begins to cool as a result of the increasingly significant adiabatic cooling effect. The higher the energy input into the thermosphere, the lower the exobase temperature. This behaviour is typical in hydrodynamic models of planetary atmospheres (refs).''***} \\
{\bf ***``When solar EUV flux is smaller than $\sim4.9$ times present EUV, all through the system the net heating from radiative transfer calculations is balanced by thermal conduction. Adiabatic cooling is negligible because of the small bulk motion velocities.''***} \\
{\bf ***``In the 5.3 times present EUV case, adiabatic cooling becomes effective near the exobase, and the significance of thermal conduction cooling begins to decrease. As the energy flux continues to increase, the adiabatic cooling effect becomes greater, and the temperature begins to decrease with altitude int the upper part of the thermosphere.''***} \\
{\bf ***``In the large solar EUV flux cases, thermal conduction acts to heat the uppermost part of the thermosphere (comparable to the heating from radiative transfer processes) rather than cooling it.''***} \\
{\bf ``Figure 8 suggests that the response of the Earth's thermosphere to extreme solar EUV conditions can be divided into the two regimes defined in the intro section by a critical solar EUV flux (5 times present EUV).''} \\
{\bf ``In regime I, the peak temperature occurs at the exobase. Our simulation results show that atomic oxygen maintains its status as the dominant species at the exobase in regime I, with increasingly stronger competition from atomic nitrogen, with their number densities becoming comparable at a solar EUV flux close to 5 times present EUV.''} \\
{\bf ***``The thermosphere enters regime II (hydrodynamic flow) when the solar EUV flux increases beyond the critical flux. In this regime, the peak temperature no longer occurs at the exobase, ad the exobase temperature decreases with increasing solar EUV flux. Atomic nitrogen becomes the dominant species at the exobase and ions (O$^+$ dominant) become increasingly important and eventually dominate the exobase.''***} \\
{\bf ``Analysis of the energy budget of the thermosphere under extreme solar EUV fluxes suggests that the electron collision heating mechanism maintains its status as the dominant heating mechanism in the upper thermosphere.''} \\
{\bf ***``A critical flux ($\sim10$ times present EUV) can still be found, with electron collision heating set to zero, beyond which the exobase would begin to cool when the solar EUV flux is further increased.''***} \\
{\bf ``Our simulations show that the thermosphere's response remains the same when solar EUV flux is small -- the hydrostatic equilibrium assumption is sound when the energy input into the thermosphereis limited.''} \\
{\bf ***``As the solar EUV flux approaches the critical flux, the exobase begins to expand and warm up dramatically. When the exobase altitude and temperature increase, the atmospheric blowoff state is quickly reached, and eventually the exobase vanishes. These results suggest that the Earth's thermosphere could experience a fast transition into the atmospheric blowoff state when exposed to certain critical solar EUV conditions, and that hydrostatic equilibrium does not apply in those cases.''***} \\
{\bf ***``Our results show that (1) planetary atmospheres can be in the hydrodynamic flow regime without satisfying the blowoff criteria and (2) adiabatic cooling associated with the hydrodynamic flow in planetary thermospheres actually prevents the atmosphere from reaching the blowoff state.''***} \\
{\bf ``Comparisons between the hydrodynamic fluxes with the Jeans escape fluxes in more than one case show that Jeans escape is orders of magnitude smaller than the hydrodynamic flow.''} \\
{\bf ***``In our 1-D model, the whole thermosphere is expanding under extreme solar EUV radiation and the adiabatic cooling associated with the hydrodynamic flow plays a similar thermostat role, and could have protected the atmospheres from atmospheric blowoff at early stages of Solar System history when the solar EUV flux was very high.''***} \\
{\bf ``The IR cooling agents in our model (15 $\mu$m CO2, 5.3 $\mu$m NO, and 63 $\mu$m O) have been generally accepted as the dominant radiative cooling agents in the thermospheres of Earth, Mars, and Venus (refs).''} \\
{\bf ***``For greater than 5 times present EUV, the hydrodynamic flow (driven by fast Jeans escape of major gases at the exobase level) should start, and the adiabatic cooling effect should be significant when Earth's thermosphere is exposed to extreme EUV radiation levels.''***} \\
{\bf ``Recent calculations for early Venus (ref) and Venus-like exoplanets in the HZs of M stars (ref) show that the exobase temperatures could reach more than 8000 K when exposed to 100$\times$ present EUV level. Figure 6 suggests these planetary thermospheres may be in the hydrodynamic flow regime.''} \\
{\bf ``It is possible that stronger radiative cooling from more abundant IR agents and/or stronger gravity of more massive terrestrial planets may keep their thermospheres in hydrostatic equilibrium.''} \\
{\bf ``The composition of early Earth's atmosphere almost certainly was different from that of today and this composition change could have had significant impacts on the structure of both the lower and the upper atmosphere.''} \\
{\bf ***``We note that the transition temperature (which itself depends on the composition of the atmosphere and can be affected by possible efficient nonthermal escape processes) from regime I to regime II of planetary atmospheres should be more important than the critical solar EUV flux, which is likely to change considering model uncertainties.''***} \\
{\bf ***``The Earth's thermosphere could experience a fast transition into atmospheric blowoff state if hydrostatic equilibrium is forced in the model. The transition occurs at certain critical solar EUV fluxes. When hydrodynamic flow and the adiabatic cooling effect are included, atmospheric blowoff is prevented, and the Earth's exobase temperatures decrease with increasing solar EUV beyond a criticial solar EUV flux. The transition between regimes occurs when exobase temperature reaches 7000 to 8000 K if atomic O and N dominate the upper atmosphere.''***} \\
{\bf ***``We propose that hydrodynamic flow and associated adiabatic cooling should be important in the thermospheres of a broad range of terrestrial planets (early and/or close-in) and that the adiabatic cooling effect (acting as a thermostat) must be included in the energy equation in order to estimate their thermospheric structures and evolutionary paths correctly.''***} \\

\subsection{Thermal escape from super-Earth atmospheres in the habitable zones of M stars \citep{tian09}}

\subsection{Atmospheric escape from Solar System terrestrial planets and exoplanets \citep{tian15a}}

\subsection{Water contents of Earth-mass planets around M-dwarfs \citep{tian15b}}
{\bf Summarize this.}

\subsection{Modelling repeated M dwarf flaring at an Earth-like planet in the habitable zone: atmospheric effects for an unmagnetized planet \citep{tilley19}}

\subsection{The habitability of a stagnant-lid Earth \citep{tosi17}}
The habitability (i.e., surface liquid water) of a stagnant-lid Earth is modelled, since stagnant lid may be the most common tectonic regime of terrestrial planets, particularly using a 1D parameterized convection model and a 1D convective-radiative atmosphere to investigate the climate. The results initially show a large amount of partial melt and rapid outgassing of H2O (although its partial pressure is limited since it is highly soluble in basaltic melts; controlled by initial water concentration) and CO2 (most carbon is outgassed; controlled by mantle oxygen fugacity). Overall, the surface temperatures favour liquid water over geological timescales of the stagnant-lid planet at 1 AU, although the HZ width may vary, dependent on CO2 outgassing and mantle oxygen fugacity. {\bf Another great reference to come back to for stagnant lid!} \\
{\it NOTES:} \\
{\bf ``The tectonic behaviour of a rocky body can be strongly affected by the specific thermal conditions present after planetary formation and by the particular thermochemical history experienced by the interior (refs).''} \\
{\bf ***``Based on limited evidence of solar system, the stagnant lid may conservatively be considered as the most common tectonic mode under which terrestrial bodies operate.''} \\
{\bf ***``Because of the strong exponential dependence of mantle viscosity on temperature, the relatively cold upper layers of a rocky body naturally tend to be highly stiff and form a single, immobile plate, i.e., a stagnant lid. The lid does not participate in mantle dynamics and does not allow surface materials to be directly recycled into the deep interior (refs).''} \\
{\bf ``While plate tectonics generally allows for melt production and outgassing over the entire stellar evolution, for stagnant-lid bodies these processes tend to cease within a few Gyr.''} \\
{\bf ***``Bodies more massive than the Earth might escape their stagnant-lid regime by producing a significant amount of negatively buoyant crust (ref).''} \\
{\bf ``The steep slope of the melting temperature characterizing the mantle of planets with large cores (i.e., high average density) strongly prevents the generation of partial melt and, in turn, CO2 outgassing, thereby drastically reducing the potential habitability of this kind of terrestrial bodies.''} \\
{\bf ``Our model simulates an Earth-like planet over 4.5 Gyr, starting from a post-accretion scenario when core formation and magam ocean solidification are completed.''} \\
{\bf ``The additional heat loss due to the transport of melt from the mantle source region to the surface is the so-called heat-piping effect (ref).''} \\
{\bf ***Fig. 2 shows ``solidus temperature of peridotite for different water contents***} \\
{\bf ``The surface temperature is held constant throughout the evolution in the model, since the effects of taking into account the evolution of surface temp are negligible for the interior.''} \\
{\bf ``Upon melting and subsequent melt extraction, incompatible elements, such as H2O and radiogenic elements, are enriched in the crust and depleted in the mantle, i.e., partitioning of incompatible elements due to partial melting.''} \\
{\bf ***``The buoyant melt percolates from the source region through the lithosphere and crust via porous flow or forming dykes and sills, and eventually part of this melt is extruded at the surface.''} \\
{\bf ***``Whether the water contained in the extruded melts is outgassing into the atmosphere or retained in the solidifying melt depends on its solubility in surface lavas at the evolving pressure and temperature conditions of the atmosphere (ref), whereby the effect of pressure dominates.''} \\
{\bf ``When surface melts are super-saturated, the excess pressure of the corresponding volatile is released into the atmosphere.''} \\
{\bf ``The solubility of H2O is much larger than that of CO2 (by more than two orders of magnitude at atmosphere-relevant pressures below 100 bar). As we show, it turns out that the outgassing of water can be significantly limited by its high solubility in the melts, while all extracted CO2 is easily released into the atmosphere.''} \\
{\bf ``Modelling of CO2 extraction and outgassing is complicated by the fact that carbon is not directly soluble in silicate melts, but occurs in separate phases depending on the pressure, temperature, and oxygen fugacity (ref).''} \\
{\bf ***``In stagnant-lid bodies where surface materials cannot be recycled into the interior via subduction, the mantle is likely to be characterized by relatively reducing conditions throughout its evolution.''} \\
{\bf ***``Three main parameters exert a first-order influence on outgassing history of a planet: initial mantle temperature, initial water concentration of mantle, and mantle oxygen fugacity.''} \\
{\bf ***``The present-day upper mantle of the Earth is largely depleted in incompatible elements and, based on studies of MORBs, also relatively dry with water concentration of 50-200 ppm (ref).''} \\
{\bf ``A surface albedo of 0.22 is assumed, which is larger than the mean observed value of present Earth (0.13) to mimic the reflectivity of clouds in the planetary atmosphere.''} \\
{\bf ``The thermal history is generally characterized by an initial heating phase during which convective cooling is not efficient enough to remove the internal heat generated by the decay of radioactive elements, a behaviour that is characteristic of the early evolution of the interior of stagnant-lid bodies (refs).''} \\
{\bf ``A low viscosity causes convection, and hence heat loss, to be more efficient (Eq. 14) with the consequence that models that have the highest initial water content -- and hence the lowest reference viscosity -- are also characterized by the shortest heating phase and, in turn, by the lowest temperature at the end of the evolution.''} \\
{\bf ***``Because of the strong exponential dependence of the viscosity on temperature, in fact, an increase in mantle temperature is accompanied by a viscosity reduction that promotes convection and leads to a more rapid heat loss. Upon cooling, the viscosity increases, slows convection down, and renders heat transfer less efficient.''} \\
{\bf ***``The initial phase of mantle heating leads to the production of a large volume of partial melt, which causes the crust to grow more rapidly the higher the initial H2O concentration is.''} \\
{\bf ``The accompanying increase in the temperature drop across the bottom thermal boundary layer could thus favour the formation of plumes and the production of partial melt, which could potentially lengthen the degassing lifetime of the planet.''} \\
{\bf ``As expected for a mantle undergoing partial melting over its entire evolution, the concentration of water decreases continuously over time, reaching about 80\% of its initial value after 4.5 Gyr. The rate of water extraction diminishes at the time when subcrustal erosion starts as a consequence of the recycling of wet crust into the mantle.''} \\
{\bf ``In general, as the degree of partial melting increases because of mantle heating, the concentration of water that partitions into the melt decreases, and vice versa.''} \\
{\bf ***``Despite the high values of water concentration in the surface melt that are achieved during the second outgassing phase, the increase in the partial pressure of atmospheric water is relatively small compared to the first phase because the overall volume of partial melt produced declines significantly with time, and because the continuous increase of the total atmospheric pressure makes further degassing, of H2O in particular, increasingly difficult.''} \\
{\bf ``CO2 outgassing instead depends on the assumed oxygen fugacity of the mantle, and it is outgassed much more easily. The partial pressure of outgassed CO2 rises monotonically as long as partial melt is produced, which, in these models, occurs over the entire evolution.''} \\
{\bf ***``As expected, the higher the initial water concentration, the higher are the final partial pressures of the two gases. For CO2, a high water concentration in the mantle causes a strong decrease of solidus temperature, which facilitates the production of large volumes of partial melt. For H2O, in addition to the above reason, a high initial concentration clearly makes the amount of water available for outgassing high as well with the consequence that there is a range of final partial pressures.''} \\
{\bf ***``Water outgassing is suppressed even more strongly as more and more oxidizing conditions are ssumed for the mantle. A higher oxygen fugacity leads in fact to a higher amount of outgassed CO2 whose increasing pressure tends to prevent the concentration of H2O in surface melts from exceeding its saturation level.''} \\
{\bf ``As long as the mantle oxygen fugacity is relatively low, the amount of outgassed CO2 is very limited.''} \\
{\bf ``The final partial pressure of CO2, which is much less soluble than water in basalts, is largely determined by the choice of the oxygen fugacity.''} \\
{\bf ``The stagnant-lid planets show habitable surface conditions over most of their history. Only during the early evolution up to 500 Myr, some scenarios -- with relatively low initial mantle concentrations of water or low oxygen fugacities -- show temperatures below 273 K.''} \\
{\bf ***``The greenhouse effect of water is so strong that the difference in CO2 of about 0.7 bar arising from the use of different initial values of mantle water concentration does not exert a significant effect.''} \\
{\bf ``Most of the water is stored in the ocean and the atmospheric abundance defined via the saturation vapour pressure is comparatively low at the surface temperatures obtained for an orbital distance of 1 AU.''} \\
{\bf ***``After 4.5 Gyr, a water amount of about 9 bar, which corresponds to about 3\% of an Earth ocean, would form an ocean on the stagnant-lid planet for this scenario, which corresponds to an equivalent depth of 85 m.''} \\ 
{\bf ``The outer edge of the HZ strongly varies in time in response to the outgassing from the interior, mainly of CO2, whose abundance increases during the evolution.''} \\
{\bf ***``The width of the HZ increases with time owing to the increase in atmospheric CO2 and is larger when accounting for the solar evolution because of the $r^{-2}$ dependence of the stellar flux at the position of the planet.''} \\
{\bf ``While the amount of CO2 outgassed from the interior increases with the oxygen fugacity, the amount of water decreases because of its increased solubility in surface lavas.''} \\
{\bf ``In stagnant-lid bodies the presence of a thick, immobile lithosphere limits the production of partial melt, making this possible only at large depths where the solidus is relatively high. As the mantle cools and the lithosphere thickens, melting tends first to wane and then to vanish completely within a few Gyr \citep{kite09}.''} \\
{\bf ``In a planet with an active lid, where thin tectonic plates form a lithosphere, melting can take place over a wide pressure range up to shallow depths of a few km benath MORs with the consequence that outgassing can typically occur over the entire lifetime of the star despite cooling of the interior \citep{kite09}.''} \\
{\bf ``The recycling of water on Earth is considered an oxidizing agent reintroduced into the mantle chiefly at subduction zones (ref).''} \\
{\bf ``Carbon in the upper mantle of Earth is stored in a number of accessory phases rather than being bound to silicates as in the case of water (ref).''} \\
{\bf ``For Earth, plate tectonics could have followed an initial stagnant-lid state (ref) and could even be a transient rather than an end-member state \citep{oneill16}.''} \\
{\bf ``While on Venus CO2 resides mostly in the atmosphere, on Earth it is largely contained in carbonate rocks.''} \\
{\bf ``A thick primordial atmosphere (neglected here) in excess of tens to hundreds of bars of H2O and CO2 could be generated by catastrophic outgassing of a magma ocean (refs).''} \\
{\bf ``Depending on the pressure of a primordial atmosphere and on its resilience to escape, volcanic outgassing of H2O could be easily suppressed from the very beginning of the evolution, and subsequent outgassing of CO2 could also be hindered.''} \\
{\bf ``Life as we know it has three basic requirements, namely (i) an energy source; (ii) a solvent; and (iii) the potential for molecular complexity (ref).''} \\
{\bf ***``Water loss is thought to be efficient if stratospheric water vapour volume mixing ratios exceed a critical value of about $3 \times 10^{-3}$ (ref).''} \\
{\bf ***``The inner boundary of the HZ of the stagnant lid planets generally lids further away from the star the smaller the water reservoirs obtained via outgassing from the interior are.''} \\
{\bf ***``For even smaller water reservoirs than those found here, the HZ could move inwards even further, as shown by, for example, \cite{abe11}, since the greenhouse effect of water would be very low at the resulting low atmospheric volume mixing ratios.''} \\
{\bf ***``\cite{kasting93} found that the runaway greenhouse limit for a planet with a water reservoir of one Earth ocean does not depend on the amount of CO2 in the atmosphere, while the water loss limit occurs at smaller orbital distances if the atmosphere contains more CO2.''} \\
{\bf ``The outer boundaries of the HZ determined for stagnant-lid planets depend mainly on the oxygen fugacity.''} \\
{\bf ``Liquid water on the surface, whether in the form of a global or regional ocean, could provide a sink for the atmospheric CO2. Such a sink is temperature dependent with a higher amount of CO2 trapped in the ocean for colder temperatures (refs).''} \\
{\bf ``The low-temperature alteration of the uppermost part of the ocean floor in the modern Earth, in which carbonates form from basaltic rock.''} \\
{\bf ``A synopsis of data from the literature suggests that although young volcanic crustal rock may not take up a major part of the CO2 and water, the oceanic crust remains an active sink as it ages and continues to react for some tens of Myr (ref).''} \\
{\bf ``The mode of crust production is absent for stagnant-lid planets, as mid-ocean ridges would also be absent: the crust would be built by intrusions and, in a top-down manner rather than lateral addition, by extrusives.''} \\
{\bf ***``In a planet with plate tectonics, the availability of carbon sinks would be even more important to prevent build-up of massive CO2 atmosphere because the increase in oxygen fugacity may reinforce CO2 degassing.''} \\
 
\subsection{The habitability of Proxima Centauri b II. possible climates and observability \citep{turbet16}}

\subsection{Revised mass-radius relationships for water-rich rocky planets more irradiated than the runaway greenhouse limit \citep{turbet20}}
Mass-radius relationships are given, as equations and tables, for rocky exoplanets that are interior to the runaway greenhouse limit of their star's HZ, now accounting for the "radius inflation effect" caused by large steam atmospheres. \\
{\it NOTES:} \\
{\bf ``Bulk water content is usually underestimated with previous relationships. This result applies to TRAPPIST 1-b, c, d, which can accommodate a water mass fraction of at most 2, 0.3, and 0.08\%, respectively, assuming planetary core with a terrestrial composition.''} \\
{\bf ***``Our results demonstrate that non-H2/He-dominated atmospheres can have a first-order effect on the mass-radius relationships, even for rocky planets receiving moderate irradiation.''} \\
{\bf ``Most of the seven TRAPPIST-1 planets are likely enriched in volatiles (e.g., water) up to several tens of percent of planetary mass.''} \\
{\bf ***``Turbet et al.(2019) in fact recently showed that water-rich planets receiving more irradiation than the runaway greenhouse irradiation threshold should suffer from a strong atmospheric expansion compared to planets receiving less than this threshold -- the runaway greenhouse radius inflation effect.''} \\
{\bf ***``Planets endowed with a steam H2O atmosphere have a significantly larger radius than icy or liquid ocean planets, for a given water-to-rock ratio.''} \\
{\bf ``Our simulations show that transit radius pressure varies little across the range of parameters we explored, and is roughly equal to $10^{-1}$ Pa.''} \\
{\bf ``The main result of this work is summarized in Figure 2, which shows how mass-radius relationships can vary depending on if water is treated as a condensed layer \citep{zeng16} or as an atmosphere (this work).''} \\
{\bf ***``Mass-radius relationships in the steam atmosphere configuration give -- for a given planet mass -- a significantly larger radius than in the condensed water configuration.''} \\
{\bf ``Fig. 2: Mass-radius relationships for various interior compositions and water content, assuming water is in the condensed form (left panel) and water forms an atmosphere (right panel)... Based on the irradiation they receive compared to the theoretical runaway greenhouse limit (refs), TRAPPIST-1e, f, g, and h should be compared with the mass-radius relationship on the left, while TRAPPIST-1b, c, d should be compared with those on the right.''} \\
{\bf ***``It has been speculated that some planets in the TRAPPIST-1 system may be enriched in water, possibly up to tens of percent for some of them (refs).''} \\
{\bf ***``Our results suggest that the three innermost planets of TRAPPIST-1 -- and more particularly, TRAPPIST-1b and d, for which TTV measurements point toward particularly low bulk densities (ref) -- do not necessarily need to be highly enriched with water to reach their measured density.''} \\
{\bf ``(ref) evaluated that the current rate of water loss can be as high as 0.19, 0.06, and 0.18\% per Gyr by mass for TRAPPIST-1b, c, d, respectively.''} \\
{\bf ***``Putting these pieces of information together, it is likely that the three inner TRAPPIST-1 planets may all be completely dry today.''} \\
{\bf ***``The runaway greenhouse transition allow planets to jump from one mass-radius relationship to another, which makes it possible to break the composition degeneracy.''} \\
{\bf ``A direct consequence is that any significant deviation of planetary densities from an interior isocomposition mass-radius relationship would be a strong indication that (i) there is either today large reservoirs of water or volatiles on at least some planets of TRAPPIST-1, or that (ii) there are significant differences in TRAPPIST-1 planets' core composition.''} \\
{\bf ***``For a given water-to-rock mass ratio, our revised mass-radius relationships lead to planet bulk densities much lower than calculated when most water is assumed to be in condensed form.''} \\
{\bf ***``These results demonstrate that non-H2/He dominated atmospheres can have a first-order effect on the mass-radius relationship even for Earth-mass planets receiving moderate irradiation.''} \\
{\bf Appendix D contains a ``quick guide on how to build mass-radius relationships for water-rich rocky planets more irradiated than the runaway greenhouse limit} \\

\subsection{Internal structure of massive terrestrial planets \citep{valencia06}}
{\bf Summarize this.}

\subsection{The trouble with water: condensation, circulation, and climate \citep{vallis20}}
A review of the hydrological cycle on Earth (water) and Titan (methane), focusing on the atmosphere (condensation, circulation, etc.) and surface processes that can impact climate. Equations governing the water cycle in Earth's atmosphere are outlined in Sections 2, 3, 4. \\
{\it NOTES:} \\
{\bf ``Condensation releases heat, and that heat affects the circulation of the atmosphere and makes the circulation different from what it might be in a dry atmosphere, especially in the tropics.''} \\
{\bf ***Section 2 and 3 contains the basic equations for working with the hydrological cycle, including the Clausius-Clapeyron expression, saturation pressure, relative humidity, etc.} \\
{\bf ``In Earth's atmosphere the temperature falls in the vertical by roughly 6 degrees per km, and consequently the saturation vapour pressure has an approximate e-folding height of about 2.5 km.''} \\
{\bf ``Fast condensation upon saturation is normally a good approximation to make, and is not the main source of problems involving water vapour and the large-scale circulation.''} \\
{\bf ``A parcel can become saturated either by moving to a lower temperature (conserving specific humidity but with saturation pressure falling) or by moving to a higher pressure (conserving specific humidity and saturation pressure, but with saturation humidity falling.''} \\
{\bf ``Equation 8 (relative humidity) explains many large scale features of relative humidity, including very low levels of relative humidity in the stratosphere: Since temperature tends to increase above the tropopause, the tropopause is a cold trap and relative humidity is very low beyond it.''} \\
{\bf ``The relative humidity on Earth varies considerably with longitude in the mid-latitudes because of the chaotic advection by baroclinic eddies.''} \\
{\bf ``Clouds are widely regarded as the greatest single uncertainty in global warming calculations.''} \\
{\bf ***``The results show that over the ocean the changes roughly correspond to the precipitation pattern itself -- high in the tropics, low in the subtropics - whereas that pattern is not at all visible over land.''} \\
{\bf ``The simple point to make is that circulation changes can be large and can overwhelm thermodynamic changes, especially but not only over land. Further, those circulation changes may be induced by the latent heat release itself.''} \\
{\bf ***``In general, a hydrology cycle will exist in a planetary atmosphere if that atmosphere contains a condensible; that is, a gas that is likely to undergo phase changes at the typical temperatures within the atmosphere, from the gaseous to the solid or liquid phase.''} \\
{\bf ***``A useful rough equivalence for water is that an evaporative flux of 100 W/m$^2$ corresponds to about 1 m/year evaporation or precipitation. If the surface were dry, the drying time for Earth's atmosphere would be of order a year or less.''} \\

\subsection{Water in star-forming regions: Physics and chemistry from clouds to disks as probed by Herschel spectroscopy \citep{vandishoeck21}}

\subsection{Setting the stage: planet formation and volatile delivery \citep{venturini20}}
{\it NOTES:} \\
{\bf ``Protoplanetary disks last typically between 1 and 10 Myr (refs).''} \\
{\bf ``Recent measurements of isotopic ages and paleomagnetic analyses of chondrites suggest that the solar nebula was probably gone after $\sim 3.8-4.5$ Myr (refs), in agreement with what is inferred from protoplanetary disk observations.''} \\
{\bf ``The central idea of the core accretion model is that a planet forms first by the accumulation of solids or heavy elements into a core, followed by the binding of an atmosphere on top (refs). Since the dominant gas constituents of the disk are H2 and He, this is usually considered the main composition of primordial atmospheres.''} \\
{\bf ``If the embryo reaches a mass of $\gtrsim 0.5-0.8 M_{\oplus}$ while still embedded in the gas disc, it can bind a primordial atmosphere of $\sim10^{-4}-10^{-2} M_{\oplus}$ (ref). The primordial atmosphere can be lost by boil-off in the $\sim10^5$ years following the disk's dispersal, depending on the planet's mass and proximity to the star (ref).''} \\
{\bf ``In a smooth disk, the pressure increases outwards. As a consequence, the gas moves at a slower rate than the pebbles. Hence, from the pebbles reference frame, the gas acts like a headwind. This provokes a strong orbital decay towards the central star.''} \\
{\bf ``Because of this radial drift barrier, the growth of dust aggregates larger than cm/m has been a puzzle for planet formation theory for decades (ref).''} \\
{\bf ``Dust particles move slower towards the star, piling up particles coming from farther out. These pile ups or clumps of dust eventually collapse into planetesimals by self-gravity.''} \\
{\bf ``(ref) found that corotation torques can abate migration significantly, and even reverse it.''} \\
{\bf ***``Water could be produced within a dry protoplanet from a reaction between H2 and FeO (ref), or retained in olivine grains via adsorption (ref).''} \\
{\bf ***``Earth's mantle water budget could be 0.3-8 Earth oceans today (and possible between 10-50 Earth oceans in the past). The core is more poorly constrained, and could contain between less than 0.1 Earth oceans and 80 Earth oceans. A maximum amount of water on Earth would be $\sim0.1$\%-$0.2$\% $M_{\oplus}$.''} \\
{\bf ***``Water could have been adsorped in situ, from the primordial nebula, by olivine grains. Alternatively, water could have been produced in situ by a reaction between the H2 of a primitive atmosphere with the iron-oxides of the magma ocean (ref).''} \\
{\bf ***``This Ikoma-Genda mechanism (magma ocean) could have operated on Earth. In their scenario, once the protosolar disc is gone, the planet cools and the ocean forms by condensation of the steam atmosphere in about a thousand years.''} \\
{\bf ``Measurements from lavas (representative of Earth's mantle, were water could have been kept pristine) point toward a lower D/H ratio, closer to the protostellar value.''} \\
{\bf ***``Indeed, calculations based on the abundance of terrestrial Ne isotopes suggest that about 10\% of Earth's water could have originated from a primordial H2 atmosphere (ref).''} \\
{\bf Figure 3 shows a ``scheme of the main processes that would contribute to the origin of water on Earth''} \\
{\bf `***`Studies suggest that planetesimals and/or embryos from the outer asteroid belt were the main source of Earth's volatiles. Numerical studies give support to the hypothesis of water being delivered to Earth mianly by C-type asteroids (the parent bodies of carbonaceous chondrites).''} \\
{\bf `***`Studies show that when the gas disc dissipates, the terrestrial planets form on a timescale ranging from $\sim30$ to 100 Myrs. The gas giants, located farther out and already formed, act as dynamical perturbers that excite the bodies of the amin belt, dispering/scattering water-rich bodies of the outer belt towards the inner regions of the disc where terrestrial planet formation takes place. (ref) showed that the fraction of water on Earth could be justified by this mechanism, and other works on the evolution of the solar system support this idea (refs).''} \\
{\bf ``(ref) analyzed the water delivery to Earth in the Grand Tack context and showed that after disk dispersal, a fraction of the scattered C-type asteroids could reach the Earth and could explain its current water content.''} \\
{\bf ``The Grand Tack is therefore successful in explaining the volatile content of Earth, the low mass of Mars, and the low-density and composition dichotomy of the current asteroid belt (inner S-type and outer C-type asteroids).''} \\
{\bf ``Some studies have pointed out the low probability of the Grand Tack occurring (refs), but the solar system is a sample size of only 1.''} \\
{\bf ``Alternatives exist; (ref) propose a different mechanism to pollute the inner regions of the disc with water-rich bodies before terrestrial planets' main growth.''} \\
{\bf ***``A major problem that occurs is how to leave the inner regions of a planetary system dry, since icy pebbles drift and very likely reach the inner regions as the disc cools, polluting it with volatiles (refs).''} \\
{\bf ``The formation of Jupiter can possibly explain this: water-rich planetesimals were probably scattered to the forming rocky planets.''} \\
{\bf ``Perhaps it is not pebbles versus planetesimals, but a dominance of pebble accretion at the earliest stages to make the core grow fast, followed by a later accretion of planetesimals to delay gas accretion for a few Myr. Hence, this hybrid model might also help to understand the abundance of intermediate-mass exoplanets.''} \\
{\bf ``The Kepler mission revealed that one of the most common type of exoplanets are those with a radius between that of Earth and Neptune (ref).''} \\
{\bf ``The dearth of the radii distribution, which is bimodal, is known as the photoevaporation valley.''} \\
{\bf ``In the core-powered mass loss scenario (ref), the heat remaining from formation strips off tenuous atmospheres once the disc dissipates.''} \\
{\bf ``In the view of photoevaporation models, the first peak of the radii distribution coresponds to naked rocky cores, while the planets from the second peak should possess some gaseous envelope.''} \\
{\bf ``Owen \& Wu(2017) show that the timescale to lose the envelope by photoevaporation is the longest if the envelope represents $\sim1-10$\% of the planet's mass. Planets with such atmospheres are then more stable against photoevaporation and would constitute the second peak of the Fulton distribution.''} \\
{\bf ``Heat of the core strips atmospheres if they are less than 5\% the mass of the core. Atmospheres with masses above this threshold will compactify, increasing their binding energy and making mass loss more difficult, while atmospheres with initial mass below the threshold will expand, and once the process is triggered, the smaller is the binding energy, the easier it is for the remaining atmosphere to be removed.''} \\
{\bf ``The position of the valley is extremely sensitive to the composition of the core.''} \\
{\bf ``Small mass planets ($M \lesssim 10 M_{\oplus}$) with rocky cores and water envelopes could contribute to populate the non-empty valley and the second peak of the distribution.''} \\
{\bf ***``But this poses a problem for formation models, which tend to show that planets with non-negligible H-He atmospheres contain large portions of water (refs).''} \\
{\bf ``More massive discs are able to form larger dry super-Earths, but, at the same time, form larger icy objects which migrate to the inner regions more efficiently (ref).''} \\
{\bf ***``New constraints for planet formation are rising from ALMA observations of disc structure and composition (ref).''} \\
{\bf ***``The conclusion strengthen the view that volatiles (and particularly water) on Earth were delivered mainly from chondritic material (asteroids and not comets) during and/or after the disc dispersal (refs).''} \\
{\bf ``It is widely accepted that Jupiter acted as a barrier that prevented carbonaceous chondritic and water-rich material from reaching $\sim1$ AU dring the gas disc phase (refs).''} \\
{\bf ``Both cosmochemical data and formation theory point towards the major role of Jupiter in keeping the inner solar system dry.''} \\
{\bf ``A combination of radii determination and thermal evolution models suggest that exoplanets with periods less than $\sim100$ days and radii smaller than that of Neptune are water-poor.''} \\
{\bf ***``\citep{lichtenberg19} recently showed that when the heating from radioactive decay is included in planet formation with planetesimal accretion, the final outcome is much drier planets.''} \\
{\bf ``Water might also be lost during planet evolution, for exaple due to the remnant heat from the core (ref).''} \\
 
\subsection{A negative feedback mechanism for the long-term stabilization of Earth's surface temperature \citep{walker81}}
First paper to propose silicate weathering thermostat?

\subsection{Investigating ocean island mantle source heterogeneity with boron isotopes in melt inclusions \citep{walowski19}}
{\bf Summarize this.}

\subsection{Europium as a lodestar: diagnosis of radiogenic heat production in terrestrial exoplanets. Spectroscopic determination of Eu abundances in $\alpha$ Centauri AB \citep{wang20}}

\subsection{The dynamics of a rapidly escaping atmosphere: applications to the evolution of Earth \& Venus \citep{watson81}}

\subsection{Climates of warm Earth-like planets I. 3D model simulations \citep{way18}}

\subsection{The role of hydrological process in ocean-atmosphere interaction \citep{webster94}}

\subsection{The California-Kepler Survey V. Peas in a pod: planets in a Kepler multi-planet system are similar in size and regularly spaced \citep{weiss18}}

\subsection{Abundant atmospheric methane from volcanism on terrestrial planets is unlikely and strengthens the case for methane as a biosignature \citep{wogan20}}

\subsection{Equilibrium chemistry down to 100 K: Impact of silicates and phyllosilicates on the carbon to oxygen ratio \citep{woitke18}}

\subsection{Coexistence of CH4, CO2, and H2O in exoplanet atmospheres \citep{woitke20}}

\subsection{The framework for 0-D atmospheric modelling (F0AM) v3.1 \citep{wolfe16}}

\subsection{A planetary system around the millisecond pulsar PSR1257+12 \citep{wolszczan92}}
First exoplanet discovery paper

\subsection{Water loss from terrestrial planets with CO\textsubscript{2}-rich atmospheres \citep{wordsworth13}}
Previous studies of CO\textsubscript{2}-influenced water loss only focused on the lower atmosphere; this study includes calculations in the middle and upper atmosphere as well, and indicates that many Earth-like HZ exoplanets will be stable waterworlds with high surface temperatures for much of their lifetimes. This is detrimental for silicate weathering, which requires exposed land, and for habitability, unless seafloor weathering efficiently removes CO\textsubscript{2} from the atmosphere. CO\textsubscript{2} only causes an increase in temperature and water loss for a narrow region of parameter space, due to the cold-trap protection. Water loss is strongly limited for planets orbiting G stars (such as the Earth), and planets orbiting M stars will likely lose $1-10$ terrestrial oceans (TO), strongly dependent on orbital distance, but the amount depends on whether N\textsubscript{2}/CO\textsubscript{2} is lost during early evolution. {\bf MUCH more detail to this paper than stated here; should come back and read many times. Good explanation of waterworld habitability in Discussion section. The atmospheric evolution model could help with additions to my model eventually.}

\subsection{Abiotic oxygen-dominated atmospheres on terrestrial habitable zone planets \citep{wordsworth14}}
O\textsubscript{2} should not be taken as a ``smoking-gun'' biosignature, as it may become the dominant atmospheric gas, as water photolysis leads to significant build-up in atmospheres with low concentrations of non-condensing gases due to inefficiency of the cold-trap mechanism. This is strongly dependent on the planet's accretion history, internal chemistry, atmospheric dynamics, and orbital state. Specifically tested in this paper, a planet with a low N\textsubscript{2} atmosphere, compared to Earth, will have a moist upper atmosphere, meaning water photolysis and loss will be more likely (this is especially a problem around young dwarf stars very active in the XUV).

\subsection{Redox evolution via gravitational differentiation on low-mass planets: implications for abiotic oxygen, water loss, and habitability \citep{wordsworth18}}
The study of potential abiotic oxygen build-up is extended from \citep{wordsworth14}, using \citep{schaefer16} as a basis, including both interior-atmosphere exchange and atmospheric loss. While mainly focused on discovered planets orbiting M dwarfs, there are implications for abiotic build-up on planets orbiting G stars as well. Planets orbiting farther from their stars are less likely to build up abiotic O\textsubscript{2} atmospheres, and, in contrast to \citep{luger15}, oxygen liberated in water photolysis during pre-MS evolution reacts strongly with the molten iron mantle; once a planet has solidified, however, it can still build up significant abiotic oxygen in its atmosphere if a cold trap is not present. Exoplanets receiving lower amounts of stellar flux from M dwarfs are the least likely to build up abiotic O\textsubscript{2} atmospheres. While the results are promising, the authors suggest a nominal N\textsubscript{2}-H\textsubscript{2}O-CO\textsubscript{2} HZ to detect biotic oxygen and possibly life. "Fig. 12 captures an exoplanet's atmospheric state immediately after the magma ocean phase has finished." {\bf Read again with more geochemistry background later, especially the summaries at the end of sections/subsections. Note: I don't understand Figure 7's escape rates for Earth vs. super-Earth, nor its explanation -- come back to this, or ask Wordsworth? BOTH the atmospheric modelling $+$ escape, and the magma ocean modelling, could be useful later, although they use extensive geochemistry.}

\subsection{The stellar-activity-rotation relationship and the evolution of stellar dynamos \citep{wright11}}

\subsection{Solar-type dynamo behaviour in fully convective stars without a tachocline \citep{wright16}}
The rotation-activity relationship (X-ray luminosity vs. Rossby number) for solar-type and early M (partly convective) stars is overlaid with four slow-rotating, fully-convective stars. The results indicate that the relationship also holds for fully convective stars as well, with an early saturation phase and an exponential decrease with time as the stars spin down. Therefore, the relationship is the same for partly convective and fully convective stars. \\
{\it NOTES:} \\
{\bf This paper contains some good background info on the solar-type star dynamo, magnetic field production, and X-ray emission.} \\
{\bf ``X-ray emission is a reliable proxy for magnetic activity.''} \\
{\bf ``In solar-type stars, X-ray emission is observed to increase monotonically with increasing stellar rotational velocity for periods exceeding a few days... This relationship has been observed in stars from late F-type through to early M-type, that is, those with radiative cores and convective envelopes.''} \\
{\bf ``For very fast rotators, the rotation-activity relationship has been found to break down, with X-ray luminosity reaching a saturation level of approximately $L_X/L_{bol} \approx 10^{-3}$, independent of spectral type.''} \\
{\bf ``Fully convective stars are common and, given their spin-down times of a few billion years, at least half of all such stars are expected to be slow rotators.''} \\
{\bf ***``The results show that fully convective stars, at least when they have spun down sufficiently, operate a dynamo that exhibits a rotation-activity relationship that is indistinguishable from that of solar-type stars.''} \\
{\bf ``It seems unlikely that both partly and fully convective stars would have the same rotation-activity relationship (requiring both their dynamo efficiency and rotational dependence to behave in the same way) without their dynamo mechanism sharing a major feature.''} \\

\subsection{The stellar rotation-activity relationship in fully convective M dwarfs \citep{wright18}}

\subsection{Detectability of atmospheric features of Earth-like planets in the habitable zone around M dwarfs \citep{wunderlich19}}
{\bf Summarize this.}

\subsection{Detectability of biosignatures on LHS 1140 b \citep{wunderlich20}}

\subsection{Impact of stellar superflares on planetary habitability \citep{yamashiki19}}

\subsection{Stabilizing cloud feedback dramatically expands the habitable zone of tidally locked planets \citep{yang13}}
The first study to use 3D global climate models (GCMs) to study the effect of water clouds on the habitability of (tidally-locked) planets orbiting M dwarfs. The clouds are shown to provide a stabilizing feedback, extending the inner edge of the HZ (through stronger convection and higher albedo) to twice the previously calculated stellar flux values, dramatically increasing the frequency of HZ planets.

\subsection{Observability of ultraviolet N I lines in the atmosphere of transiting Earth-like planets \citep{young20}}

\subsection{The MUSCLES treasury survey. II. Intrinsic Ly$\alpha$ and extreme ultraviolet spectra of K and M dwarfs with exoplanets \citep{youngblood16}}
Eleven low-mass exoplanet-bearing stars (4 K dwarfs, 7 M dwarfs) were observed, and MCMC was used to reconstruct the extreme-UV spectrum based on Ly$\alpha$ observations. The authors find a positive correlation between Ly$\alpha$ surface flux and Mg II surface flux, and a negative correlation between Ly$\alpha$ and stellar rotation period. \\
{\it NOTES:} \\
{\bf ``Most targets are optically inactive, but all exhibit significant UV activity.''} \\
{\bf ``The Ly$\alpha$ surface flux positively correlates with the Mg II surface flux and negatively correlates with the stellar rotation period.''} \\
{\bf ``Ultraviolet (UV) photons control the upper atmospheric heating and chemistry of exoplanets.''} \\
{\bf ``Joshi (2003) showedthat close-in, tidally locked exoplanets can maintain a habitable climate as long as significant surface pressure exists to support atmospheric heat transport to the nightside.''} \\
{\bf ``The high-energy flux with which M dwarfs irradiate their exoplanets may be similar to the young Sun (ref).''} \\
{\bf ``UV-driven chemistry can mimic certain biosignatures, complicating the analysis.''} \\
{\bf ***``Ly$\alpha$ and other bright chromospheric and transition region resonance lines in the far UV (912-1700 Angstroms) photodissociate CO2 and H2O, resulting in a buildup of abiotic O2. O2 dissociatio by 1200-2400 Angstrom photons leads to a three-body reaction involving O and O2 that accumulates abiotic O3 in the exoplanet's atmosphere.''***} \\
{\bf ****``Ly$\alpha$ comprises $\sim37-75$\% of the total 1150-3100 Angstrom flux from most M dwarfs (ref), and is also one proxy used to estimate the extreme UV (100-912 Angstroms) flux, which heats exoplanetary upper atmospheres, driving exoplanet mass loss (refs).''***} \\
{\bf ***Table 2 shows emission line fluxes of the 11 MUSCLES targets (7 M dwarfs, 4 K dwarfs)***} \\
{\bf ***Figure 5 shows EUV spectra of 2 M dwarfs and 1 K dwarf from sample***} \\
{\bf ***``The correlations in Figure 9 suggest that slowly rotating M dwarfs will exhibit less Ly$\alpha$ flux than slowly rotating K dwarfs, but beyond the cross-over point of the relation, a faster-rotating M dwarf may exhibit more Ly$\alpha$ flux than a similarly rotating K dwarf.''***} \\
{\bf ****``The Ly$\alpha$ and extreme-UV fluxes should be positively correlated because they both depend on magnetic heating rate.''***} \\
{\bf ***``\citet{claire12} showed that the fraction of Ly$\alpha$ photons with respect to all photons at $\lambda < 1700$ Angstroms is $\sim40$\% throughout the Sun's history, and this supports Ly$\alpha$ flux as a good predictor of the extreme-UV flux.''***} \\
{\bf ``For the Sun, de Wit et al.(2005) did not find that Ly$\alpha$ is the best emission line to predict the extreme-UV spectrum.''} \\
{\bf ***``The intrinsic Ly$\alpha$ surface flux correlates positively with Mg II, another chromospheric emission line, for both K and M dwarfs.''***} \\

\subsection{Abrupt climate transition of icy worlds from snowball to moist or runaway greenhouse \citep{yang17}}

\subsection{The ability of significant tidal stress to initiate plate tectonics \citep{zanazzi19}}
{\bf Summarize this.}

\subsection{Rapid formation of super-Earths around low-mass stars \citep{zawadzki21}}
{\bf XXXX Important reference for determining initial conditions of simulation??? XXXX}

\subsection{Mass-radius relation for rocky planets based on PREM \citep{zeng16}}
{\it NOTES:} \\
{\bf ***``$R/R_{\oplus} = (1.07 - 0.21 \times CMF) \times (M/M_{\oplus}^{1/3.7}$, where CMF stands for Core Mass Fraction. It is applicable to $1\sim8 M_{\oplus}$ and CMF of $0.0\sim0.4$.''} \\
{\bf ``The Mg/Si ratio only matters towards the low-pressure end, which is captured in our EOS by using the PREM density variation in the upper-mantle pressure range.''} \\
{\bf ***Table 2 is the mass-radius table for various compositions; Section 3.1 contains the equations for the mass-radius relationship; all information is available at \url{https://www.cfa.harvard.edu/~lzeng/planetmodels.html}} \\
{\bf ***``This study is backed up by recent studies of disintegrated planet debris in polluted white dwarf spectra (refs). These studies show that the accreted extrasolar planet debris generally resemble bulk Earth composition ($>85$\% by mass composed of O, Mg, Si, Fe), similar Fe/Si and Mg/Si ratio and are carbon-poor. It indicates formation processes similar to those controlling the formation and evolution of objects in the inner solar system (ref).''} \\
{\bf ``For more massive planets, the effect of Mg/Si tends to be smaller.''} \\
{\bf ***``The core mass fractions of Earth and Venus (ref) are around 0.3.''} \\
{\bf ``These dense exoplanets between 2 and 5 $M_{\oplus}$ so far appear to agree with the mass-radius relation with CMF $\approx 0.26$, suggesting that they are like the Earth in terms of their proportions of mantle and core.''} \\

\subsection{Role of temperature-dependent viscosity and surface plates in spherical shell models of mantle convection \citep{zhong00}}

\subsection{Implications of a long-lived basal magma ocean in generating Earth's ancient magnetic field \citep{ziegler13}}

\subsection{Toward the minimum inner edge distance of the habitable zone \citep{zsom13}}
{\bf Summarize this.}

\section{Equations for Talks}
\begin{equation}
    F_{\mbox{regas/degas}} \propto P_{\mbox{seafloor}} = g \rho_w d_w
\end{equation}

\begin{equation}
    F_{\mbox{loss}} \propto \mbox{exp}\left( \frac{-t}{t_{\mbox{loss}}} \right)
\end{equation}

\begin{equation}
    \frac{d \tilde{T}}{d \tau_{CA}} = \frac{\tau_{heat}}{\tau_{CA}} \left( \exp({-t/\tau_{decay}}) - \tilde{F}_0 \tilde{f}_w^{\beta} (\tilde{T} - \tilde{T}_s)^{\beta+1} \exp \left[ -
    \frac{\beta}{\tilde{T}_m} \left(\frac{1}{\tilde{T}} - 1 \right) \right] \right)
\end{equation}

\begin{equation}
    \mathrm{H}_2\mathrm{O} + \mathrm{XUV} \rightarrow 2\mathrm{H} + \mathrm{O}
\end{equation}

\begin{equation}
    W_{\mathrm{loss}} = 100 \times w_{\uparrow,0,\mathrm{CA}}
\end{equation}

\begin{table}[htb]
    \centering
    \begin{tabular}{c|c|c}
    \hline
        Input Parameters & Constants & Parameters to Vary  \\
    \hline
         Mantle water content $x$ & Mantle mass fraction ${f_{\mathrm{m}}}^8$ & Degassing exponent $\mu$ \\
         Surface water content $s$ & Mass of Earth ${M_{\oplus}}^6$ & Regassing exponent $\sigma$\\
         Mantle temperature $T$ & Radius of Earth ${R_{\oplus}}^6$ & Degassing efficiency $\chi$,~ $\chi_{\mathrm{d}}$ \\
         Loss rate ${W_{\mathrm{loss}}}^\dagger$ & Melt degassing fraction of Earth ${f_{\mathrm{degas},\oplus}}^5$ & Regassing efficiency $\chi_{\mathrm{r}}$ \\
         Loss timescale ${t_{\mathrm{loss}}}^\dagger$ & Hydration depth of Earth ${d_{\mathrm{h},\oplus}}^3$ & Radiogenic decay timescale $\tau_{\mathrm{decay}}$ \\
         M-Earth mass $M_{\mathrm{p}}$ & Depth of basaltic oceanic crust ${d_{\mathrm{b}}}^6$ & Ocean basin covering fraction $f_{\mathrm{b}} $\\
         M-Earth radius $R_{\mathrm{p}}$ & Seafloor pressure of Earth ${P_{\oplus}}^8$ & \\
         M-dwarf mass $M_{\mathrm{*}}$ & Mid-ocean ridge melting depth ${d_{\mathrm{melt}}}^6$ & \\
         Saturation fraction $f_0$ & Upper mantle density ${\rho_{\mathrm{m}}}^4$ & \\
         Saturation timescale $t_0$ & Crust density ${\rho_{\mathrm{c}}}^1$ & \\
         Orbital distance $a$ & Upper mantle critical Rayleigh number ${\mathrm{Ra}_{\mathrm{crit}}}^4$ & \\
         M-Earth albedo $A$ & Heat loss exponent ${\beta}^2$ & \\
          & Upper mantle thermal conductivity ${k}^2$ & \\
          & Upper mantle thermal expansivity ${\alpha}^7$ & \\
          & Upper mantle thermal diffusivity ${\kappa}^2$ & \\
          & Mass fraction of water in hydrated crust ${x_{\mathrm{h}}}^3$ & \\
    \hline
    \end{tabular}
    \caption{Model Parameters. $^\dagger$ indicates input parameters for simple loss parameterization. \\\hspace{\textwidth} $^1$Turcotte \& Schubert (1982),$^2$Jackson \& Pollack (1987), $^3$McGovern \& Schubert (1989), $^4$Schubert et al.(2001), $^5$Kite et al.(2009), $^6$Langmuir \& Broecker (2012), $^7$Schaefer \& Sasselov (2015), $^8$Komacek \& Abbot (2016)} 
\end{table}

\begin{equation}
    w_{\mathrm{loss}} = \color{red}\phi_{\mathrm{loss}} \color{black} \exp \left(\frac{-t}{\color{red}\tau_{\mathrm{XUV}}} \color{black} \right)
\end{equation}

Combined T- \& P-dependent degassing \& regassing: \\
Mantle: \\
\begin{equation}\label{eqn:dWm/dt}
\begin{split}
    \frac{d W_{\mathrm{m}}}{dt} & = L_{\mathrm{MOR}} S(T) \Biggl[x_{\mathrm{h}} \rho_{\mathrm{c}} \chi_{\mathrm{r}} \min \biggl[d_{\mathrm{h}}(T) \left( \frac{P}{P_\oplus} \right)^\sigma, ~d_{\mathrm{b}} \biggr] \\
    & - x \rho_{\mathrm{m}} d_{\mathrm{melt}} \min \biggl[ f_{\mathrm{degas},\oplus} \left(\frac{P}{P_\oplus} \right)^{-\mu}, ~1 \biggr] \left(\frac{T-T_{\mathrm{sol,wet}}(x)}{T_{\mathrm{liq,dry}} - T_{\mathrm{sol,dry}}} \right)^{\theta} \Biggr],
\end{split}
\end{equation}
where
\begin{equation}
    d_{\mathrm{h}}(T) = h^{(1-3\beta)} (T-T_{\mathrm{s}})^{-(1+\beta)} (T_{\mathrm{serp}} - T_{\mathrm{s}}) \left( \frac{\eta(T,x) \kappa \mathrm{Ra}_{\mathrm{crit}}}{\alpha \rho_{\mathrm{m}} g} \right) 
\end{equation}

Surface:
\begin{equation}\label{eqn:dWs/dt}
\begin{split}
    \frac{d W_{\mathrm{s}}}{dt} & = L_{\mathrm{MOR}} S(T) \Biggl[x \rho_{\mathrm{m}} d_{\mathrm{melt}} \min \biggl[ f_{\mathrm{degas},\oplus} \left(\frac{P}{P_\oplus} \right)^{-\mu}, ~1 \biggr] \left(\frac{T-T_{\mathrm{sol,wet}}(x)}{T_{\mathrm{liq,dry}} - T_{\mathrm{sol,dry}}} \right)^{\theta} \\
    & - x_{\mathrm{h}} \rho_{\mathrm{c}} \chi_{\mathrm{r}} \min \biggl[d_{\mathrm{h}}(T) \left( \frac{P}{P_\oplus} \right)^\sigma, ~d_{\mathrm{b}} \biggr] \Biggr] - \frac{\epsilon_{\mathrm{XUV}} \pi F_{\mathrm{XUV}} R_{\mathrm{p}}^3}{G M_{\mathrm{p}}},
\end{split}
\end{equation}


%% If you wish to include an acknowledgments section in your paper,
%% separate it off from the body of the text using the \acknowledgments
%% command.
\acknowledgments

Thanks to McGill for funding, etc. Outside people that have helped: Thaddeus Komacek, Don Baker,...

%% To help institutions obtain information on the effectiveness of their 
%% telescopes the AAS Journals has created a group of keywords for telescope 
%% facilities.
%
%% Following the acknowledgments section, use the following syntax and the
%% \facility{} or \facilities{} macros to list the keywords of facilities used 
%% in the research for the paper.  Each keyword is check against the master 
%% list during copy editing.  Individual instruments can be provided in 
%% parentheses, after the keyword, but they are not verified.

\vspace{5mm}
\facilities{HST(STIS), Swift(XRT and UVOT), AAVSO, CTIO:1.3m,
CTIO:1.5m,CXO}

%% Similar to \facility{}, there is the optional \software command to allow 
%% authors a place to specify which programs were used during the creation of 
%% the manusscript. Authors should list each code and include either a
%% citation or url to the code inside ()s when available.

\software{astropy \citep{astropycollab},  
          }

%% Appendix material should be preceded with a single \appendix command.
%% There should be a \section command for each appendix. Mark appendix
%% subsections with the same markup you use in the main body of the paper.

%% Each Appendix (indicated with \section) will be lettered A, B, C, etc.
%% The equation counter will reset when it encounters the \appendix
%% command and will number appendix equations (A1), (A2), etc. The
%% Figure and Table counter will not reset.

\appendix

%% The reference list follows the main body and any appendices.
%% Use LaTeX's thebibliography environment to mark up your reference list.
%% Note \begin{thebibliography} is followed by an empty set of
%% curly braces.  If you forget this, LaTeX will generate the error
%% "Perhaps a missing \item?".
%%
%% thebibliography produces citations in the text using \bibitem-\cite
%% cross-referencing. Each reference is preceded by a
%% \bibitem command that defines in curly braces the KEY that corresponds
%% to the KEY in the \cite commands (see the first section above).
%% Make sure that you provide a unique KEY for every \bibitem or else the
%% paper will not LaTeX. The square brackets should contain
%% the citation text that LaTeX will insert in
%% place of the \cite commands.

%% We have used macros to produce journal name abbreviations.
%% \aastex provides a number of these for the more frequently-cited journals.
%% See the Author Guide for a list of them.

%% Note that the style of the \bibitem labels (in []) is slightly
%% different from previous examples.  The natbib system solves a host
%% of citation expression problems, but it is necessary to clearly
%% delimit the year from the author name used in the citation.
%% See the natbib documentation for more details and options.

\begin{thebibliography}{}

\bibitem[Abbot et al.(2012)]{abbot12} Abbot, D.~S., Cowan, N.~B., \& Ciesla, F.~J. 2012, \apj, 756, 178

\bibitem[Abe et al.(2011)]{abe11} Abe, Y., Abe-Ouchi, A, Sleep, N.~H., \& Zahnle, K.~J. 2011, Astrobiology, 11, 443

\bibitem[Acu\~{n}a et al.(2021)]{acuna21} Acu\~{n}a, L., Deleuil, M., Mousis, O., Marcq, E., et al. 2021, arXiv e-prints, 2101.08172

\bibitem[Airapetian et al.(2017)]{airapetian17} Airapetian, V.~S., Glocer, A., Khazanov, G.~V., Loyd, R.~O.~P., France, K., Sojka, J., Danchi, W.~C., \& Liemohn, M.~W. 2017, \apjl, 836, L3

\bibitem[Alibert(2014)]{alibert14} Alibert, Y. 2014, \aap, 561, A41

\bibitem[Anglada-Escud\'{e} et al.(2016)]{anglada16} Anglada-Escud\'{e}, G., Amado, P.~J., Barnes, J., et al. 2016, Nature, 536, 437

\bibitem[Ardia et al.(2012)]{ardia12} Ardia, P., Hirschmann, M.~M., Withers, A.~C., Tenner, T.~J. 2012, Earth \& Planetary Science Letters, 345, 104

\bibitem[Astropy Collaboration et al.(2013)]{astropycollab} Astropy Collaboration, Robitaille, T.~P., Tollerud, E.~J., et al.\ 2013, \aap, 558, A33 

\bibitem[Atri(2020)]{atri20} Atri, D. 2020, MNRAS Letters, 492, L28

\bibitem[Atri \& Carberry Mogan(2020)]{atri20b} Atri, D., \& Carberry Mogan, S.~R. 2020, arXiv e-prints, 2009.04310

\bibitem[Baraffe et al.(1998)]{baraffe98} Baraffe, I., Chabrier, G., Allard, F., \& Hauschildt, P.~H. 1998, \aap, 337, 403

\bibitem[Baraffe et al.(2015)]{baraffe15} Baraffe, I., Homeier, D., Allard, F., \& Chabrier, G. 2015, \aap, 577, A42

\bibitem[Barnes(2017)]{barnes17} Barnes, R. 2017, Celestial Mechanics and Dynamical Astronomy, 129, 509

\bibitem[Barnes et al.(2018)]{barnes18} Barnes, R., Deitrick, R., Luger, R., Driscoll, P.~E., Quinn, T.~R., Fleming, D.~P., et al. 2018, arXiv:1608.06919

\bibitem[Barth et al.(2020)]{barth20} Barth, P., Carone, L., Barnes, R., Noack, L., Molli\`{e}re, P., \& Henning, T. 2020, arXiv e-prints, 2008.09599

\bibitem[B\'{e}dard(1989)]{bedard89} B\'{e}dard, J.~H. 1989, Earth and Planetary Science Letters, 91, 359

\bibitem[B\'{e}dard(2018)]{bedard18} B\'{e}dard, J.~H. 2018, Geoscience Frontiers, 9, 19

\bibitem[Benneke et al.(2019)]{benneke19} Benneke, B., Wong, I., Piaulet, C., Knutson, H.~A., Crossfield, I.~J.~M., et al. 2019, arXiv:1909.04642

\bibitem[Bolmont et al.(2016)]{bolmont16} Bolmont, E., Selsis, F., Owen, J.~E., Ribas, I., Raymond, S.~N., Leconte, J., \& Gillon, M. 2016, \mnras, 464, 3728

\bibitem[Bonsor et al.(2020)]{bonsor20} Bonsor, A., Carter, P.~J., Hollands, M., G\"{a}nsicke, B.~T., Leinhardt, Z., \& Harrison, J.~H.~D. 2020, MNRAS, 492, 2683

\bibitem[Bonsor et al.(2021)]{bonsor21} Bonsor, A., Jofr\'{e}, P., Shorttle, O., Rogers, L.~K., et al. 2021, arXiv e-prints, 2102.02843

\bibitem[Brady \& Gislason(1997)]{brady97} Brady, P.~V., \& Gislason, S.~R. 1997, Geochimica et Cosmochimica Acta, 61, 965

\bibitem[Budde et al.(2019)]{budde19} Budde, G., Burkhardt, C., \& Kleine, T. 2019, Nature Astronomy Letters, https://doi.org/10.1038/s41550-019-0779-y

\bibitem[Burger et al.(2020)]{burger20} Burger, C., Maindl, T.~I., \& Sch{\"a}fer, C. 2020, IAU Symposium, 345, 287

\bibitem[Chen et al.(2021)]{chen21} Chen, H., Zhan, Z., Youngblood, A., Wolf, E.~T., et al. 2021, arXiv e-prints, 2101.04507

\bibitem[Claire et al.(2012)]{claire12} Claire, M.~W., Sheets, J., Cohen, M., Ribas, I., Meadows, V.~S., \& Catling, D.~C. 2012, \apj, 757, 95

\bibitem[Cohen et al.(2015)]{cohen15} Cohen, O., Ma, Y., Drake, J.~J., Glocer, A., Garraffo, C., Bell, J.~M., \& Gombosi, T.~I. 2015, \apj, 806, 41

\bibitem[Cooper et al.(2020)]{cooper20} Cooper, G.~F., Macpherson, C.~G., Blundy, J.~D., Maunder, B., et al. 2020, Nature, 582, 525

\bibitem[Cowan et al.(2009)]{cowan09} Cowan, N.~B., Agol, E., Meadows, V.~S., Robinson, T., et al. 2009, \apj, 700, 915

\bibitem[Cowan \& Abbot(2014)]{cowan14} Cowan, N.~B., \& Abbot, D.~S. 2014, \apj, 781, 27

\bibitem[Dehant et al.(2019)]{dehant19} Dehant, V., Debaille, V., Dobos, V., Gaillard, F., et al. 2019, \ssr, 215, 42

\bibitem[Dencs \& Regaly(2019)]{dencs19} Dencs, Z., \& Reg{\'a}ly, Zs. 2019, \mnras, 487, 2191

\bibitem[Dong et al.(2017)]{dong17} Dong, C., Lingam, M., Ma, Y., \& Cohen, O. 2017, \apjl, 837, L26

\bibitem[Dong et al.(2019)]{dong19} Dong, C., Huang, Z., \& Lingam, M. 2019, arXiv e-prints, 1907.07459

\bibitem[Dong et al.(2020)]{dong20} Dong, C., Jin, M., \& Lingam, M. 2020, arxiv e-prints, 2005.13190

\bibitem[Dressing \& Charbonneau(2015)]{dressing15} Dressing, C.~D., \& Charbonneau, D. 2015, \apj, 807, 45

\bibitem[Edmonds \& Woods(2018)]{edmonds18} Edmonds, M., \& Woods, A.~W. 2018, Journal of Volcanology and Geothermal Research, 368, 13

\bibitem[Elkins-Tanton et al.(2007)]{elkins07} Elkins-Tanton, L.~T., Smrekar, S.~E., Hess, P.~.C, \& Parmentier, E.~M. 2007, Journal of Geophysical Research (Planets), 112, E04S06

\bibitem[Elkins-Tanton \& Seager(2008)]{elkins08} Elkins-Tanton, L.~T., \& Seager, S. 2008, \apj, 685, 1237

\bibitem[Estrela et al.(2020)]{estrela20} Estrela, R., Palit, S., \& Valio, A. 2020, arXiv e-prints, 2008.09147

\bibitem[Fei et al.(2017)]{fei17} Fei, H., Yamazaki, D., Sakurai, M, Miyajima, N., Ohfuji, H., Katsura, T., Yamamota, T. 2017, Sci.~Adv., 2017;3: e1603024

\bibitem[Fernandes et al.(2019)]{fernandes19} Fernandes, C.~S., Van Grootel, V., Salmon, S.~J.~A.~J., Aringer, B., Burgasser, A.~J., Scuflaire, R., Brassard, P., \& Fontaine, G. 2019, \apj, 879, 94

\bibitem[Fleming et al.(2020)]{fleming20} Fleming, D.~P., Barnes, R., Luger, R., VanderPlas, J.~T. 2020, \apj, 891, 155

\bibitem[Flesch et al.(2000)]{flesch00} Flesch, L.~M., Holt, W.~E., Haines, A.~J., Shen-Tu, B. 2000, Science, 287, 834

\bibitem[Foley \& Becker(2009)]{foley09} Foley, B.~J., \& Becker, T.~W. 2009, Geochem.~Geophys.~Geosyst, 10, Q08001

\bibitem[Foley et al.(2012)]{foley12} Foley, B.~J., Bercovici, D., Landuyt, W. 2012, Earth \& Planetary Science Letters, 331, 281

\bibitem[Foley(2015)]{foley15} Foley, B.~J. 2015, \apj, 812, 36

\bibitem[Foley \& Driscoll(2016)]{foley16} Foley, B.~J., \& Driscoll, P.~E. 2016, Geochem.~Geophys.~Geosyst., 17, 1885

\bibitem[Foley(2018)]{foley18a} Foley, B.~J. 2018, Phil.~Trans.~R.~Soc.~A, 376: 20170409

\bibitem[Foley \& Smye(2018)]{foley18b} Foley, B.~J., \& Smye, A.~J. 2018, Astrobiology, 18, 873

\bibitem[France et al.(2018)]{france18} France, K., Arulanantham, N., Fossati, L., Lanza, A.~F., Loyd, R.~O.~P., Redfield, S., \& Schneider, P.~C. 2018, \apjs, 239, 16

\bibitem[France et al.(2020)]{france20} France, K., Duvvuri, G., Egan, H., Koskinen, T., et al. 2020, \aj, 160, 237 

\bibitem[Fulton \& Petigura(2019)]{fulton19} Fulton, B.~J., \& Petigura, E.~A. 2019, arxiv e-prints, arXiv:1805.01453v3

\bibitem[Gallet \& Bouvier(2015)]{gallet15} Gallet, F., \& Bouvier, J. 2015, \aap, 577, A98

\bibitem[Gao et al.(2015)]{gao15} Gao, P., Hu., R., Robinson, T.~D., Li, C., \& Yung, Y.~K. 2015, \apj, 806, 249

\bibitem[Garcia-Sage et al.(2017)]{garcia17} Garcia-Sage, K., Glocer, A., Drake, J.~J., Gronoff, G., \& Cohen, O. 2017, \apjl, 844, L13

\bibitem[Gebauer et al.(2018)]{gebauer18} Gebauer, S., Grenfell, J.~L., Lehmann, R., \& Rauer, H. 2018, Astrobiology, 18, 856

\bibitem[Gialluca et al.(2021)]{gialluca21} Gialluca, M.~T., Robinson, T.~D., Rugheimer, S., \& Wunderlich, F. 2021, arXiv e-prints, 2101.04139

\bibitem[Gillon et al.(2017)]{gillon17} Gillon, M., Triaud, A.~H.~M.~J., Demory, B.-O., et al. 2017, Nature, 542, 456

\bibitem[Gillon et al.(2020)]{gillon20} Gillon, M., Meadows, V., Agol, E., Burgasser, A.~J., Deming, D., et al. 2020, arxiv e-prints, 2002.04798v2

\bibitem[Godolt et al.(2019)]{godolt19} Godolt, M., Tosi, N., Stracke, B., Lee~Grenfell, J., Ruedas, T., Spohn, T., \& Rauer, H. 2019, \aap, 625, A12

\bibitem[Gronoff et al.(2020)]{gronoff20} Gronoff, G. et al. 2020, arXiv e-prints, 2003.03231v1

\bibitem[Gu et al.(2020)]{gu20} Gu, L., Fan, S., Li, J., Bartlett, S., et al. 2020, arXiv e-prints, 2012.10556

\bibitem[Guendelman \& Kaspi(2019)]{guendelman19} Guendelman, I., \& Kaspi, Y. 2019, \apj, 881, 67

\bibitem[Hakim et al.(2020)]{hakim20} Hakim, K., Bower, D.~J., Tian, M., Deitrick, R., et al. 2020, arXiv e-prints, 2008.11620

\bibitem[Haldemann et al.(2020)]{haldemann20} Haldemann, J., Alibert, Y., Mordasini, C., \& Benz, W. 2020, arXiv e-prints, 2009.10098

\bibitem[Hamano et al.(2013)]{hamano13} Hamano, K., Abe, Y., \& Genda, H. 2013, Nature, 497, 607

\bibitem[Hara \& Suzuki(2020)]{hara20} Hara, T., \& Suzuki, A. 2020, arXiv e-prints, 2009.04040

\bibitem[Harter et al.(2020)]{harter20} Harter, S.~K., Ricci, L., Zhang, S., \& Zhu, Z. 2020, arXiv e-prints, 2011.08279

\bibitem[Hauri et al.(2006)]{hauri06} Hauri, E.~H., Gaetani, G.~A., \& Green, T.~H. 2006, E\&PSL, 248, 715

\bibitem[Helled et al.(2021)]{helled21} Helled, R., Werner, S., Dorn, C., Guillot, T., et al. 2021, arXiv e-prints, 2103.08481

\bibitem[Henry(2004)]{henry04} Henry, T.~J. 2004, ASP Conference Series 318: Spectroscopically and
Spatially Resolving the Components of the Close Binary
Stars, 159

\bibitem[Herbort et al.(2020)]{herbort20} Herbort, O., Woitke, P., Helling, Ch., \& Zerkle, A. 2020, \aap, 636, A71

\bibitem[Hirschmann(2006)]{hirschmann06} Hirschmann, M.~M. 2006, Annu.~Rev.~Earth~Planet.~Sci., 34:629–53

\bibitem[Hirschmann \& Dasgupta(2009)]{hirschmann09} Hirschmann, M.~M., \& Dasgupta, R. 2009, Chemical Geology, 262, 4 

\bibitem[Hirschmann \& Kohlstedt(2012)]{hirschmann12} Hirschmann, M.~M., \& Kohlstedt, D. 2012, Physics Today, 65, 40

\bibitem[H\"{o}ning et al.(2019a)]{honing19a} H\"{o}ning, D., Tosi, N., Hansen-Goos, H., Spohn, T. 2019, Physics of the Earth and Planetary Interiors, doi: https://doi.org/10.1016/j.pepi.2019.01.001

\bibitem[H\"{o}ning et al.(2019b)]{honing19b} H\"{o}ning, D., Tosi, N., \& Spohn, T. 2019, \aap, 627, A48

\bibitem[Hoskin et al.(2020)]{hoskin20} Hoskin, M.~J., Toloza, O., G\"{a}nsicke, B.~T., Raddi, R., et al. 2020, arXiv e-prints, 2009.05053

\bibitem[Howard et al.(2020)]{howard20} Howard, W.~S., Corbett, H., Law, N.~M., Ratzloff, J.~K., et al. 2020, \apj, 902, 115

\bibitem[Howe et al.(2020)]{howe20} Howe, A.~R., Adams, F.~C., \& Meyer, M~.R. 2020, arxiv e-prints, arXiv:1912.08820v4

\bibitem[Iacono-Marziano et al.(2012)]{iacono12} Iacono-Marziano, G., Morizet, Y., Le Trong, E., Gaillard, F. 2012, Geochimica et Cosmochimica Acta, 97, 1

\bibitem[Ikoma \& Genda(2006)]{ikoma06} Ikoma, M., \& Genda, H. 2006, \apj, 648, 696

\bibitem[Jarrard(2003)]{jarrard03} Jarrard, R.~D. 2003, Geochem.~Geophys.~Geosyst., 4, 8905

\bibitem[Johnstone et al.(2019)]{johnstone19} Johnstone, C.~P., Khodachenko, M.~L., L\"{u}ftinger, T., Kislyakova, K.~G., Lammer, H., \& G\"{u}del, M. 2019, \aap, 624, L10

\bibitem[Kaiser et al.(2020)]{kaiser20} Kaiser, B.~C., Clemens, J.~C., Blouin, S., Dufour, P., et al. 2020, arXiv e-prints, 2012.12900

\bibitem[Kaltenegger \& Traub(2009)]{kaltenegger09} Kaltenegger, L., \& Traub, W.~A. 2009, \apj, 698, 519

\bibitem[Kaltenegger \& Rugheimer(2020)]{kaltenegger20} Kaltenegger, L., Lin, Z., \& Rugheimer, S. 2020, arXiv e-prints, 2010.01734

\bibitem[Kaltenegger et al.(2021)]{kaltenegger21} Kaltenegger, L., Pepper, J., Christodoulou, P.~M., Stassun, K., et al. 2021, arXiv e-prints, 2101.07898

\bibitem[Kasting(1988)]{kasting88} Kasting, J.~F. 1988, Icarus, 74, 472

\bibitem[Kasting et al.(1993)]{kasting93} Kasting, J.~F., Whitmire, D.~P., \& Reynolds, R.~T. 1993, Icarus, 101, 108

\bibitem[Katyal et al.(2019)]{katyal19} Katyal, N., Nikolaou, A., Godolt, M., Lee~Grenfell, J., Tosi, N., Schreier, F., \& Rauer, H. 2019, arXiv e-prints, 1903.04623

\bibitem[Katz et al.(2003)]{katz03} Katz, R.~F., Spiegelman, M., \& Langmuir, C.~H. 2003, GGG, 4, 1073

\bibitem[Keles et al.(2020)]{keles20} Keles, E., Grenfell, J.~L., Godolt, M., Stracke, B., Rauer, H. 2020, arxiv e-prints, 2006.05207

\bibitem[Kerins(2020)]{kerins20} Kerins, E. 2020, arXiv e-prints, 2010.04089

\bibitem[Kimura \& Ikoma(2020)]{kimura20} Kimura, T., \& Ikoma, M. 2020, arxiv e-prints, 2006.09068

\bibitem[King(1995)]{king95} King, S.~D. 1995, Models of Mantle Viscosity, doi = {10.1029/RF002p0227}

\bibitem[King et al.(2019)]{king19} King, G.~W., Wheatley, P.~J., Bourrier, V., \& Ehrenreich, D. 2019, \mnras, 484, L49

\bibitem[King \& Wheatley(2020)]{king20} King, G.~W., \& Wheatley, P.~J. 2020, arXiv e-prints, 2007.13731

\bibitem[Kislyakova et al.(2014)]{kislyakova14} Kislyakova, K.~.G., Johnstone, C.~P., Odert, P., et al. 2014, \aap, 562, A116

\bibitem[Kislyakova et al.(2020)]{kislyakova20} Kislyakova, K.~.G., Holmstr{\"o}m, M., Lammer, H., \& Erkaev, N.~V. 2020, arxiv e-prints, 2003.13412

\bibitem[Kite et al.(2009)]{kite09} Kite, E.~S., Manga, M., \& Gaidos, E. 2009, \apj, 700, 1732

\bibitem[Kite \& Ford(2018)]{kite18} Kite, E.~S. \& Ford, E.~B. 2018, \apj, 864, 75

\bibitem[Kite et al.(2019)]{kite19} Kite, E.~S., Fegley Jr., B., Schaefer, L., \& Ford, E.~B. 2019, \apjl, 887, L33

\bibitem[Kite \& Barnett(2020)]{kite20} Kite, E.~S., \& Barnett, M. 2020, arxiv e-prints, 2006.02589

\bibitem[Kite \& Schaefer(2021)]{kite21} Kite, E.~S., \& Schaefer, L. 2021, arXiv e-prints, 2103.07753

\bibitem[Koizumi et al.(2020)]{koizumi20} Koizumi, Y., Kuzuhara, M., Omiya, M., Hirano, T., et al. 2020, arXiv e-prints, 2011.14521

\bibitem[Komacek \& Abbot(2016)]{komacek16} Komacek, T.~D., \& Abbot, D.~S. 2016, \apj, 832, 54

\bibitem[Kopparapu et al.(2013)]{kopparapu13} Kopparapu, R.~K., Ramirez, R., Kasting, J.~F., Eymet, V., et al. 2013, \apj, 765, 131

\bibitem[Kopparapu et al.(2014)]{kopparapu14} Kopparapu, R.~K., Ramirez, R.~M., SchottelKotte, J., Kasting, J.~F., Domagal-Goldman, S., \& Eymet, V. 2014, \apjl, 787, L29

\bibitem[Kopparapu et al.(2017)]{kopparapu17} Kopparapu, R.~K., Wolf, E.~T., Arney, G., Batalha, N.~.E, et al. 2017, \apj, 845, 5

\bibitem[Korenaga(2010)]{korenaga10} Korenaga, J. 2010, \apjl, 725, L43

\bibitem[Korenaga et al.(2017)]{korenaga17} Korenaga, J., Planavsky, N.~J., \& Evans, D.~A.~D. 2017, Phil.~Trans.~R.~Soc.~A, 375:20150393

\bibitem[Kostov et al.(2019)]{kostov19} Kostov, V.~B., Schlieder, J.~E., Barclay, T., Quintana, E.~V., Col\'{o}n, K.~D., et al. 2019, arXiv e-prints, 1903.08017

\bibitem[Kurokawa et al.(2018)]{kurokawa18} Kurokawa, H., Foriel, J., Laneuville, M., Houser, C., Usui, T. 2018, Earth \& Planetary Science Letters, 497, 149

\bibitem[Lammer et al.(2014)]{lammer14} Lammer, H., St\"{o}kl, A., Erkaev, N.~V., Dorfi, E.~A., Odert, P., G\"{u}del, M., Kulikov, Yu.~N., Kislyakova, K.~G., \& Leitzinger, M. 2014, \mnras, 439, 3225

\bibitem[Lammer et al.(2018)]{lammer18} Lammer, H., Zerkle, A.~L., Gebauer, S., Tosi, N., Noack, L., et al. 2018, \aapr, 26, 2

\bibitem[Langmuir \& Broecker(2012)]{langmuir12} Langmuir, C.~H., \& Broecker, W. 2012, How to Build a Habitable Planet (Princeton: Princeton Univ. Press)

\bibitem[Leconte et al.(2013)]{leconte13} Leconte, J., Forget, F., Charnay, B., Wordsworth, R., Selsis, F., Millour, E., \& Spiga, A. 2013, \aap. 554, A69

\bibitem[Leconte(2018)]{leconte18} Leconte, J. 2018, Nature Geoscience, 11, 168

\bibitem[Lehmer et al.(2020)]{lehmer20} Lehmer, O.~R., Catling, D.~C., \& Krissansen-Totton, J. 2020, arXiv e-prints, 2012.00819

\bibitem[Lenardic et al.(1993)]{lenardic93} Lenardic, A., Kaula, W.~M., \& Bindschadler, D.~L. 1993, Journal of Geophysical Research, 98, 18967

\bibitem[Lenardic et al.(2016)]{lenardic16} Lenardic, A., Jellinek, A.~M., Foley, B., O'Neill, C., \& Moore, W.B. 2016, Journal of Geophysical Research, J.~Geophys.~Res.~Planets, 121, 1831

\bibitem[Lenardic(2018a)]{lenardic18a} Lenardic, A. 2018, Phil. Trans. R. Soc. A, 376: 20170416.

\bibitem[Lenardic(2018b)]{lenardic18b} Lenardic, A. 2018, Handbook of Exoplanets, \url{https://doi.org/10.1007/978-3-319-55333-7_65}

\bibitem[Leung et al.(2020)]{leung20} Leung, M., Meadows, V.~S., \& Lustig-Yaeger, J. 2020, arxiv e-prints, arXiv:2004.13731v1

\bibitem[Li et al.(2008)]{li08} Li, Z.~X.~A., Lee, C.-T.~A., Peslier, A.~H., Lenardic, A., \& Mackwell, S.~J. 2008, Journal of Geophysical Research, 113, B09210

\bibitem[Lichtenberg et al.(2019)]{lichtenberg19} Lichtenberg, T., Golabek, G.~J., Burn, R., Meyer, M.~R., Alibert, Y., Gerya, T.~V., Mordasini, C. 2019, Nature Astronomy Letters, https://doi.org/10.1038/s41550-018-0688-5

\bibitem[Lichtenberg et al.(2021)]{lichtenberg21} Lichtenberg, T., Bower, D.~J., Hammond, M., Boukrouche, R., et al. 2021, arXiv e-prints, 2101.10991

\bibitem[Lingam \& Loeb(2019)]{lingam19} Lingam, M., \& Loeb, A. 2019, arxiv e-prints, 1905.11410

\bibitem[Lingam(2020)]{lingam20} Lingam, M. 2020, \aj, 159, 144

\bibitem[Linsky et al.(2014)]{linsky14} Linsky, J.~L., Fontenla, J., \& France, K. 2014, \apj, 780, 61

\bibitem[Lisse et al.(2020)]{lisse20} Lisse, C.~M., Desch, S.~J., Unterborn, C.~T., et al. 2020, arxiv e-prints, 2006.07403

\bibitem[Liu \& Hasterok(2016)]{liu16} Liu, L., \& Hasterok, D. 2016, Science, 353, 1515

\bibitem[Luger \& Barnes(2015)]{luger15} Luger, R., \& Barnes, R. 2015, Astrobiology, 15, 119

\bibitem[Lustig-Yaeger et al.(2018)]{lustig18} Lustig-Yaeger, J., Meadows, V.~S., Tovar Mendoza, G., et al. 2018, \aj, 156, 301

\bibitem[Madhusudhan et al.(2020)]{madhusudhan20} Madhusudhan, N., Nixon, M.~C., Welbanks, L., Piette, A.~A.~A., \& Booth, R.~A. 2020, \apjl, 891, L7

\bibitem[Malik et al.(2019)]{malik19} Malik, M., Kempton, E.~M.-R., Koll, D.~D.~B., Mansfield, M., Bean, J.~L., Kite, E. 2019, \apj, 886, 142

\bibitem[Manabe \& Strickler(1964)]{manabe64} Manabe, S., \& Strickler, R.~F. 1964, Journal of Atmospheric Sciences, 21, 361

\bibitem[Manabe \& Wetherald(1967]{manabe67} Manabe, S., \& Wetherald, R.~T. 1967, Journal of Atmospheric Sciences, 24, 241

\bibitem[Marty(2012)]{marty12} Marty, B. 2012, Earth \& Planetary Science Letters, 313, 56

\bibitem[McGovern \& Schubert(1989)]{mcgovern89} McGovern, P.~J., \& Schubert, G. 1989, Earth and Planetary Science Letters, 96, 27

\bibitem[Meadows et al.(2018)]{meadows18a} Meadows, V.~S., Arney, G.~N., Schwieterman, E.~W., Lustig-Yaeger, J., et al. 2018, Astrobiology, 18, 133

\bibitem[Meadows \& Barnes(2018)]{meadows18b} Meadows,  v.~s., \& Barnes, R.~K. 2017, Handbook of Exoplanets: Factors Affecting Exoplanet Habitability (Springer Nature)

\bibitem[Meech \& Raymond(2019)]{meech19} Meech, K., \& Raymond, S.~N. 2019, arxiv e-prints, arXiv:1912.04361v1

\bibitem[Meier et al.(2021)]{meier21} Meier, T.~G., Bower, D.~J., Lichtenberg, T., et al. 2021, arXiv e-prints, 2103.02374

\bibitem[Melbourne et al.(2020)]{melbourne20} Melbourne, K., Youngblood, A., France, K., Froning, C.~S., et al. 2020, arXiv e-prints, 2009.07869

\bibitem[Menou(2015)]{menou15} Menou, K. 2015, Earth and Planetary Science Letters, 429, 20

\bibitem[Mercer \& Stamatellos(2020)]{mercer20} Mercer, A., \& Stamatellos, D. 2020, arxiv e-prints, 2001.10062

\bibitem[Modak et al.(2016)]{modak16} Modak, A., Bala, G., Cao, L., \& Caldeira, K. 2016, Environmental Research Letters, 11, 044013

\bibitem[Moore(2001)]{moore01} Moore, W.~B. 2001, Icarus, 154, 548

\bibitem[Moore(2003)]{moore03} Moore, W.~B. 2003, Journal of Geophysicsal Research, 108, 5096

\bibitem[Moresi \& Solomatov(1998)]{moresi98} Moresi, L., \& Solomatov, V. 1998, Geophys.~J.~Int., 133, 669

\bibitem[Morley et al.(2017)]{morley17} Morley, C.~V., Kreidberg, L., Rustamkulov, Z., Robinson, T., \& Fortney, J.~J. 2017, \apj, 850, 121

\bibitem[Muralidharan et al.(2008)]{muralidharan08} Muralidharan, K., Deymier, P., Stimpfl, M., de Leeuw, N.~H., Drake, M.~J. 2008, Icarus, 198, 400

\bibitem[Nakagawa et al.(2015)]{nakagawa15} Nakagawa, T., Nakakuki, T., \& Iwamori, H. 2015, Geochem.~Geophys.~Geosys.~, 16, 1449

\bibitem[Nakajima et al.(1992)]{nakajima92} Nakajima, S., Hayashi, Y.-Y., \& Abe, Y.\ 1992, Journal of Atmospheric Sciences, 49, 2256

\bibitem[Nakayama et al.(2019)]{nakayama19} Nakayama, A., Kodama, T., Ikoma, M., \& Abe, Y. 2019, arXiv:1907.00827v2

\bibitem[Nimmo et al.(2020)]{nimmo20} Nimmo, F., Primack, J., Faber, S.~M., Ramirez-Ruiz, E., \& Safarzadeh, M. 2020, arXiv e-prints, 2011.04791

\bibitem[Noack, Breuer, \& Spohn(2012)]{noack12} Noack, L., Breuer, D., \& Spohn, T. 2012, \icarus, 217, 484

\bibitem[Noack \& Breuer(2014)]{noack14} Noack, L., \& Breuer, D. 2014, \planss, 98, 41

\bibitem[Noack et al.(2016)]{noack16} Noack, L., H{\"o}ning, D., Rivoldini, A., Heistracher, C., Zimov, N., Journaux, B., Lammer, H., Van Hoolst, T., Bredeh{\"o}ft, J.~H. 2016, Icarus, 277, 215

\bibitem[Noack et al.(2017)]{noack17} Noack, L., Snellen, I., Rauer, H. 2017, Space Sci.~Rev., 212, 877

\bibitem[Novella et al.(2014)]{novella14} Novella, D., Frost, D.~J., Hauri, E.~H., Bureau, H., Raepsaet, C., Roberge, M. 2014, Earth and Planetary Science Letters, 400, 1

\bibitem[Okuya et al.(2019)]{okuya19} Okuya, A., Fujii, Y., \& Ida, S. 2019, arxiv e-prints, 1906.058441

\bibitem[O'Neill \& Lenardic(2007)]{oneill07} O'Neill, C., \& Lenardic, A. 2007, Geophysical Research Letters, 34, L19204

\bibitem[O'Neill et al.(2016)]{oneill16} O'Neill, C., Lenardic, A., Weller, M., Moresi, L., Quenette, S., Zhang, S. 2016, Physics of the Earth and Planetary Interiors, 255, 80

\bibitem[O'Neill(2020)]{oneill20} O'Neill, C. 2020, arxiv e-prints, 2006.05654

\bibitem[Ortenzi et al.(2020)]{ortenzi20} Ortenzi, G., Noack, L., Sohl, F., Guimond, C.~M., et al. 2020, Nature Scientific Reports, 10:10907

\bibitem[Owen \& Mohanty(2016)]{owen16} Owen, J.~E., \& Mohanty, S. 2016, \mnras, 459, 4088

\bibitem[Owen(2019)]{owen19} Owen, J.~E. 2019, Annual Review of Earth and Planetary Sciences, 47, 67

\bibitem[Owen et al.(2020)]{owen20} Owen, J.~E., Shaikhislamov, I.~F., Lammer, H., Fossati, L., Khodachenkom, M.~L. 2020, arXiv e-prints, 2010.15091

\bibitem[Papale(1997)]{papale97} Papale, P. 1997, Contrib. Mineral Petrol, 126, 237

\bibitem[Papale(1999)]{papale99} Papale, P. 1999, American Mineralogist, 84, 477

\bibitem[Parke Loyd et al.(2018)]{parke18} Parke Loyd, R.~O., Shkolnik, E.~L., Schneider, A.~C., Barman, T.~S., Meadows, V.~S., Pagano, I., \& Peacock, S. 2018, \apj, 867, 70

\bibitem[Peacock et al.(2019a)]{peacock19a} Peacock, S., Barman, T., Shkolnik, E.~L., Hauschildt, P.~H., Baron, E. 2019, \apj, 871, 235

\bibitem[Peacock et al.(2019b)]{peacock19b} Peacock, S., Barman, T., Shkolnik, E.~L., Hauschildt, P.~H., Baron, E., \& Fuhrmeister, B. 2019, \apj, 886, 77

\bibitem[Pearson et al.(2014)]{pearson14} Pearson, D.~G., Brenker, F.~E., Nestola, F., McNeill, J., Nasdala, L., Hutchison, M.~T., Matveev, S., Mather, K., Silversmit, G., Schmitz, S., Vekemans, B., \& Vincze, L 2014, Nature, 507, 221

\bibitem[Petersen et al.(2015)]{petersen15} Petersen, R.~I., Stegman, D.~R., Tackley, P.~J. 2015, Physics of the Earth and Planetary Interiors, 241, 65

\bibitem[Pierrehumbert(2010)]{pierrehumbert10} Pierrehumbert, R.~T.\ 2010, Principles of Planetary Climate, by R.~T. Pierrehumbert.  Cambridge, UK: Cambridge University Press. ISBN: 9780521865562, 2010

\bibitem[Pinotti \& Porto de Mello(2020)]{pinotti20} Pinotti, R., \& Porto de Mello, G.~F. 2020, arXiv e-prints, 2003.13107

\bibitem[Pope et al.(2012)]{pope12} Pope, E.~C., Bird, D.~K., \& Rosing, M.~T. 2012, PNAS, 109, 4371

\bibitem[Quintana et al.(2016)]{quintana16} Quintana, E.~V., Barclay, T., Borucki, W.~J., Rowe, J.~F., et al. 2016, \apj, 821, 126

\bibitem[Ramirez \& Kaltenegger(2017)]{ramirez17} Ramirez, R.~M., \& Kaltenegger, L. 2017, \apjl, 837, L4

\bibitem[Ranjan et al.(2017)]{ranjan17} Ranjan, S., Wordsworth, R., \& Sasselov, D.~D. 2017, \apj, 843, 110

\bibitem[Raymond et al.(2004)]{raymond04} Raymond, S.~N., Quinn, T., \& Lunine, J.~I. 2004, \icarus, 168, 1

\bibitem[Raymond et al.(2009)]{raymond09} Raymond, S.~N., O'Brien, D.~P., Morbidelli, A., Kaib, N.~A. 2009, \icarus, 203, 644

\bibitem[Reese et al.(1998)]{reese98} Reese, C.~C., Solomatov, V.~S., Moresi, L.-N. 1998, Journal of Geophysical Research, 103, 13643

\bibitem[Ribas et al.(2005)]{ribas05} Ribas, I., Guinan, E.~F., G{\"u}del, M., \& Audard, M. 2005, ApJ, 622, 680

\bibitem[Ribas et al.(2010)]{ribas10} Ribas, I., Porto de Mello, G.~F., Ferreira, L.~D., H{\'e}brard, E., Selsis, F., et al. 2010, \apj, 714, 384

\bibitem[Ribas et al.(2016)]{ribas16} Ribas, I., Bolmont, E., Selsis, F., Reiners, A., Leconte, J., Raymond, S.~N., et al. 2016, \aap, 596, A111

\bibitem[Ribas et al.(2017)]{ribas17} Ribas, I., Gregg, M.~D., Boyajian, T.~S., \& Bolmont, E. 2017, \aap, 603, A58

\bibitem[Rimmer et al.(2018)]{rimmer18} Rimmer, P.~B., Xu, J., Thompson, S.~J., Gillen, E., Sutherland, J.~D., Queloz, D. 2018, Science Advances, 4, eaar3302

\bibitem[Robinson et al.(2010)]{robinson10} Robinson, T.~D., Meadows, V.~S., \& Crisp, D. 2010, \apjl, 721, L67

\bibitem[Robinson \& Catling(2012)]{robinson12} Robinson, T.~D., \& Catling, D.~C. 2012, \apj, 757, 104

\bibitem[Rogers et al.(2020)]{rogers20} Rogers, L.~K., Xu, S., Bonsor, A., Hodgkin, S., et al. 2020, MNRAS, 494, 2861

\bibitem[Rugheimer et al.(2015)]{rugheimer15} Rugheimer, S., Segura, A., Kaltenegger, L., \& Sasselov, D. 2015, \apj, 806, 137

\bibitem[Rushby et al.(2020)]{rushby20} Rushby, A.~J., Shields, A.~L., Wolf, E.~T., Lague, M., \& Burgasser, A. 2020, arXiv e-prints, 2011.03621

\bibitem[Samara et al.(2021)]{samara21} Samara, E., Patsourakos, S., \& Georgoulis, M.~K. 2021, arXiv e-prints, 2102.07837

\bibitem[Sandu, Lenardic \& McGovern(2011)]{sandu11} Sandu, C., Lenardic, A., \& McGovern, P. 2011, Journal of Geophysical Research, 116, B12404

\bibitem[Scalo et al.(2007)]{scalo07} Scalo, J., Kaltenegger, L., Segura, A.~G., Fridlund, M. Ribas, I., Kulikov, Y.~N., Grenfell, J.~L., Rauer, H., Odert, P., Leitzinger, M., Selsis, F., Khodachenko, M.~L., Eiroa, C., Kasting, J., \& Lammer, H. 2007, Astrobiology, 7, 85

\bibitem[Schaefer \& Sasselov(2015)]{schaefer15} Schaefer, L., \& Sasselov, D. 2015, \apj, 801, 40

\bibitem[Schaefer et al.(2016)]{schaefer16} Schaefer, L., Wordsworth, R.~D., Berta-Thompson, Z., \& Sasselov, D. 2016, \apj, 829, 63

\bibitem[Schaefer \& Elkins-Tanton(2018)]{schaefer18} Schaefer, L., \& Elkins-Tanton, L.~T. 2018, Phil.~Trans.~R.~Soc.~A., 376: 20180109

\bibitem[Schmidt \& Poli(1998)]{schmidt98} Schmidt, M.~W., \& Poli, S. 1998, E\&PSL, 163, 361

\bibitem[Schneider \& Shkolnik(2018)]{schneider18} Schneider, A.~C., \& Shkolnik, E.~L. 2018, \aj, 155, 122

\bibitem[Schulze et al.(2020)]{schulze20} Schulze, J.~G., Wang, J., Johnson, J.~A., Unterborn, C.~T., \& Panero, W.~R. 2020, arXiv e-prints, 2011.08893

\bibitem[Schwieterman et al.(2018)]{schwieterman18} Schwieterman, E.~W., Kiang, N~Y., Parenteau, M.~N., Harman, C.~E., et al. 2018, Astrobiology, 18, 663

\bibitem[Scoro et al.(2020)]{scoro20} Scoro, J., Valencia, D., Morbidelli, A., \& Jacobson, S. 2020, arxiv e-prints, 2002.09042v1

\bibitem[Scott et al.(1995)]{scott95} Scott, D.~K. 1995, Reviews of Geophysics, Supplement, 11-17

\bibitem[Seales \& Lenardic(2021)]{seales21} Seales, J., \& Lenardic, A. 2021, arXiv e-prints, 2102.01077

\bibitem[Segura et al.(2003)]{segura03} Segura, A., Krelove, K., Kasting, J.~F., Sommerlatt, D., Meadows, V., Crisp, D., Cohen, M., \& Mlawer, E. 2003, Astrobiology, 3, 689

\bibitem[Segura et al.(2010)]{segura10} Segura, A., Walkowicz, L.~M., Meadows, V., Kasting, J., \& Hawley, S. 2010, Astrobiology, 10, 751

\bibitem[Shah et al.(2020)]{shah20} Shah, O., Alibert, Y., Helled, R., \& Mezger, K. 2020, arXiv e-prints, 2012.06455

\bibitem[Shahar et al.(2019)]{shahar19} Shahar, A., Driscoll, P., Weinberger, A., \& Cody, G. 2019, Science, 364, 434

\bibitem[Shields, Ballard \& Johnson(2016)]{shields16} Shields, A.~L., Ballard, S., \& Johnson, J.~A. 2016, Physics Reports, 663, 1

\bibitem[Shkolnik \& Barman(2014)]{shkolnik14} Shkolnik, E.~L., \& Barman, T.~S. 2014, \aj, 148, 64

\bibitem[Sleep \& Zahnle(2001)]{sleep01} Sleep, N.~H., \& Zahnle, K. 2001, Journal of Geophysical Research, 106, 1373

\bibitem[Solomatov(1995)]{solomatov95} Solomatov, V.~S. 1995, Physics of Fluids, 7, 266

\bibitem[Spaargaren et al.(2020)]{spaargaren20} Spaargaren, R.~J., Ballmer, M.~D., Bower, D.~J., Dorn, C., \& Tackley, P.~J. 2020, arXiv e-prints, 2007.09021

\bibitem[Stamenkovi{\'c} et al.(2012)]{stamenkovic12} Stamenkovi{\'c}, V., Noack, L., Breuer, D., \& Spohn, T. 2012, \apj, 748, 41

\bibitem[Stein et al.(2004)]{stein04} Stein, c., Schmalzl, J., Hansen, U. 2004, Physics of the Earth and Planetary Interiors, 142, 225

\bibitem[Stern(2016)]{stern16} Stern, R.~J. 2016, Geoscience Frontiers, 7, 573

\bibitem[Swain et al.(2021)]{swain21} Swain, M.~R., Estrela, R., Roudier, G.~M., Sotin, C., et al. 2021, arXiv e-prints, 2103.05657

\bibitem[Tian et al.(2008)]{tian08} Tian, F., Kasting, J.~F., Liu, H.-L., \& Roble, R.~G. 2008, Journal of Geophysical Research, 113, E05008

\bibitem[Tian(2009)]{tian09} Tian, F. 2009, \apj, 703, 905

\bibitem[Tian(2015)]{tian15a} Teng, F. 2015, Annu.~Rev.~Earth Planet.~Sci., 43, 459

\bibitem[Tian \& Ida(2015)]{tian15b} Tian, F., \& Ida, S. 2015, Nature Geoscience, 8, 177

\bibitem[Tilley et al.(2019)]{tilley19} Tilley, M.~A., Segura, A., Meadows, V., Hawley, S., \& Davenport, J. 2019, Astrobiology, 19, 64

\bibitem[Tosi et al.(2017)]{tosi17} Tosi, N., Godolt, M., Stracke, B., Ruedas, T., et al. 2017, \aap, 605, A71

\bibitem[Turbet et al.(2016)]{turbet16} Turbet, M., Leconte, J., Selsis, F., Bolmont, E., Forget, F., Ribas, I., Raymond, S.~N., \& Anglada-Escud{\'e} 2016, \aap, 596, A112

\bibitem[Turbet et al.(2020)]{turbet20} Turbet, M., Bolmont, E., Ehrenreich, D., Gratier, P., Leconte, J., Selsis, F., Hara, N., Lovis, C. 2020, arXiv e-prints, 1911.08878v3

\bibitem[Valencia et al.(2006)]{valencia06} Valencia, D., O'Connell, R.~J., \& Sasselov, D. 2006, \icarus, 181, 545

\bibitem[Vallis(2020)]{vallis20} Vallis, G.~K. 2020, arxiv e-prints, 2006.02364

\bibitem[van Dishoeck et al.(2021)]{vandishoeck21} van Dishoeck, E.~F., Kristensen, L.~E., Mottram, J.~C., Benz, A.~O., et al. 2021, arXiv e-prints, 2102.02225

\bibitem[Venturini et al.(2020)]{venturini20} Venturini, J., Ronco, M.~P., Guilera, O.~M. 2020, arxiv e-prints, 2006.07127

\bibitem[Walker et al.(1981)]{walker81} Walker, J.~C.~G., Hays, P.~B., \& Kasting, J.~F., Journal of Geophysical Research, 86, 9776

\bibitem[Walowski et al.(2019)]{walowski19} Walowski, K.~J., Kirstein, L.~A., De~Hoog, J.~C.~M., Elliott, T.~R., Savov, I.~P., Jones, R.~E., EIMF 2019, Earth and Planetary Science Letters, 508, 97

\bibitem[Wang et al.(2020)]{wang20} Wang, H.~S., Morel, T., Quanz, S.~P., \& Mojzsis, S.~J. 2020, arXiv e-prints, 2010.04632

\bibitem[Watson et al.(1981)]{watson81} Watson, A.~J., Donahue, T.~M., \& Walker, J.~C.~G. 1981, Icarus, 48, 150

\bibitem[Way et al.(2018)]{way18} Way, M.~J., Del Genio, A.~D., Aleinov, I., Clune, T.~L., Kelley, M., \& Kiang, N.~Y. 2018, \apjs, 239, 24

\bibitem[Webster(1994)]{webster94} Webster, P.~J. 1994, Reviews of Geophysics, 32, 427

\bibitem[Weiss et al.(2018)]{weiss18} Weiss, L.~M., Marcy, G.~W., Petigura, E.~A., Fulton, B.~J., et al. 2018, \aj, 155, 48

\bibitem[Woitke et al.(2018)]{woitke18} Woitke, P., Helling, Ch. Hunter, G.~H., Millard, J.~D., et al. 2018, \aap, 614, A1

\bibitem[Woitke et al.(2020)]{woitke20} Woitke, P., Herbort, O., Helling, Ch., St\"{u}eken, E., et al. 2020, arXiv e-prints, 2010.12241

\bibitem[Wogan et al.(2020)]{wogan20} Wogan, N., Krissansen-Totton, J., \& Catling, D.~C. 2020, arXiv e-prints, 2009.07761

\bibitem[Wolfe et al.(2016)]{wolfe16} Wolfe, G.~M., Marvin, M.~R., Roberts, S.~J., Travis, K.~R., \& Liao, J. 2016, Geosci.~Model Dev., 9, 3309

\bibitem[Wolszczan \& Frail(1992)]{wolszczan92} Wolszczan, A., \& Frail, D.~A., 1992, Nature, 355, 145

\bibitem[Wordsworth \& Pierrehumbert(2013)]{wordsworth13} Wordsworth, R.~D., \& Pierrehumbert, R.~T. 2013, \apj, 778, 154

\bibitem[Wordsworth \& Pierrehumbert(2014)]{wordsworth14} Wordsworth, R., \& Pierrehumbert, R. 2014, \apjl, 785, L20

\bibitem[Wordsworth, Schaefer \& Fischer(2018)]{wordsworth18} Wordsworth, R.~D., Schaefer, L.~K., \& Fischer, R.~A. 2018, \aj, 155, 195

\bibitem[Wright et al.(2011)]{wright11} Wright, N.~J., Drake, J.~J., Mamajek, E.~E., \& Henry, G.~W. 2011, \apj, 743, 48

\bibitem[Wright \& Drake(2016)]{wright16} Wright, N.~J., \& Drake, J.~J. 2016, Nature, 535, 526

\bibitem[Wright et al.(2018)]{wright18} Wright, N.~J., Newton, E.~R., Williams, P.~K.~G., Drake, J.~J., \& Yadav, R.~.K. 2018, \mnras, 479, 2351

\bibitem[Wunderlich et al.(2019)]{wunderlich19} Wunderlich, F., Godolt, M., Grenfell, J.~L., St\"{a}dt, S., Smith, A.~M.~S., Gebauer, S., Schreier, F., Hedelt, P., \& Rauer, H. 2019, arXiv e-prints, 1905.02560

\bibitem[Wunderlich et al.(2020)]{wunderlich20} Wunderlich, F., Scheucher, M., Grenfell, J.~L., Schreier, F., et al. 2020, arXiv e-prints, 2012.11426

\bibitem[Yamashiki et al.(2019)]{yamashiki19} Yamashiki, Y.~A., Maehara, H., Airapetian, V., Notsu, Y., et al. 2019, arxiv e-prints, 1906.06797

\bibitem[Yang et al.(2013)]{yang13} Yang, J., Cowan, N.~B., \& Abbot, D.~S. 2013, \apjl, 771, L45 

\bibitem[Yang et al.(2017)]{yang17} Yang, J., Ding, F., Ramirez, R.~M., Peltier, W.~R., Hu, Y., \& Liu, Y. 2017, Nature Geoscience, 10, 556

\bibitem[Young et al.(2020)]{young20} Young, M.~E., Fossati, L., Johnstone, C., Salz, M., et al. 2020, arXiv e-prints, 2011.05613

\bibitem[Youngblood et al.(2016)]{youngblood16} Youngblood, A., France, K., Parke Loyd, R.~O., Linsky, J.~L., Redfield, S., Christian Schneider, P., et al. 2016, \apj, 824, 101

\bibitem[Zanazzi \& Triaud(2019)]{zanazzi19} Zanazzi, J.~J., \& Triaud, A.~H.~M.~J. 2019, Icarus, 325, 55

\bibitem[Zawadzki et al.(2021)]{zawadzki21} Zawadzki, B., Carrera, D., \& Ford, E. 2021, arXiv e-prints, 2103.01239

\bibitem[Zeng et al.(2016)]{zeng16} Zeng, L., Sasselov, D., \& Jacobson, S. 2016, \apj, 819, 127

\bibitem[Zhong et al.(2000)]{zhong00} Zhong, S., Zuber, M.~T., Moresi, L., Gurnis, M. 2000, Journal of Geophysical Research, 105, 11063

\bibitem[Ziegler \& Stedman(2013)]{ziegler13} Ziegler, L.~B., \& Stegman, D.~R. 2013, Geochem.~Geophys.~Geosys., 14, 4735

\bibitem[Zsom et al.(2013)]{zsom13} Zsom, A., Seager, S., De~Wit, J., \& Stamenkovic, V. 2013, \apj, 778, 109

\end{thebibliography}

%% This command is needed to show the entire author+affilation list when
%% the collaboration and author truncation commands are used.  It has to
%% go at the end of the manuscript.
%\allauthors

%% Include this line if you are using the \added, \replaced, \deleted
%% commands to see a summary list of all changes at the end of the article.
%\listofchanges

\end{document}

% End of file `sample62.tex'.
